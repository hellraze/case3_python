\question
Петя решил поучаствовать в конкурсе рисунков, к сожалению, проблема была в том, что он совершенно не умел рисовать, но Петя был умным мальчиком, который знал бинарные отношения.
\\На декартовом произведении множества $A = \{-2, -1, 0, 1, 2, 3, 4, 5, 6, 7, 8, 9, 10, 11\}$ заданы бинарные отношения:
\begin{equation*}
R_1 = {(0, -2), (2, 0), (6, -2), (8, 0), (9, -2), (12, -2), (11, -1), (11, 1), (12, 2), (9, 2), (8, 0), (6, 2), (5, 5), (0, 5), (6, 7), (0, 6)}
\end{equation*}
\begin{center}и\end{center}
\begin{equation*}
R_2 = \{(10, -1), (10, -1), (9, 0)\}
\end{equation*}
Помогите Пете выиграть в конкурсе! Постройте композиции отношений $R_1^{-1}, R_2^{-1}$ и изобразите полученный результат на декартовой системе координат (задание можно дополнять другими композициями).
\\
---------------

Автор -- Анастасия Стеценко, М3207