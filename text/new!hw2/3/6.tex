\question
Каждый студент очень боится зимних экзаменов. Благо в университете уже давно используется рецепт подготовки к сессии:
\\
\\
«Вам надо закупиться двумя пакетиками мяты и 75 ампулами валидола, если это уже куплено, то не обойдётся без солонки, наполовину наполненной пустырником. Или можно просто взять литр корвалола»
\\
\\
Не то, чтобы всё это было категорически необходимо при подготовке, но если уж решил закрыть сессию, то к делу надо подходить серьёзно. Чтобы совместить поход в аптеку с полезным, вы решили записать рецепт в виде булевой функции, где $А$ – два пакетика мяты куплены, $B$ – Солонка найдена, $C$ – 75 ампул валидола приобретены, $D$ – Литр корвалола взят. А также:
\begin{enumerate}
    \item Составьте СДНФ и СКНФ получившейся функции.
    \item Составьте полином Жегалкина получившейся функции любыми 2 способами
\end{enumerate}

---------------

Автор -- Константин Васильев, М3213