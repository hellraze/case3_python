\documentclass[10pt]{exam}
\RequirePackage{amssymb, amsfonts, amsmath, latexsym, verbatim, xspace, setspace}
\RequirePackage{tikz, pgflibraryplotmarks}

\usepackage[T2A]{fontenc}
\usepackage[utf8]{inputenc}
\usepackage[english,russian]{babel}

\usepackage[margin=0.75in]{geometry}

\usepackage[shortlabels]{enumitem}

\newcommand{\class}{Дискретная математика}
\newcommand{\term}{Осень 2018}
\newcommand{\examnum}{Типовой расчет 1}
\newcommand{\examdate}{}

\singlespacing

\parindent 0ex

\begin{document}

\pagestyle{head}
\firstpageheader{}{}{}
\runningheader{\class}{}{\examdate}
\runningheadrule
%%% begin test
\begin{flushright}
\begin{tabular}{p{2.8in} r l}
%\textbf{\class} & \textbf{ФИО:} & \makebox[2.5in]{\hrulefill}\\
\textbf{\class} & \textbf{ФИО:} &M3101
\\

\textbf{\examdate} &&\\
%\textbf{Time Limit: \timelimit} & Teaching Assistant & \makebox[2in]{\hrulefill}
\end{tabular}\\
\end{flushright}
\rule[1ex]{\textwidth}{.1pt}


\begin{questions}
\question
Найдите и упростите P:
\begin{equation*}
\overline{P} = A \cap C \cup \overline{A} \cap \overline{C} \cup \overline{B} \cap C \cup \overline{A} \cap \overline{B}
\end{equation*}
Затем найдите элементы множества P, выраженного через множества:
\begin{equation*}
A = \{0, 3, 4, 9\}; 
B = \{1, 3, 4, 7\};
C = \{0, 1, 2, 4, 7, 8, 9\};
I = \{0, 1, 2, 3, 4, 5, 6, 7, 8, 9\}.
\end{equation*}\question
Упростите следующее выражение с учетом того, что $A\subset B \subset C \subset D \subset U; A \neq \O$
\begin{equation*}
\overline{B} \cap \overline{C} \cap D \cup \overline{A} \cap \overline{C} \cap D \cup \overline{A} \cap B
\end{equation*}

Примечание: U — универсум\question
Дано отношение на множестве $\{1, 2, 3, 4, 5\}$ 
\begin{equation*}
aRb \iff a \leq b
\end{equation*}
Напишите обоснованный ответ какими свойствами обладает или не обладает отношение и почему:   
\begin{enumerate} [a)]\setcounter{enumi}{0}
\item рефлексивность
\item антирефлексивность
\item симметричность
\item асимметричность
\item антисимметричность
\item транзитивность
\end{enumerate}

Обоснуйте свой ответ по каждому из приведенных ниже вопросов:
\begin{enumerate} [a)]\setcounter{enumi}{0}
    \item Является ли это отношение отношением эквивалентности?
    \item Является ли это отношение функциональным?
    \item Каким из отношений соответствия (одно-многозначным, много-многозначный и т.д.) оно является?
    \item К каким из отношений порядка (полного, частичного и т.д.) можно отнести данное отношение?
\end{enumerate}


\question
Установите, является ли каждое из перечисленных ниже отношений на А ($R \subseteq A \times A$) отношением эквивалентности (обоснование ответа обязательно). Для каждого отношения эквивалентности постройте классы 
эквивалентности и постройте граф отношения:
\begin{enumerate} [a)]\setcounter{enumi}{0}
\item А - множество целых чисел и отношение $R = \{(a,b)|a + b = 5\}$
\item Пусть A – множество имен. $A = \{ $Алексей, Иван, Петр, Александр, Павел, Андрей$ \}$. Тогда отношение $R $ верно на парах имен, начинающихся с одной и той же буквы, и только на них.
\item На множестве $A = \{1; 2; 3; 4; 5\}$ задано отношение $R = \{(1; 2); (1; 3); (1; 5); (2; 3); (2; 4); (2; 5); (3; 4); (3; 5); (4; 5)\}$
\end{enumerate}\question Составьте полную таблицу истинности, определите, какие переменные являются фиктивными и проверьте, является ли формула тавтологией:
$ P \rightarrow (Q \rightarrow ((P \lor Q) \rightarrow (P \land Q)))$

\end{questions}
\newpage
%%% begin test
\begin{flushright}
\begin{tabular}{p{2.8in} r l}
%\textbf{\class} & \textbf{ФИО:} & \makebox[2.5in]{\hrulefill}\\
\textbf{\class} & \textbf{ФИО:} &Аронов Данила Алексеевич
\\

\textbf{\examdate} &&\\
%\textbf{Time Limit: \timelimit} & Teaching Assistant & \makebox[2in]{\hrulefill}
\end{tabular}\\
\end{flushright}
\rule[1ex]{\textwidth}{.1pt}


\begin{questions}
\question
Найдите и упростите P:
\begin{equation*}
\overline{P} = A \cap C \cup \overline{A} \cap \overline{C} \cup \overline{B} \cap C \cup \overline{A} \cap \overline{B}
\end{equation*}
Затем найдите элементы множества P, выраженного через множества:
\begin{equation*}
A = \{0, 3, 4, 9\}; 
B = \{1, 3, 4, 7\};
C = \{0, 1, 2, 4, 7, 8, 9\};
I = \{0, 1, 2, 3, 4, 5, 6, 7, 8, 9\}.
\end{equation*}\question
Упростите следующее выражение с учетом того, что $A\subset B \subset C \subset D \subset U; A \neq \O$
\begin{equation*}
A \cap B  \cap \overline{C} \cup \overline{C} \cap D \cup B \cap C \cap D
\end{equation*}

Примечание: U — универсум\question
Дано отношение на множестве $\{1, 2, 3, 4, 5\}$ 
\begin{equation*}
aRb \iff (a+b) \bmod 2 =0
\end{equation*}
Напишите обоснованный ответ какими свойствами обладает или не обладает отношение и почему:   
\begin{enumerate} [a)]\setcounter{enumi}{0}
\item рефлексивность
\item антирефлексивность
\item симметричность
\item асимметричность
\item антисимметричность
\item транзитивность
\end{enumerate}

Обоснуйте свой ответ по каждому из приведенных ниже вопросов:
\begin{enumerate} [a)]\setcounter{enumi}{0}
    \item Является ли это отношение отношением эквивалентности?
    \item Является ли это отношение функциональным?
    \item Каким из отношений соответствия (одно-многозначным, много-многозначный и т.д.) оно является?
    \item К каким из отношений порядка (полного, частичного и т.д.) можно отнести данное отношение?
\end{enumerate}



\question
Установите, является ли каждое из перечисленных ниже отношений на А ($R \subseteq A \times A$) отношением эквивалентности (обоснование ответа обязательно). Для каждого отношения эквивалентности постройте классы 
эквивалентности и постройте граф отношения:
\begin{enumerate} [a)]\setcounter{enumi}{0}
\item На множестве $A = \{1; 2; 3\}$ задано отношение $R = \{(1; 1); (2; 2); (3; 3); (2; 1); (1; 2); (2; 3); (3; 2); (3; 1); (1; 3)\}$
\item На множестве $A = \{1; 2; 3; 4; 5\}$ задано отношение $R = \{(1; 2); (1; 3); (1; 5); (2; 3); (2; 4); (2; 5); (3; 4); (3; 5); (4; 5)\}$
\item А - множество целых чисел и отношение $R = \{(a,b)|a + b = 0\}$
\end{enumerate}\question Составьте полную таблицу истинности, определите, какие переменные являются фиктивными и проверьте, является ли формула тавтологией:
$((P \rightarrow Q) \lor R) \leftrightarrow (P \rightarrow (Q \lor R))$

\end{questions}
\newpage
%%% begin test
\begin{flushright}
\begin{tabular}{p{2.8in} r l}
%\textbf{\class} & \textbf{ФИО:} & \makebox[2.5in]{\hrulefill}\\
\textbf{\class} & \textbf{ФИО:} &Гайнутдинов Самат Маратович
\\

\textbf{\examdate} &&\\
%\textbf{Time Limit: \timelimit} & Teaching Assistant & \makebox[2in]{\hrulefill}
\end{tabular}\\
\end{flushright}
\rule[1ex]{\textwidth}{.1pt}


\begin{questions}
\question
Найдите и упростите P:
\begin{equation*}
\overline{P} = A \cap C \cup \overline{A} \cap \overline{C} \cup \overline{B} \cap C \cup \overline{A} \cap \overline{B}
\end{equation*}
Затем найдите элементы множества P, выраженного через множества:
\begin{equation*}
A = \{0, 3, 4, 9\}; 
B = \{1, 3, 4, 7\};
C = \{0, 1, 2, 4, 7, 8, 9\};
I = \{0, 1, 2, 3, 4, 5, 6, 7, 8, 9\}.
\end{equation*}\question
Упростите следующее выражение с учетом того, что $A\subset B \subset C \subset D \subset U; A \neq \O$
\begin{equation*}
A \cap C  \cap D \cup B \cap \overline{C} \cap D \cup B \cap C \cap D
\end{equation*}

Примечание: U — универсум\question
Дано отношение на множестве $\{1, 2, 3, 4, 5\}$ 
\begin{equation*}
aRb \iff |a-b| = 1
\end{equation*}
Напишите обоснованный ответ какими свойствами обладает или не обладает отношение и почему:   
\begin{enumerate} [a)]\setcounter{enumi}{0}
\item рефлексивность
\item антирефлексивность
\item симметричность
\item асимметричность
\item антисимметричность
\item транзитивность
\end{enumerate}

Обоснуйте свой ответ по каждому из приведенных ниже вопросов:
\begin{enumerate} [a)]\setcounter{enumi}{0}
    \item Является ли это отношение отношением эквивалентности?
    \item Является ли это отношение функциональным?
    \item Каким из отношений соответствия (одно-многозначным, много-многозначный и т.д.) оно является?
    \item К каким из отношений порядка (полного, частичного и т.д.) можно отнести данное отношение?
\end{enumerate}

\question
Установите, является ли каждое из перечисленных ниже отношений на А ($R \subseteq A \times A$) отношением эквивалентности (обоснование ответа обязательно). Для каждого отношения эквивалентности постройте классы 
эквивалентности и постройте граф отношения:
\begin{enumerate} [a)]\setcounter{enumi}{0}
\item Пусть A – множество имен. $A = \{ $Алексей, Иван, Петр, Александр, Павел, Андрей$ \}$. Тогда отношение $R$ верно на парах имен, начинающихся с одной и той же буквы, и только на них.
\item $A = \{-10, -9, … , 9, 10\}$ и отношение $ R = \{(a,b)|a^{2} = b^{2}\}$
\item На множестве $A = \{1; 2; 3\}$ задано отношение $R = \{(1; 1); (2; 2); (3; 3); (3; 2); (1; 2); (2; 1)\}$
\end{enumerate}\question Составьте полную таблицу истинности, определите, какие переменные являются фиктивными и проверьте, является ли формула тавтологией:
$((P \rightarrow Q) \lor R) \leftrightarrow (P \rightarrow (Q \lor R))$

\end{questions}
\newpage
%%% begin test
\begin{flushright}
\begin{tabular}{p{2.8in} r l}
%\textbf{\class} & \textbf{ФИО:} & \makebox[2.5in]{\hrulefill}\\
\textbf{\class} & \textbf{ФИО:} &Воеводский Дмитрий Денисович
\\

\textbf{\examdate} &&\\
%\textbf{Time Limit: \timelimit} & Teaching Assistant & \makebox[2in]{\hrulefill}
\end{tabular}\\
\end{flushright}
\rule[1ex]{\textwidth}{.1pt}


\begin{questions}
\question
Найдите и упростите P:
\begin{equation*}
\overline{P} = A \cap \overline{B} \cup \overline{B} \cap C \cup \overline{A} \cap \overline{B} \cup \overline{A} \cap C
\end{equation*}
Затем найдите элементы множества P, выраженного через множества:
\begin{equation*}
A = \{0, 3, 4, 9\}; 
B = \{1, 3, 4, 7\};
C = \{0, 1, 2, 4, 7, 8, 9\};
I = \{0, 1, 2, 3, 4, 5, 6, 7, 8, 9\}.
\end{equation*}\question
Упростите следующее выражение с учетом того, что $A\subset B \subset C \subset D \subset U; A \neq \O$
\begin{equation*}
A \cap B \cup \overline{A} \cap \overline{C} \cup A \cap C \cup \overline{B} \cap \overline{C}
\end{equation*}

Примечание: U — универсум\question
Дано отношение на множестве $\{1, 2, 3, 4, 5\}$ 
\begin{equation*}
aRb \iff a \leq b
\end{equation*}
Напишите обоснованный ответ какими свойствами обладает или не обладает отношение и почему:   
\begin{enumerate} [a)]\setcounter{enumi}{0}
\item рефлексивность
\item антирефлексивность
\item симметричность
\item асимметричность
\item антисимметричность
\item транзитивность
\end{enumerate}

Обоснуйте свой ответ по каждому из приведенных ниже вопросов:
\begin{enumerate} [a)]\setcounter{enumi}{0}
    \item Является ли это отношение отношением эквивалентности?
    \item Является ли это отношение функциональным?
    \item Каким из отношений соответствия (одно-многозначным, много-многозначный и т.д.) оно является?
    \item К каким из отношений порядка (полного, частичного и т.д.) можно отнести данное отношение?
\end{enumerate}


\question
Установите, является ли каждое из перечисленных ниже отношений на А ($R \subseteq A \times A$) отношением эквивалентности (обоснование ответа обязательно). Для каждого отношения эквивалентности постройте классы эквивалентности и постройте граф отношения:
\begin{enumerate} [a)]\setcounter{enumi}{0}
\item $F(x)=x^{2}+1$, где $x \in A = [-2, 4]$ и отношение $R = \{(a,b)|F(a) = F(b)\}$
\item А - множество целых чисел и отношение $R = \{(a,b)|a + b = 5\}$
\item На множестве $A = \{1; 2; 3\}$ задано отношение $R = \{(1; 1); (2; 2); (3; 3); (3; 2); (1; 2); (2; 1)\}$

\end{enumerate}\question Составьте полную таблицу истинности, определите, какие переменные являются фиктивными и проверьте, является ли формула тавтологией:

$(P \rightarrow (Q \land R)) \leftrightarrow ((P \rightarrow Q) \land (P \rightarrow R))$

\end{questions}
\newpage
%%% begin test
\begin{flushright}
\begin{tabular}{p{2.8in} r l}
%\textbf{\class} & \textbf{ФИО:} & \makebox[2.5in]{\hrulefill}\\
\textbf{\class} & \textbf{ФИО:} &Лебедь Михаил Сергеевич
\\

\textbf{\examdate} &&\\
%\textbf{Time Limit: \timelimit} & Teaching Assistant & \makebox[2in]{\hrulefill}
\end{tabular}\\
\end{flushright}
\rule[1ex]{\textwidth}{.1pt}


\begin{questions}
\question
Найдите и упростите P:
\begin{equation*}
\overline{P} = \overline{A} \cap B \cup \overline{A} \cap C \cup A \cap \overline{B} \cup \overline{B} \cap C
\end{equation*}
Затем найдите элементы множества P, выраженного через множества:
\begin{equation*}
A = \{0, 3, 4, 9\}; 
B = \{1, 3, 4, 7\};
C = \{0, 1, 2, 4, 7, 8, 9\};
I = \{0, 1, 2, 3, 4, 5, 6, 7, 8, 9\}.
\end{equation*}\question
Упростите следующее выражение с учетом того, что $A\subset B \subset C \subset D \subset U; A \neq \O$
\begin{equation*}
A \cap C  \cap D \cup B \cap \overline{C} \cap D \cup B \cap C \cap D
\end{equation*}

Примечание: U — универсум\question
Дано отношение на множестве $\{1, 2, 3, 4, 5\}$ 
\begin{equation*}
aRb \iff b > a
\end{equation*}
Напишите обоснованный ответ какими свойствами обладает или не обладает отношение и почему:   
\begin{enumerate} [a)]\setcounter{enumi}{0}
\item рефлексивность
\item антирефлексивность
\item симметричность
\item асимметричность
\item антисимметричность
\item транзитивность
\end{enumerate}

Обоснуйте свой ответ по каждому из приведенных ниже вопросов:
\begin{enumerate} [a)]\setcounter{enumi}{0}
    \item Является ли это отношение отношением эквивалентности?
    \item Является ли это отношение функциональным?
    \item Каким из отношений соответствия (одно-многозначным, много-многозначный и т.д.) оно является?
    \item К каким из отношений порядка (полного, частичного и т.д.) можно отнести данное отношение?
\end{enumerate}

\question
Установите, является ли каждое из перечисленных ниже отношений на А ($R \subseteq A \times A$) отношением эквивалентности (обоснование ответа обязательно). Для каждого отношения эквивалентности постройте классы 
эквивалентности и постройте граф отношения:
\begin{enumerate} [a)]\setcounter{enumi}{0}
\item $A = \{-10, -9, … , 9, 10\}$ и отношение $R = \{(a,b)|a^{2} = b^{2}\}$
\item $A = \{a, b, c, d, p, t\}$ задано отношение $R = \{(a, a), (b, b), (b, c), (b, d), (c, b), (c, c), (c, d), (d, b), (d, c), (d, d), (p,p), (t,t)\}$
\item Пусть A – множество имен. $A = \{ $Алексей, Иван, Петр, Александр, Павел, Андрей$ \}$. Тогда отношение $R$ верно на парах имен, начинающихся с одной и той же буквы, и только на них.
\end{enumerate}\question Составьте полную таблицу истинности, определите, какие переменные являются фиктивными и проверьте, является ли формула тавтологией:
$(( P \rightarrow Q) \land (Q \rightarrow P)) \rightarrow (P \rightarrow R)$

\end{questions}
\newpage
%%% begin test
\begin{flushright}
\begin{tabular}{p{2.8in} r l}
%\textbf{\class} & \textbf{ФИО:} & \makebox[2.5in]{\hrulefill}\\
\textbf{\class} & \textbf{ФИО:} &Головин Максим Тимурович
\\

\textbf{\examdate} &&\\
%\textbf{Time Limit: \timelimit} & Teaching Assistant & \makebox[2in]{\hrulefill}
\end{tabular}\\
\end{flushright}
\rule[1ex]{\textwidth}{.1pt}


\begin{questions}
\question
Найдите и упростите P:
\begin{equation*}
\overline{P} = A \cap C \cup \overline{A} \cap \overline{C} \cup \overline{B} \cap C \cup \overline{A} \cap \overline{B}
\end{equation*}
Затем найдите элементы множества P, выраженного через множества:
\begin{equation*}
A = \{0, 3, 4, 9\}; 
B = \{1, 3, 4, 7\};
C = \{0, 1, 2, 4, 7, 8, 9\};
I = \{0, 1, 2, 3, 4, 5, 6, 7, 8, 9\}.
\end{equation*}\question
Упростите следующее выражение с учетом того, что $A\subset B \subset C \subset D \subset U; A \neq \O$
\begin{equation*}
\overline{A} \cap \overline{B} \cup B \cap \overline{C} \cup \overline{C} \cap D
\end{equation*}

Примечание: U — универсум\question
Дано отношение на множестве $\{1, 2, 3, 4, 5\}$ 
\begin{equation*}
aRb \iff b > a
\end{equation*}
Напишите обоснованный ответ какими свойствами обладает или не обладает отношение и почему:   
\begin{enumerate} [a)]\setcounter{enumi}{0}
\item рефлексивность
\item антирефлексивность
\item симметричность
\item асимметричность
\item антисимметричность
\item транзитивность
\end{enumerate}

Обоснуйте свой ответ по каждому из приведенных ниже вопросов:
\begin{enumerate} [a)]\setcounter{enumi}{0}
    \item Является ли это отношение отношением эквивалентности?
    \item Является ли это отношение функциональным?
    \item Каким из отношений соответствия (одно-многозначным, много-многозначный и т.д.) оно является?
    \item К каким из отношений порядка (полного, частичного и т.д.) можно отнести данное отношение?
\end{enumerate}

\question
Установите, является ли каждое из перечисленных ниже отношений на А ($R \subseteq A \times A$) отношением эквивалентности (обоснование ответа обязательно). Для каждого отношения эквивалентности постройте классы 
эквивалентности и постройте граф отношения:
\begin{enumerate} [a)]\setcounter{enumi}{0}
\item $A = \{-10, -9, … , 9, 10\}$ и отношение $R = \{(a,b)|a^{2} = b^{2}\}$
\item $A = \{a, b, c, d, p, t\}$ задано отношение $R = \{(a, a), (b, b), (b, c), (b, d), (c, b), (c, c), (c, d), (d, b), (d, c), (d, d), (p,p), (t,t)\}$
\item Пусть A – множество имен. $A = \{ $Алексей, Иван, Петр, Александр, Павел, Андрей$ \}$. Тогда отношение $R$ верно на парах имен, начинающихся с одной и той же буквы, и только на них.
\end{enumerate}\question Составьте полную таблицу истинности, определите, какие переменные являются фиктивными и проверьте, является ли формула тавтологией:
$ P \rightarrow (Q \rightarrow ((P \lor Q) \rightarrow (P \land Q)))$

\end{questions}
\newpage
%%% begin test
\begin{flushright}
\begin{tabular}{p{2.8in} r l}
%\textbf{\class} & \textbf{ФИО:} & \makebox[2.5in]{\hrulefill}\\
\textbf{\class} & \textbf{ФИО:} &Григорович Вячеслав Дмитриевич
\\

\textbf{\examdate} &&\\
%\textbf{Time Limit: \timelimit} & Teaching Assistant & \makebox[2in]{\hrulefill}
\end{tabular}\\
\end{flushright}
\rule[1ex]{\textwidth}{.1pt}


\begin{questions}
\question
Найдите и упростите P:
\begin{equation*}
\overline{P} = A \cap \overline{C} \cup A \cap \overline{B} \cup B \cap \overline{C} \cup A \cap C
\end{equation*}
Затем найдите элементы множества P, выраженного через множества:
\begin{equation*}
A = \{0, 3, 4, 9\}; 
B = \{1, 3, 4, 7\};
C = \{0, 1, 2, 4, 7, 8, 9\};
I = \{0, 1, 2, 3, 4, 5, 6, 7, 8, 9\}.
\end{equation*}\question
Упростите следующее выражение с учетом того, что $A\subset B \subset C \subset D \subset U; A \neq \O$
\begin{equation*}
A \cap  \overline{C} \cup B \cap \overline{D} \cup  \overline{A} \cap C \cap  \overline{D}
\end{equation*}

Примечание: U — универсум\question
Дано отношение на множестве $\{1, 2, 3, 4, 5\}$ 
\begin{equation*}
aRb \iff  \text{НОД}(a,b) =1
\end{equation*}
Напишите обоснованный ответ какими свойствами обладает или не обладает отношение и почему:   
\begin{enumerate} [a)]\setcounter{enumi}{0}
\item рефлексивность
\item антирефлексивность
\item симметричность
\item асимметричность
\item антисимметричность
\item транзитивность
\end{enumerate}

Обоснуйте свой ответ по каждому из приведенных ниже вопросов:
\begin{enumerate} [a)]\setcounter{enumi}{0}
    \item Является ли это отношение отношением эквивалентности?
    \item Является ли это отношение функциональным?
    \item Каким из отношений соответствия (одно-многозначным, много-многозначный и т.д.) оно является?
    \item К каким из отношений порядка (полного, частичного и т.д.) можно отнести данное отношение?
\end{enumerate}


\question
Установите, является ли каждое из перечисленных ниже отношений на А ($R \subseteq A \times A$) отношением эквивалентности (обоснование ответа обязательно). Для каждого отношения эквивалентности 
постройте классы эквивалентности и постройте граф отношения:
\begin{enumerate}[a)]\setcounter{enumi}{0}
\item А - множество целых чисел и отношение $R = \{(a,b)|a + b = 0\}$
\item $A = \{-10, -9, …, 9, 10\}$ и отношение $R = \{(a,b)|a^{3} = b^{3}\}$
\item На множестве $A = \{1; 2; 3\}$ задано отношение $R = \{(1; 1); (2; 2); (3; 3); (2; 1); (1; 2); (2; 3); (3; 2); (3; 1); (1; 3)\}$

\end{enumerate}\question Составьте полную таблицу истинности, определите, какие переменные являются фиктивными и проверьте, является ли формула тавтологией:

$(P \rightarrow (Q \land R)) \leftrightarrow ((P \rightarrow Q) \land (P \rightarrow R))$

\end{questions}
\newpage
%%% begin test
\begin{flushright}
\begin{tabular}{p{2.8in} r l}
%\textbf{\class} & \textbf{ФИО:} & \makebox[2.5in]{\hrulefill}\\
\textbf{\class} & \textbf{ФИО:} &Дятлов Максим Олегович
\\

\textbf{\examdate} &&\\
%\textbf{Time Limit: \timelimit} & Teaching Assistant & \makebox[2in]{\hrulefill}
\end{tabular}\\
\end{flushright}
\rule[1ex]{\textwidth}{.1pt}


\begin{questions}
\question
Найдите и упростите P:
\begin{equation*}
\overline{P} = A \cap \overline{B} \cup \overline{B} \cap C \cup \overline{A} \cap \overline{B} \cup \overline{A} \cap C
\end{equation*}
Затем найдите элементы множества P, выраженного через множества:
\begin{equation*}
A = \{0, 3, 4, 9\}; 
B = \{1, 3, 4, 7\};
C = \{0, 1, 2, 4, 7, 8, 9\};
I = \{0, 1, 2, 3, 4, 5, 6, 7, 8, 9\}.
\end{equation*}\question
Упростите следующее выражение с учетом того, что $A\subset B \subset C \subset D \subset U; A \neq \O$
\begin{equation*}
A \cap B \cup \overline{A} \cap \overline{C} \cup A \cap C \cup \overline{B} \cap \overline{C}
\end{equation*}

Примечание: U — универсум\question
Дано отношение на множестве $\{1, 2, 3, 4, 5\}$ 
\begin{equation*}
aRb \iff a \geq b^2
\end{equation*}
Напишите обоснованный ответ какими свойствами обладает или не обладает отношение и почему:   
\begin{enumerate} [a)]\setcounter{enumi}{0}
\item рефлексивность
\item антирефлексивность
\item симметричность
\item асимметричность
\item антисимметричность
\item транзитивность
\end{enumerate}

Обоснуйте свой ответ по каждому из приведенных ниже вопросов:
\begin{enumerate} [a)]\setcounter{enumi}{0}
    \item Является ли это отношение отношением эквивалентности?
    \item Является ли это отношение функциональным?
    \item Каким из отношений соответствия (одно-многозначным, много-многозначный и т.д.) оно является?
    \item К каким из отношений порядка (полного, частичного и т.д.) можно отнести данное отношение?
\end{enumerate}


\question
Установите, является ли каждое из перечисленных ниже отношений на А ($R \subseteq A \times A$) отношением эквивалентности (обоснование ответа обязательно). Для каждого отношения эквивалентности постройте классы 
эквивалентности и постройте граф отношения:
\begin{enumerate} [a)]\setcounter{enumi}{0}
\item $A = \{-10, -9, … , 9, 10\}$ и отношение $R = \{(a,b)|a^{2} = b^{2}\}$
\item $A = \{a, b, c, d, p, t\}$ задано отношение $R = \{(a, a), (b, b), (b, c), (b, d), (c, b), (c, c), (c, d), (d, b), (d, c), (d, d), (p,p), (t,t)\}$
\item Пусть A – множество имен. $A = \{ $Алексей, Иван, Петр, Александр, Павел, Андрей$ \}$. Тогда отношение $R$ верно на парах имен, начинающихся с одной и той же буквы, и только на них.
\end{enumerate}\question Составьте полную таблицу истинности, определите, какие переменные являются фиктивными и проверьте, является ли формула тавтологией:

$(P \rightarrow (Q \land R)) \leftrightarrow ((P \rightarrow Q) \land (P \rightarrow R))$

\end{questions}
\newpage
%%% begin test
\begin{flushright}
\begin{tabular}{p{2.8in} r l}
%\textbf{\class} & \textbf{ФИО:} & \makebox[2.5in]{\hrulefill}\\
\textbf{\class} & \textbf{ФИО:} &Иванов Александр Сергеевич
\\

\textbf{\examdate} &&\\
%\textbf{Time Limit: \timelimit} & Teaching Assistant & \makebox[2in]{\hrulefill}
\end{tabular}\\
\end{flushright}
\rule[1ex]{\textwidth}{.1pt}


\begin{questions}
\question
Найдите и упростите P:
\begin{equation*}
\overline{P} = \overline{A} \cap B \cup \overline{A} \cap C \cup A \cap \overline{B} \cup \overline{B} \cap C
\end{equation*}
Затем найдите элементы множества P, выраженного через множества:
\begin{equation*}
A = \{0, 3, 4, 9\}; 
B = \{1, 3, 4, 7\};
C = \{0, 1, 2, 4, 7, 8, 9\};
I = \{0, 1, 2, 3, 4, 5, 6, 7, 8, 9\}.
\end{equation*}\question
Упростите следующее выражение с учетом того, что $A\subset B \subset C \subset D \subset U; A \neq \O$
\begin{equation*}
\overline{A} \cap \overline{C} \cap D \cup \overline{B} \cap \overline{C} \cap D \cup A \cap B
\end{equation*}

Примечание: U — универсум\question
Дано отношение на множестве $\{1, 2, 3, 4, 5\}$ 
\begin{equation*}
aRb \iff a \leq b
\end{equation*}
Напишите обоснованный ответ какими свойствами обладает или не обладает отношение и почему:   
\begin{enumerate} [a)]\setcounter{enumi}{0}
\item рефлексивность
\item антирефлексивность
\item симметричность
\item асимметричность
\item антисимметричность
\item транзитивность
\end{enumerate}

Обоснуйте свой ответ по каждому из приведенных ниже вопросов:
\begin{enumerate} [a)]\setcounter{enumi}{0}
    \item Является ли это отношение отношением эквивалентности?
    \item Является ли это отношение функциональным?
    \item Каким из отношений соответствия (одно-многозначным, много-многозначный и т.д.) оно является?
    \item К каким из отношений порядка (полного, частичного и т.д.) можно отнести данное отношение?
\end{enumerate}


\question
Установите, является ли каждое из перечисленных ниже отношений на А ($R \subseteq A \times A$) отношением эквивалентности (обоснование ответа обязательно). Для каждого отношения эквивалентности постройте классы 
эквивалентности и постройте граф отношения:
\begin{enumerate} [a)]\setcounter{enumi}{0}
\item А - множество целых чисел и отношение $R = \{(a,b)|a + b = 5\}$
\item Пусть A – множество имен. $A = \{ $Алексей, Иван, Петр, Александр, Павел, Андрей$ \}$. Тогда отношение $R $ верно на парах имен, начинающихся с одной и той же буквы, и только на них.
\item На множестве $A = \{1; 2; 3; 4; 5\}$ задано отношение $R = \{(1; 2); (1; 3); (1; 5); (2; 3); (2; 4); (2; 5); (3; 4); (3; 5); (4; 5)\}$
\end{enumerate}\question Составьте полную таблицу истинности, определите, какие переменные являются фиктивными и проверьте, является ли формула тавтологией:

$(P \rightarrow (Q \land R)) \leftrightarrow ((P \rightarrow Q) \land (P \rightarrow R))$

\end{questions}
\newpage
%%% begin test
\begin{flushright}
\begin{tabular}{p{2.8in} r l}
%\textbf{\class} & \textbf{ФИО:} & \makebox[2.5in]{\hrulefill}\\
\textbf{\class} & \textbf{ФИО:} &Иоффе Александр Алексеевич
\\

\textbf{\examdate} &&\\
%\textbf{Time Limit: \timelimit} & Teaching Assistant & \makebox[2in]{\hrulefill}
\end{tabular}\\
\end{flushright}
\rule[1ex]{\textwidth}{.1pt}


\begin{questions}
\question
Найдите и упростите P:
\begin{equation*}
\overline{P} = \overline{A} \cap B \cup \overline{A} \cap C \cup A \cap \overline{B} \cup \overline{B} \cap C
\end{equation*}
Затем найдите элементы множества P, выраженного через множества:
\begin{equation*}
A = \{0, 3, 4, 9\}; 
B = \{1, 3, 4, 7\};
C = \{0, 1, 2, 4, 7, 8, 9\};
I = \{0, 1, 2, 3, 4, 5, 6, 7, 8, 9\}.
\end{equation*}\question
Упростите следующее выражение с учетом того, что $A\subset B \subset C \subset D \subset U; A \neq \O$
\begin{equation*}
A \cap C  \cap D \cup B \cap \overline{C} \cap D \cup B \cap C \cap D
\end{equation*}

Примечание: U — универсум\question
Для следующего отношения на множестве $\{1, 2, 3, 4, 5\}$ 
\begin{equation*}
aRb \iff 0 < a-b<2
\end{equation*}
Напишите обоснованный ответ какими свойствами обладает или не обладает отношение и почему:   
\begin{enumerate} [a)]\setcounter{enumi}{0}
\item рефлексивность
\item антирефлексивность
\item симметричность
\item асимметричность
\item антисимметричность
\item транзитивность
\end{enumerate}

Обоснуйте свой ответ по каждому из приведенных ниже вопросов:
\begin{enumerate} [a)]\setcounter{enumi}{0}
    \item Является ли это отношение отношением эквивалентности?
    \item Является ли это отношение функциональным?
    \item Каким из отношений соответствия (одно-многозначным, много-многозначный и т.д.) оно является?
    \item К каким из отношений порядка (полного, частичного и т.д.) можно отнести данное отношение?
\end{enumerate}
\question
Установите, является ли каждое из перечисленных ниже отношений на А ($R \subseteq A \times A$) отношением эквивалентности (обоснование ответа обязательно). Для каждого отношения эквивалентности постройте классы 
эквивалентности и постройте граф отношения:
\begin{enumerate} [a)]\setcounter{enumi}{0}
\item $A = \{-10, -9, … , 9, 10\}$ и отношение $R = \{(a,b)|a^{2} = b^{2}\}$
\item $A = \{a, b, c, d, p, t\}$ задано отношение $R = \{(a, a), (b, b), (b, c), (b, d), (c, b), (c, c), (c, d), (d, b), (d, c), (d, d), (p,p), (t,t)\}$
\item Пусть A – множество имен. $A = \{ $Алексей, Иван, Петр, Александр, Павел, Андрей$ \}$. Тогда отношение $R$ верно на парах имен, начинающихся с одной и той же буквы, и только на них.
\end{enumerate}\question Составьте полную таблицу истинности, определите, какие переменные являются фиктивными и проверьте, является ли формула тавтологией:
$ P \rightarrow (Q \rightarrow ((P \lor Q) \rightarrow (P \land Q)))$

\end{questions}
\newpage
%%% begin test
\begin{flushright}
\begin{tabular}{p{2.8in} r l}
%\textbf{\class} & \textbf{ФИО:} & \makebox[2.5in]{\hrulefill}\\
\textbf{\class} & \textbf{ФИО:} &Корчагин Артём Александрович
\\

\textbf{\examdate} &&\\
%\textbf{Time Limit: \timelimit} & Teaching Assistant & \makebox[2in]{\hrulefill}
\end{tabular}\\
\end{flushright}
\rule[1ex]{\textwidth}{.1pt}


\begin{questions}
\question
Найдите и упростите P:
\begin{equation*}
\overline{P} = \overline{A} \cap B \cup \overline{A} \cap C \cup A \cap \overline{B} \cup \overline{B} \cap C
\end{equation*}
Затем найдите элементы множества P, выраженного через множества:
\begin{equation*}
A = \{0, 3, 4, 9\}; 
B = \{1, 3, 4, 7\};
C = \{0, 1, 2, 4, 7, 8, 9\};
I = \{0, 1, 2, 3, 4, 5, 6, 7, 8, 9\}.
\end{equation*}\question
Упростите следующее выражение с учетом того, что $A\subset B \subset C \subset D \subset U; A \neq \O$
\begin{equation*}
\overline{A} \cap \overline{B} \cup B \cap \overline{C} \cup \overline{C} \cap D
\end{equation*}

Примечание: U — универсум\question
Дано отношение на множестве $\{1, 2, 3, 4, 5\}$ 
\begin{equation*}
aRb \iff a \geq b^2
\end{equation*}
Напишите обоснованный ответ какими свойствами обладает или не обладает отношение и почему:   
\begin{enumerate} [a)]\setcounter{enumi}{0}
\item рефлексивность
\item антирефлексивность
\item симметричность
\item асимметричность
\item антисимметричность
\item транзитивность
\end{enumerate}

Обоснуйте свой ответ по каждому из приведенных ниже вопросов:
\begin{enumerate} [a)]\setcounter{enumi}{0}
    \item Является ли это отношение отношением эквивалентности?
    \item Является ли это отношение функциональным?
    \item Каким из отношений соответствия (одно-многозначным, много-многозначный и т.д.) оно является?
    \item К каким из отношений порядка (полного, частичного и т.д.) можно отнести данное отношение?
\end{enumerate}


\question
Установите, является ли каждое из перечисленных ниже отношений на А ($R \subseteq A \times A$) отношением эквивалентности (обоснование ответа обязательно). Для каждого отношения эквивалентности постройте классы 
эквивалентности и постройте граф отношения:
\begin{enumerate} [a)]\setcounter{enumi}{0}
\item А - множество целых чисел и отношение $R = \{(a,b)|a + b = 5\}$
\item Пусть A – множество имен. $A = \{ $Алексей, Иван, Петр, Александр, Павел, Андрей$ \}$. Тогда отношение $R $ верно на парах имен, начинающихся с одной и той же буквы, и только на них.
\item На множестве $A = \{1; 2; 3; 4; 5\}$ задано отношение $R = \{(1; 2); (1; 3); (1; 5); (2; 3); (2; 4); (2; 5); (3; 4); (3; 5); (4; 5)\}$
\end{enumerate}\question Составьте полную таблицу истинности, определите, какие переменные являются фиктивными и проверьте, является ли формула тавтологией:

$(P \rightarrow (Q \land R)) \leftrightarrow ((P \rightarrow Q) \land (P \rightarrow R))$

\end{questions}
\newpage
%%% begin test
\begin{flushright}
\begin{tabular}{p{2.8in} r l}
%\textbf{\class} & \textbf{ФИО:} & \makebox[2.5in]{\hrulefill}\\
\textbf{\class} & \textbf{ФИО:} &Кулябин Денис Юрьевич
\\

\textbf{\examdate} &&\\
%\textbf{Time Limit: \timelimit} & Teaching Assistant & \makebox[2in]{\hrulefill}
\end{tabular}\\
\end{flushright}
\rule[1ex]{\textwidth}{.1pt}


\begin{questions}
\question
Найдите и упростите P:
\begin{equation*}
\overline{P} = B \cap \overline{C} \cup A \cap B \cup \overline{A} \cap C \cup \overline{A} \cap B
\end{equation*}
Затем найдите элементы множества P, выраженного через множества:
\begin{equation*}
A = \{0, 3, 4, 9\}; 
B = \{1, 3, 4, 7\};
C = \{0, 1, 2, 4, 7, 8, 9\};
I = \{0, 1, 2, 3, 4, 5, 6, 7, 8, 9\}.
\end{equation*}\question
Упростите следующее выражение с учетом того, что $A\subset B \subset C \subset D \subset U; A \neq \O$
\begin{equation*}
A \cap C  \cap D \cup B \cap \overline{C} \cap D \cup B \cap C \cap D
\end{equation*}

Примечание: U — универсум\question
Дано отношение на множестве $\{1, 2, 3, 4, 5\}$ 
\begin{equation*}
aRb \iff a \leq b
\end{equation*}
Напишите обоснованный ответ какими свойствами обладает или не обладает отношение и почему:   
\begin{enumerate} [a)]\setcounter{enumi}{0}
\item рефлексивность
\item антирефлексивность
\item симметричность
\item асимметричность
\item антисимметричность
\item транзитивность
\end{enumerate}

Обоснуйте свой ответ по каждому из приведенных ниже вопросов:
\begin{enumerate} [a)]\setcounter{enumi}{0}
    \item Является ли это отношение отношением эквивалентности?
    \item Является ли это отношение функциональным?
    \item Каким из отношений соответствия (одно-многозначным, много-многозначный и т.д.) оно является?
    \item К каким из отношений порядка (полного, частичного и т.д.) можно отнести данное отношение?
\end{enumerate}


\question
Установите, является ли каждое из перечисленных ниже отношений на А ($R \subseteq A \times A$) отношением эквивалентности (обоснование ответа обязательно). Для каждого отношения эквивалентности постройте классы 
эквивалентности и постройте граф отношения:
\begin{enumerate} [a)]\setcounter{enumi}{0}
\item $A = \{a, b, c, d, p, t\}$ задано отношение $R = \{(a, a), (b, b), (b, c), (b, d), (c, b), (c, c), (c, d), (d, b), (d, c), (d, d), (p,p), (t,t)\}$
\item $A = \{-10, -9, … , 9, 10\}$ и отношение $R = \{(a,b)|a^{3} = b^{3}\}$

\item $F(x)=x^{2}+1$, где $x \in A = [-2, 4]$ и отношение $R = \{(a,b)|F(a) = F(b)\}$
\end{enumerate}\question Составьте полную таблицу истинности, определите, какие переменные являются фиктивными и проверьте, является ли формула тавтологией:
$((P \rightarrow Q) \land (R \rightarrow S) \land \neg (Q \lor S)) \rightarrow \neg (P \lor R)$

\end{questions}
\newpage
%%% begin test
\begin{flushright}
\begin{tabular}{p{2.8in} r l}
%\textbf{\class} & \textbf{ФИО:} & \makebox[2.5in]{\hrulefill}\\
\textbf{\class} & \textbf{ФИО:} &Лебедь Михаил Сергеевич
\\

\textbf{\examdate} &&\\
%\textbf{Time Limit: \timelimit} & Teaching Assistant & \makebox[2in]{\hrulefill}
\end{tabular}\\
\end{flushright}
\rule[1ex]{\textwidth}{.1pt}


\begin{questions}
\question
Найдите и упростите P:
\begin{equation*}
\overline{P} = \overline{A} \cap B \cup \overline{A} \cap C \cup A \cap \overline{B} \cup \overline{B} \cap C
\end{equation*}
Затем найдите элементы множества P, выраженного через множества:
\begin{equation*}
A = \{0, 3, 4, 9\}; 
B = \{1, 3, 4, 7\};
C = \{0, 1, 2, 4, 7, 8, 9\};
I = \{0, 1, 2, 3, 4, 5, 6, 7, 8, 9\}.
\end{equation*}\question
Упростите следующее выражение с учетом того, что $A\subset B \subset C \subset D \subset U; A \neq \O$
\begin{equation*}
A \cap  \overline{C} \cup B \cap \overline{D} \cup  \overline{A} \cap C \cap  \overline{D}
\end{equation*}

Примечание: U — универсум\question
Для следующего отношения на множестве $\{1, 2, 3, 4, 5\}$ 
\begin{equation*}
aRb \iff 0 < a-b<2
\end{equation*}
Напишите обоснованный ответ какими свойствами обладает или не обладает отношение и почему:   
\begin{enumerate} [a)]\setcounter{enumi}{0}
\item рефлексивность
\item антирефлексивность
\item симметричность
\item асимметричность
\item антисимметричность
\item транзитивность
\end{enumerate}

Обоснуйте свой ответ по каждому из приведенных ниже вопросов:
\begin{enumerate} [a)]\setcounter{enumi}{0}
    \item Является ли это отношение отношением эквивалентности?
    \item Является ли это отношение функциональным?
    \item Каким из отношений соответствия (одно-многозначным, много-многозначный и т.д.) оно является?
    \item К каким из отношений порядка (полного, частичного и т.д.) можно отнести данное отношение?
\end{enumerate}
\question
Установите, является ли каждое из перечисленных ниже отношений на А ($R \subseteq A \times A$) отношением эквивалентности (обоснование ответа обязательно). Для каждого отношения эквивалентности постройте классы эквивалентности и постройте граф отношения:
\begin{enumerate} [a)]\setcounter{enumi}{0}
\item $F(x)=x^{2}+1$, где $x \in A = [-2, 4]$ и отношение $R = \{(a,b)|F(a) = F(b)\}$
\item А - множество целых чисел и отношение $R = \{(a,b)|a + b = 5\}$
\item На множестве $A = \{1; 2; 3\}$ задано отношение $R = \{(1; 1); (2; 2); (3; 3); (3; 2); (1; 2); (2; 1)\}$

\end{enumerate}\question Составьте полную таблицу истинности, определите, какие переменные являются фиктивными и проверьте, является ли формула тавтологией:
$(P \rightarrow (Q \rightarrow R)) \rightarrow ((P \rightarrow Q) \rightarrow (P \rightarrow R))$

\end{questions}
\newpage
%%% begin test
\begin{flushright}
\begin{tabular}{p{2.8in} r l}
%\textbf{\class} & \textbf{ФИО:} & \makebox[2.5in]{\hrulefill}\\
\textbf{\class} & \textbf{ФИО:} &Мамедов Мансур Солтан-Махмуд Оглы
\\

\textbf{\examdate} &&\\
%\textbf{Time Limit: \timelimit} & Teaching Assistant & \makebox[2in]{\hrulefill}
\end{tabular}\\
\end{flushright}
\rule[1ex]{\textwidth}{.1pt}


\begin{questions}
\question
Найдите и упростите P:
\begin{equation*}
\overline{P} = \overline{A} \cap B \cup \overline{A} \cap C \cup A \cap \overline{B} \cup \overline{B} \cap C
\end{equation*}
Затем найдите элементы множества P, выраженного через множества:
\begin{equation*}
A = \{0, 3, 4, 9\}; 
B = \{1, 3, 4, 7\};
C = \{0, 1, 2, 4, 7, 8, 9\};
I = \{0, 1, 2, 3, 4, 5, 6, 7, 8, 9\}.
\end{equation*}\question
Упростите следующее выражение с учетом того, что $A\subset B \subset C \subset D \subset U; A \neq \O$
\begin{equation*}
A \cap  \overline{C} \cup B \cap \overline{D} \cup  \overline{A} \cap C \cap  \overline{D}
\end{equation*}

Примечание: U — универсум\question
Дано отношение на множестве $\{1, 2, 3, 4, 5\}$ 
\begin{equation*}
aRb \iff  \text{НОД}(a,b) =1
\end{equation*}
Напишите обоснованный ответ какими свойствами обладает или не обладает отношение и почему:   
\begin{enumerate} [a)]\setcounter{enumi}{0}
\item рефлексивность
\item антирефлексивность
\item симметричность
\item асимметричность
\item антисимметричность
\item транзитивность
\end{enumerate}

Обоснуйте свой ответ по каждому из приведенных ниже вопросов:
\begin{enumerate} [a)]\setcounter{enumi}{0}
    \item Является ли это отношение отношением эквивалентности?
    \item Является ли это отношение функциональным?
    \item Каким из отношений соответствия (одно-многозначным, много-многозначный и т.д.) оно является?
    \item К каким из отношений порядка (полного, частичного и т.д.) можно отнести данное отношение?
\end{enumerate}


\question
Установите, является ли каждое из перечисленных ниже отношений на А ($R \subseteq A \times A$) отношением эквивалентности (обоснование ответа обязательно). Для каждого отношения эквивалентности постройте классы 
эквивалентности и постройте граф отношения:
\begin{enumerate} [a)]\setcounter{enumi}{0}
\item $A = \{-10, -9, … , 9, 10\}$ и отношение $R = \{(a,b)|a^{2} = b^{2}\}$
\item $A = \{a, b, c, d, p, t\}$ задано отношение $R = \{(a, a), (b, b), (b, c), (b, d), (c, b), (c, c), (c, d), (d, b), (d, c), (d, d), (p,p), (t,t)\}$
\item Пусть A – множество имен. $A = \{ $Алексей, Иван, Петр, Александр, Павел, Андрей$ \}$. Тогда отношение $R$ верно на парах имен, начинающихся с одной и той же буквы, и только на них.
\end{enumerate}\question Составьте полную таблицу истинности, определите, какие переменные являются фиктивными и проверьте, является ли формула тавтологией:
$(P \rightarrow (Q \rightarrow R)) \rightarrow ((P \rightarrow Q) \rightarrow (P \rightarrow R))$

\end{questions}
\newpage
%%% begin test
\begin{flushright}
\begin{tabular}{p{2.8in} r l}
%\textbf{\class} & \textbf{ФИО:} & \makebox[2.5in]{\hrulefill}\\
\textbf{\class} & \textbf{ФИО:} &Масянов Андрей Денисович
\\

\textbf{\examdate} &&\\
%\textbf{Time Limit: \timelimit} & Teaching Assistant & \makebox[2in]{\hrulefill}
\end{tabular}\\
\end{flushright}
\rule[1ex]{\textwidth}{.1pt}


\begin{questions}
\question
Найдите и упростите P:
\begin{equation*}
\overline{P} = A \cap \overline{B} \cup A \cap C \cup B \cap C \cup \overline{A} \cap C
\end{equation*}
Затем найдите элементы множества P, выраженного через множества:
\begin{equation*}
A = \{0, 3, 4, 9\}; 
B = \{1, 3, 4, 7\};
C = \{0, 1, 2, 4, 7, 8, 9\};
I = \{0, 1, 2, 3, 4, 5, 6, 7, 8, 9\}.
\end{equation*}\question
Упростите следующее выражение с учетом того, что $A\subset B \subset C \subset D \subset U; A \neq \O$
\begin{equation*}
\overline{A} \cap \overline{B} \cup B \cap \overline{C} \cup \overline{C} \cap D
\end{equation*}

Примечание: U — универсум\question
Для следующего отношения на множестве $\{1, 2, 3, 4, 5\}$ 
\begin{equation*}
aRb \iff 0 < a-b<2
\end{equation*}
Напишите обоснованный ответ какими свойствами обладает или не обладает отношение и почему:   
\begin{enumerate} [a)]\setcounter{enumi}{0}
\item рефлексивность
\item антирефлексивность
\item симметричность
\item асимметричность
\item антисимметричность
\item транзитивность
\end{enumerate}

Обоснуйте свой ответ по каждому из приведенных ниже вопросов:
\begin{enumerate} [a)]\setcounter{enumi}{0}
    \item Является ли это отношение отношением эквивалентности?
    \item Является ли это отношение функциональным?
    \item Каким из отношений соответствия (одно-многозначным, много-многозначный и т.д.) оно является?
    \item К каким из отношений порядка (полного, частичного и т.д.) можно отнести данное отношение?
\end{enumerate}
\question
Установите, является ли каждое из перечисленных ниже отношений на А ($R \subseteq A \times A$) отношением эквивалентности (обоснование ответа обязательно). Для каждого отношения эквивалентности постройте классы 
эквивалентности и постройте граф отношения:
\begin{enumerate} [a)]\setcounter{enumi}{0}
\item Пусть A – множество имен. $A = \{ $Алексей, Иван, Петр, Александр, Павел, Андрей$ \}$. Тогда отношение $R$ верно на парах имен, начинающихся с одной и той же буквы, и только на них.
\item $A = \{-10, -9, … , 9, 10\}$ и отношение $ R = \{(a,b)|a^{2} = b^{2}\}$
\item На множестве $A = \{1; 2; 3\}$ задано отношение $R = \{(1; 1); (2; 2); (3; 3); (3; 2); (1; 2); (2; 1)\}$
\end{enumerate}\question Составьте полную таблицу истинности, определите, какие переменные являются фиктивными и проверьте, является ли формула тавтологией:
$((P \rightarrow Q) \lor R) \leftrightarrow (P \rightarrow (Q \lor R))$

\end{questions}
\newpage
%%% begin test
\begin{flushright}
\begin{tabular}{p{2.8in} r l}
%\textbf{\class} & \textbf{ФИО:} & \makebox[2.5in]{\hrulefill}\\
\textbf{\class} & \textbf{ФИО:} &Мельник Денис Александрович
\\

\textbf{\examdate} &&\\
%\textbf{Time Limit: \timelimit} & Teaching Assistant & \makebox[2in]{\hrulefill}
\end{tabular}\\
\end{flushright}
\rule[1ex]{\textwidth}{.1pt}


\begin{questions}
\question
Найдите и упростите P:
\begin{equation*}
\overline{P} = B \cap \overline{C} \cup A \cap B \cup \overline{A} \cap C \cup \overline{A} \cap B
\end{equation*}
Затем найдите элементы множества P, выраженного через множества:
\begin{equation*}
A = \{0, 3, 4, 9\}; 
B = \{1, 3, 4, 7\};
C = \{0, 1, 2, 4, 7, 8, 9\};
I = \{0, 1, 2, 3, 4, 5, 6, 7, 8, 9\}.
\end{equation*}\question
Упростите следующее выражение с учетом того, что $A\subset B \subset C \subset D \subset U; A \neq \O$
\begin{equation*}
A \cap B \cup \overline{A} \cap \overline{C} \cup A \cap C \cup \overline{B} \cap \overline{C}
\end{equation*}

Примечание: U — универсум\question
Для следующего отношения на множестве $\{1, 2, 3, 4, 5\}$ 
\begin{equation*}
aRb \iff 0 < a-b<2
\end{equation*}
Напишите обоснованный ответ какими свойствами обладает или не обладает отношение и почему:   
\begin{enumerate} [a)]\setcounter{enumi}{0}
\item рефлексивность
\item антирефлексивность
\item симметричность
\item асимметричность
\item антисимметричность
\item транзитивность
\end{enumerate}

Обоснуйте свой ответ по каждому из приведенных ниже вопросов:
\begin{enumerate} [a)]\setcounter{enumi}{0}
    \item Является ли это отношение отношением эквивалентности?
    \item Является ли это отношение функциональным?
    \item Каким из отношений соответствия (одно-многозначным, много-многозначный и т.д.) оно является?
    \item К каким из отношений порядка (полного, частичного и т.д.) можно отнести данное отношение?
\end{enumerate}
\question
Установите, является ли каждое из перечисленных ниже отношений на А ($R \subseteq A \times A$) отношением эквивалентности (обоснование ответа обязательно). Для каждого отношения эквивалентности постройте классы 
эквивалентности и постройте граф отношения:
\begin{enumerate} [a)]\setcounter{enumi}{0}
\item $A = \{a, b, c, d, p, t\}$ задано отношение $R = \{(a, a), (b, b), (b, c), (b, d), (c, b), (c, c), (c, d), (d, b), (d, c), (d, d), (p,p), (t,t)\}$
\item $A = \{-10, -9, … , 9, 10\}$ и отношение $R = \{(a,b)|a^{3} = b^{3}\}$

\item $F(x)=x^{2}+1$, где $x \in A = [-2, 4]$ и отношение $R = \{(a,b)|F(a) = F(b)\}$
\end{enumerate}\question Составьте полную таблицу истинности, определите, какие переменные являются фиктивными и проверьте, является ли формула тавтологией:
$(( P \rightarrow Q) \land (Q \rightarrow P)) \rightarrow (P \rightarrow R)$

\end{questions}
\newpage
%%% begin test
\begin{flushright}
\begin{tabular}{p{2.8in} r l}
%\textbf{\class} & \textbf{ФИО:} & \makebox[2.5in]{\hrulefill}\\
\textbf{\class} & \textbf{ФИО:} &Михеев Артем Романович
\\

\textbf{\examdate} &&\\
%\textbf{Time Limit: \timelimit} & Teaching Assistant & \makebox[2in]{\hrulefill}
\end{tabular}\\
\end{flushright}
\rule[1ex]{\textwidth}{.1pt}


\begin{questions}
\question
Найдите и упростите P:
\begin{equation*}
\overline{P} = A \cap \overline{B} \cup A \cap C \cup B \cap C \cup \overline{A} \cap C
\end{equation*}
Затем найдите элементы множества P, выраженного через множества:
\begin{equation*}
A = \{0, 3, 4, 9\}; 
B = \{1, 3, 4, 7\};
C = \{0, 1, 2, 4, 7, 8, 9\};
I = \{0, 1, 2, 3, 4, 5, 6, 7, 8, 9\}.
\end{equation*}\question
Упростите следующее выражение с учетом того, что $A\subset B \subset C \subset D \subset U; A \neq \O$
\begin{equation*}
\overline{B} \cap \overline{C} \cap D \cup \overline{A} \cap \overline{C} \cap D \cup \overline{A} \cap B
\end{equation*}

Примечание: U — универсум\question
Дано отношение на множестве $\{1, 2, 3, 4, 5\}$ 
\begin{equation*}
aRb \iff b > a
\end{equation*}
Напишите обоснованный ответ какими свойствами обладает или не обладает отношение и почему:   
\begin{enumerate} [a)]\setcounter{enumi}{0}
\item рефлексивность
\item антирефлексивность
\item симметричность
\item асимметричность
\item антисимметричность
\item транзитивность
\end{enumerate}

Обоснуйте свой ответ по каждому из приведенных ниже вопросов:
\begin{enumerate} [a)]\setcounter{enumi}{0}
    \item Является ли это отношение отношением эквивалентности?
    \item Является ли это отношение функциональным?
    \item Каким из отношений соответствия (одно-многозначным, много-многозначный и т.д.) оно является?
    \item К каким из отношений порядка (полного, частичного и т.д.) можно отнести данное отношение?
\end{enumerate}

\question
Установите, является ли каждое из перечисленных ниже отношений на А ($R \subseteq A \times A$) отношением эквивалентности (обоснование ответа обязательно). Для каждого отношения эквивалентности 
постройте классы эквивалентности и постройте граф отношения:
\begin{enumerate}[a)]\setcounter{enumi}{0}
\item А - множество целых чисел и отношение $R = \{(a,b)|a + b = 0\}$
\item $A = \{-10, -9, …, 9, 10\}$ и отношение $R = \{(a,b)|a^{3} = b^{3}\}$
\item На множестве $A = \{1; 2; 3\}$ задано отношение $R = \{(1; 1); (2; 2); (3; 3); (2; 1); (1; 2); (2; 3); (3; 2); (3; 1); (1; 3)\}$

\end{enumerate}\question Составьте полную таблицу истинности, определите, какие переменные являются фиктивными и проверьте, является ли формула тавтологией:
$((P \rightarrow Q) \land (R \rightarrow S) \land \neg (Q \lor S)) \rightarrow \neg (P \lor R)$

\end{questions}
\newpage
%%% begin test
\begin{flushright}
\begin{tabular}{p{2.8in} r l}
%\textbf{\class} & \textbf{ФИО:} & \makebox[2.5in]{\hrulefill}\\
\textbf{\class} & \textbf{ФИО:} &Моисеев Дмитрий Владимирович
\\

\textbf{\examdate} &&\\
%\textbf{Time Limit: \timelimit} & Teaching Assistant & \makebox[2in]{\hrulefill}
\end{tabular}\\
\end{flushright}
\rule[1ex]{\textwidth}{.1pt}


\begin{questions}
\question
Найдите и упростите P:
\begin{equation*}
\overline{P} = A \cap \overline{B} \cup A \cap C \cup B \cap C \cup \overline{A} \cap C
\end{equation*}
Затем найдите элементы множества P, выраженного через множества:
\begin{equation*}
A = \{0, 3, 4, 9\}; 
B = \{1, 3, 4, 7\};
C = \{0, 1, 2, 4, 7, 8, 9\};
I = \{0, 1, 2, 3, 4, 5, 6, 7, 8, 9\}.
\end{equation*}\question
Упростите следующее выражение с учетом того, что $A\subset B \subset C \subset D \subset U; A \neq \O$
\begin{equation*}
A \cap B  \cap \overline{C} \cup \overline{C} \cap D \cup B \cap C \cap D
\end{equation*}

Примечание: U — универсум\question
Дано отношение на множестве $\{1, 2, 3, 4, 5\}$ 
\begin{equation*}
aRb \iff |a-b| = 1
\end{equation*}
Напишите обоснованный ответ какими свойствами обладает или не обладает отношение и почему:   
\begin{enumerate} [a)]\setcounter{enumi}{0}
\item рефлексивность
\item антирефлексивность
\item симметричность
\item асимметричность
\item антисимметричность
\item транзитивность
\end{enumerate}

Обоснуйте свой ответ по каждому из приведенных ниже вопросов:
\begin{enumerate} [a)]\setcounter{enumi}{0}
    \item Является ли это отношение отношением эквивалентности?
    \item Является ли это отношение функциональным?
    \item Каким из отношений соответствия (одно-многозначным, много-многозначный и т.д.) оно является?
    \item К каким из отношений порядка (полного, частичного и т.д.) можно отнести данное отношение?
\end{enumerate}

\question
Установите, является ли каждое из перечисленных ниже отношений на А ($R \subseteq A \times A$) отношением эквивалентности (обоснование ответа обязательно). Для каждого отношения эквивалентности 
постройте классы эквивалентности и постройте граф отношения:
\begin{enumerate}[a)]\setcounter{enumi}{0}
\item А - множество целых чисел и отношение $R = \{(a,b)|a + b = 0\}$
\item $A = \{-10, -9, …, 9, 10\}$ и отношение $R = \{(a,b)|a^{3} = b^{3}\}$
\item На множестве $A = \{1; 2; 3\}$ задано отношение $R = \{(1; 1); (2; 2); (3; 3); (2; 1); (1; 2); (2; 3); (3; 2); (3; 1); (1; 3)\}$

\end{enumerate}\question Составьте полную таблицу истинности, определите, какие переменные являются фиктивными и проверьте, является ли формула тавтологией:
$ P \rightarrow (Q \rightarrow ((P \lor Q) \rightarrow (P \land Q)))$

\end{questions}
\newpage
%%% begin test
\begin{flushright}
\begin{tabular}{p{2.8in} r l}
%\textbf{\class} & \textbf{ФИО:} & \makebox[2.5in]{\hrulefill}\\
\textbf{\class} & \textbf{ФИО:} &Муров Глеб Андреевич
\\

\textbf{\examdate} &&\\
%\textbf{Time Limit: \timelimit} & Teaching Assistant & \makebox[2in]{\hrulefill}
\end{tabular}\\
\end{flushright}
\rule[1ex]{\textwidth}{.1pt}


\begin{questions}
\question
Найдите и упростите P:
\begin{equation*}
\overline{P} = A \cap C \cup \overline{A} \cap \overline{C} \cup \overline{B} \cap C \cup \overline{A} \cap \overline{B}
\end{equation*}
Затем найдите элементы множества P, выраженного через множества:
\begin{equation*}
A = \{0, 3, 4, 9\}; 
B = \{1, 3, 4, 7\};
C = \{0, 1, 2, 4, 7, 8, 9\};
I = \{0, 1, 2, 3, 4, 5, 6, 7, 8, 9\}.
\end{equation*}\question
Упростите следующее выражение с учетом того, что $A\subset B \subset C \subset D \subset U; A \neq \O$
\begin{equation*}
A \cap C  \cap D \cup B \cap \overline{C} \cap D \cup B \cap C \cap D
\end{equation*}

Примечание: U — универсум\question
Для следующего отношения на множестве $\{1, 2, 3, 4, 5\}$ 
\begin{equation*}
aRb \iff 0 < a-b<2
\end{equation*}
Напишите обоснованный ответ какими свойствами обладает или не обладает отношение и почему:   
\begin{enumerate} [a)]\setcounter{enumi}{0}
\item рефлексивность
\item антирефлексивность
\item симметричность
\item асимметричность
\item антисимметричность
\item транзитивность
\end{enumerate}

Обоснуйте свой ответ по каждому из приведенных ниже вопросов:
\begin{enumerate} [a)]\setcounter{enumi}{0}
    \item Является ли это отношение отношением эквивалентности?
    \item Является ли это отношение функциональным?
    \item Каким из отношений соответствия (одно-многозначным, много-многозначный и т.д.) оно является?
    \item К каким из отношений порядка (полного, частичного и т.д.) можно отнести данное отношение?
\end{enumerate}
\question
Установите, является ли каждое из перечисленных ниже отношений на А ($R \subseteq A \times A$) отношением эквивалентности (обоснование ответа обязательно). Для каждого отношения эквивалентности постройте классы 
эквивалентности и постройте граф отношения:
\begin{enumerate} [a)]\setcounter{enumi}{0}
\item А - множество целых чисел и отношение $R = \{(a,b)|a + b = 5\}$
\item Пусть A – множество имен. $A = \{ $Алексей, Иван, Петр, Александр, Павел, Андрей$ \}$. Тогда отношение $R $ верно на парах имен, начинающихся с одной и той же буквы, и только на них.
\item На множестве $A = \{1; 2; 3; 4; 5\}$ задано отношение $R = \{(1; 2); (1; 3); (1; 5); (2; 3); (2; 4); (2; 5); (3; 4); (3; 5); (4; 5)\}$
\end{enumerate}\question Составьте полную таблицу истинности, определите, какие переменные являются фиктивными и проверьте, является ли формула тавтологией:
$((P \rightarrow Q) \land (R \rightarrow S) \land \neg (Q \lor S)) \rightarrow \neg (P \lor R)$

\end{questions}
\newpage
%%% begin test
\begin{flushright}
\begin{tabular}{p{2.8in} r l}
%\textbf{\class} & \textbf{ФИО:} & \makebox[2.5in]{\hrulefill}\\
\textbf{\class} & \textbf{ФИО:} &Перевощиков Радомир Евгеньевич
\\

\textbf{\examdate} &&\\
%\textbf{Time Limit: \timelimit} & Teaching Assistant & \makebox[2in]{\hrulefill}
\end{tabular}\\
\end{flushright}
\rule[1ex]{\textwidth}{.1pt}


\begin{questions}
\question
Найдите и упростите P:
\begin{equation*}
\overline{P} = A \cap \overline{B} \cup \overline{B} \cap C \cup \overline{A} \cap \overline{B} \cup \overline{A} \cap C
\end{equation*}
Затем найдите элементы множества P, выраженного через множества:
\begin{equation*}
A = \{0, 3, 4, 9\}; 
B = \{1, 3, 4, 7\};
C = \{0, 1, 2, 4, 7, 8, 9\};
I = \{0, 1, 2, 3, 4, 5, 6, 7, 8, 9\}.
\end{equation*}\question
Упростите следующее выражение с учетом того, что $A\subset B \subset C \subset D \subset U; A \neq \O$
\begin{equation*}
A \cap B \cup \overline{A} \cap \overline{C} \cup A \cap C \cup \overline{B} \cap \overline{C}
\end{equation*}

Примечание: U — универсум\question
Дано отношение на множестве $\{1, 2, 3, 4, 5\}$ 
\begin{equation*}
aRb \iff a \leq b
\end{equation*}
Напишите обоснованный ответ какими свойствами обладает или не обладает отношение и почему:   
\begin{enumerate} [a)]\setcounter{enumi}{0}
\item рефлексивность
\item антирефлексивность
\item симметричность
\item асимметричность
\item антисимметричность
\item транзитивность
\end{enumerate}

Обоснуйте свой ответ по каждому из приведенных ниже вопросов:
\begin{enumerate} [a)]\setcounter{enumi}{0}
    \item Является ли это отношение отношением эквивалентности?
    \item Является ли это отношение функциональным?
    \item Каким из отношений соответствия (одно-многозначным, много-многозначный и т.д.) оно является?
    \item К каким из отношений порядка (полного, частичного и т.д.) можно отнести данное отношение?
\end{enumerate}


\question
Установите, является ли каждое из перечисленных ниже отношений на А ($R \subseteq A \times A$) отношением эквивалентности (обоснование ответа обязательно). Для каждого отношения эквивалентности 
постройте классы эквивалентности и постройте граф отношения:
\begin{enumerate}[a)]\setcounter{enumi}{0}
\item А - множество целых чисел и отношение $R = \{(a,b)|a + b = 0\}$
\item $A = \{-10, -9, …, 9, 10\}$ и отношение $R = \{(a,b)|a^{3} = b^{3}\}$
\item На множестве $A = \{1; 2; 3\}$ задано отношение $R = \{(1; 1); (2; 2); (3; 3); (2; 1); (1; 2); (2; 3); (3; 2); (3; 1); (1; 3)\}$

\end{enumerate}\question Составьте полную таблицу истинности, определите, какие переменные являются фиктивными и проверьте, является ли формула тавтологией:
$(P \rightarrow (Q \rightarrow R)) \rightarrow ((P \rightarrow Q) \rightarrow (P \rightarrow R))$

\end{questions}
\newpage
%%% begin test
\begin{flushright}
\begin{tabular}{p{2.8in} r l}
%\textbf{\class} & \textbf{ФИО:} & \makebox[2.5in]{\hrulefill}\\
\textbf{\class} & \textbf{ФИО:} &Пискуровский Матвей Григорьевич
\\

\textbf{\examdate} &&\\
%\textbf{Time Limit: \timelimit} & Teaching Assistant & \makebox[2in]{\hrulefill}
\end{tabular}\\
\end{flushright}
\rule[1ex]{\textwidth}{.1pt}


\begin{questions}
\question
Найдите и упростите P:
\begin{equation*}
\overline{P} = A \cap \overline{C} \cup A \cap \overline{B} \cup B \cap \overline{C} \cup A \cap C
\end{equation*}
Затем найдите элементы множества P, выраженного через множества:
\begin{equation*}
A = \{0, 3, 4, 9\}; 
B = \{1, 3, 4, 7\};
C = \{0, 1, 2, 4, 7, 8, 9\};
I = \{0, 1, 2, 3, 4, 5, 6, 7, 8, 9\}.
\end{equation*}\question
Упростите следующее выражение с учетом того, что $A\subset B \subset C \subset D \subset U; A \neq \O$
\begin{equation*}
A \cap B \cup \overline{A} \cap \overline{C} \cup A \cap C \cup \overline{B} \cap \overline{C}
\end{equation*}

Примечание: U — универсум\question
Дано отношение на множестве $\{1, 2, 3, 4, 5\}$ 
\begin{equation*}
aRb \iff b > a
\end{equation*}
Напишите обоснованный ответ какими свойствами обладает или не обладает отношение и почему:   
\begin{enumerate} [a)]\setcounter{enumi}{0}
\item рефлексивность
\item антирефлексивность
\item симметричность
\item асимметричность
\item антисимметричность
\item транзитивность
\end{enumerate}

Обоснуйте свой ответ по каждому из приведенных ниже вопросов:
\begin{enumerate} [a)]\setcounter{enumi}{0}
    \item Является ли это отношение отношением эквивалентности?
    \item Является ли это отношение функциональным?
    \item Каким из отношений соответствия (одно-многозначным, много-многозначный и т.д.) оно является?
    \item К каким из отношений порядка (полного, частичного и т.д.) можно отнести данное отношение?
\end{enumerate}

\question
Установите, является ли каждое из перечисленных ниже отношений на А ($R \subseteq A \times A$) отношением эквивалентности (обоснование ответа обязательно). Для каждого отношения эквивалентности 
постройте классы эквивалентности и постройте граф отношения:
\begin{enumerate}[a)]\setcounter{enumi}{0}
\item А - множество целых чисел и отношение $R = \{(a,b)|a + b = 0\}$
\item $A = \{-10, -9, …, 9, 10\}$ и отношение $R = \{(a,b)|a^{3} = b^{3}\}$
\item На множестве $A = \{1; 2; 3\}$ задано отношение $R = \{(1; 1); (2; 2); (3; 3); (2; 1); (1; 2); (2; 3); (3; 2); (3; 1); (1; 3)\}$

\end{enumerate}\question Составьте полную таблицу истинности, определите, какие переменные являются фиктивными и проверьте, является ли формула тавтологией:
$(( P \land \neg Q) \rightarrow (R \land \neg R)) \rightarrow (P \rightarrow Q)$

\end{questions}
\newpage
%%% begin test
\begin{flushright}
\begin{tabular}{p{2.8in} r l}
%\textbf{\class} & \textbf{ФИО:} & \makebox[2.5in]{\hrulefill}\\
\textbf{\class} & \textbf{ФИО:} &Сергеев Егор Дмитриевич
\\

\textbf{\examdate} &&\\
%\textbf{Time Limit: \timelimit} & Teaching Assistant & \makebox[2in]{\hrulefill}
\end{tabular}\\
\end{flushright}
\rule[1ex]{\textwidth}{.1pt}


\begin{questions}
\question
Найдите и упростите P:
\begin{equation*}
\overline{P} = A \cap \overline{B} \cup A \cap C \cup B \cap C \cup \overline{A} \cap C
\end{equation*}
Затем найдите элементы множества P, выраженного через множества:
\begin{equation*}
A = \{0, 3, 4, 9\}; 
B = \{1, 3, 4, 7\};
C = \{0, 1, 2, 4, 7, 8, 9\};
I = \{0, 1, 2, 3, 4, 5, 6, 7, 8, 9\}.
\end{equation*}\question
Упростите следующее выражение с учетом того, что $A\subset B \subset C \subset D \subset U; A \neq \O$
\begin{equation*}
\overline{B} \cap \overline{C} \cap D \cup \overline{A} \cap \overline{C} \cap D \cup \overline{A} \cap B
\end{equation*}

Примечание: U — универсум\question
Дано отношение на множестве $\{1, 2, 3, 4, 5\}$ 
\begin{equation*}
aRb \iff (a+b) \bmod 2 =0
\end{equation*}
Напишите обоснованный ответ какими свойствами обладает или не обладает отношение и почему:   
\begin{enumerate} [a)]\setcounter{enumi}{0}
\item рефлексивность
\item антирефлексивность
\item симметричность
\item асимметричность
\item антисимметричность
\item транзитивность
\end{enumerate}

Обоснуйте свой ответ по каждому из приведенных ниже вопросов:
\begin{enumerate} [a)]\setcounter{enumi}{0}
    \item Является ли это отношение отношением эквивалентности?
    \item Является ли это отношение функциональным?
    \item Каким из отношений соответствия (одно-многозначным, много-многозначный и т.д.) оно является?
    \item К каким из отношений порядка (полного, частичного и т.д.) можно отнести данное отношение?
\end{enumerate}



\question
Установите, является ли каждое из перечисленных ниже отношений на А ($R \subseteq A \times A$) отношением эквивалентности (обоснование ответа обязательно). Для каждого отношения эквивалентности постройте классы 
эквивалентности и постройте граф отношения:
\begin{enumerate} [a)]\setcounter{enumi}{0}
\item На множестве $A = \{1; 2; 3\}$ задано отношение $R = \{(1; 1); (2; 2); (3; 3); (2; 1); (1; 2); (2; 3); (3; 2); (3; 1); (1; 3)\}$
\item На множестве $A = \{1; 2; 3; 4; 5\}$ задано отношение $R = \{(1; 2); (1; 3); (1; 5); (2; 3); (2; 4); (2; 5); (3; 4); (3; 5); (4; 5)\}$
\item А - множество целых чисел и отношение $R = \{(a,b)|a + b = 0\}$
\end{enumerate}\question Составьте полную таблицу истинности, определите, какие переменные являются фиктивными и проверьте, является ли формула тавтологией:
$(( P \rightarrow Q) \land (Q \rightarrow P)) \rightarrow (P \rightarrow R)$

\end{questions}
\newpage
%%% begin test
\begin{flushright}
\begin{tabular}{p{2.8in} r l}
%\textbf{\class} & \textbf{ФИО:} & \makebox[2.5in]{\hrulefill}\\
\textbf{\class} & \textbf{ФИО:} &Солдатов Константин Максимович
\\

\textbf{\examdate} &&\\
%\textbf{Time Limit: \timelimit} & Teaching Assistant & \makebox[2in]{\hrulefill}
\end{tabular}\\
\end{flushright}
\rule[1ex]{\textwidth}{.1pt}


\begin{questions}
\question
Найдите и упростите P:
\begin{equation*}
\overline{P} = A \cap \overline{B} \cup A \cap C \cup B \cap C \cup \overline{A} \cap C
\end{equation*}
Затем найдите элементы множества P, выраженного через множества:
\begin{equation*}
A = \{0, 3, 4, 9\}; 
B = \{1, 3, 4, 7\};
C = \{0, 1, 2, 4, 7, 8, 9\};
I = \{0, 1, 2, 3, 4, 5, 6, 7, 8, 9\}.
\end{equation*}\question
Упростите следующее выражение с учетом того, что $A\subset B \subset C \subset D \subset U; A \neq \O$
\begin{equation*}
\overline{B} \cap \overline{C} \cap D \cup \overline{A} \cap \overline{C} \cap D \cup \overline{A} \cap B
\end{equation*}

Примечание: U — универсум\question
Для следующего отношения на множестве $\{1, 2, 3, 4, 5\}$ 
\begin{equation*}
aRb \iff 0 < a-b<2
\end{equation*}
Напишите обоснованный ответ какими свойствами обладает или не обладает отношение и почему:   
\begin{enumerate} [a)]\setcounter{enumi}{0}
\item рефлексивность
\item антирефлексивность
\item симметричность
\item асимметричность
\item антисимметричность
\item транзитивность
\end{enumerate}

Обоснуйте свой ответ по каждому из приведенных ниже вопросов:
\begin{enumerate} [a)]\setcounter{enumi}{0}
    \item Является ли это отношение отношением эквивалентности?
    \item Является ли это отношение функциональным?
    \item Каким из отношений соответствия (одно-многозначным, много-многозначный и т.д.) оно является?
    \item К каким из отношений порядка (полного, частичного и т.д.) можно отнести данное отношение?
\end{enumerate}
\question
Установите, является ли каждое из перечисленных ниже отношений на А ($R \subseteq A \times A$) отношением эквивалентности (обоснование ответа обязательно). Для каждого отношения эквивалентности постройте классы эквивалентности и постройте граф отношения:
\begin{enumerate} [a)]\setcounter{enumi}{0}
\item $F(x)=x^{2}+1$, где $x \in A = [-2, 4]$ и отношение $R = \{(a,b)|F(a) = F(b)\}$
\item А - множество целых чисел и отношение $R = \{(a,b)|a + b = 5\}$
\item На множестве $A = \{1; 2; 3\}$ задано отношение $R = \{(1; 1); (2; 2); (3; 3); (3; 2); (1; 2); (2; 1)\}$

\end{enumerate}\question Составьте полную таблицу истинности, определите, какие переменные являются фиктивными и проверьте, является ли формула тавтологией:
$(P \rightarrow (Q \rightarrow R)) \rightarrow ((P \rightarrow Q) \rightarrow (P \rightarrow R))$

\end{questions}
\newpage
%%% begin test
\begin{flushright}
\begin{tabular}{p{2.8in} r l}
%\textbf{\class} & \textbf{ФИО:} & \makebox[2.5in]{\hrulefill}\\
\textbf{\class} & \textbf{ФИО:} &Сухов Владимир Игоревич
\\

\textbf{\examdate} &&\\
%\textbf{Time Limit: \timelimit} & Teaching Assistant & \makebox[2in]{\hrulefill}
\end{tabular}\\
\end{flushright}
\rule[1ex]{\textwidth}{.1pt}


\begin{questions}
\question
Найдите и упростите P:
\begin{equation*}
\overline{P} = A \cap C \cup \overline{A} \cap \overline{C} \cup \overline{B} \cap C \cup \overline{A} \cap \overline{B}
\end{equation*}
Затем найдите элементы множества P, выраженного через множества:
\begin{equation*}
A = \{0, 3, 4, 9\}; 
B = \{1, 3, 4, 7\};
C = \{0, 1, 2, 4, 7, 8, 9\};
I = \{0, 1, 2, 3, 4, 5, 6, 7, 8, 9\}.
\end{equation*}\question
Упростите следующее выражение с учетом того, что $A\subset B \subset C \subset D \subset U; A \neq \O$
\begin{equation*}
A \cap C  \cap D \cup B \cap \overline{C} \cap D \cup B \cap C \cap D
\end{equation*}

Примечание: U — универсум\question
Дано отношение на множестве $\{1, 2, 3, 4, 5\}$ 
\begin{equation*}
aRb \iff a \leq b
\end{equation*}
Напишите обоснованный ответ какими свойствами обладает или не обладает отношение и почему:   
\begin{enumerate} [a)]\setcounter{enumi}{0}
\item рефлексивность
\item антирефлексивность
\item симметричность
\item асимметричность
\item антисимметричность
\item транзитивность
\end{enumerate}

Обоснуйте свой ответ по каждому из приведенных ниже вопросов:
\begin{enumerate} [a)]\setcounter{enumi}{0}
    \item Является ли это отношение отношением эквивалентности?
    \item Является ли это отношение функциональным?
    \item Каким из отношений соответствия (одно-многозначным, много-многозначный и т.д.) оно является?
    \item К каким из отношений порядка (полного, частичного и т.д.) можно отнести данное отношение?
\end{enumerate}


\question
Установите, является ли каждое из перечисленных ниже отношений на А ($R \subseteq A \times A$) отношением эквивалентности (обоснование ответа обязательно). Для каждого отношения эквивалентности постройте классы 
эквивалентности и постройте граф отношения:
\begin{enumerate} [a)]\setcounter{enumi}{0}
\item $A = \{-10, -9, … , 9, 10\}$ и отношение $R = \{(a,b)|a^{2} = b^{2}\}$
\item $A = \{a, b, c, d, p, t\}$ задано отношение $R = \{(a, a), (b, b), (b, c), (b, d), (c, b), (c, c), (c, d), (d, b), (d, c), (d, d), (p,p), (t,t)\}$
\item Пусть A – множество имен. $A = \{ $Алексей, Иван, Петр, Александр, Павел, Андрей$ \}$. Тогда отношение $R$ верно на парах имен, начинающихся с одной и той же буквы, и только на них.
\end{enumerate}\question Составьте полную таблицу истинности, определите, какие переменные являются фиктивными и проверьте, является ли формула тавтологией:

$(P \rightarrow (Q \land R)) \leftrightarrow ((P \rightarrow Q) \land (P \rightarrow R))$

\end{questions}
\newpage
%%% begin test
\begin{flushright}
\begin{tabular}{p{2.8in} r l}
%\textbf{\class} & \textbf{ФИО:} & \makebox[2.5in]{\hrulefill}\\
\textbf{\class} & \textbf{ФИО:} &Теряев Роман Алексеевич
\\

\textbf{\examdate} &&\\
%\textbf{Time Limit: \timelimit} & Teaching Assistant & \makebox[2in]{\hrulefill}
\end{tabular}\\
\end{flushright}
\rule[1ex]{\textwidth}{.1pt}


\begin{questions}
\question
Найдите и упростите P:
\begin{equation*}
\overline{P} = B \cap \overline{C} \cup A \cap B \cup \overline{A} \cap C \cup \overline{A} \cap B
\end{equation*}
Затем найдите элементы множества P, выраженного через множества:
\begin{equation*}
A = \{0, 3, 4, 9\}; 
B = \{1, 3, 4, 7\};
C = \{0, 1, 2, 4, 7, 8, 9\};
I = \{0, 1, 2, 3, 4, 5, 6, 7, 8, 9\}.
\end{equation*}\question
Упростите следующее выражение с учетом того, что $A\subset B \subset C \subset D \subset U; A \neq \O$
\begin{equation*}
A \cap B  \cap \overline{C} \cup \overline{C} \cap D \cup B \cap C \cap D
\end{equation*}

Примечание: U — универсум\question
Дано отношение на множестве $\{1, 2, 3, 4, 5\}$ 
\begin{equation*}
aRb \iff a \leq b
\end{equation*}
Напишите обоснованный ответ какими свойствами обладает или не обладает отношение и почему:   
\begin{enumerate} [a)]\setcounter{enumi}{0}
\item рефлексивность
\item антирефлексивность
\item симметричность
\item асимметричность
\item антисимметричность
\item транзитивность
\end{enumerate}

Обоснуйте свой ответ по каждому из приведенных ниже вопросов:
\begin{enumerate} [a)]\setcounter{enumi}{0}
    \item Является ли это отношение отношением эквивалентности?
    \item Является ли это отношение функциональным?
    \item Каким из отношений соответствия (одно-многозначным, много-многозначный и т.д.) оно является?
    \item К каким из отношений порядка (полного, частичного и т.д.) можно отнести данное отношение?
\end{enumerate}


\question
Установите, является ли каждое из перечисленных ниже отношений на А ($R \subseteq A \times A$) отношением эквивалентности (обоснование ответа обязательно). Для каждого отношения эквивалентности постройте классы 
эквивалентности и постройте граф отношения:
\begin{enumerate} [a)]\setcounter{enumi}{0}
\item $A = \{-10, -9, … , 9, 10\}$ и отношение $R = \{(a,b)|a^{2} = b^{2}\}$
\item $A = \{a, b, c, d, p, t\}$ задано отношение $R = \{(a, a), (b, b), (b, c), (b, d), (c, b), (c, c), (c, d), (d, b), (d, c), (d, d), (p,p), (t,t)\}$
\item Пусть A – множество имен. $A = \{ $Алексей, Иван, Петр, Александр, Павел, Андрей$ \}$. Тогда отношение $R$ верно на парах имен, начинающихся с одной и той же буквы, и только на них.
\end{enumerate}\question Составьте полную таблицу истинности, определите, какие переменные являются фиктивными и проверьте, является ли формула тавтологией:
$(( P \rightarrow Q) \land (Q \rightarrow P)) \rightarrow (P \rightarrow R)$

\end{questions}
\newpage
%%% begin test
\begin{flushright}
\begin{tabular}{p{2.8in} r l}
%\textbf{\class} & \textbf{ФИО:} & \makebox[2.5in]{\hrulefill}\\
\textbf{\class} & \textbf{ФИО:} &Хакимов Руслан Венирович
\\

\textbf{\examdate} &&\\
%\textbf{Time Limit: \timelimit} & Teaching Assistant & \makebox[2in]{\hrulefill}
\end{tabular}\\
\end{flushright}
\rule[1ex]{\textwidth}{.1pt}


\begin{questions}
\question
Найдите и упростите P:
\begin{equation*}
\overline{P} = \overline{A} \cap B \cup \overline{A} \cap C \cup A \cap \overline{B} \cup \overline{B} \cap C
\end{equation*}
Затем найдите элементы множества P, выраженного через множества:
\begin{equation*}
A = \{0, 3, 4, 9\}; 
B = \{1, 3, 4, 7\};
C = \{0, 1, 2, 4, 7, 8, 9\};
I = \{0, 1, 2, 3, 4, 5, 6, 7, 8, 9\}.
\end{equation*}\question
Упростите следующее выражение с учетом того, что $A\subset B \subset C \subset D \subset U; A \neq \O$
\begin{equation*}
A \cap B  \cap \overline{C} \cup \overline{C} \cap D \cup B \cap C \cap D
\end{equation*}

Примечание: U — универсум\question
Для следующего отношения на множестве $\{1, 2, 3, 4, 5\}$ 
\begin{equation*}
aRb \iff 0 < a-b<2
\end{equation*}
Напишите обоснованный ответ какими свойствами обладает или не обладает отношение и почему:   
\begin{enumerate} [a)]\setcounter{enumi}{0}
\item рефлексивность
\item антирефлексивность
\item симметричность
\item асимметричность
\item антисимметричность
\item транзитивность
\end{enumerate}

Обоснуйте свой ответ по каждому из приведенных ниже вопросов:
\begin{enumerate} [a)]\setcounter{enumi}{0}
    \item Является ли это отношение отношением эквивалентности?
    \item Является ли это отношение функциональным?
    \item Каким из отношений соответствия (одно-многозначным, много-многозначный и т.д.) оно является?
    \item К каким из отношений порядка (полного, частичного и т.д.) можно отнести данное отношение?
\end{enumerate}
\question
Установите, является ли каждое из перечисленных ниже отношений на А ($R \subseteq A \times A$) отношением эквивалентности (обоснование ответа обязательно). Для каждого отношения эквивалентности постройте классы 
эквивалентности и постройте граф отношения:
\begin{enumerate} [a)]\setcounter{enumi}{0}
\item Пусть A – множество имен. $A = \{ $Алексей, Иван, Петр, Александр, Павел, Андрей$ \}$. Тогда отношение $R$ верно на парах имен, начинающихся с одной и той же буквы, и только на них.
\item $A = \{-10, -9, … , 9, 10\}$ и отношение $ R = \{(a,b)|a^{2} = b^{2}\}$
\item На множестве $A = \{1; 2; 3\}$ задано отношение $R = \{(1; 1); (2; 2); (3; 3); (3; 2); (1; 2); (2; 1)\}$
\end{enumerate}\question Составьте полную таблицу истинности, определите, какие переменные являются фиктивными и проверьте, является ли формула тавтологией:

$(P \rightarrow (Q \land R)) \leftrightarrow ((P \rightarrow Q) \land (P \rightarrow R))$

\end{questions}
\newpage
%%% begin test
\begin{flushright}
\begin{tabular}{p{2.8in} r l}
%\textbf{\class} & \textbf{ФИО:} & \makebox[2.5in]{\hrulefill}\\
\textbf{\class} & \textbf{ФИО:} &Халеев Михаил Дмитриевич
\\

\textbf{\examdate} &&\\
%\textbf{Time Limit: \timelimit} & Teaching Assistant & \makebox[2in]{\hrulefill}
\end{tabular}\\
\end{flushright}
\rule[1ex]{\textwidth}{.1pt}


\begin{questions}
\question
Найдите и упростите P:
\begin{equation*}
\overline{P} = A \cap C \cup \overline{A} \cap \overline{C} \cup \overline{B} \cap C \cup \overline{A} \cap \overline{B}
\end{equation*}
Затем найдите элементы множества P, выраженного через множества:
\begin{equation*}
A = \{0, 3, 4, 9\}; 
B = \{1, 3, 4, 7\};
C = \{0, 1, 2, 4, 7, 8, 9\};
I = \{0, 1, 2, 3, 4, 5, 6, 7, 8, 9\}.
\end{equation*}\question
Упростите следующее выражение с учетом того, что $A\subset B \subset C \subset D \subset U; A \neq \O$
\begin{equation*}
A \cap B \cup \overline{A} \cap \overline{C} \cup A \cap C \cup \overline{B} \cap \overline{C}
\end{equation*}

Примечание: U — универсум\question
Дано отношение на множестве $\{1, 2, 3, 4, 5\}$ 
\begin{equation*}
aRb \iff |a-b| = 1
\end{equation*}
Напишите обоснованный ответ какими свойствами обладает или не обладает отношение и почему:   
\begin{enumerate} [a)]\setcounter{enumi}{0}
\item рефлексивность
\item антирефлексивность
\item симметричность
\item асимметричность
\item антисимметричность
\item транзитивность
\end{enumerate}

Обоснуйте свой ответ по каждому из приведенных ниже вопросов:
\begin{enumerate} [a)]\setcounter{enumi}{0}
    \item Является ли это отношение отношением эквивалентности?
    \item Является ли это отношение функциональным?
    \item Каким из отношений соответствия (одно-многозначным, много-многозначный и т.д.) оно является?
    \item К каким из отношений порядка (полного, частичного и т.д.) можно отнести данное отношение?
\end{enumerate}

\question
Установите, является ли каждое из перечисленных ниже отношений на А ($R \subseteq A \times A$) отношением эквивалентности (обоснование ответа обязательно). Для каждого отношения эквивалентности постройте классы 
эквивалентности и постройте граф отношения:
\begin{enumerate} [a)]\setcounter{enumi}{0}
\item $A = \{a, b, c, d, p, t\}$ задано отношение $R = \{(a, a), (b, b), (b, c), (b, d), (c, b), (c, c), (c, d), (d, b), (d, c), (d, d), (p,p), (t,t)\}$
\item $A = \{-10, -9, … , 9, 10\}$ и отношение $R = \{(a,b)|a^{3} = b^{3}\}$

\item $F(x)=x^{2}+1$, где $x \in A = [-2, 4]$ и отношение $R = \{(a,b)|F(a) = F(b)\}$
\end{enumerate}\question Составьте полную таблицу истинности, определите, какие переменные являются фиктивными и проверьте, является ли формула тавтологией:
$(( P \rightarrow Q) \land (Q \rightarrow P)) \rightarrow (P \rightarrow R)$

\end{questions}
\newpage
%%% begin test
\begin{flushright}
\begin{tabular}{p{2.8in} r l}
%\textbf{\class} & \textbf{ФИО:} & \makebox[2.5in]{\hrulefill}\\
\textbf{\class} & \textbf{ФИО:} &М3102
\\

\textbf{\examdate} &&\\
%\textbf{Time Limit: \timelimit} & Teaching Assistant & \makebox[2in]{\hrulefill}
\end{tabular}\\
\end{flushright}
\rule[1ex]{\textwidth}{.1pt}


\begin{questions}
\question
Найдите и упростите P:
\begin{equation*}
\overline{P} = A \cap C \cup \overline{A} \cap \overline{C} \cup \overline{B} \cap C \cup \overline{A} \cap \overline{B}
\end{equation*}
Затем найдите элементы множества P, выраженного через множества:
\begin{equation*}
A = \{0, 3, 4, 9\}; 
B = \{1, 3, 4, 7\};
C = \{0, 1, 2, 4, 7, 8, 9\};
I = \{0, 1, 2, 3, 4, 5, 6, 7, 8, 9\}.
\end{equation*}\question
Упростите следующее выражение с учетом того, что $A\subset B \subset C \subset D \subset U; A \neq \O$
\begin{equation*}
\overline{B} \cap \overline{C} \cap D \cup \overline{A} \cap \overline{C} \cap D \cup \overline{A} \cap B
\end{equation*}

Примечание: U — универсум\question
Дано отношение на множестве $\{1, 2, 3, 4, 5\}$ 
\begin{equation*}
aRb \iff  \text{НОД}(a,b) =1
\end{equation*}
Напишите обоснованный ответ какими свойствами обладает или не обладает отношение и почему:   
\begin{enumerate} [a)]\setcounter{enumi}{0}
\item рефлексивность
\item антирефлексивность
\item симметричность
\item асимметричность
\item антисимметричность
\item транзитивность
\end{enumerate}

Обоснуйте свой ответ по каждому из приведенных ниже вопросов:
\begin{enumerate} [a)]\setcounter{enumi}{0}
    \item Является ли это отношение отношением эквивалентности?
    \item Является ли это отношение функциональным?
    \item Каким из отношений соответствия (одно-многозначным, много-многозначный и т.д.) оно является?
    \item К каким из отношений порядка (полного, частичного и т.д.) можно отнести данное отношение?
\end{enumerate}


\question
Установите, является ли каждое из перечисленных ниже отношений на А ($R \subseteq A \times A$) отношением эквивалентности (обоснование ответа обязательно). Для каждого отношения эквивалентности постройте классы 
эквивалентности и постройте граф отношения:
\begin{enumerate} [a)]\setcounter{enumi}{0}
\item На множестве $A = \{1; 2; 3\}$ задано отношение $R = \{(1; 1); (2; 2); (3; 3); (2; 1); (1; 2); (2; 3); (3; 2); (3; 1); (1; 3)\}$
\item На множестве $A = \{1; 2; 3; 4; 5\}$ задано отношение $R = \{(1; 2); (1; 3); (1; 5); (2; 3); (2; 4); (2; 5); (3; 4); (3; 5); (4; 5)\}$
\item А - множество целых чисел и отношение $R = \{(a,b)|a + b = 0\}$
\end{enumerate}\question Составьте полную таблицу истинности, определите, какие переменные являются фиктивными и проверьте, является ли формула тавтологией:
$(( P \rightarrow Q) \land (Q \rightarrow P)) \rightarrow (P \rightarrow R)$

\end{questions}
\newpage
%%% begin test
\begin{flushright}
\begin{tabular}{p{2.8in} r l}
%\textbf{\class} & \textbf{ФИО:} & \makebox[2.5in]{\hrulefill}\\
\textbf{\class} & \textbf{ФИО:} &Белая Виктория Александровна
\\

\textbf{\examdate} &&\\
%\textbf{Time Limit: \timelimit} & Teaching Assistant & \makebox[2in]{\hrulefill}
\end{tabular}\\
\end{flushright}
\rule[1ex]{\textwidth}{.1pt}


\begin{questions}
\question
Найдите и упростите P:
\begin{equation*}
\overline{P} = A \cap B \cup \overline{A} \cap \overline{B} \cup A \cap C \cup \overline{B} \cap C
\end{equation*}
Затем найдите элементы множества P, выраженного через множества:
\begin{equation*}
A = \{0, 3, 4, 9\}; 
B = \{1, 3, 4, 7\};
C = \{0, 1, 2, 4, 7, 8, 9\};
I = \{0, 1, 2, 3, 4, 5, 6, 7, 8, 9\}.
\end{equation*}\question
Упростите следующее выражение с учетом того, что $A\subset B \subset C \subset D \subset U; A \neq \O$
\begin{equation*}
\overline{B} \cap \overline{C} \cap D \cup \overline{A} \cap \overline{C} \cap D \cup \overline{A} \cap B
\end{equation*}

Примечание: U — универсум\question
Дано отношение на множестве $\{1, 2, 3, 4, 5\}$ 
\begin{equation*}
aRb \iff a \geq b^2
\end{equation*}
Напишите обоснованный ответ какими свойствами обладает или не обладает отношение и почему:   
\begin{enumerate} [a)]\setcounter{enumi}{0}
\item рефлексивность
\item антирефлексивность
\item симметричность
\item асимметричность
\item антисимметричность
\item транзитивность
\end{enumerate}

Обоснуйте свой ответ по каждому из приведенных ниже вопросов:
\begin{enumerate} [a)]\setcounter{enumi}{0}
    \item Является ли это отношение отношением эквивалентности?
    \item Является ли это отношение функциональным?
    \item Каким из отношений соответствия (одно-многозначным, много-многозначный и т.д.) оно является?
    \item К каким из отношений порядка (полного, частичного и т.д.) можно отнести данное отношение?
\end{enumerate}


\question
Установите, является ли каждое из перечисленных ниже отношений на А ($R \subseteq A \times A$) отношением эквивалентности (обоснование ответа обязательно). Для каждого отношения эквивалентности постройте классы эквивалентности и постройте граф отношения:
\begin{enumerate} [a)]\setcounter{enumi}{0}
\item $F(x)=x^{2}+1$, где $x \in A = [-2, 4]$ и отношение $R = \{(a,b)|F(a) = F(b)\}$
\item А - множество целых чисел и отношение $R = \{(a,b)|a + b = 5\}$
\item На множестве $A = \{1; 2; 3\}$ задано отношение $R = \{(1; 1); (2; 2); (3; 3); (3; 2); (1; 2); (2; 1)\}$

\end{enumerate}\question Составьте полную таблицу истинности, определите, какие переменные являются фиктивными и проверьте, является ли формула тавтологией:
$((P \rightarrow Q) \lor R) \leftrightarrow (P \rightarrow (Q \lor R))$

\end{questions}
\newpage
%%% begin test
\begin{flushright}
\begin{tabular}{p{2.8in} r l}
%\textbf{\class} & \textbf{ФИО:} & \makebox[2.5in]{\hrulefill}\\
\textbf{\class} & \textbf{ФИО:} &Власов Роман Алексеевич
\\

\textbf{\examdate} &&\\
%\textbf{Time Limit: \timelimit} & Teaching Assistant & \makebox[2in]{\hrulefill}
\end{tabular}\\
\end{flushright}
\rule[1ex]{\textwidth}{.1pt}


\begin{questions}
\question
Найдите и упростите P:
\begin{equation*}
\overline{P} = \overline{A} \cap B \cup \overline{A} \cap C \cup A \cap \overline{B} \cup \overline{B} \cap C
\end{equation*}
Затем найдите элементы множества P, выраженного через множества:
\begin{equation*}
A = \{0, 3, 4, 9\}; 
B = \{1, 3, 4, 7\};
C = \{0, 1, 2, 4, 7, 8, 9\};
I = \{0, 1, 2, 3, 4, 5, 6, 7, 8, 9\}.
\end{equation*}\question
Упростите следующее выражение с учетом того, что $A\subset B \subset C \subset D \subset U; A \neq \O$
\begin{equation*}
A \cap  \overline{C} \cup B \cap \overline{D} \cup  \overline{A} \cap C \cap  \overline{D}
\end{equation*}

Примечание: U — универсум\question
Дано отношение на множестве $\{1, 2, 3, 4, 5\}$ 
\begin{equation*}
aRb \iff |a-b| = 1
\end{equation*}
Напишите обоснованный ответ какими свойствами обладает или не обладает отношение и почему:   
\begin{enumerate} [a)]\setcounter{enumi}{0}
\item рефлексивность
\item антирефлексивность
\item симметричность
\item асимметричность
\item антисимметричность
\item транзитивность
\end{enumerate}

Обоснуйте свой ответ по каждому из приведенных ниже вопросов:
\begin{enumerate} [a)]\setcounter{enumi}{0}
    \item Является ли это отношение отношением эквивалентности?
    \item Является ли это отношение функциональным?
    \item Каким из отношений соответствия (одно-многозначным, много-многозначный и т.д.) оно является?
    \item К каким из отношений порядка (полного, частичного и т.д.) можно отнести данное отношение?
\end{enumerate}

\question
Установите, является ли каждое из перечисленных ниже отношений на А ($R \subseteq A \times A$) отношением эквивалентности (обоснование ответа обязательно). Для каждого отношения эквивалентности 
постройте классы эквивалентности и постройте граф отношения:
\begin{enumerate}[a)]\setcounter{enumi}{0}
\item А - множество целых чисел и отношение $R = \{(a,b)|a + b = 0\}$
\item $A = \{-10, -9, …, 9, 10\}$ и отношение $R = \{(a,b)|a^{3} = b^{3}\}$
\item На множестве $A = \{1; 2; 3\}$ задано отношение $R = \{(1; 1); (2; 2); (3; 3); (2; 1); (1; 2); (2; 3); (3; 2); (3; 1); (1; 3)\}$

\end{enumerate}\question Составьте полную таблицу истинности, определите, какие переменные являются фиктивными и проверьте, является ли формула тавтологией:
$(P \rightarrow (Q \rightarrow R)) \rightarrow ((P \rightarrow Q) \rightarrow (P \rightarrow R))$

\end{questions}
\newpage
%%% begin test
\begin{flushright}
\begin{tabular}{p{2.8in} r l}
%\textbf{\class} & \textbf{ФИО:} & \makebox[2.5in]{\hrulefill}\\
\textbf{\class} & \textbf{ФИО:} &Войнов Лев Витальевич
\\

\textbf{\examdate} &&\\
%\textbf{Time Limit: \timelimit} & Teaching Assistant & \makebox[2in]{\hrulefill}
\end{tabular}\\
\end{flushright}
\rule[1ex]{\textwidth}{.1pt}


\begin{questions}
\question
Найдите и упростите P:
\begin{equation*}
\overline{P} = A \cap B \cup \overline{A} \cap \overline{B} \cup A \cap C \cup \overline{B} \cap C
\end{equation*}
Затем найдите элементы множества P, выраженного через множества:
\begin{equation*}
A = \{0, 3, 4, 9\}; 
B = \{1, 3, 4, 7\};
C = \{0, 1, 2, 4, 7, 8, 9\};
I = \{0, 1, 2, 3, 4, 5, 6, 7, 8, 9\}.
\end{equation*}\question
Упростите следующее выражение с учетом того, что $A\subset B \subset C \subset D \subset U; A \neq \O$
\begin{equation*}
A \cap B \cup \overline{A} \cap \overline{C} \cup A \cap C \cup \overline{B} \cap \overline{C}
\end{equation*}

Примечание: U — универсум\question
Дано отношение на множестве $\{1, 2, 3, 4, 5\}$ 
\begin{equation*}
aRb \iff a \geq b^2
\end{equation*}
Напишите обоснованный ответ какими свойствами обладает или не обладает отношение и почему:   
\begin{enumerate} [a)]\setcounter{enumi}{0}
\item рефлексивность
\item антирефлексивность
\item симметричность
\item асимметричность
\item антисимметричность
\item транзитивность
\end{enumerate}

Обоснуйте свой ответ по каждому из приведенных ниже вопросов:
\begin{enumerate} [a)]\setcounter{enumi}{0}
    \item Является ли это отношение отношением эквивалентности?
    \item Является ли это отношение функциональным?
    \item Каким из отношений соответствия (одно-многозначным, много-многозначный и т.д.) оно является?
    \item К каким из отношений порядка (полного, частичного и т.д.) можно отнести данное отношение?
\end{enumerate}


\question
Установите, является ли каждое из перечисленных ниже отношений на А ($R \subseteq A \times A$) отношением эквивалентности (обоснование ответа обязательно). Для каждого отношения эквивалентности постройте классы эквивалентности и постройте граф отношения:
\begin{enumerate} [a)]\setcounter{enumi}{0}
\item $F(x)=x^{2}+1$, где $x \in A = [-2, 4]$ и отношение $R = \{(a,b)|F(a) = F(b)\}$
\item А - множество целых чисел и отношение $R = \{(a,b)|a + b = 5\}$
\item На множестве $A = \{1; 2; 3\}$ задано отношение $R = \{(1; 1); (2; 2); (3; 3); (3; 2); (1; 2); (2; 1)\}$

\end{enumerate}\question Составьте полную таблицу истинности, определите, какие переменные являются фиктивными и проверьте, является ли формула тавтологией:

$(P \rightarrow (Q \land R)) \leftrightarrow ((P \rightarrow Q) \land (P \rightarrow R))$

\end{questions}
\newpage
%%% begin test
\begin{flushright}
\begin{tabular}{p{2.8in} r l}
%\textbf{\class} & \textbf{ФИО:} & \makebox[2.5in]{\hrulefill}\\
\textbf{\class} & \textbf{ФИО:} &Высоцкая Валерия
\\

\textbf{\examdate} &&\\
%\textbf{Time Limit: \timelimit} & Teaching Assistant & \makebox[2in]{\hrulefill}
\end{tabular}\\
\end{flushright}
\rule[1ex]{\textwidth}{.1pt}


\begin{questions}
\question
Найдите и упростите P:
\begin{equation*}
\overline{P} = A \cap B \cup \overline{A} \cap \overline{B} \cup A \cap C \cup \overline{B} \cap C
\end{equation*}
Затем найдите элементы множества P, выраженного через множества:
\begin{equation*}
A = \{0, 3, 4, 9\}; 
B = \{1, 3, 4, 7\};
C = \{0, 1, 2, 4, 7, 8, 9\};
I = \{0, 1, 2, 3, 4, 5, 6, 7, 8, 9\}.
\end{equation*}\question
Упростите следующее выражение с учетом того, что $A\subset B \subset C \subset D \subset U; A \neq \O$
\begin{equation*}
A \cap  \overline{C} \cup B \cap \overline{D} \cup  \overline{A} \cap C \cap  \overline{D}
\end{equation*}

Примечание: U — универсум\question
Для следующего отношения на множестве $\{1, 2, 3, 4, 5\}$ 
\begin{equation*}
aRb \iff 0 < a-b<2
\end{equation*}
Напишите обоснованный ответ какими свойствами обладает или не обладает отношение и почему:   
\begin{enumerate} [a)]\setcounter{enumi}{0}
\item рефлексивность
\item антирефлексивность
\item симметричность
\item асимметричность
\item антисимметричность
\item транзитивность
\end{enumerate}

Обоснуйте свой ответ по каждому из приведенных ниже вопросов:
\begin{enumerate} [a)]\setcounter{enumi}{0}
    \item Является ли это отношение отношением эквивалентности?
    \item Является ли это отношение функциональным?
    \item Каким из отношений соответствия (одно-многозначным, много-многозначный и т.д.) оно является?
    \item К каким из отношений порядка (полного, частичного и т.д.) можно отнести данное отношение?
\end{enumerate}
\question
Установите, является ли каждое из перечисленных ниже отношений на А ($R \subseteq A \times A$) отношением эквивалентности (обоснование ответа обязательно). Для каждого отношения эквивалентности постройте классы 
эквивалентности и постройте граф отношения:
\begin{enumerate} [a)]\setcounter{enumi}{0}
\item $A = \{-10, -9, … , 9, 10\}$ и отношение $R = \{(a,b)|a^{2} = b^{2}\}$
\item $A = \{a, b, c, d, p, t\}$ задано отношение $R = \{(a, a), (b, b), (b, c), (b, d), (c, b), (c, c), (c, d), (d, b), (d, c), (d, d), (p,p), (t,t)\}$
\item Пусть A – множество имен. $A = \{ $Алексей, Иван, Петр, Александр, Павел, Андрей$ \}$. Тогда отношение $R$ верно на парах имен, начинающихся с одной и той же буквы, и только на них.
\end{enumerate}\question Составьте полную таблицу истинности, определите, какие переменные являются фиктивными и проверьте, является ли формула тавтологией:
$ P \rightarrow (Q \rightarrow ((P \lor Q) \rightarrow (P \land Q)))$

\end{questions}
\newpage
%%% begin test
\begin{flushright}
\begin{tabular}{p{2.8in} r l}
%\textbf{\class} & \textbf{ФИО:} & \makebox[2.5in]{\hrulefill}\\
\textbf{\class} & \textbf{ФИО:} &Иванов Никита Сергеевич
\\

\textbf{\examdate} &&\\
%\textbf{Time Limit: \timelimit} & Teaching Assistant & \makebox[2in]{\hrulefill}
\end{tabular}\\
\end{flushright}
\rule[1ex]{\textwidth}{.1pt}


\begin{questions}
\question
Найдите и упростите P:
\begin{equation*}
\overline{P} = A \cap C \cup \overline{A} \cap \overline{C} \cup \overline{B} \cap C \cup \overline{A} \cap \overline{B}
\end{equation*}
Затем найдите элементы множества P, выраженного через множества:
\begin{equation*}
A = \{0, 3, 4, 9\}; 
B = \{1, 3, 4, 7\};
C = \{0, 1, 2, 4, 7, 8, 9\};
I = \{0, 1, 2, 3, 4, 5, 6, 7, 8, 9\}.
\end{equation*}\question
Упростите следующее выражение с учетом того, что $A\subset B \subset C \subset D \subset U; A \neq \O$
\begin{equation*}
A \cap C  \cap D \cup B \cap \overline{C} \cap D \cup B \cap C \cap D
\end{equation*}

Примечание: U — универсум\question
Для следующего отношения на множестве $\{1, 2, 3, 4, 5\}$ 
\begin{equation*}
aRb \iff 0 < a-b<2
\end{equation*}
Напишите обоснованный ответ какими свойствами обладает или не обладает отношение и почему:   
\begin{enumerate} [a)]\setcounter{enumi}{0}
\item рефлексивность
\item антирефлексивность
\item симметричность
\item асимметричность
\item антисимметричность
\item транзитивность
\end{enumerate}

Обоснуйте свой ответ по каждому из приведенных ниже вопросов:
\begin{enumerate} [a)]\setcounter{enumi}{0}
    \item Является ли это отношение отношением эквивалентности?
    \item Является ли это отношение функциональным?
    \item Каким из отношений соответствия (одно-многозначным, много-многозначный и т.д.) оно является?
    \item К каким из отношений порядка (полного, частичного и т.д.) можно отнести данное отношение?
\end{enumerate}
\question
Установите, является ли каждое из перечисленных ниже отношений на А ($R \subseteq A \times A$) отношением эквивалентности (обоснование ответа обязательно). Для каждого отношения эквивалентности 
постройте классы эквивалентности и постройте граф отношения:
\begin{enumerate}[a)]\setcounter{enumi}{0}
\item А - множество целых чисел и отношение $R = \{(a,b)|a + b = 0\}$
\item $A = \{-10, -9, …, 9, 10\}$ и отношение $R = \{(a,b)|a^{3} = b^{3}\}$
\item На множестве $A = \{1; 2; 3\}$ задано отношение $R = \{(1; 1); (2; 2); (3; 3); (2; 1); (1; 2); (2; 3); (3; 2); (3; 1); (1; 3)\}$

\end{enumerate}\question Составьте полную таблицу истинности, определите, какие переменные являются фиктивными и проверьте, является ли формула тавтологией:

$(P \rightarrow (Q \land R)) \leftrightarrow ((P \rightarrow Q) \land (P \rightarrow R))$

\end{questions}
\newpage
%%% begin test
\begin{flushright}
\begin{tabular}{p{2.8in} r l}
%\textbf{\class} & \textbf{ФИО:} & \makebox[2.5in]{\hrulefill}\\
\textbf{\class} & \textbf{ФИО:} &Капанин Дмитрий Алексеевич
\\

\textbf{\examdate} &&\\
%\textbf{Time Limit: \timelimit} & Teaching Assistant & \makebox[2in]{\hrulefill}
\end{tabular}\\
\end{flushright}
\rule[1ex]{\textwidth}{.1pt}


\begin{questions}
\question
Найдите и упростите P:
\begin{equation*}
\overline{P} = \overline{A} \cap B \cup \overline{A} \cap C \cup A \cap \overline{B} \cup \overline{B} \cap C
\end{equation*}
Затем найдите элементы множества P, выраженного через множества:
\begin{equation*}
A = \{0, 3, 4, 9\}; 
B = \{1, 3, 4, 7\};
C = \{0, 1, 2, 4, 7, 8, 9\};
I = \{0, 1, 2, 3, 4, 5, 6, 7, 8, 9\}.
\end{equation*}\question
Упростите следующее выражение с учетом того, что $A\subset B \subset C \subset D \subset U; A \neq \O$
\begin{equation*}
A \cap  \overline{C} \cup B \cap \overline{D} \cup  \overline{A} \cap C \cap  \overline{D}
\end{equation*}

Примечание: U — универсум\question
Для следующего отношения на множестве $\{1, 2, 3, 4, 5\}$ 
\begin{equation*}
aRb \iff 0 < a-b<2
\end{equation*}
Напишите обоснованный ответ какими свойствами обладает или не обладает отношение и почему:   
\begin{enumerate} [a)]\setcounter{enumi}{0}
\item рефлексивность
\item антирефлексивность
\item симметричность
\item асимметричность
\item антисимметричность
\item транзитивность
\end{enumerate}

Обоснуйте свой ответ по каждому из приведенных ниже вопросов:
\begin{enumerate} [a)]\setcounter{enumi}{0}
    \item Является ли это отношение отношением эквивалентности?
    \item Является ли это отношение функциональным?
    \item Каким из отношений соответствия (одно-многозначным, много-многозначный и т.д.) оно является?
    \item К каким из отношений порядка (полного, частичного и т.д.) можно отнести данное отношение?
\end{enumerate}
\question
Установите, является ли каждое из перечисленных ниже отношений на А ($R \subseteq A \times A$) отношением эквивалентности (обоснование ответа обязательно). Для каждого отношения эквивалентности постройте классы 
эквивалентности и постройте граф отношения:
\begin{enumerate} [a)]\setcounter{enumi}{0}
\item Пусть A – множество имен. $A = \{ $Алексей, Иван, Петр, Александр, Павел, Андрей$ \}$. Тогда отношение $R$ верно на парах имен, начинающихся с одной и той же буквы, и только на них.
\item $A = \{-10, -9, … , 9, 10\}$ и отношение $ R = \{(a,b)|a^{2} = b^{2}\}$
\item На множестве $A = \{1; 2; 3\}$ задано отношение $R = \{(1; 1); (2; 2); (3; 3); (3; 2); (1; 2); (2; 1)\}$
\end{enumerate}\question Составьте полную таблицу истинности, определите, какие переменные являются фиктивными и проверьте, является ли формула тавтологией:
$(( P \rightarrow Q) \land (Q \rightarrow P)) \rightarrow (P \rightarrow R)$

\end{questions}
\newpage
%%% begin test
\begin{flushright}
\begin{tabular}{p{2.8in} r l}
%\textbf{\class} & \textbf{ФИО:} & \makebox[2.5in]{\hrulefill}\\
\textbf{\class} & \textbf{ФИО:} &Комаров Александр Дмитриевич
\\

\textbf{\examdate} &&\\
%\textbf{Time Limit: \timelimit} & Teaching Assistant & \makebox[2in]{\hrulefill}
\end{tabular}\\
\end{flushright}
\rule[1ex]{\textwidth}{.1pt}


\begin{questions}
\question
Найдите и упростите P:
\begin{equation*}
\overline{P} = A \cap B \cup \overline{A} \cap \overline{B} \cup A \cap C \cup \overline{B} \cap C
\end{equation*}
Затем найдите элементы множества P, выраженного через множества:
\begin{equation*}
A = \{0, 3, 4, 9\}; 
B = \{1, 3, 4, 7\};
C = \{0, 1, 2, 4, 7, 8, 9\};
I = \{0, 1, 2, 3, 4, 5, 6, 7, 8, 9\}.
\end{equation*}\question
Упростите следующее выражение с учетом того, что $A\subset B \subset C \subset D \subset U; A \neq \O$
\begin{equation*}
A \cap  \overline{C} \cup B \cap \overline{D} \cup  \overline{A} \cap C \cap  \overline{D}
\end{equation*}

Примечание: U — универсум\question
Дано отношение на множестве $\{1, 2, 3, 4, 5\}$ 
\begin{equation*}
aRb \iff b > a
\end{equation*}
Напишите обоснованный ответ какими свойствами обладает или не обладает отношение и почему:   
\begin{enumerate} [a)]\setcounter{enumi}{0}
\item рефлексивность
\item антирефлексивность
\item симметричность
\item асимметричность
\item антисимметричность
\item транзитивность
\end{enumerate}

Обоснуйте свой ответ по каждому из приведенных ниже вопросов:
\begin{enumerate} [a)]\setcounter{enumi}{0}
    \item Является ли это отношение отношением эквивалентности?
    \item Является ли это отношение функциональным?
    \item Каким из отношений соответствия (одно-многозначным, много-многозначный и т.д.) оно является?
    \item К каким из отношений порядка (полного, частичного и т.д.) можно отнести данное отношение?
\end{enumerate}

\question
Установите, является ли каждое из перечисленных ниже отношений на А ($R \subseteq A \times A$) отношением эквивалентности (обоснование ответа обязательно). Для каждого отношения эквивалентности постройте классы 
эквивалентности и постройте граф отношения:
\begin{enumerate} [a)]\setcounter{enumi}{0}
\item $A = \{a, b, c, d, p, t\}$ задано отношение $R = \{(a, a), (b, b), (b, c), (b, d), (c, b), (c, c), (c, d), (d, b), (d, c), (d, d), (p,p), (t,t)\}$
\item $A = \{-10, -9, … , 9, 10\}$ и отношение $R = \{(a,b)|a^{3} = b^{3}\}$

\item $F(x)=x^{2}+1$, где $x \in A = [-2, 4]$ и отношение $R = \{(a,b)|F(a) = F(b)\}$
\end{enumerate}\question Составьте полную таблицу истинности, определите, какие переменные являются фиктивными и проверьте, является ли формула тавтологией:
$((P \rightarrow Q) \land (R \rightarrow S) \land \neg (Q \lor S)) \rightarrow \neg (P \lor R)$

\end{questions}
\newpage
%%% begin test
\begin{flushright}
\begin{tabular}{p{2.8in} r l}
%\textbf{\class} & \textbf{ФИО:} & \makebox[2.5in]{\hrulefill}\\
\textbf{\class} & \textbf{ФИО:} &Корнилов Никита Вадимович
\\

\textbf{\examdate} &&\\
%\textbf{Time Limit: \timelimit} & Teaching Assistant & \makebox[2in]{\hrulefill}
\end{tabular}\\
\end{flushright}
\rule[1ex]{\textwidth}{.1pt}


\begin{questions}
\question
Найдите и упростите P:
\begin{equation*}
\overline{P} = A \cap B \cup \overline{A} \cap \overline{B} \cup A \cap C \cup \overline{B} \cap C
\end{equation*}
Затем найдите элементы множества P, выраженного через множества:
\begin{equation*}
A = \{0, 3, 4, 9\}; 
B = \{1, 3, 4, 7\};
C = \{0, 1, 2, 4, 7, 8, 9\};
I = \{0, 1, 2, 3, 4, 5, 6, 7, 8, 9\}.
\end{equation*}\question
Упростите следующее выражение с учетом того, что $A\subset B \subset C \subset D \subset U; A \neq \O$
\begin{equation*}
A \cap C  \cap D \cup B \cap \overline{C} \cap D \cup B \cap C \cap D
\end{equation*}

Примечание: U — универсум\question
Дано отношение на множестве $\{1, 2, 3, 4, 5\}$ 
\begin{equation*}
aRb \iff (a+b) \bmod 2 =0
\end{equation*}
Напишите обоснованный ответ какими свойствами обладает или не обладает отношение и почему:   
\begin{enumerate} [a)]\setcounter{enumi}{0}
\item рефлексивность
\item антирефлексивность
\item симметричность
\item асимметричность
\item антисимметричность
\item транзитивность
\end{enumerate}

Обоснуйте свой ответ по каждому из приведенных ниже вопросов:
\begin{enumerate} [a)]\setcounter{enumi}{0}
    \item Является ли это отношение отношением эквивалентности?
    \item Является ли это отношение функциональным?
    \item Каким из отношений соответствия (одно-многозначным, много-многозначный и т.д.) оно является?
    \item К каким из отношений порядка (полного, частичного и т.д.) можно отнести данное отношение?
\end{enumerate}



\question
Установите, является ли каждое из перечисленных ниже отношений на А ($R \subseteq A \times A$) отношением эквивалентности (обоснование ответа обязательно). Для каждого отношения эквивалентности постройте классы 
эквивалентности и постройте граф отношения:
\begin{enumerate} [a)]\setcounter{enumi}{0}
\item $A = \{-10, -9, … , 9, 10\}$ и отношение $R = \{(a,b)|a^{2} = b^{2}\}$
\item $A = \{a, b, c, d, p, t\}$ задано отношение $R = \{(a, a), (b, b), (b, c), (b, d), (c, b), (c, c), (c, d), (d, b), (d, c), (d, d), (p,p), (t,t)\}$
\item Пусть A – множество имен. $A = \{ $Алексей, Иван, Петр, Александр, Павел, Андрей$ \}$. Тогда отношение $R$ верно на парах имен, начинающихся с одной и той же буквы, и только на них.
\end{enumerate}\question Составьте полную таблицу истинности, определите, какие переменные являются фиктивными и проверьте, является ли формула тавтологией:
$ P \rightarrow (Q \rightarrow ((P \lor Q) \rightarrow (P \land Q)))$

\end{questions}
\newpage
%%% begin test
\begin{flushright}
\begin{tabular}{p{2.8in} r l}
%\textbf{\class} & \textbf{ФИО:} & \makebox[2.5in]{\hrulefill}\\
\textbf{\class} & \textbf{ФИО:} &Круглов Георгий Николаевич
\\

\textbf{\examdate} &&\\
%\textbf{Time Limit: \timelimit} & Teaching Assistant & \makebox[2in]{\hrulefill}
\end{tabular}\\
\end{flushright}
\rule[1ex]{\textwidth}{.1pt}


\begin{questions}
\question
Найдите и упростите P:
\begin{equation*}
\overline{P} = A \cap \overline{B} \cup \overline{B} \cap C \cup \overline{A} \cap \overline{B} \cup \overline{A} \cap C
\end{equation*}
Затем найдите элементы множества P, выраженного через множества:
\begin{equation*}
A = \{0, 3, 4, 9\}; 
B = \{1, 3, 4, 7\};
C = \{0, 1, 2, 4, 7, 8, 9\};
I = \{0, 1, 2, 3, 4, 5, 6, 7, 8, 9\}.
\end{equation*}\question
Упростите следующее выражение с учетом того, что $A\subset B \subset C \subset D \subset U; A \neq \O$
\begin{equation*}
A \cap C  \cap D \cup B \cap \overline{C} \cap D \cup B \cap C \cap D
\end{equation*}

Примечание: U — универсум\question
Дано отношение на множестве $\{1, 2, 3, 4, 5\}$ 
\begin{equation*}
aRb \iff |a-b| = 1
\end{equation*}
Напишите обоснованный ответ какими свойствами обладает или не обладает отношение и почему:   
\begin{enumerate} [a)]\setcounter{enumi}{0}
\item рефлексивность
\item антирефлексивность
\item симметричность
\item асимметричность
\item антисимметричность
\item транзитивность
\end{enumerate}

Обоснуйте свой ответ по каждому из приведенных ниже вопросов:
\begin{enumerate} [a)]\setcounter{enumi}{0}
    \item Является ли это отношение отношением эквивалентности?
    \item Является ли это отношение функциональным?
    \item Каким из отношений соответствия (одно-многозначным, много-многозначный и т.д.) оно является?
    \item К каким из отношений порядка (полного, частичного и т.д.) можно отнести данное отношение?
\end{enumerate}

\question
Установите, является ли каждое из перечисленных ниже отношений на А ($R \subseteq A \times A$) отношением эквивалентности (обоснование ответа обязательно). Для каждого отношения эквивалентности 
постройте классы эквивалентности и постройте граф отношения:
\begin{enumerate}[a)]\setcounter{enumi}{0}
\item А - множество целых чисел и отношение $R = \{(a,b)|a + b = 0\}$
\item $A = \{-10, -9, …, 9, 10\}$ и отношение $R = \{(a,b)|a^{3} = b^{3}\}$
\item На множестве $A = \{1; 2; 3\}$ задано отношение $R = \{(1; 1); (2; 2); (3; 3); (2; 1); (1; 2); (2; 3); (3; 2); (3; 1); (1; 3)\}$

\end{enumerate}\question Составьте полную таблицу истинности, определите, какие переменные являются фиктивными и проверьте, является ли формула тавтологией:

$(P \rightarrow (Q \land R)) \leftrightarrow ((P \rightarrow Q) \land (P \rightarrow R))$

\end{questions}
\newpage
%%% begin test
\begin{flushright}
\begin{tabular}{p{2.8in} r l}
%\textbf{\class} & \textbf{ФИО:} & \makebox[2.5in]{\hrulefill}\\
\textbf{\class} & \textbf{ФИО:} &Крыжанков Степан Сергеевич
\\

\textbf{\examdate} &&\\
%\textbf{Time Limit: \timelimit} & Teaching Assistant & \makebox[2in]{\hrulefill}
\end{tabular}\\
\end{flushright}
\rule[1ex]{\textwidth}{.1pt}


\begin{questions}
\question
Найдите и упростите P:
\begin{equation*}
\overline{P} = \overline{A} \cap B \cup \overline{A} \cap C \cup A \cap \overline{B} \cup \overline{B} \cap C
\end{equation*}
Затем найдите элементы множества P, выраженного через множества:
\begin{equation*}
A = \{0, 3, 4, 9\}; 
B = \{1, 3, 4, 7\};
C = \{0, 1, 2, 4, 7, 8, 9\};
I = \{0, 1, 2, 3, 4, 5, 6, 7, 8, 9\}.
\end{equation*}\question
Упростите следующее выражение с учетом того, что $A\subset B \subset C \subset D \subset U; A \neq \O$
\begin{equation*}
A \cap  \overline{C} \cup B \cap \overline{D} \cup  \overline{A} \cap C \cap  \overline{D}
\end{equation*}

Примечание: U — универсум\question
Дано отношение на множестве $\{1, 2, 3, 4, 5\}$ 
\begin{equation*}
aRb \iff  \text{НОД}(a,b) =1
\end{equation*}
Напишите обоснованный ответ какими свойствами обладает или не обладает отношение и почему:   
\begin{enumerate} [a)]\setcounter{enumi}{0}
\item рефлексивность
\item антирефлексивность
\item симметричность
\item асимметричность
\item антисимметричность
\item транзитивность
\end{enumerate}

Обоснуйте свой ответ по каждому из приведенных ниже вопросов:
\begin{enumerate} [a)]\setcounter{enumi}{0}
    \item Является ли это отношение отношением эквивалентности?
    \item Является ли это отношение функциональным?
    \item Каким из отношений соответствия (одно-многозначным, много-многозначный и т.д.) оно является?
    \item К каким из отношений порядка (полного, частичного и т.д.) можно отнести данное отношение?
\end{enumerate}


\question
Установите, является ли каждое из перечисленных ниже отношений на А ($R \subseteq A \times A$) отношением эквивалентности (обоснование ответа обязательно). Для каждого отношения эквивалентности 
постройте классы эквивалентности и постройте граф отношения:
\begin{enumerate}[a)]\setcounter{enumi}{0}
\item А - множество целых чисел и отношение $R = \{(a,b)|a + b = 0\}$
\item $A = \{-10, -9, …, 9, 10\}$ и отношение $R = \{(a,b)|a^{3} = b^{3}\}$
\item На множестве $A = \{1; 2; 3\}$ задано отношение $R = \{(1; 1); (2; 2); (3; 3); (2; 1); (1; 2); (2; 3); (3; 2); (3; 1); (1; 3)\}$

\end{enumerate}\question Составьте полную таблицу истинности, определите, какие переменные являются фиктивными и проверьте, является ли формула тавтологией:
$((P \rightarrow Q) \land (R \rightarrow S) \land \neg (Q \lor S)) \rightarrow \neg (P \lor R)$

\end{questions}
\newpage
%%% begin test
\begin{flushright}
\begin{tabular}{p{2.8in} r l}
%\textbf{\class} & \textbf{ФИО:} & \makebox[2.5in]{\hrulefill}\\
\textbf{\class} & \textbf{ФИО:} &Кукуев Артём Владимирович
\\

\textbf{\examdate} &&\\
%\textbf{Time Limit: \timelimit} & Teaching Assistant & \makebox[2in]{\hrulefill}
\end{tabular}\\
\end{flushright}
\rule[1ex]{\textwidth}{.1pt}


\begin{questions}
\question
Найдите и упростите P:
\begin{equation*}
\overline{P} = A \cap \overline{B} \cup A \cap C \cup B \cap C \cup \overline{A} \cap C
\end{equation*}
Затем найдите элементы множества P, выраженного через множества:
\begin{equation*}
A = \{0, 3, 4, 9\}; 
B = \{1, 3, 4, 7\};
C = \{0, 1, 2, 4, 7, 8, 9\};
I = \{0, 1, 2, 3, 4, 5, 6, 7, 8, 9\}.
\end{equation*}\question
Упростите следующее выражение с учетом того, что $A\subset B \subset C \subset D \subset U; A \neq \O$
\begin{equation*}
A \cap  \overline{C} \cup B \cap \overline{D} \cup  \overline{A} \cap C \cap  \overline{D}
\end{equation*}

Примечание: U — универсум\question
Дано отношение на множестве $\{1, 2, 3, 4, 5\}$ 
\begin{equation*}
aRb \iff b > a
\end{equation*}
Напишите обоснованный ответ какими свойствами обладает или не обладает отношение и почему:   
\begin{enumerate} [a)]\setcounter{enumi}{0}
\item рефлексивность
\item антирефлексивность
\item симметричность
\item асимметричность
\item антисимметричность
\item транзитивность
\end{enumerate}

Обоснуйте свой ответ по каждому из приведенных ниже вопросов:
\begin{enumerate} [a)]\setcounter{enumi}{0}
    \item Является ли это отношение отношением эквивалентности?
    \item Является ли это отношение функциональным?
    \item Каким из отношений соответствия (одно-многозначным, много-многозначный и т.д.) оно является?
    \item К каким из отношений порядка (полного, частичного и т.д.) можно отнести данное отношение?
\end{enumerate}

\question
Установите, является ли каждое из перечисленных ниже отношений на А ($R \subseteq A \times A$) отношением эквивалентности (обоснование ответа обязательно). Для каждого отношения эквивалентности постройте классы 
эквивалентности и постройте граф отношения:
\begin{enumerate} [a)]\setcounter{enumi}{0}
\item $A = \{-10, -9, … , 9, 10\}$ и отношение $R = \{(a,b)|a^{2} = b^{2}\}$
\item $A = \{a, b, c, d, p, t\}$ задано отношение $R = \{(a, a), (b, b), (b, c), (b, d), (c, b), (c, c), (c, d), (d, b), (d, c), (d, d), (p,p), (t,t)\}$
\item Пусть A – множество имен. $A = \{ $Алексей, Иван, Петр, Александр, Павел, Андрей$ \}$. Тогда отношение $R$ верно на парах имен, начинающихся с одной и той же буквы, и только на них.
\end{enumerate}\question Составьте полную таблицу истинности, определите, какие переменные являются фиктивными и проверьте, является ли формула тавтологией:
$(( P \rightarrow Q) \land (Q \rightarrow P)) \rightarrow (P \rightarrow R)$

\end{questions}
\newpage
%%% begin test
\begin{flushright}
\begin{tabular}{p{2.8in} r l}
%\textbf{\class} & \textbf{ФИО:} & \makebox[2.5in]{\hrulefill}\\
\textbf{\class} & \textbf{ФИО:} &Кутузов Михаил Владимирович
\\

\textbf{\examdate} &&\\
%\textbf{Time Limit: \timelimit} & Teaching Assistant & \makebox[2in]{\hrulefill}
\end{tabular}\\
\end{flushright}
\rule[1ex]{\textwidth}{.1pt}


\begin{questions}
\question
Найдите и упростите P:
\begin{equation*}
\overline{P} = A \cap \overline{B} \cup \overline{B} \cap C \cup \overline{A} \cap \overline{B} \cup \overline{A} \cap C
\end{equation*}
Затем найдите элементы множества P, выраженного через множества:
\begin{equation*}
A = \{0, 3, 4, 9\}; 
B = \{1, 3, 4, 7\};
C = \{0, 1, 2, 4, 7, 8, 9\};
I = \{0, 1, 2, 3, 4, 5, 6, 7, 8, 9\}.
\end{equation*}\question
Упростите следующее выражение с учетом того, что $A\subset B \subset C \subset D \subset U; A \neq \O$
\begin{equation*}
A \cap B \cup \overline{A} \cap \overline{C} \cup A \cap C \cup \overline{B} \cap \overline{C}
\end{equation*}

Примечание: U — универсум\question
Для следующего отношения на множестве $\{1, 2, 3, 4, 5\}$ 
\begin{equation*}
aRb \iff 0 < a-b<2
\end{equation*}
Напишите обоснованный ответ какими свойствами обладает или не обладает отношение и почему:   
\begin{enumerate} [a)]\setcounter{enumi}{0}
\item рефлексивность
\item антирефлексивность
\item симметричность
\item асимметричность
\item антисимметричность
\item транзитивность
\end{enumerate}

Обоснуйте свой ответ по каждому из приведенных ниже вопросов:
\begin{enumerate} [a)]\setcounter{enumi}{0}
    \item Является ли это отношение отношением эквивалентности?
    \item Является ли это отношение функциональным?
    \item Каким из отношений соответствия (одно-многозначным, много-многозначный и т.д.) оно является?
    \item К каким из отношений порядка (полного, частичного и т.д.) можно отнести данное отношение?
\end{enumerate}
\question
Установите, является ли каждое из перечисленных ниже отношений на А ($R \subseteq A \times A$) отношением эквивалентности (обоснование ответа обязательно). Для каждого отношения эквивалентности постройте классы 
эквивалентности и постройте граф отношения:
\begin{enumerate} [a)]\setcounter{enumi}{0}
\item $A = \{a, b, c, d, p, t\}$ задано отношение $R = \{(a, a), (b, b), (b, c), (b, d), (c, b), (c, c), (c, d), (d, b), (d, c), (d, d), (p,p), (t,t)\}$
\item $A = \{-10, -9, … , 9, 10\}$ и отношение $R = \{(a,b)|a^{3} = b^{3}\}$

\item $F(x)=x^{2}+1$, где $x \in A = [-2, 4]$ и отношение $R = \{(a,b)|F(a) = F(b)\}$
\end{enumerate}\question Составьте полную таблицу истинности, определите, какие переменные являются фиктивными и проверьте, является ли формула тавтологией:
$(P \rightarrow (Q \rightarrow R)) \rightarrow ((P \rightarrow Q) \rightarrow (P \rightarrow R))$

\end{questions}
\newpage
%%% begin test
\begin{flushright}
\begin{tabular}{p{2.8in} r l}
%\textbf{\class} & \textbf{ФИО:} & \makebox[2.5in]{\hrulefill}\\
\textbf{\class} & \textbf{ФИО:} &Логачева Елизавета Дмитриевна
\\

\textbf{\examdate} &&\\
%\textbf{Time Limit: \timelimit} & Teaching Assistant & \makebox[2in]{\hrulefill}
\end{tabular}\\
\end{flushright}
\rule[1ex]{\textwidth}{.1pt}


\begin{questions}
\question
Найдите и упростите P:
\begin{equation*}
\overline{P} = \overline{A} \cap B \cup \overline{A} \cap C \cup A \cap \overline{B} \cup \overline{B} \cap C
\end{equation*}
Затем найдите элементы множества P, выраженного через множества:
\begin{equation*}
A = \{0, 3, 4, 9\}; 
B = \{1, 3, 4, 7\};
C = \{0, 1, 2, 4, 7, 8, 9\};
I = \{0, 1, 2, 3, 4, 5, 6, 7, 8, 9\}.
\end{equation*}\question
Упростите следующее выражение с учетом того, что $A\subset B \subset C \subset D \subset U; A \neq \O$
\begin{equation*}
A \cap C  \cap D \cup B \cap \overline{C} \cap D \cup B \cap C \cap D
\end{equation*}

Примечание: U — универсум\question
Дано отношение на множестве $\{1, 2, 3, 4, 5\}$ 
\begin{equation*}
aRb \iff  \text{НОД}(a,b) =1
\end{equation*}
Напишите обоснованный ответ какими свойствами обладает или не обладает отношение и почему:   
\begin{enumerate} [a)]\setcounter{enumi}{0}
\item рефлексивность
\item антирефлексивность
\item симметричность
\item асимметричность
\item антисимметричность
\item транзитивность
\end{enumerate}

Обоснуйте свой ответ по каждому из приведенных ниже вопросов:
\begin{enumerate} [a)]\setcounter{enumi}{0}
    \item Является ли это отношение отношением эквивалентности?
    \item Является ли это отношение функциональным?
    \item Каким из отношений соответствия (одно-многозначным, много-многозначный и т.д.) оно является?
    \item К каким из отношений порядка (полного, частичного и т.д.) можно отнести данное отношение?
\end{enumerate}


\question
Установите, является ли каждое из перечисленных ниже отношений на А ($R \subseteq A \times A$) отношением эквивалентности (обоснование ответа обязательно). Для каждого отношения эквивалентности постройте классы 
эквивалентности и постройте граф отношения:
\begin{enumerate} [a)]\setcounter{enumi}{0}
\item На множестве $A = \{1; 2; 3\}$ задано отношение $R = \{(1; 1); (2; 2); (3; 3); (2; 1); (1; 2); (2; 3); (3; 2); (3; 1); (1; 3)\}$
\item На множестве $A = \{1; 2; 3; 4; 5\}$ задано отношение $R = \{(1; 2); (1; 3); (1; 5); (2; 3); (2; 4); (2; 5); (3; 4); (3; 5); (4; 5)\}$
\item А - множество целых чисел и отношение $R = \{(a,b)|a + b = 0\}$
\end{enumerate}\question Составьте полную таблицу истинности, определите, какие переменные являются фиктивными и проверьте, является ли формула тавтологией:
$((P \rightarrow Q) \land (R \rightarrow S) \land \neg (Q \lor S)) \rightarrow \neg (P \lor R)$

\end{questions}
\newpage
%%% begin test
\begin{flushright}
\begin{tabular}{p{2.8in} r l}
%\textbf{\class} & \textbf{ФИО:} & \makebox[2.5in]{\hrulefill}\\
\textbf{\class} & \textbf{ФИО:} &Минаев Юрий Евгеньевич
\\

\textbf{\examdate} &&\\
%\textbf{Time Limit: \timelimit} & Teaching Assistant & \makebox[2in]{\hrulefill}
\end{tabular}\\
\end{flushright}
\rule[1ex]{\textwidth}{.1pt}


\begin{questions}
\question
Найдите и упростите P:
\begin{equation*}
\overline{P} = B \cap \overline{C} \cup A \cap B \cup \overline{A} \cap C \cup \overline{A} \cap B
\end{equation*}
Затем найдите элементы множества P, выраженного через множества:
\begin{equation*}
A = \{0, 3, 4, 9\}; 
B = \{1, 3, 4, 7\};
C = \{0, 1, 2, 4, 7, 8, 9\};
I = \{0, 1, 2, 3, 4, 5, 6, 7, 8, 9\}.
\end{equation*}\question
Упростите следующее выражение с учетом того, что $A\subset B \subset C \subset D \subset U; A \neq \O$
\begin{equation*}
A \cap B \cup \overline{A} \cap \overline{C} \cup A \cap C \cup \overline{B} \cap \overline{C}
\end{equation*}

Примечание: U — универсум\question
Дано отношение на множестве $\{1, 2, 3, 4, 5\}$ 
\begin{equation*}
aRb \iff b > a
\end{equation*}
Напишите обоснованный ответ какими свойствами обладает или не обладает отношение и почему:   
\begin{enumerate} [a)]\setcounter{enumi}{0}
\item рефлексивность
\item антирефлексивность
\item симметричность
\item асимметричность
\item антисимметричность
\item транзитивность
\end{enumerate}

Обоснуйте свой ответ по каждому из приведенных ниже вопросов:
\begin{enumerate} [a)]\setcounter{enumi}{0}
    \item Является ли это отношение отношением эквивалентности?
    \item Является ли это отношение функциональным?
    \item Каким из отношений соответствия (одно-многозначным, много-многозначный и т.д.) оно является?
    \item К каким из отношений порядка (полного, частичного и т.д.) можно отнести данное отношение?
\end{enumerate}

\question
Установите, является ли каждое из перечисленных ниже отношений на А ($R \subseteq A \times A$) отношением эквивалентности (обоснование ответа обязательно). Для каждого отношения эквивалентности постройте классы 
эквивалентности и постройте граф отношения:
\begin{enumerate} [a)]\setcounter{enumi}{0}
\item А - множество целых чисел и отношение $R = \{(a,b)|a + b = 5\}$
\item Пусть A – множество имен. $A = \{ $Алексей, Иван, Петр, Александр, Павел, Андрей$ \}$. Тогда отношение $R $ верно на парах имен, начинающихся с одной и той же буквы, и только на них.
\item На множестве $A = \{1; 2; 3; 4; 5\}$ задано отношение $R = \{(1; 2); (1; 3); (1; 5); (2; 3); (2; 4); (2; 5); (3; 4); (3; 5); (4; 5)\}$
\end{enumerate}\question Составьте полную таблицу истинности, определите, какие переменные являются фиктивными и проверьте, является ли формула тавтологией:
$ P \rightarrow (Q \rightarrow ((P \lor Q) \rightarrow (P \land Q)))$

\end{questions}
\newpage
%%% begin test
\begin{flushright}
\begin{tabular}{p{2.8in} r l}
%\textbf{\class} & \textbf{ФИО:} & \makebox[2.5in]{\hrulefill}\\
\textbf{\class} & \textbf{ФИО:} &Михайличенко Глеб Бориславович
\\

\textbf{\examdate} &&\\
%\textbf{Time Limit: \timelimit} & Teaching Assistant & \makebox[2in]{\hrulefill}
\end{tabular}\\
\end{flushright}
\rule[1ex]{\textwidth}{.1pt}


\begin{questions}
\question
Найдите и упростите P:
\begin{equation*}
\overline{P} = A \cap C \cup \overline{A} \cap \overline{C} \cup \overline{B} \cap C \cup \overline{A} \cap \overline{B}
\end{equation*}
Затем найдите элементы множества P, выраженного через множества:
\begin{equation*}
A = \{0, 3, 4, 9\}; 
B = \{1, 3, 4, 7\};
C = \{0, 1, 2, 4, 7, 8, 9\};
I = \{0, 1, 2, 3, 4, 5, 6, 7, 8, 9\}.
\end{equation*}\question
Упростите следующее выражение с учетом того, что $A\subset B \subset C \subset D \subset U; A \neq \O$
\begin{equation*}
\overline{A} \cap \overline{B} \cup B \cap \overline{C} \cup \overline{C} \cap D
\end{equation*}

Примечание: U — универсум\question
Дано отношение на множестве $\{1, 2, 3, 4, 5\}$ 
\begin{equation*}
aRb \iff a \leq b
\end{equation*}
Напишите обоснованный ответ какими свойствами обладает или не обладает отношение и почему:   
\begin{enumerate} [a)]\setcounter{enumi}{0}
\item рефлексивность
\item антирефлексивность
\item симметричность
\item асимметричность
\item антисимметричность
\item транзитивность
\end{enumerate}

Обоснуйте свой ответ по каждому из приведенных ниже вопросов:
\begin{enumerate} [a)]\setcounter{enumi}{0}
    \item Является ли это отношение отношением эквивалентности?
    \item Является ли это отношение функциональным?
    \item Каким из отношений соответствия (одно-многозначным, много-многозначный и т.д.) оно является?
    \item К каким из отношений порядка (полного, частичного и т.д.) можно отнести данное отношение?
\end{enumerate}


\question
Установите, является ли каждое из перечисленных ниже отношений на А ($R \subseteq A \times A$) отношением эквивалентности (обоснование ответа обязательно). Для каждого отношения эквивалентности постройте классы 
эквивалентности и постройте граф отношения:
\begin{enumerate} [a)]\setcounter{enumi}{0}
\item Пусть A – множество имен. $A = \{ $Алексей, Иван, Петр, Александр, Павел, Андрей$ \}$. Тогда отношение $R$ верно на парах имен, начинающихся с одной и той же буквы, и только на них.
\item $A = \{-10, -9, … , 9, 10\}$ и отношение $ R = \{(a,b)|a^{2} = b^{2}\}$
\item На множестве $A = \{1; 2; 3\}$ задано отношение $R = \{(1; 1); (2; 2); (3; 3); (3; 2); (1; 2); (2; 1)\}$
\end{enumerate}\question Составьте полную таблицу истинности, определите, какие переменные являются фиктивными и проверьте, является ли формула тавтологией:
$(P \rightarrow (Q \rightarrow R)) \rightarrow ((P \rightarrow Q) \rightarrow (P \rightarrow R))$

\end{questions}
\newpage
%%% begin test
\begin{flushright}
\begin{tabular}{p{2.8in} r l}
%\textbf{\class} & \textbf{ФИО:} & \makebox[2.5in]{\hrulefill}\\
\textbf{\class} & \textbf{ФИО:} &Муллер Даниил Александрович
\\

\textbf{\examdate} &&\\
%\textbf{Time Limit: \timelimit} & Teaching Assistant & \makebox[2in]{\hrulefill}
\end{tabular}\\
\end{flushright}
\rule[1ex]{\textwidth}{.1pt}


\begin{questions}
\question
Найдите и упростите P:
\begin{equation*}
\overline{P} = A \cap \overline{C} \cup A \cap \overline{B} \cup B \cap \overline{C} \cup A \cap C
\end{equation*}
Затем найдите элементы множества P, выраженного через множества:
\begin{equation*}
A = \{0, 3, 4, 9\}; 
B = \{1, 3, 4, 7\};
C = \{0, 1, 2, 4, 7, 8, 9\};
I = \{0, 1, 2, 3, 4, 5, 6, 7, 8, 9\}.
\end{equation*}\question
Упростите следующее выражение с учетом того, что $A\subset B \subset C \subset D \subset U; A \neq \O$
\begin{equation*}
A \cap  \overline{C} \cup B \cap \overline{D} \cup  \overline{A} \cap C \cap  \overline{D}
\end{equation*}

Примечание: U — универсум\question
Дано отношение на множестве $\{1, 2, 3, 4, 5\}$ 
\begin{equation*}
aRb \iff a \leq b
\end{equation*}
Напишите обоснованный ответ какими свойствами обладает или не обладает отношение и почему:   
\begin{enumerate} [a)]\setcounter{enumi}{0}
\item рефлексивность
\item антирефлексивность
\item симметричность
\item асимметричность
\item антисимметричность
\item транзитивность
\end{enumerate}

Обоснуйте свой ответ по каждому из приведенных ниже вопросов:
\begin{enumerate} [a)]\setcounter{enumi}{0}
    \item Является ли это отношение отношением эквивалентности?
    \item Является ли это отношение функциональным?
    \item Каким из отношений соответствия (одно-многозначным, много-многозначный и т.д.) оно является?
    \item К каким из отношений порядка (полного, частичного и т.д.) можно отнести данное отношение?
\end{enumerate}


\question
Установите, является ли каждое из перечисленных ниже отношений на А ($R \subseteq A \times A$) отношением эквивалентности (обоснование ответа обязательно). Для каждого отношения эквивалентности постройте классы 
эквивалентности и постройте граф отношения:
\begin{enumerate} [a)]\setcounter{enumi}{0}
\item На множестве $A = \{1; 2; 3\}$ задано отношение $R = \{(1; 1); (2; 2); (3; 3); (2; 1); (1; 2); (2; 3); (3; 2); (3; 1); (1; 3)\}$
\item На множестве $A = \{1; 2; 3; 4; 5\}$ задано отношение $R = \{(1; 2); (1; 3); (1; 5); (2; 3); (2; 4); (2; 5); (3; 4); (3; 5); (4; 5)\}$
\item А - множество целых чисел и отношение $R = \{(a,b)|a + b = 0\}$
\end{enumerate}\question Составьте полную таблицу истинности, определите, какие переменные являются фиктивными и проверьте, является ли формула тавтологией:

$(P \rightarrow (Q \land R)) \leftrightarrow ((P \rightarrow Q) \land (P \rightarrow R))$

\end{questions}
\newpage
%%% begin test
\begin{flushright}
\begin{tabular}{p{2.8in} r l}
%\textbf{\class} & \textbf{ФИО:} & \makebox[2.5in]{\hrulefill}\\
\textbf{\class} & \textbf{ФИО:} &Мурашов Иван Григорьевич
\\

\textbf{\examdate} &&\\
%\textbf{Time Limit: \timelimit} & Teaching Assistant & \makebox[2in]{\hrulefill}
\end{tabular}\\
\end{flushright}
\rule[1ex]{\textwidth}{.1pt}


\begin{questions}
\question
Найдите и упростите P:
\begin{equation*}
\overline{P} = A \cap \overline{B} \cup \overline{B} \cap C \cup \overline{A} \cap \overline{B} \cup \overline{A} \cap C
\end{equation*}
Затем найдите элементы множества P, выраженного через множества:
\begin{equation*}
A = \{0, 3, 4, 9\}; 
B = \{1, 3, 4, 7\};
C = \{0, 1, 2, 4, 7, 8, 9\};
I = \{0, 1, 2, 3, 4, 5, 6, 7, 8, 9\}.
\end{equation*}\question
Упростите следующее выражение с учетом того, что $A\subset B \subset C \subset D \subset U; A \neq \O$
\begin{equation*}
\overline{A} \cap \overline{C} \cap D \cup \overline{B} \cap \overline{C} \cap D \cup A \cap B
\end{equation*}

Примечание: U — универсум\question
Дано отношение на множестве $\{1, 2, 3, 4, 5\}$ 
\begin{equation*}
aRb \iff b > a
\end{equation*}
Напишите обоснованный ответ какими свойствами обладает или не обладает отношение и почему:   
\begin{enumerate} [a)]\setcounter{enumi}{0}
\item рефлексивность
\item антирефлексивность
\item симметричность
\item асимметричность
\item антисимметричность
\item транзитивность
\end{enumerate}

Обоснуйте свой ответ по каждому из приведенных ниже вопросов:
\begin{enumerate} [a)]\setcounter{enumi}{0}
    \item Является ли это отношение отношением эквивалентности?
    \item Является ли это отношение функциональным?
    \item Каким из отношений соответствия (одно-многозначным, много-многозначный и т.д.) оно является?
    \item К каким из отношений порядка (полного, частичного и т.д.) можно отнести данное отношение?
\end{enumerate}

\question
Установите, является ли каждое из перечисленных ниже отношений на А ($R \subseteq A \times A$) отношением эквивалентности (обоснование ответа обязательно). Для каждого отношения эквивалентности 
постройте классы эквивалентности и постройте граф отношения:
\begin{enumerate}[a)]\setcounter{enumi}{0}
\item А - множество целых чисел и отношение $R = \{(a,b)|a + b = 0\}$
\item $A = \{-10, -9, …, 9, 10\}$ и отношение $R = \{(a,b)|a^{3} = b^{3}\}$
\item На множестве $A = \{1; 2; 3\}$ задано отношение $R = \{(1; 1); (2; 2); (3; 3); (2; 1); (1; 2); (2; 3); (3; 2); (3; 1); (1; 3)\}$

\end{enumerate}\question Составьте полную таблицу истинности, определите, какие переменные являются фиктивными и проверьте, является ли формула тавтологией:
$ P \rightarrow (Q \rightarrow ((P \lor Q) \rightarrow (P \land Q)))$

\end{questions}
\newpage
%%% begin test
\begin{flushright}
\begin{tabular}{p{2.8in} r l}
%\textbf{\class} & \textbf{ФИО:} & \makebox[2.5in]{\hrulefill}\\
\textbf{\class} & \textbf{ФИО:} &Окорочкова Мария Валентиновна
\\

\textbf{\examdate} &&\\
%\textbf{Time Limit: \timelimit} & Teaching Assistant & \makebox[2in]{\hrulefill}
\end{tabular}\\
\end{flushright}
\rule[1ex]{\textwidth}{.1pt}


\begin{questions}
\question
Найдите и упростите P:
\begin{equation*}
\overline{P} = B \cap \overline{C} \cup A \cap B \cup \overline{A} \cap C \cup \overline{A} \cap B
\end{equation*}
Затем найдите элементы множества P, выраженного через множества:
\begin{equation*}
A = \{0, 3, 4, 9\}; 
B = \{1, 3, 4, 7\};
C = \{0, 1, 2, 4, 7, 8, 9\};
I = \{0, 1, 2, 3, 4, 5, 6, 7, 8, 9\}.
\end{equation*}\question
Упростите следующее выражение с учетом того, что $A\subset B \subset C \subset D \subset U; A \neq \O$
\begin{equation*}
A \cap  \overline{C} \cup B \cap \overline{D} \cup  \overline{A} \cap C \cap  \overline{D}
\end{equation*}

Примечание: U — универсум\question
Дано отношение на множестве $\{1, 2, 3, 4, 5\}$ 
\begin{equation*}
aRb \iff (a+b) \bmod 2 =0
\end{equation*}
Напишите обоснованный ответ какими свойствами обладает или не обладает отношение и почему:   
\begin{enumerate} [a)]\setcounter{enumi}{0}
\item рефлексивность
\item антирефлексивность
\item симметричность
\item асимметричность
\item антисимметричность
\item транзитивность
\end{enumerate}

Обоснуйте свой ответ по каждому из приведенных ниже вопросов:
\begin{enumerate} [a)]\setcounter{enumi}{0}
    \item Является ли это отношение отношением эквивалентности?
    \item Является ли это отношение функциональным?
    \item Каким из отношений соответствия (одно-многозначным, много-многозначный и т.д.) оно является?
    \item К каким из отношений порядка (полного, частичного и т.д.) можно отнести данное отношение?
\end{enumerate}



\question
Установите, является ли каждое из перечисленных ниже отношений на А ($R \subseteq A \times A$) отношением эквивалентности (обоснование ответа обязательно). Для каждого отношения эквивалентности постройте классы 
эквивалентности и постройте граф отношения:
\begin{enumerate} [a)]\setcounter{enumi}{0}
\item А - множество целых чисел и отношение $R = \{(a,b)|a + b = 5\}$
\item Пусть A – множество имен. $A = \{ $Алексей, Иван, Петр, Александр, Павел, Андрей$ \}$. Тогда отношение $R $ верно на парах имен, начинающихся с одной и той же буквы, и только на них.
\item На множестве $A = \{1; 2; 3; 4; 5\}$ задано отношение $R = \{(1; 2); (1; 3); (1; 5); (2; 3); (2; 4); (2; 5); (3; 4); (3; 5); (4; 5)\}$
\end{enumerate}\question Составьте полную таблицу истинности, определите, какие переменные являются фиктивными и проверьте, является ли формула тавтологией:
$(( P \rightarrow Q) \land (Q \rightarrow P)) \rightarrow (P \rightarrow R)$

\end{questions}
\newpage
%%% begin test
\begin{flushright}
\begin{tabular}{p{2.8in} r l}
%\textbf{\class} & \textbf{ФИО:} & \makebox[2.5in]{\hrulefill}\\
\textbf{\class} & \textbf{ФИО:} &Полухин Максим Денисович
\\

\textbf{\examdate} &&\\
%\textbf{Time Limit: \timelimit} & Teaching Assistant & \makebox[2in]{\hrulefill}
\end{tabular}\\
\end{flushright}
\rule[1ex]{\textwidth}{.1pt}


\begin{questions}
\question
Найдите и упростите P:
\begin{equation*}
\overline{P} = A \cap \overline{B} \cup A \cap C \cup B \cap C \cup \overline{A} \cap C
\end{equation*}
Затем найдите элементы множества P, выраженного через множества:
\begin{equation*}
A = \{0, 3, 4, 9\}; 
B = \{1, 3, 4, 7\};
C = \{0, 1, 2, 4, 7, 8, 9\};
I = \{0, 1, 2, 3, 4, 5, 6, 7, 8, 9\}.
\end{equation*}\question
Упростите следующее выражение с учетом того, что $A\subset B \subset C \subset D \subset U; A \neq \O$
\begin{equation*}
\overline{A} \cap \overline{B} \cup B \cap \overline{C} \cup \overline{C} \cap D
\end{equation*}

Примечание: U — универсум\question
Дано отношение на множестве $\{1, 2, 3, 4, 5\}$ 
\begin{equation*}
aRb \iff a \leq b
\end{equation*}
Напишите обоснованный ответ какими свойствами обладает или не обладает отношение и почему:   
\begin{enumerate} [a)]\setcounter{enumi}{0}
\item рефлексивность
\item антирефлексивность
\item симметричность
\item асимметричность
\item антисимметричность
\item транзитивность
\end{enumerate}

Обоснуйте свой ответ по каждому из приведенных ниже вопросов:
\begin{enumerate} [a)]\setcounter{enumi}{0}
    \item Является ли это отношение отношением эквивалентности?
    \item Является ли это отношение функциональным?
    \item Каким из отношений соответствия (одно-многозначным, много-многозначный и т.д.) оно является?
    \item К каким из отношений порядка (полного, частичного и т.д.) можно отнести данное отношение?
\end{enumerate}


\question
Установите, является ли каждое из перечисленных ниже отношений на А ($R \subseteq A \times A$) отношением эквивалентности (обоснование ответа обязательно). Для каждого отношения эквивалентности 
постройте классы эквивалентности и постройте граф отношения:
\begin{enumerate}[a)]\setcounter{enumi}{0}
\item А - множество целых чисел и отношение $R = \{(a,b)|a + b = 0\}$
\item $A = \{-10, -9, …, 9, 10\}$ и отношение $R = \{(a,b)|a^{3} = b^{3}\}$
\item На множестве $A = \{1; 2; 3\}$ задано отношение $R = \{(1; 1); (2; 2); (3; 3); (2; 1); (1; 2); (2; 3); (3; 2); (3; 1); (1; 3)\}$

\end{enumerate}\question Составьте полную таблицу истинности, определите, какие переменные являются фиктивными и проверьте, является ли формула тавтологией:
$((P \rightarrow Q) \land (R \rightarrow S) \land \neg (Q \lor S)) \rightarrow \neg (P \lor R)$

\end{questions}
\newpage
%%% begin test
\begin{flushright}
\begin{tabular}{p{2.8in} r l}
%\textbf{\class} & \textbf{ФИО:} & \makebox[2.5in]{\hrulefill}\\
\textbf{\class} & \textbf{ФИО:} &Сазиков Михаил Алексеевич
\\

\textbf{\examdate} &&\\
%\textbf{Time Limit: \timelimit} & Teaching Assistant & \makebox[2in]{\hrulefill}
\end{tabular}\\
\end{flushright}
\rule[1ex]{\textwidth}{.1pt}


\begin{questions}
\question
Найдите и упростите P:
\begin{equation*}
\overline{P} = A \cap \overline{B} \cup A \cap C \cup B \cap C \cup \overline{A} \cap C
\end{equation*}
Затем найдите элементы множества P, выраженного через множества:
\begin{equation*}
A = \{0, 3, 4, 9\}; 
B = \{1, 3, 4, 7\};
C = \{0, 1, 2, 4, 7, 8, 9\};
I = \{0, 1, 2, 3, 4, 5, 6, 7, 8, 9\}.
\end{equation*}\question
Упростите следующее выражение с учетом того, что $A\subset B \subset C \subset D \subset U; A \neq \O$
\begin{equation*}
A \cap  \overline{C} \cup B \cap \overline{D} \cup  \overline{A} \cap C \cap  \overline{D}
\end{equation*}

Примечание: U — универсум\question
Дано отношение на множестве $\{1, 2, 3, 4, 5\}$ 
\begin{equation*}
aRb \iff  \text{НОД}(a,b) =1
\end{equation*}
Напишите обоснованный ответ какими свойствами обладает или не обладает отношение и почему:   
\begin{enumerate} [a)]\setcounter{enumi}{0}
\item рефлексивность
\item антирефлексивность
\item симметричность
\item асимметричность
\item антисимметричность
\item транзитивность
\end{enumerate}

Обоснуйте свой ответ по каждому из приведенных ниже вопросов:
\begin{enumerate} [a)]\setcounter{enumi}{0}
    \item Является ли это отношение отношением эквивалентности?
    \item Является ли это отношение функциональным?
    \item Каким из отношений соответствия (одно-многозначным, много-многозначный и т.д.) оно является?
    \item К каким из отношений порядка (полного, частичного и т.д.) можно отнести данное отношение?
\end{enumerate}


\question
Установите, является ли каждое из перечисленных ниже отношений на А ($R \subseteq A \times A$) отношением эквивалентности (обоснование ответа обязательно). Для каждого отношения эквивалентности постройте классы 
эквивалентности и постройте граф отношения:
\begin{enumerate} [a)]\setcounter{enumi}{0}
\item $A = \{-10, -9, … , 9, 10\}$ и отношение $R = \{(a,b)|a^{2} = b^{2}\}$
\item $A = \{a, b, c, d, p, t\}$ задано отношение $R = \{(a, a), (b, b), (b, c), (b, d), (c, b), (c, c), (c, d), (d, b), (d, c), (d, d), (p,p), (t,t)\}$
\item Пусть A – множество имен. $A = \{ $Алексей, Иван, Петр, Александр, Павел, Андрей$ \}$. Тогда отношение $R$ верно на парах имен, начинающихся с одной и той же буквы, и только на них.
\end{enumerate}\question Составьте полную таблицу истинности, определите, какие переменные являются фиктивными и проверьте, является ли формула тавтологией:
$((P \rightarrow Q) \land (R \rightarrow S) \land \neg (Q \lor S)) \rightarrow \neg (P \lor R)$

\end{questions}
\newpage
%%% begin test
\begin{flushright}
\begin{tabular}{p{2.8in} r l}
%\textbf{\class} & \textbf{ФИО:} & \makebox[2.5in]{\hrulefill}\\
\textbf{\class} & \textbf{ФИО:} &Семенов Михаил Юрьевич
\\

\textbf{\examdate} &&\\
%\textbf{Time Limit: \timelimit} & Teaching Assistant & \makebox[2in]{\hrulefill}
\end{tabular}\\
\end{flushright}
\rule[1ex]{\textwidth}{.1pt}


\begin{questions}
\question
Найдите и упростите P:
\begin{equation*}
\overline{P} = A \cap B \cup \overline{A} \cap \overline{B} \cup A \cap C \cup \overline{B} \cap C
\end{equation*}
Затем найдите элементы множества P, выраженного через множества:
\begin{equation*}
A = \{0, 3, 4, 9\}; 
B = \{1, 3, 4, 7\};
C = \{0, 1, 2, 4, 7, 8, 9\};
I = \{0, 1, 2, 3, 4, 5, 6, 7, 8, 9\}.
\end{equation*}\question
Упростите следующее выражение с учетом того, что $A\subset B \subset C \subset D \subset U; A \neq \O$
\begin{equation*}
A \cap B  \cap \overline{C} \cup \overline{C} \cap D \cup B \cap C \cap D
\end{equation*}

Примечание: U — универсум\question
Дано отношение на множестве $\{1, 2, 3, 4, 5\}$ 
\begin{equation*}
aRb \iff (a+b) \bmod 2 =0
\end{equation*}
Напишите обоснованный ответ какими свойствами обладает или не обладает отношение и почему:   
\begin{enumerate} [a)]\setcounter{enumi}{0}
\item рефлексивность
\item антирефлексивность
\item симметричность
\item асимметричность
\item антисимметричность
\item транзитивность
\end{enumerate}

Обоснуйте свой ответ по каждому из приведенных ниже вопросов:
\begin{enumerate} [a)]\setcounter{enumi}{0}
    \item Является ли это отношение отношением эквивалентности?
    \item Является ли это отношение функциональным?
    \item Каким из отношений соответствия (одно-многозначным, много-многозначный и т.д.) оно является?
    \item К каким из отношений порядка (полного, частичного и т.д.) можно отнести данное отношение?
\end{enumerate}



\question
Установите, является ли каждое из перечисленных ниже отношений на А ($R \subseteq A \times A$) отношением эквивалентности (обоснование ответа обязательно). Для каждого отношения эквивалентности постройте классы 
эквивалентности и постройте граф отношения:
\begin{enumerate} [a)]\setcounter{enumi}{0}
\item $A = \{a, b, c, d, p, t\}$ задано отношение $R = \{(a, a), (b, b), (b, c), (b, d), (c, b), (c, c), (c, d), (d, b), (d, c), (d, d), (p,p), (t,t)\}$
\item $A = \{-10, -9, … , 9, 10\}$ и отношение $R = \{(a,b)|a^{3} = b^{3}\}$

\item $F(x)=x^{2}+1$, где $x \in A = [-2, 4]$ и отношение $R = \{(a,b)|F(a) = F(b)\}$
\end{enumerate}\question Составьте полную таблицу истинности, определите, какие переменные являются фиктивными и проверьте, является ли формула тавтологией:
$((P \rightarrow Q) \lor R) \leftrightarrow (P \rightarrow (Q \lor R))$

\end{questions}
\newpage
%%% begin test
\begin{flushright}
\begin{tabular}{p{2.8in} r l}
%\textbf{\class} & \textbf{ФИО:} & \makebox[2.5in]{\hrulefill}\\
\textbf{\class} & \textbf{ФИО:} &Семёнов Роман Сергеевич
\\

\textbf{\examdate} &&\\
%\textbf{Time Limit: \timelimit} & Teaching Assistant & \makebox[2in]{\hrulefill}
\end{tabular}\\
\end{flushright}
\rule[1ex]{\textwidth}{.1pt}


\begin{questions}
\question
Найдите и упростите P:
\begin{equation*}
\overline{P} = B \cap \overline{C} \cup A \cap B \cup \overline{A} \cap C \cup \overline{A} \cap B
\end{equation*}
Затем найдите элементы множества P, выраженного через множества:
\begin{equation*}
A = \{0, 3, 4, 9\}; 
B = \{1, 3, 4, 7\};
C = \{0, 1, 2, 4, 7, 8, 9\};
I = \{0, 1, 2, 3, 4, 5, 6, 7, 8, 9\}.
\end{equation*}\question
Упростите следующее выражение с учетом того, что $A\subset B \subset C \subset D \subset U; A \neq \O$
\begin{equation*}
A \cap  \overline{C} \cup B \cap \overline{D} \cup  \overline{A} \cap C \cap  \overline{D}
\end{equation*}

Примечание: U — универсум\question
Дано отношение на множестве $\{1, 2, 3, 4, 5\}$ 
\begin{equation*}
aRb \iff b > a
\end{equation*}
Напишите обоснованный ответ какими свойствами обладает или не обладает отношение и почему:   
\begin{enumerate} [a)]\setcounter{enumi}{0}
\item рефлексивность
\item антирефлексивность
\item симметричность
\item асимметричность
\item антисимметричность
\item транзитивность
\end{enumerate}

Обоснуйте свой ответ по каждому из приведенных ниже вопросов:
\begin{enumerate} [a)]\setcounter{enumi}{0}
    \item Является ли это отношение отношением эквивалентности?
    \item Является ли это отношение функциональным?
    \item Каким из отношений соответствия (одно-многозначным, много-многозначный и т.д.) оно является?
    \item К каким из отношений порядка (полного, частичного и т.д.) можно отнести данное отношение?
\end{enumerate}

\question
Установите, является ли каждое из перечисленных ниже отношений на А ($R \subseteq A \times A$) отношением эквивалентности (обоснование ответа обязательно). Для каждого отношения эквивалентности постройте классы 
эквивалентности и постройте граф отношения:
\begin{enumerate} [a)]\setcounter{enumi}{0}
\item $A = \{a, b, c, d, p, t\}$ задано отношение $R = \{(a, a), (b, b), (b, c), (b, d), (c, b), (c, c), (c, d), (d, b), (d, c), (d, d), (p,p), (t,t)\}$
\item $A = \{-10, -9, … , 9, 10\}$ и отношение $R = \{(a,b)|a^{3} = b^{3}\}$

\item $F(x)=x^{2}+1$, где $x \in A = [-2, 4]$ и отношение $R = \{(a,b)|F(a) = F(b)\}$
\end{enumerate}\question Составьте полную таблицу истинности, определите, какие переменные являются фиктивными и проверьте, является ли формула тавтологией:
$ P \rightarrow (Q \rightarrow ((P \lor Q) \rightarrow (P \land Q)))$

\end{questions}
\newpage
%%% begin test
\begin{flushright}
\begin{tabular}{p{2.8in} r l}
%\textbf{\class} & \textbf{ФИО:} & \makebox[2.5in]{\hrulefill}\\
\textbf{\class} & \textbf{ФИО:} &Соколов Денис Андреевич
\\

\textbf{\examdate} &&\\
%\textbf{Time Limit: \timelimit} & Teaching Assistant & \makebox[2in]{\hrulefill}
\end{tabular}\\
\end{flushright}
\rule[1ex]{\textwidth}{.1pt}


\begin{questions}
\question
Найдите и упростите P:
\begin{equation*}
\overline{P} = A \cap \overline{B} \cup A \cap C \cup B \cap C \cup \overline{A} \cap C
\end{equation*}
Затем найдите элементы множества P, выраженного через множества:
\begin{equation*}
A = \{0, 3, 4, 9\}; 
B = \{1, 3, 4, 7\};
C = \{0, 1, 2, 4, 7, 8, 9\};
I = \{0, 1, 2, 3, 4, 5, 6, 7, 8, 9\}.
\end{equation*}\question
Упростите следующее выражение с учетом того, что $A\subset B \subset C \subset D \subset U; A \neq \O$
\begin{equation*}
\overline{B} \cap \overline{C} \cap D \cup \overline{A} \cap \overline{C} \cap D \cup \overline{A} \cap B
\end{equation*}

Примечание: U — универсум\question
Дано отношение на множестве $\{1, 2, 3, 4, 5\}$ 
\begin{equation*}
aRb \iff b > a
\end{equation*}
Напишите обоснованный ответ какими свойствами обладает или не обладает отношение и почему:   
\begin{enumerate} [a)]\setcounter{enumi}{0}
\item рефлексивность
\item антирефлексивность
\item симметричность
\item асимметричность
\item антисимметричность
\item транзитивность
\end{enumerate}

Обоснуйте свой ответ по каждому из приведенных ниже вопросов:
\begin{enumerate} [a)]\setcounter{enumi}{0}
    \item Является ли это отношение отношением эквивалентности?
    \item Является ли это отношение функциональным?
    \item Каким из отношений соответствия (одно-многозначным, много-многозначный и т.д.) оно является?
    \item К каким из отношений порядка (полного, частичного и т.д.) можно отнести данное отношение?
\end{enumerate}

\question
Установите, является ли каждое из перечисленных ниже отношений на А ($R \subseteq A \times A$) отношением эквивалентности (обоснование ответа обязательно). Для каждого отношения эквивалентности постройте классы 
эквивалентности и постройте граф отношения:
\begin{enumerate} [a)]\setcounter{enumi}{0}
\item Пусть A – множество имен. $A = \{ $Алексей, Иван, Петр, Александр, Павел, Андрей$ \}$. Тогда отношение $R$ верно на парах имен, начинающихся с одной и той же буквы, и только на них.
\item $A = \{-10, -9, … , 9, 10\}$ и отношение $ R = \{(a,b)|a^{2} = b^{2}\}$
\item На множестве $A = \{1; 2; 3\}$ задано отношение $R = \{(1; 1); (2; 2); (3; 3); (3; 2); (1; 2); (2; 1)\}$
\end{enumerate}\question Составьте полную таблицу истинности, определите, какие переменные являются фиктивными и проверьте, является ли формула тавтологией:
$(P \rightarrow (Q \rightarrow R)) \rightarrow ((P \rightarrow Q) \rightarrow (P \rightarrow R))$

\end{questions}
\newpage
%%% begin test
\begin{flushright}
\begin{tabular}{p{2.8in} r l}
%\textbf{\class} & \textbf{ФИО:} & \makebox[2.5in]{\hrulefill}\\
\textbf{\class} & \textbf{ФИО:} &Тихомиров Дмитрий Алексеевич
\\

\textbf{\examdate} &&\\
%\textbf{Time Limit: \timelimit} & Teaching Assistant & \makebox[2in]{\hrulefill}
\end{tabular}\\
\end{flushright}
\rule[1ex]{\textwidth}{.1pt}


\begin{questions}
\question
Найдите и упростите P:
\begin{equation*}
\overline{P} = A \cap \overline{C} \cup A \cap \overline{B} \cup B \cap \overline{C} \cup A \cap C
\end{equation*}
Затем найдите элементы множества P, выраженного через множества:
\begin{equation*}
A = \{0, 3, 4, 9\}; 
B = \{1, 3, 4, 7\};
C = \{0, 1, 2, 4, 7, 8, 9\};
I = \{0, 1, 2, 3, 4, 5, 6, 7, 8, 9\}.
\end{equation*}\question
Упростите следующее выражение с учетом того, что $A\subset B \subset C \subset D \subset U; A \neq \O$
\begin{equation*}
A \cap  \overline{C} \cup B \cap \overline{D} \cup  \overline{A} \cap C \cap  \overline{D}
\end{equation*}

Примечание: U — универсум\question
Дано отношение на множестве $\{1, 2, 3, 4, 5\}$ 
\begin{equation*}
aRb \iff b > a
\end{equation*}
Напишите обоснованный ответ какими свойствами обладает или не обладает отношение и почему:   
\begin{enumerate} [a)]\setcounter{enumi}{0}
\item рефлексивность
\item антирефлексивность
\item симметричность
\item асимметричность
\item антисимметричность
\item транзитивность
\end{enumerate}

Обоснуйте свой ответ по каждому из приведенных ниже вопросов:
\begin{enumerate} [a)]\setcounter{enumi}{0}
    \item Является ли это отношение отношением эквивалентности?
    \item Является ли это отношение функциональным?
    \item Каким из отношений соответствия (одно-многозначным, много-многозначный и т.д.) оно является?
    \item К каким из отношений порядка (полного, частичного и т.д.) можно отнести данное отношение?
\end{enumerate}

\question
Установите, является ли каждое из перечисленных ниже отношений на А ($R \subseteq A \times A$) отношением эквивалентности (обоснование ответа обязательно). Для каждого отношения эквивалентности постройте классы 
эквивалентности и постройте граф отношения:
\begin{enumerate} [a)]\setcounter{enumi}{0}
\item $A = \{a, b, c, d, p, t\}$ задано отношение $R = \{(a, a), (b, b), (b, c), (b, d), (c, b), (c, c), (c, d), (d, b), (d, c), (d, d), (p,p), (t,t)\}$
\item $A = \{-10, -9, … , 9, 10\}$ и отношение $R = \{(a,b)|a^{3} = b^{3}\}$

\item $F(x)=x^{2}+1$, где $x \in A = [-2, 4]$ и отношение $R = \{(a,b)|F(a) = F(b)\}$
\end{enumerate}\question Составьте полную таблицу истинности, определите, какие переменные являются фиктивными и проверьте, является ли формула тавтологией:
$((P \rightarrow Q) \land (R \rightarrow S) \land \neg (Q \lor S)) \rightarrow \neg (P \lor R)$

\end{questions}
\newpage
%%% begin test
\begin{flushright}
\begin{tabular}{p{2.8in} r l}
%\textbf{\class} & \textbf{ФИО:} & \makebox[2.5in]{\hrulefill}\\
\textbf{\class} & \textbf{ФИО:} &Фадеев Артём Владимирович
\\

\textbf{\examdate} &&\\
%\textbf{Time Limit: \timelimit} & Teaching Assistant & \makebox[2in]{\hrulefill}
\end{tabular}\\
\end{flushright}
\rule[1ex]{\textwidth}{.1pt}


\begin{questions}
\question
Найдите и упростите P:
\begin{equation*}
\overline{P} = \overline{A} \cap B \cup \overline{A} \cap C \cup A \cap \overline{B} \cup \overline{B} \cap C
\end{equation*}
Затем найдите элементы множества P, выраженного через множества:
\begin{equation*}
A = \{0, 3, 4, 9\}; 
B = \{1, 3, 4, 7\};
C = \{0, 1, 2, 4, 7, 8, 9\};
I = \{0, 1, 2, 3, 4, 5, 6, 7, 8, 9\}.
\end{equation*}\question
Упростите следующее выражение с учетом того, что $A\subset B \subset C \subset D \subset U; A \neq \O$
\begin{equation*}
\overline{A} \cap \overline{C} \cap D \cup \overline{B} \cap \overline{C} \cap D \cup A \cap B
\end{equation*}

Примечание: U — универсум\question
Дано отношение на множестве $\{1, 2, 3, 4, 5\}$ 
\begin{equation*}
aRb \iff (a+b) \bmod 2 =0
\end{equation*}
Напишите обоснованный ответ какими свойствами обладает или не обладает отношение и почему:   
\begin{enumerate} [a)]\setcounter{enumi}{0}
\item рефлексивность
\item антирефлексивность
\item симметричность
\item асимметричность
\item антисимметричность
\item транзитивность
\end{enumerate}

Обоснуйте свой ответ по каждому из приведенных ниже вопросов:
\begin{enumerate} [a)]\setcounter{enumi}{0}
    \item Является ли это отношение отношением эквивалентности?
    \item Является ли это отношение функциональным?
    \item Каким из отношений соответствия (одно-многозначным, много-многозначный и т.д.) оно является?
    \item К каким из отношений порядка (полного, частичного и т.д.) можно отнести данное отношение?
\end{enumerate}



\question
Установите, является ли каждое из перечисленных ниже отношений на А ($R \subseteq A \times A$) отношением эквивалентности (обоснование ответа обязательно). Для каждого отношения эквивалентности постройте классы 
эквивалентности и постройте граф отношения:
\begin{enumerate} [a)]\setcounter{enumi}{0}
\item На множестве $A = \{1; 2; 3\}$ задано отношение $R = \{(1; 1); (2; 2); (3; 3); (2; 1); (1; 2); (2; 3); (3; 2); (3; 1); (1; 3)\}$
\item На множестве $A = \{1; 2; 3; 4; 5\}$ задано отношение $R = \{(1; 2); (1; 3); (1; 5); (2; 3); (2; 4); (2; 5); (3; 4); (3; 5); (4; 5)\}$
\item А - множество целых чисел и отношение $R = \{(a,b)|a + b = 0\}$
\end{enumerate}\question Составьте полную таблицу истинности, определите, какие переменные являются фиктивными и проверьте, является ли формула тавтологией:

$(P \rightarrow (Q \land R)) \leftrightarrow ((P \rightarrow Q) \land (P \rightarrow R))$

\end{questions}
\newpage
%%% begin test
\begin{flushright}
\begin{tabular}{p{2.8in} r l}
%\textbf{\class} & \textbf{ФИО:} & \makebox[2.5in]{\hrulefill}\\
\textbf{\class} & \textbf{ФИО:} &Чечулин Лев Олегович
\\

\textbf{\examdate} &&\\
%\textbf{Time Limit: \timelimit} & Teaching Assistant & \makebox[2in]{\hrulefill}
\end{tabular}\\
\end{flushright}
\rule[1ex]{\textwidth}{.1pt}


\begin{questions}
\question
Найдите и упростите P:
\begin{equation*}
\overline{P} = A \cap B \cup \overline{A} \cap \overline{B} \cup A \cap C \cup \overline{B} \cap C
\end{equation*}
Затем найдите элементы множества P, выраженного через множества:
\begin{equation*}
A = \{0, 3, 4, 9\}; 
B = \{1, 3, 4, 7\};
C = \{0, 1, 2, 4, 7, 8, 9\};
I = \{0, 1, 2, 3, 4, 5, 6, 7, 8, 9\}.
\end{equation*}\question
Упростите следующее выражение с учетом того, что $A\subset B \subset C \subset D \subset U; A \neq \O$
\begin{equation*}
\overline{A} \cap \overline{C} \cap D \cup \overline{B} \cap \overline{C} \cap D \cup A \cap B
\end{equation*}

Примечание: U — универсум\question
Для следующего отношения на множестве $\{1, 2, 3, 4, 5\}$ 
\begin{equation*}
aRb \iff 0 < a-b<2
\end{equation*}
Напишите обоснованный ответ какими свойствами обладает или не обладает отношение и почему:   
\begin{enumerate} [a)]\setcounter{enumi}{0}
\item рефлексивность
\item антирефлексивность
\item симметричность
\item асимметричность
\item антисимметричность
\item транзитивность
\end{enumerate}

Обоснуйте свой ответ по каждому из приведенных ниже вопросов:
\begin{enumerate} [a)]\setcounter{enumi}{0}
    \item Является ли это отношение отношением эквивалентности?
    \item Является ли это отношение функциональным?
    \item Каким из отношений соответствия (одно-многозначным, много-многозначный и т.д.) оно является?
    \item К каким из отношений порядка (полного, частичного и т.д.) можно отнести данное отношение?
\end{enumerate}
\question
Установите, является ли каждое из перечисленных ниже отношений на А ($R \subseteq A \times A$) отношением эквивалентности (обоснование ответа обязательно). Для каждого отношения эквивалентности постройте классы 
эквивалентности и постройте граф отношения:
\begin{enumerate} [a)]\setcounter{enumi}{0}
\item На множестве $A = \{1; 2; 3\}$ задано отношение $R = \{(1; 1); (2; 2); (3; 3); (2; 1); (1; 2); (2; 3); (3; 2); (3; 1); (1; 3)\}$
\item На множестве $A = \{1; 2; 3; 4; 5\}$ задано отношение $R = \{(1; 2); (1; 3); (1; 5); (2; 3); (2; 4); (2; 5); (3; 4); (3; 5); (4; 5)\}$
\item А - множество целых чисел и отношение $R = \{(a,b)|a + b = 0\}$
\end{enumerate}\question Составьте полную таблицу истинности, определите, какие переменные являются фиктивными и проверьте, является ли формула тавтологией:

$(P \rightarrow (Q \land R)) \leftrightarrow ((P \rightarrow Q) \land (P \rightarrow R))$

\end{questions}
\newpage
%%% begin test
\begin{flushright}
\begin{tabular}{p{2.8in} r l}
%\textbf{\class} & \textbf{ФИО:} & \makebox[2.5in]{\hrulefill}\\
\textbf{\class} & \textbf{ФИО:} &Шпис Петр Сергеевич
\\

\textbf{\examdate} &&\\
%\textbf{Time Limit: \timelimit} & Teaching Assistant & \makebox[2in]{\hrulefill}
\end{tabular}\\
\end{flushright}
\rule[1ex]{\textwidth}{.1pt}


\begin{questions}
\question
Найдите и упростите P:
\begin{equation*}
\overline{P} = A \cap \overline{C} \cup A \cap \overline{B} \cup B \cap \overline{C} \cup A \cap C
\end{equation*}
Затем найдите элементы множества P, выраженного через множества:
\begin{equation*}
A = \{0, 3, 4, 9\}; 
B = \{1, 3, 4, 7\};
C = \{0, 1, 2, 4, 7, 8, 9\};
I = \{0, 1, 2, 3, 4, 5, 6, 7, 8, 9\}.
\end{equation*}\question
Упростите следующее выражение с учетом того, что $A\subset B \subset C \subset D \subset U; A \neq \O$
\begin{equation*}
A \cap C  \cap D \cup B \cap \overline{C} \cap D \cup B \cap C \cap D
\end{equation*}

Примечание: U — универсум\question
Дано отношение на множестве $\{1, 2, 3, 4, 5\}$ 
\begin{equation*}
aRb \iff a \geq b^2
\end{equation*}
Напишите обоснованный ответ какими свойствами обладает или не обладает отношение и почему:   
\begin{enumerate} [a)]\setcounter{enumi}{0}
\item рефлексивность
\item антирефлексивность
\item симметричность
\item асимметричность
\item антисимметричность
\item транзитивность
\end{enumerate}

Обоснуйте свой ответ по каждому из приведенных ниже вопросов:
\begin{enumerate} [a)]\setcounter{enumi}{0}
    \item Является ли это отношение отношением эквивалентности?
    \item Является ли это отношение функциональным?
    \item Каким из отношений соответствия (одно-многозначным, много-многозначный и т.д.) оно является?
    \item К каким из отношений порядка (полного, частичного и т.д.) можно отнести данное отношение?
\end{enumerate}


\question
Установите, является ли каждое из перечисленных ниже отношений на А ($R \subseteq A \times A$) отношением эквивалентности (обоснование ответа обязательно). Для каждого отношения эквивалентности постройте классы эквивалентности и постройте граф отношения:
\begin{enumerate} [a)]\setcounter{enumi}{0}
\item $F(x)=x^{2}+1$, где $x \in A = [-2, 4]$ и отношение $R = \{(a,b)|F(a) = F(b)\}$
\item А - множество целых чисел и отношение $R = \{(a,b)|a + b = 5\}$
\item На множестве $A = \{1; 2; 3\}$ задано отношение $R = \{(1; 1); (2; 2); (3; 3); (3; 2); (1; 2); (2; 1)\}$

\end{enumerate}\question Составьте полную таблицу истинности, определите, какие переменные являются фиктивными и проверьте, является ли формула тавтологией:
$(( P \rightarrow Q) \land (Q \rightarrow P)) \rightarrow (P \rightarrow R)$

\end{questions}
\newpage
%%% begin test
\begin{flushright}
\begin{tabular}{p{2.8in} r l}
%\textbf{\class} & \textbf{ФИО:} & \makebox[2.5in]{\hrulefill}\\
\textbf{\class} & \textbf{ФИО:} &М3103
\\

\textbf{\examdate} &&\\
%\textbf{Time Limit: \timelimit} & Teaching Assistant & \makebox[2in]{\hrulefill}
\end{tabular}\\
\end{flushright}
\rule[1ex]{\textwidth}{.1pt}


\begin{questions}
\question
Найдите и упростите P:
\begin{equation*}
\overline{P} = A \cap \overline{C} \cup A \cap \overline{B} \cup B \cap \overline{C} \cup A \cap C
\end{equation*}
Затем найдите элементы множества P, выраженного через множества:
\begin{equation*}
A = \{0, 3, 4, 9\}; 
B = \{1, 3, 4, 7\};
C = \{0, 1, 2, 4, 7, 8, 9\};
I = \{0, 1, 2, 3, 4, 5, 6, 7, 8, 9\}.
\end{equation*}\question
Упростите следующее выражение с учетом того, что $A\subset B \subset C \subset D \subset U; A \neq \O$
\begin{equation*}
A \cap  \overline{C} \cup B \cap \overline{D} \cup  \overline{A} \cap C \cap  \overline{D}
\end{equation*}

Примечание: U — универсум\question
Дано отношение на множестве $\{1, 2, 3, 4, 5\}$ 
\begin{equation*}
aRb \iff  \text{НОД}(a,b) =1
\end{equation*}
Напишите обоснованный ответ какими свойствами обладает или не обладает отношение и почему:   
\begin{enumerate} [a)]\setcounter{enumi}{0}
\item рефлексивность
\item антирефлексивность
\item симметричность
\item асимметричность
\item антисимметричность
\item транзитивность
\end{enumerate}

Обоснуйте свой ответ по каждому из приведенных ниже вопросов:
\begin{enumerate} [a)]\setcounter{enumi}{0}
    \item Является ли это отношение отношением эквивалентности?
    \item Является ли это отношение функциональным?
    \item Каким из отношений соответствия (одно-многозначным, много-многозначный и т.д.) оно является?
    \item К каким из отношений порядка (полного, частичного и т.д.) можно отнести данное отношение?
\end{enumerate}


\question
Установите, является ли каждое из перечисленных ниже отношений на А ($R \subseteq A \times A$) отношением эквивалентности (обоснование ответа обязательно). Для каждого отношения эквивалентности постройте классы эквивалентности и постройте граф отношения:
\begin{enumerate} [a)]\setcounter{enumi}{0}
\item $F(x)=x^{2}+1$, где $x \in A = [-2, 4]$ и отношение $R = \{(a,b)|F(a) = F(b)\}$
\item А - множество целых чисел и отношение $R = \{(a,b)|a + b = 5\}$
\item На множестве $A = \{1; 2; 3\}$ задано отношение $R = \{(1; 1); (2; 2); (3; 3); (3; 2); (1; 2); (2; 1)\}$

\end{enumerate}\question Составьте полную таблицу истинности, определите, какие переменные являются фиктивными и проверьте, является ли формула тавтологией:

$(P \rightarrow (Q \land R)) \leftrightarrow ((P \rightarrow Q) \land (P \rightarrow R))$

\end{questions}
\newpage
%%% begin test
\begin{flushright}
\begin{tabular}{p{2.8in} r l}
%\textbf{\class} & \textbf{ФИО:} & \makebox[2.5in]{\hrulefill}\\
\textbf{\class} & \textbf{ФИО:} &Валуйский Михаил Игоревич
\\

\textbf{\examdate} &&\\
%\textbf{Time Limit: \timelimit} & Teaching Assistant & \makebox[2in]{\hrulefill}
\end{tabular}\\
\end{flushright}
\rule[1ex]{\textwidth}{.1pt}


\begin{questions}
\question
Найдите и упростите P:
\begin{equation*}
\overline{P} = A \cap \overline{B} \cup A \cap C \cup B \cap C \cup \overline{A} \cap C
\end{equation*}
Затем найдите элементы множества P, выраженного через множества:
\begin{equation*}
A = \{0, 3, 4, 9\}; 
B = \{1, 3, 4, 7\};
C = \{0, 1, 2, 4, 7, 8, 9\};
I = \{0, 1, 2, 3, 4, 5, 6, 7, 8, 9\}.
\end{equation*}\question
Упростите следующее выражение с учетом того, что $A\subset B \subset C \subset D \subset U; A \neq \O$
\begin{equation*}
\overline{A} \cap \overline{B} \cup B \cap \overline{C} \cup \overline{C} \cap D
\end{equation*}

Примечание: U — универсум\question
Дано отношение на множестве $\{1, 2, 3, 4, 5\}$ 
\begin{equation*}
aRb \iff a \geq b^2
\end{equation*}
Напишите обоснованный ответ какими свойствами обладает или не обладает отношение и почему:   
\begin{enumerate} [a)]\setcounter{enumi}{0}
\item рефлексивность
\item антирефлексивность
\item симметричность
\item асимметричность
\item антисимметричность
\item транзитивность
\end{enumerate}

Обоснуйте свой ответ по каждому из приведенных ниже вопросов:
\begin{enumerate} [a)]\setcounter{enumi}{0}
    \item Является ли это отношение отношением эквивалентности?
    \item Является ли это отношение функциональным?
    \item Каким из отношений соответствия (одно-многозначным, много-многозначный и т.д.) оно является?
    \item К каким из отношений порядка (полного, частичного и т.д.) можно отнести данное отношение?
\end{enumerate}


\question
Установите, является ли каждое из перечисленных ниже отношений на А ($R \subseteq A \times A$) отношением эквивалентности (обоснование ответа обязательно). Для каждого отношения эквивалентности постройте классы эквивалентности и постройте граф отношения:
\begin{enumerate} [a)]\setcounter{enumi}{0}
\item $F(x)=x^{2}+1$, где $x \in A = [-2, 4]$ и отношение $R = \{(a,b)|F(a) = F(b)\}$
\item А - множество целых чисел и отношение $R = \{(a,b)|a + b = 5\}$
\item На множестве $A = \{1; 2; 3\}$ задано отношение $R = \{(1; 1); (2; 2); (3; 3); (3; 2); (1; 2); (2; 1)\}$

\end{enumerate}\question Составьте полную таблицу истинности, определите, какие переменные являются фиктивными и проверьте, является ли формула тавтологией:

$(P \rightarrow (Q \land R)) \leftrightarrow ((P \rightarrow Q) \land (P \rightarrow R))$

\end{questions}
\newpage
%%% begin test
\begin{flushright}
\begin{tabular}{p{2.8in} r l}
%\textbf{\class} & \textbf{ФИО:} & \makebox[2.5in]{\hrulefill}\\
\textbf{\class} & \textbf{ФИО:} &Вяткина Софья Андреевна
\\

\textbf{\examdate} &&\\
%\textbf{Time Limit: \timelimit} & Teaching Assistant & \makebox[2in]{\hrulefill}
\end{tabular}\\
\end{flushright}
\rule[1ex]{\textwidth}{.1pt}


\begin{questions}
\question
Найдите и упростите P:
\begin{equation*}
\overline{P} = B \cap \overline{C} \cup A \cap B \cup \overline{A} \cap C \cup \overline{A} \cap B
\end{equation*}
Затем найдите элементы множества P, выраженного через множества:
\begin{equation*}
A = \{0, 3, 4, 9\}; 
B = \{1, 3, 4, 7\};
C = \{0, 1, 2, 4, 7, 8, 9\};
I = \{0, 1, 2, 3, 4, 5, 6, 7, 8, 9\}.
\end{equation*}\question
Упростите следующее выражение с учетом того, что $A\subset B \subset C \subset D \subset U; A \neq \O$
\begin{equation*}
\overline{A} \cap \overline{C} \cap D \cup \overline{B} \cap \overline{C} \cap D \cup A \cap B
\end{equation*}

Примечание: U — универсум\question
Дано отношение на множестве $\{1, 2, 3, 4, 5\}$ 
\begin{equation*}
aRb \iff |a-b| = 1
\end{equation*}
Напишите обоснованный ответ какими свойствами обладает или не обладает отношение и почему:   
\begin{enumerate} [a)]\setcounter{enumi}{0}
\item рефлексивность
\item антирефлексивность
\item симметричность
\item асимметричность
\item антисимметричность
\item транзитивность
\end{enumerate}

Обоснуйте свой ответ по каждому из приведенных ниже вопросов:
\begin{enumerate} [a)]\setcounter{enumi}{0}
    \item Является ли это отношение отношением эквивалентности?
    \item Является ли это отношение функциональным?
    \item Каким из отношений соответствия (одно-многозначным, много-многозначный и т.д.) оно является?
    \item К каким из отношений порядка (полного, частичного и т.д.) можно отнести данное отношение?
\end{enumerate}

\question
Установите, является ли каждое из перечисленных ниже отношений на А ($R \subseteq A \times A$) отношением эквивалентности (обоснование ответа обязательно). Для каждого отношения эквивалентности постройте классы 
эквивалентности и постройте граф отношения:
\begin{enumerate} [a)]\setcounter{enumi}{0}
\item Пусть A – множество имен. $A = \{ $Алексей, Иван, Петр, Александр, Павел, Андрей$ \}$. Тогда отношение $R$ верно на парах имен, начинающихся с одной и той же буквы, и только на них.
\item $A = \{-10, -9, … , 9, 10\}$ и отношение $ R = \{(a,b)|a^{2} = b^{2}\}$
\item На множестве $A = \{1; 2; 3\}$ задано отношение $R = \{(1; 1); (2; 2); (3; 3); (3; 2); (1; 2); (2; 1)\}$
\end{enumerate}\question Составьте полную таблицу истинности, определите, какие переменные являются фиктивными и проверьте, является ли формула тавтологией:

$(P \rightarrow (Q \land R)) \leftrightarrow ((P \rightarrow Q) \land (P \rightarrow R))$

\end{questions}
\newpage
%%% begin test
\begin{flushright}
\begin{tabular}{p{2.8in} r l}
%\textbf{\class} & \textbf{ФИО:} & \makebox[2.5in]{\hrulefill}\\
\textbf{\class} & \textbf{ФИО:} &Голякова Татьяна Олеговна
\\

\textbf{\examdate} &&\\
%\textbf{Time Limit: \timelimit} & Teaching Assistant & \makebox[2in]{\hrulefill}
\end{tabular}\\
\end{flushright}
\rule[1ex]{\textwidth}{.1pt}


\begin{questions}
\question
Найдите и упростите P:
\begin{equation*}
\overline{P} = A \cap \overline{B} \cup \overline{B} \cap C \cup \overline{A} \cap \overline{B} \cup \overline{A} \cap C
\end{equation*}
Затем найдите элементы множества P, выраженного через множества:
\begin{equation*}
A = \{0, 3, 4, 9\}; 
B = \{1, 3, 4, 7\};
C = \{0, 1, 2, 4, 7, 8, 9\};
I = \{0, 1, 2, 3, 4, 5, 6, 7, 8, 9\}.
\end{equation*}\question
Упростите следующее выражение с учетом того, что $A\subset B \subset C \subset D \subset U; A \neq \O$
\begin{equation*}
A \cap C  \cap D \cup B \cap \overline{C} \cap D \cup B \cap C \cap D
\end{equation*}

Примечание: U — универсум\question
Дано отношение на множестве $\{1, 2, 3, 4, 5\}$ 
\begin{equation*}
aRb \iff a \geq b^2
\end{equation*}
Напишите обоснованный ответ какими свойствами обладает или не обладает отношение и почему:   
\begin{enumerate} [a)]\setcounter{enumi}{0}
\item рефлексивность
\item антирефлексивность
\item симметричность
\item асимметричность
\item антисимметричность
\item транзитивность
\end{enumerate}

Обоснуйте свой ответ по каждому из приведенных ниже вопросов:
\begin{enumerate} [a)]\setcounter{enumi}{0}
    \item Является ли это отношение отношением эквивалентности?
    \item Является ли это отношение функциональным?
    \item Каким из отношений соответствия (одно-многозначным, много-многозначный и т.д.) оно является?
    \item К каким из отношений порядка (полного, частичного и т.д.) можно отнести данное отношение?
\end{enumerate}


\question
Установите, является ли каждое из перечисленных ниже отношений на А ($R \subseteq A \times A$) отношением эквивалентности (обоснование ответа обязательно). Для каждого отношения эквивалентности постройте классы 
эквивалентности и постройте граф отношения:
\begin{enumerate} [a)]\setcounter{enumi}{0}
\item $A = \{-10, -9, … , 9, 10\}$ и отношение $R = \{(a,b)|a^{2} = b^{2}\}$
\item $A = \{a, b, c, d, p, t\}$ задано отношение $R = \{(a, a), (b, b), (b, c), (b, d), (c, b), (c, c), (c, d), (d, b), (d, c), (d, d), (p,p), (t,t)\}$
\item Пусть A – множество имен. $A = \{ $Алексей, Иван, Петр, Александр, Павел, Андрей$ \}$. Тогда отношение $R$ верно на парах имен, начинающихся с одной и той же буквы, и только на них.
\end{enumerate}\question Составьте полную таблицу истинности, определите, какие переменные являются фиктивными и проверьте, является ли формула тавтологией:
$((P \rightarrow Q) \land (R \rightarrow S) \land \neg (Q \lor S)) \rightarrow \neg (P \lor R)$

\end{questions}
\newpage
%%% begin test
\begin{flushright}
\begin{tabular}{p{2.8in} r l}
%\textbf{\class} & \textbf{ФИО:} & \makebox[2.5in]{\hrulefill}\\
\textbf{\class} & \textbf{ФИО:} &Гумин Даниил Дмитриевич
\\

\textbf{\examdate} &&\\
%\textbf{Time Limit: \timelimit} & Teaching Assistant & \makebox[2in]{\hrulefill}
\end{tabular}\\
\end{flushright}
\rule[1ex]{\textwidth}{.1pt}


\begin{questions}
\question
Найдите и упростите P:
\begin{equation*}
\overline{P} = \overline{A} \cap B \cup \overline{A} \cap C \cup A \cap \overline{B} \cup \overline{B} \cap C
\end{equation*}
Затем найдите элементы множества P, выраженного через множества:
\begin{equation*}
A = \{0, 3, 4, 9\}; 
B = \{1, 3, 4, 7\};
C = \{0, 1, 2, 4, 7, 8, 9\};
I = \{0, 1, 2, 3, 4, 5, 6, 7, 8, 9\}.
\end{equation*}\question
Упростите следующее выражение с учетом того, что $A\subset B \subset C \subset D \subset U; A \neq \O$
\begin{equation*}
A \cap B  \cap \overline{C} \cup \overline{C} \cap D \cup B \cap C \cap D
\end{equation*}

Примечание: U — универсум\question
Дано отношение на множестве $\{1, 2, 3, 4, 5\}$ 
\begin{equation*}
aRb \iff  \text{НОД}(a,b) =1
\end{equation*}
Напишите обоснованный ответ какими свойствами обладает или не обладает отношение и почему:   
\begin{enumerate} [a)]\setcounter{enumi}{0}
\item рефлексивность
\item антирефлексивность
\item симметричность
\item асимметричность
\item антисимметричность
\item транзитивность
\end{enumerate}

Обоснуйте свой ответ по каждому из приведенных ниже вопросов:
\begin{enumerate} [a)]\setcounter{enumi}{0}
    \item Является ли это отношение отношением эквивалентности?
    \item Является ли это отношение функциональным?
    \item Каким из отношений соответствия (одно-многозначным, много-многозначный и т.д.) оно является?
    \item К каким из отношений порядка (полного, частичного и т.д.) можно отнести данное отношение?
\end{enumerate}


\question
Установите, является ли каждое из перечисленных ниже отношений на А ($R \subseteq A \times A$) отношением эквивалентности (обоснование ответа обязательно). Для каждого отношения эквивалентности постройте классы 
эквивалентности и постройте граф отношения:
\begin{enumerate} [a)]\setcounter{enumi}{0}
\item Пусть A – множество имен. $A = \{ $Алексей, Иван, Петр, Александр, Павел, Андрей$ \}$. Тогда отношение $R$ верно на парах имен, начинающихся с одной и той же буквы, и только на них.
\item $A = \{-10, -9, … , 9, 10\}$ и отношение $ R = \{(a,b)|a^{2} = b^{2}\}$
\item На множестве $A = \{1; 2; 3\}$ задано отношение $R = \{(1; 1); (2; 2); (3; 3); (3; 2); (1; 2); (2; 1)\}$
\end{enumerate}\question Составьте полную таблицу истинности, определите, какие переменные являются фиктивными и проверьте, является ли формула тавтологией:
$(P \rightarrow (Q \rightarrow R)) \rightarrow ((P \rightarrow Q) \rightarrow (P \rightarrow R))$

\end{questions}
\newpage
%%% begin test
\begin{flushright}
\begin{tabular}{p{2.8in} r l}
%\textbf{\class} & \textbf{ФИО:} & \makebox[2.5in]{\hrulefill}\\
\textbf{\class} & \textbf{ФИО:} &Демидович Эдуард Максимович
\\

\textbf{\examdate} &&\\
%\textbf{Time Limit: \timelimit} & Teaching Assistant & \makebox[2in]{\hrulefill}
\end{tabular}\\
\end{flushright}
\rule[1ex]{\textwidth}{.1pt}


\begin{questions}
\question
Найдите и упростите P:
\begin{equation*}
\overline{P} = B \cap \overline{C} \cup A \cap B \cup \overline{A} \cap C \cup \overline{A} \cap B
\end{equation*}
Затем найдите элементы множества P, выраженного через множества:
\begin{equation*}
A = \{0, 3, 4, 9\}; 
B = \{1, 3, 4, 7\};
C = \{0, 1, 2, 4, 7, 8, 9\};
I = \{0, 1, 2, 3, 4, 5, 6, 7, 8, 9\}.
\end{equation*}\question
Упростите следующее выражение с учетом того, что $A\subset B \subset C \subset D \subset U; A \neq \O$
\begin{equation*}
A \cap B  \cap \overline{C} \cup \overline{C} \cap D \cup B \cap C \cap D
\end{equation*}

Примечание: U — универсум\question
Дано отношение на множестве $\{1, 2, 3, 4, 5\}$ 
\begin{equation*}
aRb \iff a \leq b
\end{equation*}
Напишите обоснованный ответ какими свойствами обладает или не обладает отношение и почему:   
\begin{enumerate} [a)]\setcounter{enumi}{0}
\item рефлексивность
\item антирефлексивность
\item симметричность
\item асимметричность
\item антисимметричность
\item транзитивность
\end{enumerate}

Обоснуйте свой ответ по каждому из приведенных ниже вопросов:
\begin{enumerate} [a)]\setcounter{enumi}{0}
    \item Является ли это отношение отношением эквивалентности?
    \item Является ли это отношение функциональным?
    \item Каким из отношений соответствия (одно-многозначным, много-многозначный и т.д.) оно является?
    \item К каким из отношений порядка (полного, частичного и т.д.) можно отнести данное отношение?
\end{enumerate}


\question
Установите, является ли каждое из перечисленных ниже отношений на А ($R \subseteq A \times A$) отношением эквивалентности (обоснование ответа обязательно). Для каждого отношения эквивалентности постройте классы эквивалентности и постройте граф отношения:
\begin{enumerate} [a)]\setcounter{enumi}{0}
\item $F(x)=x^{2}+1$, где $x \in A = [-2, 4]$ и отношение $R = \{(a,b)|F(a) = F(b)\}$
\item А - множество целых чисел и отношение $R = \{(a,b)|a + b = 5\}$
\item На множестве $A = \{1; 2; 3\}$ задано отношение $R = \{(1; 1); (2; 2); (3; 3); (3; 2); (1; 2); (2; 1)\}$

\end{enumerate}\question Составьте полную таблицу истинности, определите, какие переменные являются фиктивными и проверьте, является ли формула тавтологией:

$(P \rightarrow (Q \land R)) \leftrightarrow ((P \rightarrow Q) \land (P \rightarrow R))$

\end{questions}
\newpage
%%% begin test
\begin{flushright}
\begin{tabular}{p{2.8in} r l}
%\textbf{\class} & \textbf{ФИО:} & \makebox[2.5in]{\hrulefill}\\
\textbf{\class} & \textbf{ФИО:} &Змушко Андрей Сергеевич
\\

\textbf{\examdate} &&\\
%\textbf{Time Limit: \timelimit} & Teaching Assistant & \makebox[2in]{\hrulefill}
\end{tabular}\\
\end{flushright}
\rule[1ex]{\textwidth}{.1pt}


\begin{questions}
\question
Найдите и упростите P:
\begin{equation*}
\overline{P} = \overline{A} \cap B \cup \overline{A} \cap C \cup A \cap \overline{B} \cup \overline{B} \cap C
\end{equation*}
Затем найдите элементы множества P, выраженного через множества:
\begin{equation*}
A = \{0, 3, 4, 9\}; 
B = \{1, 3, 4, 7\};
C = \{0, 1, 2, 4, 7, 8, 9\};
I = \{0, 1, 2, 3, 4, 5, 6, 7, 8, 9\}.
\end{equation*}\question
Упростите следующее выражение с учетом того, что $A\subset B \subset C \subset D \subset U; A \neq \O$
\begin{equation*}
\overline{B} \cap \overline{C} \cap D \cup \overline{A} \cap \overline{C} \cap D \cup \overline{A} \cap B
\end{equation*}

Примечание: U — универсум\question
Дано отношение на множестве $\{1, 2, 3, 4, 5\}$ 
\begin{equation*}
aRb \iff |a-b| = 1
\end{equation*}
Напишите обоснованный ответ какими свойствами обладает или не обладает отношение и почему:   
\begin{enumerate} [a)]\setcounter{enumi}{0}
\item рефлексивность
\item антирефлексивность
\item симметричность
\item асимметричность
\item антисимметричность
\item транзитивность
\end{enumerate}

Обоснуйте свой ответ по каждому из приведенных ниже вопросов:
\begin{enumerate} [a)]\setcounter{enumi}{0}
    \item Является ли это отношение отношением эквивалентности?
    \item Является ли это отношение функциональным?
    \item Каким из отношений соответствия (одно-многозначным, много-многозначный и т.д.) оно является?
    \item К каким из отношений порядка (полного, частичного и т.д.) можно отнести данное отношение?
\end{enumerate}

\question
Установите, является ли каждое из перечисленных ниже отношений на А ($R \subseteq A \times A$) отношением эквивалентности (обоснование ответа обязательно). Для каждого отношения эквивалентности постройте классы 
эквивалентности и постройте граф отношения:
\begin{enumerate} [a)]\setcounter{enumi}{0}
\item Пусть A – множество имен. $A = \{ $Алексей, Иван, Петр, Александр, Павел, Андрей$ \}$. Тогда отношение $R$ верно на парах имен, начинающихся с одной и той же буквы, и только на них.
\item $A = \{-10, -9, … , 9, 10\}$ и отношение $ R = \{(a,b)|a^{2} = b^{2}\}$
\item На множестве $A = \{1; 2; 3\}$ задано отношение $R = \{(1; 1); (2; 2); (3; 3); (3; 2); (1; 2); (2; 1)\}$
\end{enumerate}\question Составьте полную таблицу истинности, определите, какие переменные являются фиктивными и проверьте, является ли формула тавтологией:
$(( P \rightarrow Q) \land (Q \rightarrow P)) \rightarrow (P \rightarrow R)$

\end{questions}
\newpage
%%% begin test
\begin{flushright}
\begin{tabular}{p{2.8in} r l}
%\textbf{\class} & \textbf{ФИО:} & \makebox[2.5in]{\hrulefill}\\
\textbf{\class} & \textbf{ФИО:} &Комова Анна Владимировна
\\

\textbf{\examdate} &&\\
%\textbf{Time Limit: \timelimit} & Teaching Assistant & \makebox[2in]{\hrulefill}
\end{tabular}\\
\end{flushright}
\rule[1ex]{\textwidth}{.1pt}


\begin{questions}
\question
Найдите и упростите P:
\begin{equation*}
\overline{P} = A \cap \overline{C} \cup A \cap \overline{B} \cup B \cap \overline{C} \cup A \cap C
\end{equation*}
Затем найдите элементы множества P, выраженного через множества:
\begin{equation*}
A = \{0, 3, 4, 9\}; 
B = \{1, 3, 4, 7\};
C = \{0, 1, 2, 4, 7, 8, 9\};
I = \{0, 1, 2, 3, 4, 5, 6, 7, 8, 9\}.
\end{equation*}\question
Упростите следующее выражение с учетом того, что $A\subset B \subset C \subset D \subset U; A \neq \O$
\begin{equation*}
A \cap B  \cap \overline{C} \cup \overline{C} \cap D \cup B \cap C \cap D
\end{equation*}

Примечание: U — универсум\question
Дано отношение на множестве $\{1, 2, 3, 4, 5\}$ 
\begin{equation*}
aRb \iff b > a
\end{equation*}
Напишите обоснованный ответ какими свойствами обладает или не обладает отношение и почему:   
\begin{enumerate} [a)]\setcounter{enumi}{0}
\item рефлексивность
\item антирефлексивность
\item симметричность
\item асимметричность
\item антисимметричность
\item транзитивность
\end{enumerate}

Обоснуйте свой ответ по каждому из приведенных ниже вопросов:
\begin{enumerate} [a)]\setcounter{enumi}{0}
    \item Является ли это отношение отношением эквивалентности?
    \item Является ли это отношение функциональным?
    \item Каким из отношений соответствия (одно-многозначным, много-многозначный и т.д.) оно является?
    \item К каким из отношений порядка (полного, частичного и т.д.) можно отнести данное отношение?
\end{enumerate}

\question
Установите, является ли каждое из перечисленных ниже отношений на А ($R \subseteq A \times A$) отношением эквивалентности (обоснование ответа обязательно). Для каждого отношения эквивалентности постройте классы 
эквивалентности и постройте граф отношения:
\begin{enumerate} [a)]\setcounter{enumi}{0}
\item А - множество целых чисел и отношение $R = \{(a,b)|a + b = 5\}$
\item Пусть A – множество имен. $A = \{ $Алексей, Иван, Петр, Александр, Павел, Андрей$ \}$. Тогда отношение $R $ верно на парах имен, начинающихся с одной и той же буквы, и только на них.
\item На множестве $A = \{1; 2; 3; 4; 5\}$ задано отношение $R = \{(1; 2); (1; 3); (1; 5); (2; 3); (2; 4); (2; 5); (3; 4); (3; 5); (4; 5)\}$
\end{enumerate}\question Составьте полную таблицу истинности, определите, какие переменные являются фиктивными и проверьте, является ли формула тавтологией:
$ P \rightarrow (Q \rightarrow ((P \lor Q) \rightarrow (P \land Q)))$

\end{questions}
\newpage
%%% begin test
\begin{flushright}
\begin{tabular}{p{2.8in} r l}
%\textbf{\class} & \textbf{ФИО:} & \makebox[2.5in]{\hrulefill}\\
\textbf{\class} & \textbf{ФИО:} &Корехов Илья Андреевич
\\

\textbf{\examdate} &&\\
%\textbf{Time Limit: \timelimit} & Teaching Assistant & \makebox[2in]{\hrulefill}
\end{tabular}\\
\end{flushright}
\rule[1ex]{\textwidth}{.1pt}


\begin{questions}
\question
Найдите и упростите P:
\begin{equation*}
\overline{P} = A \cap C \cup \overline{A} \cap \overline{C} \cup \overline{B} \cap C \cup \overline{A} \cap \overline{B}
\end{equation*}
Затем найдите элементы множества P, выраженного через множества:
\begin{equation*}
A = \{0, 3, 4, 9\}; 
B = \{1, 3, 4, 7\};
C = \{0, 1, 2, 4, 7, 8, 9\};
I = \{0, 1, 2, 3, 4, 5, 6, 7, 8, 9\}.
\end{equation*}\question
Упростите следующее выражение с учетом того, что $A\subset B \subset C \subset D \subset U; A \neq \O$
\begin{equation*}
\overline{B} \cap \overline{C} \cap D \cup \overline{A} \cap \overline{C} \cap D \cup \overline{A} \cap B
\end{equation*}

Примечание: U — универсум\question
Дано отношение на множестве $\{1, 2, 3, 4, 5\}$ 
\begin{equation*}
aRb \iff a \geq b^2
\end{equation*}
Напишите обоснованный ответ какими свойствами обладает или не обладает отношение и почему:   
\begin{enumerate} [a)]\setcounter{enumi}{0}
\item рефлексивность
\item антирефлексивность
\item симметричность
\item асимметричность
\item антисимметричность
\item транзитивность
\end{enumerate}

Обоснуйте свой ответ по каждому из приведенных ниже вопросов:
\begin{enumerate} [a)]\setcounter{enumi}{0}
    \item Является ли это отношение отношением эквивалентности?
    \item Является ли это отношение функциональным?
    \item Каким из отношений соответствия (одно-многозначным, много-многозначный и т.д.) оно является?
    \item К каким из отношений порядка (полного, частичного и т.д.) можно отнести данное отношение?
\end{enumerate}


\question
Установите, является ли каждое из перечисленных ниже отношений на А ($R \subseteq A \times A$) отношением эквивалентности (обоснование ответа обязательно). Для каждого отношения эквивалентности постройте классы 
эквивалентности и постройте граф отношения:
\begin{enumerate} [a)]\setcounter{enumi}{0}
\item На множестве $A = \{1; 2; 3\}$ задано отношение $R = \{(1; 1); (2; 2); (3; 3); (2; 1); (1; 2); (2; 3); (3; 2); (3; 1); (1; 3)\}$
\item На множестве $A = \{1; 2; 3; 4; 5\}$ задано отношение $R = \{(1; 2); (1; 3); (1; 5); (2; 3); (2; 4); (2; 5); (3; 4); (3; 5); (4; 5)\}$
\item А - множество целых чисел и отношение $R = \{(a,b)|a + b = 0\}$
\end{enumerate}\question Составьте полную таблицу истинности, определите, какие переменные являются фиктивными и проверьте, является ли формула тавтологией:
$ P \rightarrow (Q \rightarrow ((P \lor Q) \rightarrow (P \land Q)))$

\end{questions}
\newpage
%%% begin test
\begin{flushright}
\begin{tabular}{p{2.8in} r l}
%\textbf{\class} & \textbf{ФИО:} & \makebox[2.5in]{\hrulefill}\\
\textbf{\class} & \textbf{ФИО:} &Куликов Олег Леонидович
\\

\textbf{\examdate} &&\\
%\textbf{Time Limit: \timelimit} & Teaching Assistant & \makebox[2in]{\hrulefill}
\end{tabular}\\
\end{flushright}
\rule[1ex]{\textwidth}{.1pt}


\begin{questions}
\question
Найдите и упростите P:
\begin{equation*}
\overline{P} = A \cap \overline{C} \cup A \cap \overline{B} \cup B \cap \overline{C} \cup A \cap C
\end{equation*}
Затем найдите элементы множества P, выраженного через множества:
\begin{equation*}
A = \{0, 3, 4, 9\}; 
B = \{1, 3, 4, 7\};
C = \{0, 1, 2, 4, 7, 8, 9\};
I = \{0, 1, 2, 3, 4, 5, 6, 7, 8, 9\}.
\end{equation*}\question
Упростите следующее выражение с учетом того, что $A\subset B \subset C \subset D \subset U; A \neq \O$
\begin{equation*}
\overline{A} \cap \overline{B} \cup B \cap \overline{C} \cup \overline{C} \cap D
\end{equation*}

Примечание: U — универсум\question
Дано отношение на множестве $\{1, 2, 3, 4, 5\}$ 
\begin{equation*}
aRb \iff  \text{НОД}(a,b) =1
\end{equation*}
Напишите обоснованный ответ какими свойствами обладает или не обладает отношение и почему:   
\begin{enumerate} [a)]\setcounter{enumi}{0}
\item рефлексивность
\item антирефлексивность
\item симметричность
\item асимметричность
\item антисимметричность
\item транзитивность
\end{enumerate}

Обоснуйте свой ответ по каждому из приведенных ниже вопросов:
\begin{enumerate} [a)]\setcounter{enumi}{0}
    \item Является ли это отношение отношением эквивалентности?
    \item Является ли это отношение функциональным?
    \item Каким из отношений соответствия (одно-многозначным, много-многозначный и т.д.) оно является?
    \item К каким из отношений порядка (полного, частичного и т.д.) можно отнести данное отношение?
\end{enumerate}


\question
Установите, является ли каждое из перечисленных ниже отношений на А ($R \subseteq A \times A$) отношением эквивалентности (обоснование ответа обязательно). Для каждого отношения эквивалентности постройте классы 
эквивалентности и постройте граф отношения:
\begin{enumerate} [a)]\setcounter{enumi}{0}
\item $A = \{a, b, c, d, p, t\}$ задано отношение $R = \{(a, a), (b, b), (b, c), (b, d), (c, b), (c, c), (c, d), (d, b), (d, c), (d, d), (p,p), (t,t)\}$
\item $A = \{-10, -9, … , 9, 10\}$ и отношение $R = \{(a,b)|a^{3} = b^{3}\}$

\item $F(x)=x^{2}+1$, где $x \in A = [-2, 4]$ и отношение $R = \{(a,b)|F(a) = F(b)\}$
\end{enumerate}\question Составьте полную таблицу истинности, определите, какие переменные являются фиктивными и проверьте, является ли формула тавтологией:
$((P \rightarrow Q) \land (R \rightarrow S) \land \neg (Q \lor S)) \rightarrow \neg (P \lor R)$

\end{questions}
\newpage
%%% begin test
\begin{flushright}
\begin{tabular}{p{2.8in} r l}
%\textbf{\class} & \textbf{ФИО:} & \makebox[2.5in]{\hrulefill}\\
\textbf{\class} & \textbf{ФИО:} &Ли Евгений Владимирович
\\

\textbf{\examdate} &&\\
%\textbf{Time Limit: \timelimit} & Teaching Assistant & \makebox[2in]{\hrulefill}
\end{tabular}\\
\end{flushright}
\rule[1ex]{\textwidth}{.1pt}


\begin{questions}
\question
Найдите и упростите P:
\begin{equation*}
\overline{P} = A \cap C \cup \overline{A} \cap \overline{C} \cup \overline{B} \cap C \cup \overline{A} \cap \overline{B}
\end{equation*}
Затем найдите элементы множества P, выраженного через множества:
\begin{equation*}
A = \{0, 3, 4, 9\}; 
B = \{1, 3, 4, 7\};
C = \{0, 1, 2, 4, 7, 8, 9\};
I = \{0, 1, 2, 3, 4, 5, 6, 7, 8, 9\}.
\end{equation*}\question
Упростите следующее выражение с учетом того, что $A\subset B \subset C \subset D \subset U; A \neq \O$
\begin{equation*}
\overline{A} \cap \overline{C} \cap D \cup \overline{B} \cap \overline{C} \cap D \cup A \cap B
\end{equation*}

Примечание: U — универсум\question
Для следующего отношения на множестве $\{1, 2, 3, 4, 5\}$ 
\begin{equation*}
aRb \iff 0 < a-b<2
\end{equation*}
Напишите обоснованный ответ какими свойствами обладает или не обладает отношение и почему:   
\begin{enumerate} [a)]\setcounter{enumi}{0}
\item рефлексивность
\item антирефлексивность
\item симметричность
\item асимметричность
\item антисимметричность
\item транзитивность
\end{enumerate}

Обоснуйте свой ответ по каждому из приведенных ниже вопросов:
\begin{enumerate} [a)]\setcounter{enumi}{0}
    \item Является ли это отношение отношением эквивалентности?
    \item Является ли это отношение функциональным?
    \item Каким из отношений соответствия (одно-многозначным, много-многозначный и т.д.) оно является?
    \item К каким из отношений порядка (полного, частичного и т.д.) можно отнести данное отношение?
\end{enumerate}
\question
Установите, является ли каждое из перечисленных ниже отношений на А ($R \subseteq A \times A$) отношением эквивалентности (обоснование ответа обязательно). Для каждого отношения эквивалентности постройте классы 
эквивалентности и постройте граф отношения:
\begin{enumerate} [a)]\setcounter{enumi}{0}
\item А - множество целых чисел и отношение $R = \{(a,b)|a + b = 5\}$
\item Пусть A – множество имен. $A = \{ $Алексей, Иван, Петр, Александр, Павел, Андрей$ \}$. Тогда отношение $R $ верно на парах имен, начинающихся с одной и той же буквы, и только на них.
\item На множестве $A = \{1; 2; 3; 4; 5\}$ задано отношение $R = \{(1; 2); (1; 3); (1; 5); (2; 3); (2; 4); (2; 5); (3; 4); (3; 5); (4; 5)\}$
\end{enumerate}\question Составьте полную таблицу истинности, определите, какие переменные являются фиктивными и проверьте, является ли формула тавтологией:

$(P \rightarrow (Q \land R)) \leftrightarrow ((P \rightarrow Q) \land (P \rightarrow R))$

\end{questions}
\newpage
%%% begin test
\begin{flushright}
\begin{tabular}{p{2.8in} r l}
%\textbf{\class} & \textbf{ФИО:} & \makebox[2.5in]{\hrulefill}\\
\textbf{\class} & \textbf{ФИО:} &Мелещенко Иннокентий Олегович
\\

\textbf{\examdate} &&\\
%\textbf{Time Limit: \timelimit} & Teaching Assistant & \makebox[2in]{\hrulefill}
\end{tabular}\\
\end{flushright}
\rule[1ex]{\textwidth}{.1pt}


\begin{questions}
\question
Найдите и упростите P:
\begin{equation*}
\overline{P} = A \cap \overline{C} \cup A \cap \overline{B} \cup B \cap \overline{C} \cup A \cap C
\end{equation*}
Затем найдите элементы множества P, выраженного через множества:
\begin{equation*}
A = \{0, 3, 4, 9\}; 
B = \{1, 3, 4, 7\};
C = \{0, 1, 2, 4, 7, 8, 9\};
I = \{0, 1, 2, 3, 4, 5, 6, 7, 8, 9\}.
\end{equation*}\question
Упростите следующее выражение с учетом того, что $A\subset B \subset C \subset D \subset U; A \neq \O$
\begin{equation*}
A \cap  \overline{C} \cup B \cap \overline{D} \cup  \overline{A} \cap C \cap  \overline{D}
\end{equation*}

Примечание: U — универсум\question
Дано отношение на множестве $\{1, 2, 3, 4, 5\}$ 
\begin{equation*}
aRb \iff (a+b) \bmod 2 =0
\end{equation*}
Напишите обоснованный ответ какими свойствами обладает или не обладает отношение и почему:   
\begin{enumerate} [a)]\setcounter{enumi}{0}
\item рефлексивность
\item антирефлексивность
\item симметричность
\item асимметричность
\item антисимметричность
\item транзитивность
\end{enumerate}

Обоснуйте свой ответ по каждому из приведенных ниже вопросов:
\begin{enumerate} [a)]\setcounter{enumi}{0}
    \item Является ли это отношение отношением эквивалентности?
    \item Является ли это отношение функциональным?
    \item Каким из отношений соответствия (одно-многозначным, много-многозначный и т.д.) оно является?
    \item К каким из отношений порядка (полного, частичного и т.д.) можно отнести данное отношение?
\end{enumerate}



\question
Установите, является ли каждое из перечисленных ниже отношений на А ($R \subseteq A \times A$) отношением эквивалентности (обоснование ответа обязательно). Для каждого отношения эквивалентности постройте классы 
эквивалентности и постройте граф отношения:
\begin{enumerate} [a)]\setcounter{enumi}{0}
\item $A = \{-10, -9, … , 9, 10\}$ и отношение $R = \{(a,b)|a^{2} = b^{2}\}$
\item $A = \{a, b, c, d, p, t\}$ задано отношение $R = \{(a, a), (b, b), (b, c), (b, d), (c, b), (c, c), (c, d), (d, b), (d, c), (d, d), (p,p), (t,t)\}$
\item Пусть A – множество имен. $A = \{ $Алексей, Иван, Петр, Александр, Павел, Андрей$ \}$. Тогда отношение $R$ верно на парах имен, начинающихся с одной и той же буквы, и только на них.
\end{enumerate}\question Составьте полную таблицу истинности, определите, какие переменные являются фиктивными и проверьте, является ли формула тавтологией:
$((P \rightarrow Q) \land (R \rightarrow S) \land \neg (Q \lor S)) \rightarrow \neg (P \lor R)$

\end{questions}
\newpage
%%% begin test
\begin{flushright}
\begin{tabular}{p{2.8in} r l}
%\textbf{\class} & \textbf{ФИО:} & \makebox[2.5in]{\hrulefill}\\
\textbf{\class} & \textbf{ФИО:} &Пологов Никита Евгеньевич
\\

\textbf{\examdate} &&\\
%\textbf{Time Limit: \timelimit} & Teaching Assistant & \makebox[2in]{\hrulefill}
\end{tabular}\\
\end{flushright}
\rule[1ex]{\textwidth}{.1pt}


\begin{questions}
\question
Найдите и упростите P:
\begin{equation*}
\overline{P} = B \cap \overline{C} \cup A \cap B \cup \overline{A} \cap C \cup \overline{A} \cap B
\end{equation*}
Затем найдите элементы множества P, выраженного через множества:
\begin{equation*}
A = \{0, 3, 4, 9\}; 
B = \{1, 3, 4, 7\};
C = \{0, 1, 2, 4, 7, 8, 9\};
I = \{0, 1, 2, 3, 4, 5, 6, 7, 8, 9\}.
\end{equation*}\question
Упростите следующее выражение с учетом того, что $A\subset B \subset C \subset D \subset U; A \neq \O$
\begin{equation*}
\overline{A} \cap \overline{C} \cap D \cup \overline{B} \cap \overline{C} \cap D \cup A \cap B
\end{equation*}

Примечание: U — универсум\question
Для следующего отношения на множестве $\{1, 2, 3, 4, 5\}$ 
\begin{equation*}
aRb \iff 0 < a-b<2
\end{equation*}
Напишите обоснованный ответ какими свойствами обладает или не обладает отношение и почему:   
\begin{enumerate} [a)]\setcounter{enumi}{0}
\item рефлексивность
\item антирефлексивность
\item симметричность
\item асимметричность
\item антисимметричность
\item транзитивность
\end{enumerate}

Обоснуйте свой ответ по каждому из приведенных ниже вопросов:
\begin{enumerate} [a)]\setcounter{enumi}{0}
    \item Является ли это отношение отношением эквивалентности?
    \item Является ли это отношение функциональным?
    \item Каким из отношений соответствия (одно-многозначным, много-многозначный и т.д.) оно является?
    \item К каким из отношений порядка (полного, частичного и т.д.) можно отнести данное отношение?
\end{enumerate}
\question
Установите, является ли каждое из перечисленных ниже отношений на А ($R \subseteq A \times A$) отношением эквивалентности (обоснование ответа обязательно). Для каждого отношения эквивалентности постройте классы 
эквивалентности и постройте граф отношения:
\begin{enumerate} [a)]\setcounter{enumi}{0}
\item На множестве $A = \{1; 2; 3\}$ задано отношение $R = \{(1; 1); (2; 2); (3; 3); (2; 1); (1; 2); (2; 3); (3; 2); (3; 1); (1; 3)\}$
\item На множестве $A = \{1; 2; 3; 4; 5\}$ задано отношение $R = \{(1; 2); (1; 3); (1; 5); (2; 3); (2; 4); (2; 5); (3; 4); (3; 5); (4; 5)\}$
\item А - множество целых чисел и отношение $R = \{(a,b)|a + b = 0\}$
\end{enumerate}\question Составьте полную таблицу истинности, определите, какие переменные являются фиктивными и проверьте, является ли формула тавтологией:
$((P \rightarrow Q) \land (R \rightarrow S) \land \neg (Q \lor S)) \rightarrow \neg (P \lor R)$

\end{questions}
\newpage
%%% begin test
\begin{flushright}
\begin{tabular}{p{2.8in} r l}
%\textbf{\class} & \textbf{ФИО:} & \makebox[2.5in]{\hrulefill}\\
\textbf{\class} & \textbf{ФИО:} &Руковишников Михаил Александрович
\\

\textbf{\examdate} &&\\
%\textbf{Time Limit: \timelimit} & Teaching Assistant & \makebox[2in]{\hrulefill}
\end{tabular}\\
\end{flushright}
\rule[1ex]{\textwidth}{.1pt}


\begin{questions}
\question
Найдите и упростите P:
\begin{equation*}
\overline{P} = A \cap \overline{B} \cup \overline{B} \cap C \cup \overline{A} \cap \overline{B} \cup \overline{A} \cap C
\end{equation*}
Затем найдите элементы множества P, выраженного через множества:
\begin{equation*}
A = \{0, 3, 4, 9\}; 
B = \{1, 3, 4, 7\};
C = \{0, 1, 2, 4, 7, 8, 9\};
I = \{0, 1, 2, 3, 4, 5, 6, 7, 8, 9\}.
\end{equation*}\question
Упростите следующее выражение с учетом того, что $A\subset B \subset C \subset D \subset U; A \neq \O$
\begin{equation*}
A \cap C  \cap D \cup B \cap \overline{C} \cap D \cup B \cap C \cap D
\end{equation*}

Примечание: U — универсум\question
Дано отношение на множестве $\{1, 2, 3, 4, 5\}$ 
\begin{equation*}
aRb \iff a \leq b
\end{equation*}
Напишите обоснованный ответ какими свойствами обладает или не обладает отношение и почему:   
\begin{enumerate} [a)]\setcounter{enumi}{0}
\item рефлексивность
\item антирефлексивность
\item симметричность
\item асимметричность
\item антисимметричность
\item транзитивность
\end{enumerate}

Обоснуйте свой ответ по каждому из приведенных ниже вопросов:
\begin{enumerate} [a)]\setcounter{enumi}{0}
    \item Является ли это отношение отношением эквивалентности?
    \item Является ли это отношение функциональным?
    \item Каким из отношений соответствия (одно-многозначным, много-многозначный и т.д.) оно является?
    \item К каким из отношений порядка (полного, частичного и т.д.) можно отнести данное отношение?
\end{enumerate}


\question
Установите, является ли каждое из перечисленных ниже отношений на А ($R \subseteq A \times A$) отношением эквивалентности (обоснование ответа обязательно). Для каждого отношения эквивалентности постройте классы 
эквивалентности и постройте граф отношения:
\begin{enumerate} [a)]\setcounter{enumi}{0}
\item Пусть A – множество имен. $A = \{ $Алексей, Иван, Петр, Александр, Павел, Андрей$ \}$. Тогда отношение $R$ верно на парах имен, начинающихся с одной и той же буквы, и только на них.
\item $A = \{-10, -9, … , 9, 10\}$ и отношение $ R = \{(a,b)|a^{2} = b^{2}\}$
\item На множестве $A = \{1; 2; 3\}$ задано отношение $R = \{(1; 1); (2; 2); (3; 3); (3; 2); (1; 2); (2; 1)\}$
\end{enumerate}\question Составьте полную таблицу истинности, определите, какие переменные являются фиктивными и проверьте, является ли формула тавтологией:
$(( P \rightarrow Q) \land (Q \rightarrow P)) \rightarrow (P \rightarrow R)$

\end{questions}
\newpage
%%% begin test
\begin{flushright}
\begin{tabular}{p{2.8in} r l}
%\textbf{\class} & \textbf{ФИО:} & \makebox[2.5in]{\hrulefill}\\
\textbf{\class} & \textbf{ФИО:} &Садохов Вадим Алексеевич
\\

\textbf{\examdate} &&\\
%\textbf{Time Limit: \timelimit} & Teaching Assistant & \makebox[2in]{\hrulefill}
\end{tabular}\\
\end{flushright}
\rule[1ex]{\textwidth}{.1pt}


\begin{questions}
\question
Найдите и упростите P:
\begin{equation*}
\overline{P} = B \cap \overline{C} \cup A \cap B \cup \overline{A} \cap C \cup \overline{A} \cap B
\end{equation*}
Затем найдите элементы множества P, выраженного через множества:
\begin{equation*}
A = \{0, 3, 4, 9\}; 
B = \{1, 3, 4, 7\};
C = \{0, 1, 2, 4, 7, 8, 9\};
I = \{0, 1, 2, 3, 4, 5, 6, 7, 8, 9\}.
\end{equation*}\question
Упростите следующее выражение с учетом того, что $A\subset B \subset C \subset D \subset U; A \neq \O$
\begin{equation*}
A \cap C  \cap D \cup B \cap \overline{C} \cap D \cup B \cap C \cap D
\end{equation*}

Примечание: U — универсум\question
Дано отношение на множестве $\{1, 2, 3, 4, 5\}$ 
\begin{equation*}
aRb \iff  \text{НОД}(a,b) =1
\end{equation*}
Напишите обоснованный ответ какими свойствами обладает или не обладает отношение и почему:   
\begin{enumerate} [a)]\setcounter{enumi}{0}
\item рефлексивность
\item антирефлексивность
\item симметричность
\item асимметричность
\item антисимметричность
\item транзитивность
\end{enumerate}

Обоснуйте свой ответ по каждому из приведенных ниже вопросов:
\begin{enumerate} [a)]\setcounter{enumi}{0}
    \item Является ли это отношение отношением эквивалентности?
    \item Является ли это отношение функциональным?
    \item Каким из отношений соответствия (одно-многозначным, много-многозначный и т.д.) оно является?
    \item К каким из отношений порядка (полного, частичного и т.д.) можно отнести данное отношение?
\end{enumerate}


\question
Установите, является ли каждое из перечисленных ниже отношений на А ($R \subseteq A \times A$) отношением эквивалентности (обоснование ответа обязательно). Для каждого отношения эквивалентности постройте классы 
эквивалентности и постройте граф отношения:
\begin{enumerate} [a)]\setcounter{enumi}{0}
\item $A = \{a, b, c, d, p, t\}$ задано отношение $R = \{(a, a), (b, b), (b, c), (b, d), (c, b), (c, c), (c, d), (d, b), (d, c), (d, d), (p,p), (t,t)\}$
\item $A = \{-10, -9, … , 9, 10\}$ и отношение $R = \{(a,b)|a^{3} = b^{3}\}$

\item $F(x)=x^{2}+1$, где $x \in A = [-2, 4]$ и отношение $R = \{(a,b)|F(a) = F(b)\}$
\end{enumerate}\question Составьте полную таблицу истинности, определите, какие переменные являются фиктивными и проверьте, является ли формула тавтологией:
$((P \rightarrow Q) \lor R) \leftrightarrow (P \rightarrow (Q \lor R))$

\end{questions}
\newpage
%%% begin test
\begin{flushright}
\begin{tabular}{p{2.8in} r l}
%\textbf{\class} & \textbf{ФИО:} & \makebox[2.5in]{\hrulefill}\\
\textbf{\class} & \textbf{ФИО:} &Садыкова Алсу Дамировна
\\

\textbf{\examdate} &&\\
%\textbf{Time Limit: \timelimit} & Teaching Assistant & \makebox[2in]{\hrulefill}
\end{tabular}\\
\end{flushright}
\rule[1ex]{\textwidth}{.1pt}


\begin{questions}
\question
Найдите и упростите P:
\begin{equation*}
\overline{P} = A \cap \overline{C} \cup A \cap \overline{B} \cup B \cap \overline{C} \cup A \cap C
\end{equation*}
Затем найдите элементы множества P, выраженного через множества:
\begin{equation*}
A = \{0, 3, 4, 9\}; 
B = \{1, 3, 4, 7\};
C = \{0, 1, 2, 4, 7, 8, 9\};
I = \{0, 1, 2, 3, 4, 5, 6, 7, 8, 9\}.
\end{equation*}\question
Упростите следующее выражение с учетом того, что $A\subset B \subset C \subset D \subset U; A \neq \O$
\begin{equation*}
\overline{A} \cap \overline{C} \cap D \cup \overline{B} \cap \overline{C} \cap D \cup A \cap B
\end{equation*}

Примечание: U — универсум\question
Дано отношение на множестве $\{1, 2, 3, 4, 5\}$ 
\begin{equation*}
aRb \iff a \leq b
\end{equation*}
Напишите обоснованный ответ какими свойствами обладает или не обладает отношение и почему:   
\begin{enumerate} [a)]\setcounter{enumi}{0}
\item рефлексивность
\item антирефлексивность
\item симметричность
\item асимметричность
\item антисимметричность
\item транзитивность
\end{enumerate}

Обоснуйте свой ответ по каждому из приведенных ниже вопросов:
\begin{enumerate} [a)]\setcounter{enumi}{0}
    \item Является ли это отношение отношением эквивалентности?
    \item Является ли это отношение функциональным?
    \item Каким из отношений соответствия (одно-многозначным, много-многозначный и т.д.) оно является?
    \item К каким из отношений порядка (полного, частичного и т.д.) можно отнести данное отношение?
\end{enumerate}


\question
Установите, является ли каждое из перечисленных ниже отношений на А ($R \subseteq A \times A$) отношением эквивалентности (обоснование ответа обязательно). Для каждого отношения эквивалентности постройте классы 
эквивалентности и постройте граф отношения:
\begin{enumerate} [a)]\setcounter{enumi}{0}
\item А - множество целых чисел и отношение $R = \{(a,b)|a + b = 5\}$
\item Пусть A – множество имен. $A = \{ $Алексей, Иван, Петр, Александр, Павел, Андрей$ \}$. Тогда отношение $R $ верно на парах имен, начинающихся с одной и той же буквы, и только на них.
\item На множестве $A = \{1; 2; 3; 4; 5\}$ задано отношение $R = \{(1; 2); (1; 3); (1; 5); (2; 3); (2; 4); (2; 5); (3; 4); (3; 5); (4; 5)\}$
\end{enumerate}\question Составьте полную таблицу истинности, определите, какие переменные являются фиктивными и проверьте, является ли формула тавтологией:
$(P \rightarrow (Q \rightarrow R)) \rightarrow ((P \rightarrow Q) \rightarrow (P \rightarrow R))$

\end{questions}
\newpage
%%% begin test
\begin{flushright}
\begin{tabular}{p{2.8in} r l}
%\textbf{\class} & \textbf{ФИО:} & \makebox[2.5in]{\hrulefill}\\
\textbf{\class} & \textbf{ФИО:} &Салдусов Владимир Гаряевич
\\

\textbf{\examdate} &&\\
%\textbf{Time Limit: \timelimit} & Teaching Assistant & \makebox[2in]{\hrulefill}
\end{tabular}\\
\end{flushright}
\rule[1ex]{\textwidth}{.1pt}


\begin{questions}
\question
Найдите и упростите P:
\begin{equation*}
\overline{P} = B \cap \overline{C} \cup A \cap B \cup \overline{A} \cap C \cup \overline{A} \cap B
\end{equation*}
Затем найдите элементы множества P, выраженного через множества:
\begin{equation*}
A = \{0, 3, 4, 9\}; 
B = \{1, 3, 4, 7\};
C = \{0, 1, 2, 4, 7, 8, 9\};
I = \{0, 1, 2, 3, 4, 5, 6, 7, 8, 9\}.
\end{equation*}\question
Упростите следующее выражение с учетом того, что $A\subset B \subset C \subset D \subset U; A \neq \O$
\begin{equation*}
A \cap  \overline{C} \cup B \cap \overline{D} \cup  \overline{A} \cap C \cap  \overline{D}
\end{equation*}

Примечание: U — универсум\question
Дано отношение на множестве $\{1, 2, 3, 4, 5\}$ 
\begin{equation*}
aRb \iff a \leq b
\end{equation*}
Напишите обоснованный ответ какими свойствами обладает или не обладает отношение и почему:   
\begin{enumerate} [a)]\setcounter{enumi}{0}
\item рефлексивность
\item антирефлексивность
\item симметричность
\item асимметричность
\item антисимметричность
\item транзитивность
\end{enumerate}

Обоснуйте свой ответ по каждому из приведенных ниже вопросов:
\begin{enumerate} [a)]\setcounter{enumi}{0}
    \item Является ли это отношение отношением эквивалентности?
    \item Является ли это отношение функциональным?
    \item Каким из отношений соответствия (одно-многозначным, много-многозначный и т.д.) оно является?
    \item К каким из отношений порядка (полного, частичного и т.д.) можно отнести данное отношение?
\end{enumerate}


\question
Установите, является ли каждое из перечисленных ниже отношений на А ($R \subseteq A \times A$) отношением эквивалентности (обоснование ответа обязательно). Для каждого отношения эквивалентности постройте классы 
эквивалентности и постройте граф отношения:
\begin{enumerate} [a)]\setcounter{enumi}{0}
\item $A = \{a, b, c, d, p, t\}$ задано отношение $R = \{(a, a), (b, b), (b, c), (b, d), (c, b), (c, c), (c, d), (d, b), (d, c), (d, d), (p,p), (t,t)\}$
\item $A = \{-10, -9, … , 9, 10\}$ и отношение $R = \{(a,b)|a^{3} = b^{3}\}$

\item $F(x)=x^{2}+1$, где $x \in A = [-2, 4]$ и отношение $R = \{(a,b)|F(a) = F(b)\}$
\end{enumerate}\question Составьте полную таблицу истинности, определите, какие переменные являются фиктивными и проверьте, является ли формула тавтологией:
$(( P \rightarrow Q) \land (Q \rightarrow P)) \rightarrow (P \rightarrow R)$

\end{questions}
\newpage
%%% begin test
\begin{flushright}
\begin{tabular}{p{2.8in} r l}
%\textbf{\class} & \textbf{ФИО:} & \makebox[2.5in]{\hrulefill}\\
\textbf{\class} & \textbf{ФИО:} &Сластенин Григорий Сергеевич
\\

\textbf{\examdate} &&\\
%\textbf{Time Limit: \timelimit} & Teaching Assistant & \makebox[2in]{\hrulefill}
\end{tabular}\\
\end{flushright}
\rule[1ex]{\textwidth}{.1pt}


\begin{questions}
\question
Найдите и упростите P:
\begin{equation*}
\overline{P} = B \cap \overline{C} \cup A \cap B \cup \overline{A} \cap C \cup \overline{A} \cap B
\end{equation*}
Затем найдите элементы множества P, выраженного через множества:
\begin{equation*}
A = \{0, 3, 4, 9\}; 
B = \{1, 3, 4, 7\};
C = \{0, 1, 2, 4, 7, 8, 9\};
I = \{0, 1, 2, 3, 4, 5, 6, 7, 8, 9\}.
\end{equation*}\question
Упростите следующее выражение с учетом того, что $A\subset B \subset C \subset D \subset U; A \neq \O$
\begin{equation*}
A \cap B \cup \overline{A} \cap \overline{C} \cup A \cap C \cup \overline{B} \cap \overline{C}
\end{equation*}

Примечание: U — универсум\question
Дано отношение на множестве $\{1, 2, 3, 4, 5\}$ 
\begin{equation*}
aRb \iff |a-b| = 1
\end{equation*}
Напишите обоснованный ответ какими свойствами обладает или не обладает отношение и почему:   
\begin{enumerate} [a)]\setcounter{enumi}{0}
\item рефлексивность
\item антирефлексивность
\item симметричность
\item асимметричность
\item антисимметричность
\item транзитивность
\end{enumerate}

Обоснуйте свой ответ по каждому из приведенных ниже вопросов:
\begin{enumerate} [a)]\setcounter{enumi}{0}
    \item Является ли это отношение отношением эквивалентности?
    \item Является ли это отношение функциональным?
    \item Каким из отношений соответствия (одно-многозначным, много-многозначный и т.д.) оно является?
    \item К каким из отношений порядка (полного, частичного и т.д.) можно отнести данное отношение?
\end{enumerate}

\question
Установите, является ли каждое из перечисленных ниже отношений на А ($R \subseteq A \times A$) отношением эквивалентности (обоснование ответа обязательно). Для каждого отношения эквивалентности постройте классы 
эквивалентности и постройте граф отношения:
\begin{enumerate} [a)]\setcounter{enumi}{0}
\item На множестве $A = \{1; 2; 3\}$ задано отношение $R = \{(1; 1); (2; 2); (3; 3); (2; 1); (1; 2); (2; 3); (3; 2); (3; 1); (1; 3)\}$
\item На множестве $A = \{1; 2; 3; 4; 5\}$ задано отношение $R = \{(1; 2); (1; 3); (1; 5); (2; 3); (2; 4); (2; 5); (3; 4); (3; 5); (4; 5)\}$
\item А - множество целых чисел и отношение $R = \{(a,b)|a + b = 0\}$
\end{enumerate}\question Составьте полную таблицу истинности, определите, какие переменные являются фиктивными и проверьте, является ли формула тавтологией:
$(( P \land \neg Q) \rightarrow (R \land \neg R)) \rightarrow (P \rightarrow Q)$

\end{questions}
\newpage
%%% begin test
\begin{flushright}
\begin{tabular}{p{2.8in} r l}
%\textbf{\class} & \textbf{ФИО:} & \makebox[2.5in]{\hrulefill}\\
\textbf{\class} & \textbf{ФИО:} &Слюсаренко Сергей Владимирович
\\

\textbf{\examdate} &&\\
%\textbf{Time Limit: \timelimit} & Teaching Assistant & \makebox[2in]{\hrulefill}
\end{tabular}\\
\end{flushright}
\rule[1ex]{\textwidth}{.1pt}


\begin{questions}
\question
Найдите и упростите P:
\begin{equation*}
\overline{P} = A \cap \overline{C} \cup A \cap \overline{B} \cup B \cap \overline{C} \cup A \cap C
\end{equation*}
Затем найдите элементы множества P, выраженного через множества:
\begin{equation*}
A = \{0, 3, 4, 9\}; 
B = \{1, 3, 4, 7\};
C = \{0, 1, 2, 4, 7, 8, 9\};
I = \{0, 1, 2, 3, 4, 5, 6, 7, 8, 9\}.
\end{equation*}\question
Упростите следующее выражение с учетом того, что $A\subset B \subset C \subset D \subset U; A \neq \O$
\begin{equation*}
A \cap B \cup \overline{A} \cap \overline{C} \cup A \cap C \cup \overline{B} \cap \overline{C}
\end{equation*}

Примечание: U — универсум\question
Дано отношение на множестве $\{1, 2, 3, 4, 5\}$ 
\begin{equation*}
aRb \iff |a-b| = 1
\end{equation*}
Напишите обоснованный ответ какими свойствами обладает или не обладает отношение и почему:   
\begin{enumerate} [a)]\setcounter{enumi}{0}
\item рефлексивность
\item антирефлексивность
\item симметричность
\item асимметричность
\item антисимметричность
\item транзитивность
\end{enumerate}

Обоснуйте свой ответ по каждому из приведенных ниже вопросов:
\begin{enumerate} [a)]\setcounter{enumi}{0}
    \item Является ли это отношение отношением эквивалентности?
    \item Является ли это отношение функциональным?
    \item Каким из отношений соответствия (одно-многозначным, много-многозначный и т.д.) оно является?
    \item К каким из отношений порядка (полного, частичного и т.д.) можно отнести данное отношение?
\end{enumerate}

\question
Установите, является ли каждое из перечисленных ниже отношений на А ($R \subseteq A \times A$) отношением эквивалентности (обоснование ответа обязательно). Для каждого отношения эквивалентности постройте классы 
эквивалентности и постройте граф отношения:
\begin{enumerate} [a)]\setcounter{enumi}{0}
\item Пусть A – множество имен. $A = \{ $Алексей, Иван, Петр, Александр, Павел, Андрей$ \}$. Тогда отношение $R$ верно на парах имен, начинающихся с одной и той же буквы, и только на них.
\item $A = \{-10, -9, … , 9, 10\}$ и отношение $ R = \{(a,b)|a^{2} = b^{2}\}$
\item На множестве $A = \{1; 2; 3\}$ задано отношение $R = \{(1; 1); (2; 2); (3; 3); (3; 2); (1; 2); (2; 1)\}$
\end{enumerate}\question Составьте полную таблицу истинности, определите, какие переменные являются фиктивными и проверьте, является ли формула тавтологией:
$(P \rightarrow (Q \rightarrow R)) \rightarrow ((P \rightarrow Q) \rightarrow (P \rightarrow R))$

\end{questions}
\newpage
%%% begin test
\begin{flushright}
\begin{tabular}{p{2.8in} r l}
%\textbf{\class} & \textbf{ФИО:} & \makebox[2.5in]{\hrulefill}\\
\textbf{\class} & \textbf{ФИО:} &Степанов Илья Алексеевич
\\

\textbf{\examdate} &&\\
%\textbf{Time Limit: \timelimit} & Teaching Assistant & \makebox[2in]{\hrulefill}
\end{tabular}\\
\end{flushright}
\rule[1ex]{\textwidth}{.1pt}


\begin{questions}
\question
Найдите и упростите P:
\begin{equation*}
\overline{P} = A \cap \overline{C} \cup A \cap \overline{B} \cup B \cap \overline{C} \cup A \cap C
\end{equation*}
Затем найдите элементы множества P, выраженного через множества:
\begin{equation*}
A = \{0, 3, 4, 9\}; 
B = \{1, 3, 4, 7\};
C = \{0, 1, 2, 4, 7, 8, 9\};
I = \{0, 1, 2, 3, 4, 5, 6, 7, 8, 9\}.
\end{equation*}\question
Упростите следующее выражение с учетом того, что $A\subset B \subset C \subset D \subset U; A \neq \O$
\begin{equation*}
A \cap C  \cap D \cup B \cap \overline{C} \cap D \cup B \cap C \cap D
\end{equation*}

Примечание: U — универсум\question
Дано отношение на множестве $\{1, 2, 3, 4, 5\}$ 
\begin{equation*}
aRb \iff  \text{НОД}(a,b) =1
\end{equation*}
Напишите обоснованный ответ какими свойствами обладает или не обладает отношение и почему:   
\begin{enumerate} [a)]\setcounter{enumi}{0}
\item рефлексивность
\item антирефлексивность
\item симметричность
\item асимметричность
\item антисимметричность
\item транзитивность
\end{enumerate}

Обоснуйте свой ответ по каждому из приведенных ниже вопросов:
\begin{enumerate} [a)]\setcounter{enumi}{0}
    \item Является ли это отношение отношением эквивалентности?
    \item Является ли это отношение функциональным?
    \item Каким из отношений соответствия (одно-многозначным, много-многозначный и т.д.) оно является?
    \item К каким из отношений порядка (полного, частичного и т.д.) можно отнести данное отношение?
\end{enumerate}


\question
Установите, является ли каждое из перечисленных ниже отношений на А ($R \subseteq A \times A$) отношением эквивалентности (обоснование ответа обязательно). Для каждого отношения эквивалентности постройте классы 
эквивалентности и постройте граф отношения:
\begin{enumerate} [a)]\setcounter{enumi}{0}
\item На множестве $A = \{1; 2; 3\}$ задано отношение $R = \{(1; 1); (2; 2); (3; 3); (2; 1); (1; 2); (2; 3); (3; 2); (3; 1); (1; 3)\}$
\item На множестве $A = \{1; 2; 3; 4; 5\}$ задано отношение $R = \{(1; 2); (1; 3); (1; 5); (2; 3); (2; 4); (2; 5); (3; 4); (3; 5); (4; 5)\}$
\item А - множество целых чисел и отношение $R = \{(a,b)|a + b = 0\}$
\end{enumerate}\question Составьте полную таблицу истинности, определите, какие переменные являются фиктивными и проверьте, является ли формула тавтологией:

$(P \rightarrow (Q \land R)) \leftrightarrow ((P \rightarrow Q) \land (P \rightarrow R))$

\end{questions}
\newpage
%%% begin test
\begin{flushright}
\begin{tabular}{p{2.8in} r l}
%\textbf{\class} & \textbf{ФИО:} & \makebox[2.5in]{\hrulefill}\\
\textbf{\class} & \textbf{ФИО:} &Тетерина Мария Олеговна
\\

\textbf{\examdate} &&\\
%\textbf{Time Limit: \timelimit} & Teaching Assistant & \makebox[2in]{\hrulefill}
\end{tabular}\\
\end{flushright}
\rule[1ex]{\textwidth}{.1pt}


\begin{questions}
\question
Найдите и упростите P:
\begin{equation*}
\overline{P} = A \cap \overline{B} \cup \overline{B} \cap C \cup \overline{A} \cap \overline{B} \cup \overline{A} \cap C
\end{equation*}
Затем найдите элементы множества P, выраженного через множества:
\begin{equation*}
A = \{0, 3, 4, 9\}; 
B = \{1, 3, 4, 7\};
C = \{0, 1, 2, 4, 7, 8, 9\};
I = \{0, 1, 2, 3, 4, 5, 6, 7, 8, 9\}.
\end{equation*}\question
Упростите следующее выражение с учетом того, что $A\subset B \subset C \subset D \subset U; A \neq \O$
\begin{equation*}
\overline{B} \cap \overline{C} \cap D \cup \overline{A} \cap \overline{C} \cap D \cup \overline{A} \cap B
\end{equation*}

Примечание: U — универсум\question
Дано отношение на множестве $\{1, 2, 3, 4, 5\}$ 
\begin{equation*}
aRb \iff  \text{НОД}(a,b) =1
\end{equation*}
Напишите обоснованный ответ какими свойствами обладает или не обладает отношение и почему:   
\begin{enumerate} [a)]\setcounter{enumi}{0}
\item рефлексивность
\item антирефлексивность
\item симметричность
\item асимметричность
\item антисимметричность
\item транзитивность
\end{enumerate}

Обоснуйте свой ответ по каждому из приведенных ниже вопросов:
\begin{enumerate} [a)]\setcounter{enumi}{0}
    \item Является ли это отношение отношением эквивалентности?
    \item Является ли это отношение функциональным?
    \item Каким из отношений соответствия (одно-многозначным, много-многозначный и т.д.) оно является?
    \item К каким из отношений порядка (полного, частичного и т.д.) можно отнести данное отношение?
\end{enumerate}


\question
Установите, является ли каждое из перечисленных ниже отношений на А ($R \subseteq A \times A$) отношением эквивалентности (обоснование ответа обязательно). Для каждого отношения эквивалентности постройте классы 
эквивалентности и постройте граф отношения:
\begin{enumerate} [a)]\setcounter{enumi}{0}
\item На множестве $A = \{1; 2; 3\}$ задано отношение $R = \{(1; 1); (2; 2); (3; 3); (2; 1); (1; 2); (2; 3); (3; 2); (3; 1); (1; 3)\}$
\item На множестве $A = \{1; 2; 3; 4; 5\}$ задано отношение $R = \{(1; 2); (1; 3); (1; 5); (2; 3); (2; 4); (2; 5); (3; 4); (3; 5); (4; 5)\}$
\item А - множество целых чисел и отношение $R = \{(a,b)|a + b = 0\}$
\end{enumerate}\question Составьте полную таблицу истинности, определите, какие переменные являются фиктивными и проверьте, является ли формула тавтологией:
$((P \rightarrow Q) \lor R) \leftrightarrow (P \rightarrow (Q \lor R))$

\end{questions}
\newpage
%%% begin test
\begin{flushright}
\begin{tabular}{p{2.8in} r l}
%\textbf{\class} & \textbf{ФИО:} & \makebox[2.5in]{\hrulefill}\\
\textbf{\class} & \textbf{ФИО:} &Хафизов Александр Олегович
\\

\textbf{\examdate} &&\\
%\textbf{Time Limit: \timelimit} & Teaching Assistant & \makebox[2in]{\hrulefill}
\end{tabular}\\
\end{flushright}
\rule[1ex]{\textwidth}{.1pt}


\begin{questions}
\question
Найдите и упростите P:
\begin{equation*}
\overline{P} = B \cap \overline{C} \cup A \cap B \cup \overline{A} \cap C \cup \overline{A} \cap B
\end{equation*}
Затем найдите элементы множества P, выраженного через множества:
\begin{equation*}
A = \{0, 3, 4, 9\}; 
B = \{1, 3, 4, 7\};
C = \{0, 1, 2, 4, 7, 8, 9\};
I = \{0, 1, 2, 3, 4, 5, 6, 7, 8, 9\}.
\end{equation*}\question
Упростите следующее выражение с учетом того, что $A\subset B \subset C \subset D \subset U; A \neq \O$
\begin{equation*}
A \cap B  \cap \overline{C} \cup \overline{C} \cap D \cup B \cap C \cap D
\end{equation*}

Примечание: U — универсум\question
Дано отношение на множестве $\{1, 2, 3, 4, 5\}$ 
\begin{equation*}
aRb \iff (a+b) \bmod 2 =0
\end{equation*}
Напишите обоснованный ответ какими свойствами обладает или не обладает отношение и почему:   
\begin{enumerate} [a)]\setcounter{enumi}{0}
\item рефлексивность
\item антирефлексивность
\item симметричность
\item асимметричность
\item антисимметричность
\item транзитивность
\end{enumerate}

Обоснуйте свой ответ по каждому из приведенных ниже вопросов:
\begin{enumerate} [a)]\setcounter{enumi}{0}
    \item Является ли это отношение отношением эквивалентности?
    \item Является ли это отношение функциональным?
    \item Каким из отношений соответствия (одно-многозначным, много-многозначный и т.д.) оно является?
    \item К каким из отношений порядка (полного, частичного и т.д.) можно отнести данное отношение?
\end{enumerate}



\question
Установите, является ли каждое из перечисленных ниже отношений на А ($R \subseteq A \times A$) отношением эквивалентности (обоснование ответа обязательно). Для каждого отношения эквивалентности постройте классы эквивалентности и постройте граф отношения:
\begin{enumerate} [a)]\setcounter{enumi}{0}
\item $F(x)=x^{2}+1$, где $x \in A = [-2, 4]$ и отношение $R = \{(a,b)|F(a) = F(b)\}$
\item А - множество целых чисел и отношение $R = \{(a,b)|a + b = 5\}$
\item На множестве $A = \{1; 2; 3\}$ задано отношение $R = \{(1; 1); (2; 2); (3; 3); (3; 2); (1; 2); (2; 1)\}$

\end{enumerate}\question Составьте полную таблицу истинности, определите, какие переменные являются фиктивными и проверьте, является ли формула тавтологией:
$ P \rightarrow (Q \rightarrow ((P \lor Q) \rightarrow (P \land Q)))$

\end{questions}
\newpage
%%% begin test
\begin{flushright}
\begin{tabular}{p{2.8in} r l}
%\textbf{\class} & \textbf{ФИО:} & \makebox[2.5in]{\hrulefill}\\
\textbf{\class} & \textbf{ФИО:} &Холопов Денис Сергеевич
\\

\textbf{\examdate} &&\\
%\textbf{Time Limit: \timelimit} & Teaching Assistant & \makebox[2in]{\hrulefill}
\end{tabular}\\
\end{flushright}
\rule[1ex]{\textwidth}{.1pt}


\begin{questions}
\question
Найдите и упростите P:
\begin{equation*}
\overline{P} = A \cap \overline{B} \cup \overline{B} \cap C \cup \overline{A} \cap \overline{B} \cup \overline{A} \cap C
\end{equation*}
Затем найдите элементы множества P, выраженного через множества:
\begin{equation*}
A = \{0, 3, 4, 9\}; 
B = \{1, 3, 4, 7\};
C = \{0, 1, 2, 4, 7, 8, 9\};
I = \{0, 1, 2, 3, 4, 5, 6, 7, 8, 9\}.
\end{equation*}\question
Упростите следующее выражение с учетом того, что $A\subset B \subset C \subset D \subset U; A \neq \O$
\begin{equation*}
A \cap  \overline{C} \cup B \cap \overline{D} \cup  \overline{A} \cap C \cap  \overline{D}
\end{equation*}

Примечание: U — универсум\question
Дано отношение на множестве $\{1, 2, 3, 4, 5\}$ 
\begin{equation*}
aRb \iff |a-b| = 1
\end{equation*}
Напишите обоснованный ответ какими свойствами обладает или не обладает отношение и почему:   
\begin{enumerate} [a)]\setcounter{enumi}{0}
\item рефлексивность
\item антирефлексивность
\item симметричность
\item асимметричность
\item антисимметричность
\item транзитивность
\end{enumerate}

Обоснуйте свой ответ по каждому из приведенных ниже вопросов:
\begin{enumerate} [a)]\setcounter{enumi}{0}
    \item Является ли это отношение отношением эквивалентности?
    \item Является ли это отношение функциональным?
    \item Каким из отношений соответствия (одно-многозначным, много-многозначный и т.д.) оно является?
    \item К каким из отношений порядка (полного, частичного и т.д.) можно отнести данное отношение?
\end{enumerate}

\question
Установите, является ли каждое из перечисленных ниже отношений на А ($R \subseteq A \times A$) отношением эквивалентности (обоснование ответа обязательно). Для каждого отношения эквивалентности 
постройте классы эквивалентности и постройте граф отношения:
\begin{enumerate}[a)]\setcounter{enumi}{0}
\item А - множество целых чисел и отношение $R = \{(a,b)|a + b = 0\}$
\item $A = \{-10, -9, …, 9, 10\}$ и отношение $R = \{(a,b)|a^{3} = b^{3}\}$
\item На множестве $A = \{1; 2; 3\}$ задано отношение $R = \{(1; 1); (2; 2); (3; 3); (2; 1); (1; 2); (2; 3); (3; 2); (3; 1); (1; 3)\}$

\end{enumerate}\question Составьте полную таблицу истинности, определите, какие переменные являются фиктивными и проверьте, является ли формула тавтологией:
$(P \rightarrow (Q \rightarrow R)) \rightarrow ((P \rightarrow Q) \rightarrow (P \rightarrow R))$

\end{questions}
\newpage
%%% begin test
\begin{flushright}
\begin{tabular}{p{2.8in} r l}
%\textbf{\class} & \textbf{ФИО:} & \makebox[2.5in]{\hrulefill}\\
\textbf{\class} & \textbf{ФИО:} &Хряков Иван Валентинович
\\

\textbf{\examdate} &&\\
%\textbf{Time Limit: \timelimit} & Teaching Assistant & \makebox[2in]{\hrulefill}
\end{tabular}\\
\end{flushright}
\rule[1ex]{\textwidth}{.1pt}


\begin{questions}
\question
Найдите и упростите P:
\begin{equation*}
\overline{P} = \overline{A} \cap B \cup \overline{A} \cap C \cup A \cap \overline{B} \cup \overline{B} \cap C
\end{equation*}
Затем найдите элементы множества P, выраженного через множества:
\begin{equation*}
A = \{0, 3, 4, 9\}; 
B = \{1, 3, 4, 7\};
C = \{0, 1, 2, 4, 7, 8, 9\};
I = \{0, 1, 2, 3, 4, 5, 6, 7, 8, 9\}.
\end{equation*}\question
Упростите следующее выражение с учетом того, что $A\subset B \subset C \subset D \subset U; A \neq \O$
\begin{equation*}
\overline{A} \cap \overline{B} \cup B \cap \overline{C} \cup \overline{C} \cap D
\end{equation*}

Примечание: U — универсум\question
Дано отношение на множестве $\{1, 2, 3, 4, 5\}$ 
\begin{equation*}
aRb \iff  \text{НОД}(a,b) =1
\end{equation*}
Напишите обоснованный ответ какими свойствами обладает или не обладает отношение и почему:   
\begin{enumerate} [a)]\setcounter{enumi}{0}
\item рефлексивность
\item антирефлексивность
\item симметричность
\item асимметричность
\item антисимметричность
\item транзитивность
\end{enumerate}

Обоснуйте свой ответ по каждому из приведенных ниже вопросов:
\begin{enumerate} [a)]\setcounter{enumi}{0}
    \item Является ли это отношение отношением эквивалентности?
    \item Является ли это отношение функциональным?
    \item Каким из отношений соответствия (одно-многозначным, много-многозначный и т.д.) оно является?
    \item К каким из отношений порядка (полного, частичного и т.д.) можно отнести данное отношение?
\end{enumerate}


\question
Установите, является ли каждое из перечисленных ниже отношений на А ($R \subseteq A \times A$) отношением эквивалентности (обоснование ответа обязательно). Для каждого отношения эквивалентности постройте классы 
эквивалентности и постройте граф отношения:
\begin{enumerate} [a)]\setcounter{enumi}{0}
\item А - множество целых чисел и отношение $R = \{(a,b)|a + b = 5\}$
\item Пусть A – множество имен. $A = \{ $Алексей, Иван, Петр, Александр, Павел, Андрей$ \}$. Тогда отношение $R $ верно на парах имен, начинающихся с одной и той же буквы, и только на них.
\item На множестве $A = \{1; 2; 3; 4; 5\}$ задано отношение $R = \{(1; 2); (1; 3); (1; 5); (2; 3); (2; 4); (2; 5); (3; 4); (3; 5); (4; 5)\}$
\end{enumerate}\question Составьте полную таблицу истинности, определите, какие переменные являются фиктивными и проверьте, является ли формула тавтологией:
$(( P \land \neg Q) \rightarrow (R \land \neg R)) \rightarrow (P \rightarrow Q)$

\end{questions}
\newpage
%%% begin test
\begin{flushright}
\begin{tabular}{p{2.8in} r l}
%\textbf{\class} & \textbf{ФИО:} & \makebox[2.5in]{\hrulefill}\\
\textbf{\class} & \textbf{ФИО:} &Шатинский Григорий Сергеевич
\\

\textbf{\examdate} &&\\
%\textbf{Time Limit: \timelimit} & Teaching Assistant & \makebox[2in]{\hrulefill}
\end{tabular}\\
\end{flushright}
\rule[1ex]{\textwidth}{.1pt}


\begin{questions}
\question
Найдите и упростите P:
\begin{equation*}
\overline{P} = A \cap C \cup \overline{A} \cap \overline{C} \cup \overline{B} \cap C \cup \overline{A} \cap \overline{B}
\end{equation*}
Затем найдите элементы множества P, выраженного через множества:
\begin{equation*}
A = \{0, 3, 4, 9\}; 
B = \{1, 3, 4, 7\};
C = \{0, 1, 2, 4, 7, 8, 9\};
I = \{0, 1, 2, 3, 4, 5, 6, 7, 8, 9\}.
\end{equation*}\question
Упростите следующее выражение с учетом того, что $A\subset B \subset C \subset D \subset U; A \neq \O$
\begin{equation*}
A \cap  \overline{C} \cup B \cap \overline{D} \cup  \overline{A} \cap C \cap  \overline{D}
\end{equation*}

Примечание: U — универсум\question
Для следующего отношения на множестве $\{1, 2, 3, 4, 5\}$ 
\begin{equation*}
aRb \iff 0 < a-b<2
\end{equation*}
Напишите обоснованный ответ какими свойствами обладает или не обладает отношение и почему:   
\begin{enumerate} [a)]\setcounter{enumi}{0}
\item рефлексивность
\item антирефлексивность
\item симметричность
\item асимметричность
\item антисимметричность
\item транзитивность
\end{enumerate}

Обоснуйте свой ответ по каждому из приведенных ниже вопросов:
\begin{enumerate} [a)]\setcounter{enumi}{0}
    \item Является ли это отношение отношением эквивалентности?
    \item Является ли это отношение функциональным?
    \item Каким из отношений соответствия (одно-многозначным, много-многозначный и т.д.) оно является?
    \item К каким из отношений порядка (полного, частичного и т.д.) можно отнести данное отношение?
\end{enumerate}
\question
Установите, является ли каждое из перечисленных ниже отношений на А ($R \subseteq A \times A$) отношением эквивалентности (обоснование ответа обязательно). Для каждого отношения эквивалентности постройте классы 
эквивалентности и постройте граф отношения:
\begin{enumerate} [a)]\setcounter{enumi}{0}
\item А - множество целых чисел и отношение $R = \{(a,b)|a + b = 5\}$
\item Пусть A – множество имен. $A = \{ $Алексей, Иван, Петр, Александр, Павел, Андрей$ \}$. Тогда отношение $R $ верно на парах имен, начинающихся с одной и той же буквы, и только на них.
\item На множестве $A = \{1; 2; 3; 4; 5\}$ задано отношение $R = \{(1; 2); (1; 3); (1; 5); (2; 3); (2; 4); (2; 5); (3; 4); (3; 5); (4; 5)\}$
\end{enumerate}\question Составьте полную таблицу истинности, определите, какие переменные являются фиктивными и проверьте, является ли формула тавтологией:

$(P \rightarrow (Q \land R)) \leftrightarrow ((P \rightarrow Q) \land (P \rightarrow R))$

\end{questions}
\newpage
%%% begin test
\begin{flushright}
\begin{tabular}{p{2.8in} r l}
%\textbf{\class} & \textbf{ФИО:} & \makebox[2.5in]{\hrulefill}\\
\textbf{\class} & \textbf{ФИО:} &Шевченко Валерий Владимирович
\\

\textbf{\examdate} &&\\
%\textbf{Time Limit: \timelimit} & Teaching Assistant & \makebox[2in]{\hrulefill}
\end{tabular}\\
\end{flushright}
\rule[1ex]{\textwidth}{.1pt}


\begin{questions}
\question
Найдите и упростите P:
\begin{equation*}
\overline{P} = A \cap \overline{C} \cup A \cap \overline{B} \cup B \cap \overline{C} \cup A \cap C
\end{equation*}
Затем найдите элементы множества P, выраженного через множества:
\begin{equation*}
A = \{0, 3, 4, 9\}; 
B = \{1, 3, 4, 7\};
C = \{0, 1, 2, 4, 7, 8, 9\};
I = \{0, 1, 2, 3, 4, 5, 6, 7, 8, 9\}.
\end{equation*}\question
Упростите следующее выражение с учетом того, что $A\subset B \subset C \subset D \subset U; A \neq \O$
\begin{equation*}
\overline{A} \cap \overline{C} \cap D \cup \overline{B} \cap \overline{C} \cap D \cup A \cap B
\end{equation*}

Примечание: U — универсум\question
Дано отношение на множестве $\{1, 2, 3, 4, 5\}$ 
\begin{equation*}
aRb \iff  \text{НОД}(a,b) =1
\end{equation*}
Напишите обоснованный ответ какими свойствами обладает или не обладает отношение и почему:   
\begin{enumerate} [a)]\setcounter{enumi}{0}
\item рефлексивность
\item антирефлексивность
\item симметричность
\item асимметричность
\item антисимметричность
\item транзитивность
\end{enumerate}

Обоснуйте свой ответ по каждому из приведенных ниже вопросов:
\begin{enumerate} [a)]\setcounter{enumi}{0}
    \item Является ли это отношение отношением эквивалентности?
    \item Является ли это отношение функциональным?
    \item Каким из отношений соответствия (одно-многозначным, много-многозначный и т.д.) оно является?
    \item К каким из отношений порядка (полного, частичного и т.д.) можно отнести данное отношение?
\end{enumerate}


\question
Установите, является ли каждое из перечисленных ниже отношений на А ($R \subseteq A \times A$) отношением эквивалентности (обоснование ответа обязательно). Для каждого отношения эквивалентности постройте классы 
эквивалентности и постройте граф отношения:
\begin{enumerate} [a)]\setcounter{enumi}{0}
\item На множестве $A = \{1; 2; 3\}$ задано отношение $R = \{(1; 1); (2; 2); (3; 3); (2; 1); (1; 2); (2; 3); (3; 2); (3; 1); (1; 3)\}$
\item На множестве $A = \{1; 2; 3; 4; 5\}$ задано отношение $R = \{(1; 2); (1; 3); (1; 5); (2; 3); (2; 4); (2; 5); (3; 4); (3; 5); (4; 5)\}$
\item А - множество целых чисел и отношение $R = \{(a,b)|a + b = 0\}$
\end{enumerate}\question Составьте полную таблицу истинности, определите, какие переменные являются фиктивными и проверьте, является ли формула тавтологией:
$(P \rightarrow (Q \rightarrow R)) \rightarrow ((P \rightarrow Q) \rightarrow (P \rightarrow R))$

\end{questions}
\newpage
%%% begin test
\begin{flushright}
\begin{tabular}{p{2.8in} r l}
%\textbf{\class} & \textbf{ФИО:} & \makebox[2.5in]{\hrulefill}\\
\textbf{\class} & \textbf{ФИО:} &Шуляк Георгий Владимирович
\\

\textbf{\examdate} &&\\
%\textbf{Time Limit: \timelimit} & Teaching Assistant & \makebox[2in]{\hrulefill}
\end{tabular}\\
\end{flushright}
\rule[1ex]{\textwidth}{.1pt}


\begin{questions}
\question
Найдите и упростите P:
\begin{equation*}
\overline{P} = A \cap \overline{C} \cup A \cap \overline{B} \cup B \cap \overline{C} \cup A \cap C
\end{equation*}
Затем найдите элементы множества P, выраженного через множества:
\begin{equation*}
A = \{0, 3, 4, 9\}; 
B = \{1, 3, 4, 7\};
C = \{0, 1, 2, 4, 7, 8, 9\};
I = \{0, 1, 2, 3, 4, 5, 6, 7, 8, 9\}.
\end{equation*}\question
Упростите следующее выражение с учетом того, что $A\subset B \subset C \subset D \subset U; A \neq \O$
\begin{equation*}
A \cap B \cup \overline{A} \cap \overline{C} \cup A \cap C \cup \overline{B} \cap \overline{C}
\end{equation*}

Примечание: U — универсум\question
Дано отношение на множестве $\{1, 2, 3, 4, 5\}$ 
\begin{equation*}
aRb \iff b > a
\end{equation*}
Напишите обоснованный ответ какими свойствами обладает или не обладает отношение и почему:   
\begin{enumerate} [a)]\setcounter{enumi}{0}
\item рефлексивность
\item антирефлексивность
\item симметричность
\item асимметричность
\item антисимметричность
\item транзитивность
\end{enumerate}

Обоснуйте свой ответ по каждому из приведенных ниже вопросов:
\begin{enumerate} [a)]\setcounter{enumi}{0}
    \item Является ли это отношение отношением эквивалентности?
    \item Является ли это отношение функциональным?
    \item Каким из отношений соответствия (одно-многозначным, много-многозначный и т.д.) оно является?
    \item К каким из отношений порядка (полного, частичного и т.д.) можно отнести данное отношение?
\end{enumerate}

\question
Установите, является ли каждое из перечисленных ниже отношений на А ($R \subseteq A \times A$) отношением эквивалентности (обоснование ответа обязательно). Для каждого отношения эквивалентности постройте классы 
эквивалентности и постройте граф отношения:
\begin{enumerate} [a)]\setcounter{enumi}{0}
\item $A = \{-10, -9, … , 9, 10\}$ и отношение $R = \{(a,b)|a^{2} = b^{2}\}$
\item $A = \{a, b, c, d, p, t\}$ задано отношение $R = \{(a, a), (b, b), (b, c), (b, d), (c, b), (c, c), (c, d), (d, b), (d, c), (d, d), (p,p), (t,t)\}$
\item Пусть A – множество имен. $A = \{ $Алексей, Иван, Петр, Александр, Павел, Андрей$ \}$. Тогда отношение $R$ верно на парах имен, начинающихся с одной и той же буквы, и только на них.
\end{enumerate}\question Составьте полную таблицу истинности, определите, какие переменные являются фиктивными и проверьте, является ли формула тавтологией:
$ P \rightarrow (Q \rightarrow ((P \lor Q) \rightarrow (P \land Q)))$

\end{questions}
\newpage
%%% begin test
\begin{flushright}
\begin{tabular}{p{2.8in} r l}
%\textbf{\class} & \textbf{ФИО:} & \makebox[2.5in]{\hrulefill}\\
\textbf{\class} & \textbf{ФИО:} &Энкеев Баир Энкеевич
\\

\textbf{\examdate} &&\\
%\textbf{Time Limit: \timelimit} & Teaching Assistant & \makebox[2in]{\hrulefill}
\end{tabular}\\
\end{flushright}
\rule[1ex]{\textwidth}{.1pt}


\begin{questions}
\question
Найдите и упростите P:
\begin{equation*}
\overline{P} = A \cap B \cup \overline{A} \cap \overline{B} \cup A \cap C \cup \overline{B} \cap C
\end{equation*}
Затем найдите элементы множества P, выраженного через множества:
\begin{equation*}
A = \{0, 3, 4, 9\}; 
B = \{1, 3, 4, 7\};
C = \{0, 1, 2, 4, 7, 8, 9\};
I = \{0, 1, 2, 3, 4, 5, 6, 7, 8, 9\}.
\end{equation*}\question
Упростите следующее выражение с учетом того, что $A\subset B \subset C \subset D \subset U; A \neq \O$
\begin{equation*}
\overline{A} \cap \overline{B} \cup B \cap \overline{C} \cup \overline{C} \cap D
\end{equation*}

Примечание: U — универсум\question
Дано отношение на множестве $\{1, 2, 3, 4, 5\}$ 
\begin{equation*}
aRb \iff  \text{НОД}(a,b) =1
\end{equation*}
Напишите обоснованный ответ какими свойствами обладает или не обладает отношение и почему:   
\begin{enumerate} [a)]\setcounter{enumi}{0}
\item рефлексивность
\item антирефлексивность
\item симметричность
\item асимметричность
\item антисимметричность
\item транзитивность
\end{enumerate}

Обоснуйте свой ответ по каждому из приведенных ниже вопросов:
\begin{enumerate} [a)]\setcounter{enumi}{0}
    \item Является ли это отношение отношением эквивалентности?
    \item Является ли это отношение функциональным?
    \item Каким из отношений соответствия (одно-многозначным, много-многозначный и т.д.) оно является?
    \item К каким из отношений порядка (полного, частичного и т.д.) можно отнести данное отношение?
\end{enumerate}


\question
Установите, является ли каждое из перечисленных ниже отношений на А ($R \subseteq A \times A$) отношением эквивалентности (обоснование ответа обязательно). Для каждого отношения эквивалентности постройте классы 
эквивалентности и постройте граф отношения:
\begin{enumerate} [a)]\setcounter{enumi}{0}
\item $A = \{-10, -9, … , 9, 10\}$ и отношение $R = \{(a,b)|a^{2} = b^{2}\}$
\item $A = \{a, b, c, d, p, t\}$ задано отношение $R = \{(a, a), (b, b), (b, c), (b, d), (c, b), (c, c), (c, d), (d, b), (d, c), (d, d), (p,p), (t,t)\}$
\item Пусть A – множество имен. $A = \{ $Алексей, Иван, Петр, Александр, Павел, Андрей$ \}$. Тогда отношение $R$ верно на парах имен, начинающихся с одной и той же буквы, и только на них.
\end{enumerate}\question Составьте полную таблицу истинности, определите, какие переменные являются фиктивными и проверьте, является ли формула тавтологией:
$((P \rightarrow Q) \lor R) \leftrightarrow (P \rightarrow (Q \lor R))$

\end{questions}
\newpage
%%% begin test
\begin{flushright}
\begin{tabular}{p{2.8in} r l}
%\textbf{\class} & \textbf{ФИО:} & \makebox[2.5in]{\hrulefill}\\
\textbf{\class} & \textbf{ФИО:} &М3104
\\

\textbf{\examdate} &&\\
%\textbf{Time Limit: \timelimit} & Teaching Assistant & \makebox[2in]{\hrulefill}
\end{tabular}\\
\end{flushright}
\rule[1ex]{\textwidth}{.1pt}


\begin{questions}
\question
Найдите и упростите P:
\begin{equation*}
\overline{P} = \overline{A} \cap B \cup \overline{A} \cap C \cup A \cap \overline{B} \cup \overline{B} \cap C
\end{equation*}
Затем найдите элементы множества P, выраженного через множества:
\begin{equation*}
A = \{0, 3, 4, 9\}; 
B = \{1, 3, 4, 7\};
C = \{0, 1, 2, 4, 7, 8, 9\};
I = \{0, 1, 2, 3, 4, 5, 6, 7, 8, 9\}.
\end{equation*}\question
Упростите следующее выражение с учетом того, что $A\subset B \subset C \subset D \subset U; A \neq \O$
\begin{equation*}
A \cap  \overline{C} \cup B \cap \overline{D} \cup  \overline{A} \cap C \cap  \overline{D}
\end{equation*}

Примечание: U — универсум\question
Дано отношение на множестве $\{1, 2, 3, 4, 5\}$ 
\begin{equation*}
aRb \iff a \leq b
\end{equation*}
Напишите обоснованный ответ какими свойствами обладает или не обладает отношение и почему:   
\begin{enumerate} [a)]\setcounter{enumi}{0}
\item рефлексивность
\item антирефлексивность
\item симметричность
\item асимметричность
\item антисимметричность
\item транзитивность
\end{enumerate}

Обоснуйте свой ответ по каждому из приведенных ниже вопросов:
\begin{enumerate} [a)]\setcounter{enumi}{0}
    \item Является ли это отношение отношением эквивалентности?
    \item Является ли это отношение функциональным?
    \item Каким из отношений соответствия (одно-многозначным, много-многозначный и т.д.) оно является?
    \item К каким из отношений порядка (полного, частичного и т.д.) можно отнести данное отношение?
\end{enumerate}


\question
Установите, является ли каждое из перечисленных ниже отношений на А ($R \subseteq A \times A$) отношением эквивалентности (обоснование ответа обязательно). Для каждого отношения эквивалентности постройте классы 
эквивалентности и постройте граф отношения:
\begin{enumerate} [a)]\setcounter{enumi}{0}
\item На множестве $A = \{1; 2; 3\}$ задано отношение $R = \{(1; 1); (2; 2); (3; 3); (2; 1); (1; 2); (2; 3); (3; 2); (3; 1); (1; 3)\}$
\item На множестве $A = \{1; 2; 3; 4; 5\}$ задано отношение $R = \{(1; 2); (1; 3); (1; 5); (2; 3); (2; 4); (2; 5); (3; 4); (3; 5); (4; 5)\}$
\item А - множество целых чисел и отношение $R = \{(a,b)|a + b = 0\}$
\end{enumerate}\question Составьте полную таблицу истинности, определите, какие переменные являются фиктивными и проверьте, является ли формула тавтологией:
$ P \rightarrow (Q \rightarrow ((P \lor Q) \rightarrow (P \land Q)))$

\end{questions}
\newpage
%%% begin test
\begin{flushright}
\begin{tabular}{p{2.8in} r l}
%\textbf{\class} & \textbf{ФИО:} & \makebox[2.5in]{\hrulefill}\\
\textbf{\class} & \textbf{ФИО:} &Александров Даниил Евгеньевич
\\

\textbf{\examdate} &&\\
%\textbf{Time Limit: \timelimit} & Teaching Assistant & \makebox[2in]{\hrulefill}
\end{tabular}\\
\end{flushright}
\rule[1ex]{\textwidth}{.1pt}


\begin{questions}
\question
Найдите и упростите P:
\begin{equation*}
\overline{P} = A \cap \overline{C} \cup A \cap \overline{B} \cup B \cap \overline{C} \cup A \cap C
\end{equation*}
Затем найдите элементы множества P, выраженного через множества:
\begin{equation*}
A = \{0, 3, 4, 9\}; 
B = \{1, 3, 4, 7\};
C = \{0, 1, 2, 4, 7, 8, 9\};
I = \{0, 1, 2, 3, 4, 5, 6, 7, 8, 9\}.
\end{equation*}\question
Упростите следующее выражение с учетом того, что $A\subset B \subset C \subset D \subset U; A \neq \O$
\begin{equation*}
A \cap C  \cap D \cup B \cap \overline{C} \cap D \cup B \cap C \cap D
\end{equation*}

Примечание: U — универсум\question
Дано отношение на множестве $\{1, 2, 3, 4, 5\}$ 
\begin{equation*}
aRb \iff a \geq b^2
\end{equation*}
Напишите обоснованный ответ какими свойствами обладает или не обладает отношение и почему:   
\begin{enumerate} [a)]\setcounter{enumi}{0}
\item рефлексивность
\item антирефлексивность
\item симметричность
\item асимметричность
\item антисимметричность
\item транзитивность
\end{enumerate}

Обоснуйте свой ответ по каждому из приведенных ниже вопросов:
\begin{enumerate} [a)]\setcounter{enumi}{0}
    \item Является ли это отношение отношением эквивалентности?
    \item Является ли это отношение функциональным?
    \item Каким из отношений соответствия (одно-многозначным, много-многозначный и т.д.) оно является?
    \item К каким из отношений порядка (полного, частичного и т.д.) можно отнести данное отношение?
\end{enumerate}


\question
Установите, является ли каждое из перечисленных ниже отношений на А ($R \subseteq A \times A$) отношением эквивалентности (обоснование ответа обязательно). Для каждого отношения эквивалентности 
постройте классы эквивалентности и постройте граф отношения:
\begin{enumerate}[a)]\setcounter{enumi}{0}
\item А - множество целых чисел и отношение $R = \{(a,b)|a + b = 0\}$
\item $A = \{-10, -9, …, 9, 10\}$ и отношение $R = \{(a,b)|a^{3} = b^{3}\}$
\item На множестве $A = \{1; 2; 3\}$ задано отношение $R = \{(1; 1); (2; 2); (3; 3); (2; 1); (1; 2); (2; 3); (3; 2); (3; 1); (1; 3)\}$

\end{enumerate}\question Составьте полную таблицу истинности, определите, какие переменные являются фиктивными и проверьте, является ли формула тавтологией:
$(P \rightarrow (Q \rightarrow R)) \rightarrow ((P \rightarrow Q) \rightarrow (P \rightarrow R))$

\end{questions}
\newpage
%%% begin test
\begin{flushright}
\begin{tabular}{p{2.8in} r l}
%\textbf{\class} & \textbf{ФИО:} & \makebox[2.5in]{\hrulefill}\\
\textbf{\class} & \textbf{ФИО:} &Антоненко Екатерина Витальевна
\\

\textbf{\examdate} &&\\
%\textbf{Time Limit: \timelimit} & Teaching Assistant & \makebox[2in]{\hrulefill}
\end{tabular}\\
\end{flushright}
\rule[1ex]{\textwidth}{.1pt}


\begin{questions}
\question
Найдите и упростите P:
\begin{equation*}
\overline{P} = B \cap \overline{C} \cup A \cap B \cup \overline{A} \cap C \cup \overline{A} \cap B
\end{equation*}
Затем найдите элементы множества P, выраженного через множества:
\begin{equation*}
A = \{0, 3, 4, 9\}; 
B = \{1, 3, 4, 7\};
C = \{0, 1, 2, 4, 7, 8, 9\};
I = \{0, 1, 2, 3, 4, 5, 6, 7, 8, 9\}.
\end{equation*}\question
Упростите следующее выражение с учетом того, что $A\subset B \subset C \subset D \subset U; A \neq \O$
\begin{equation*}
A \cap B \cup \overline{A} \cap \overline{C} \cup A \cap C \cup \overline{B} \cap \overline{C}
\end{equation*}

Примечание: U — универсум\question
Дано отношение на множестве $\{1, 2, 3, 4, 5\}$ 
\begin{equation*}
aRb \iff  \text{НОД}(a,b) =1
\end{equation*}
Напишите обоснованный ответ какими свойствами обладает или не обладает отношение и почему:   
\begin{enumerate} [a)]\setcounter{enumi}{0}
\item рефлексивность
\item антирефлексивность
\item симметричность
\item асимметричность
\item антисимметричность
\item транзитивность
\end{enumerate}

Обоснуйте свой ответ по каждому из приведенных ниже вопросов:
\begin{enumerate} [a)]\setcounter{enumi}{0}
    \item Является ли это отношение отношением эквивалентности?
    \item Является ли это отношение функциональным?
    \item Каким из отношений соответствия (одно-многозначным, много-многозначный и т.д.) оно является?
    \item К каким из отношений порядка (полного, частичного и т.д.) можно отнести данное отношение?
\end{enumerate}


\question
Установите, является ли каждое из перечисленных ниже отношений на А ($R \subseteq A \times A$) отношением эквивалентности (обоснование ответа обязательно). Для каждого отношения эквивалентности постройте классы 
эквивалентности и постройте граф отношения:
\begin{enumerate} [a)]\setcounter{enumi}{0}
\item А - множество целых чисел и отношение $R = \{(a,b)|a + b = 5\}$
\item Пусть A – множество имен. $A = \{ $Алексей, Иван, Петр, Александр, Павел, Андрей$ \}$. Тогда отношение $R $ верно на парах имен, начинающихся с одной и той же буквы, и только на них.
\item На множестве $A = \{1; 2; 3; 4; 5\}$ задано отношение $R = \{(1; 2); (1; 3); (1; 5); (2; 3); (2; 4); (2; 5); (3; 4); (3; 5); (4; 5)\}$
\end{enumerate}\question Составьте полную таблицу истинности, определите, какие переменные являются фиктивными и проверьте, является ли формула тавтологией:
$((P \rightarrow Q) \lor R) \leftrightarrow (P \rightarrow (Q \lor R))$

\end{questions}
\newpage
%%% begin test
\begin{flushright}
\begin{tabular}{p{2.8in} r l}
%\textbf{\class} & \textbf{ФИО:} & \makebox[2.5in]{\hrulefill}\\
\textbf{\class} & \textbf{ФИО:} &Арсентьев Даниил Геннадьевич
\\

\textbf{\examdate} &&\\
%\textbf{Time Limit: \timelimit} & Teaching Assistant & \makebox[2in]{\hrulefill}
\end{tabular}\\
\end{flushright}
\rule[1ex]{\textwidth}{.1pt}


\begin{questions}
\question
Найдите и упростите P:
\begin{equation*}
\overline{P} = A \cap \overline{B} \cup A \cap C \cup B \cap C \cup \overline{A} \cap C
\end{equation*}
Затем найдите элементы множества P, выраженного через множества:
\begin{equation*}
A = \{0, 3, 4, 9\}; 
B = \{1, 3, 4, 7\};
C = \{0, 1, 2, 4, 7, 8, 9\};
I = \{0, 1, 2, 3, 4, 5, 6, 7, 8, 9\}.
\end{equation*}\question
Упростите следующее выражение с учетом того, что $A\subset B \subset C \subset D \subset U; A \neq \O$
\begin{equation*}
A \cap C  \cap D \cup B \cap \overline{C} \cap D \cup B \cap C \cap D
\end{equation*}

Примечание: U — универсум\question
Дано отношение на множестве $\{1, 2, 3, 4, 5\}$ 
\begin{equation*}
aRb \iff a \leq b
\end{equation*}
Напишите обоснованный ответ какими свойствами обладает или не обладает отношение и почему:   
\begin{enumerate} [a)]\setcounter{enumi}{0}
\item рефлексивность
\item антирефлексивность
\item симметричность
\item асимметричность
\item антисимметричность
\item транзитивность
\end{enumerate}

Обоснуйте свой ответ по каждому из приведенных ниже вопросов:
\begin{enumerate} [a)]\setcounter{enumi}{0}
    \item Является ли это отношение отношением эквивалентности?
    \item Является ли это отношение функциональным?
    \item Каким из отношений соответствия (одно-многозначным, много-многозначный и т.д.) оно является?
    \item К каким из отношений порядка (полного, частичного и т.д.) можно отнести данное отношение?
\end{enumerate}


\question
Установите, является ли каждое из перечисленных ниже отношений на А ($R \subseteq A \times A$) отношением эквивалентности (обоснование ответа обязательно). Для каждого отношения эквивалентности постройте классы 
эквивалентности и постройте граф отношения:
\begin{enumerate} [a)]\setcounter{enumi}{0}
\item $A = \{-10, -9, … , 9, 10\}$ и отношение $R = \{(a,b)|a^{2} = b^{2}\}$
\item $A = \{a, b, c, d, p, t\}$ задано отношение $R = \{(a, a), (b, b), (b, c), (b, d), (c, b), (c, c), (c, d), (d, b), (d, c), (d, d), (p,p), (t,t)\}$
\item Пусть A – множество имен. $A = \{ $Алексей, Иван, Петр, Александр, Павел, Андрей$ \}$. Тогда отношение $R$ верно на парах имен, начинающихся с одной и той же буквы, и только на них.
\end{enumerate}\question Составьте полную таблицу истинности, определите, какие переменные являются фиктивными и проверьте, является ли формула тавтологией:
$(P \rightarrow (Q \rightarrow R)) \rightarrow ((P \rightarrow Q) \rightarrow (P \rightarrow R))$

\end{questions}
\newpage
%%% begin test
\begin{flushright}
\begin{tabular}{p{2.8in} r l}
%\textbf{\class} & \textbf{ФИО:} & \makebox[2.5in]{\hrulefill}\\
\textbf{\class} & \textbf{ФИО:} &Базалий Иван Андреевич
\\

\textbf{\examdate} &&\\
%\textbf{Time Limit: \timelimit} & Teaching Assistant & \makebox[2in]{\hrulefill}
\end{tabular}\\
\end{flushright}
\rule[1ex]{\textwidth}{.1pt}


\begin{questions}
\question
Найдите и упростите P:
\begin{equation*}
\overline{P} = A \cap B \cup \overline{A} \cap \overline{B} \cup A \cap C \cup \overline{B} \cap C
\end{equation*}
Затем найдите элементы множества P, выраженного через множества:
\begin{equation*}
A = \{0, 3, 4, 9\}; 
B = \{1, 3, 4, 7\};
C = \{0, 1, 2, 4, 7, 8, 9\};
I = \{0, 1, 2, 3, 4, 5, 6, 7, 8, 9\}.
\end{equation*}\question
Упростите следующее выражение с учетом того, что $A\subset B \subset C \subset D \subset U; A \neq \O$
\begin{equation*}
A \cap  \overline{C} \cup B \cap \overline{D} \cup  \overline{A} \cap C \cap  \overline{D}
\end{equation*}

Примечание: U — универсум\question
Дано отношение на множестве $\{1, 2, 3, 4, 5\}$ 
\begin{equation*}
aRb \iff |a-b| = 1
\end{equation*}
Напишите обоснованный ответ какими свойствами обладает или не обладает отношение и почему:   
\begin{enumerate} [a)]\setcounter{enumi}{0}
\item рефлексивность
\item антирефлексивность
\item симметричность
\item асимметричность
\item антисимметричность
\item транзитивность
\end{enumerate}

Обоснуйте свой ответ по каждому из приведенных ниже вопросов:
\begin{enumerate} [a)]\setcounter{enumi}{0}
    \item Является ли это отношение отношением эквивалентности?
    \item Является ли это отношение функциональным?
    \item Каким из отношений соответствия (одно-многозначным, много-многозначный и т.д.) оно является?
    \item К каким из отношений порядка (полного, частичного и т.д.) можно отнести данное отношение?
\end{enumerate}

\question
Установите, является ли каждое из перечисленных ниже отношений на А ($R \subseteq A \times A$) отношением эквивалентности (обоснование ответа обязательно). Для каждого отношения эквивалентности постройте классы 
эквивалентности и постройте граф отношения:
\begin{enumerate} [a)]\setcounter{enumi}{0}
\item А - множество целых чисел и отношение $R = \{(a,b)|a + b = 5\}$
\item Пусть A – множество имен. $A = \{ $Алексей, Иван, Петр, Александр, Павел, Андрей$ \}$. Тогда отношение $R $ верно на парах имен, начинающихся с одной и той же буквы, и только на них.
\item На множестве $A = \{1; 2; 3; 4; 5\}$ задано отношение $R = \{(1; 2); (1; 3); (1; 5); (2; 3); (2; 4); (2; 5); (3; 4); (3; 5); (4; 5)\}$
\end{enumerate}\question Составьте полную таблицу истинности, определите, какие переменные являются фиктивными и проверьте, является ли формула тавтологией:
$ P \rightarrow (Q \rightarrow ((P \lor Q) \rightarrow (P \land Q)))$

\end{questions}
\newpage
%%% begin test
\begin{flushright}
\begin{tabular}{p{2.8in} r l}
%\textbf{\class} & \textbf{ФИО:} & \makebox[2.5in]{\hrulefill}\\
\textbf{\class} & \textbf{ФИО:} &Беззубцева Анастасия Андреевна
\\

\textbf{\examdate} &&\\
%\textbf{Time Limit: \timelimit} & Teaching Assistant & \makebox[2in]{\hrulefill}
\end{tabular}\\
\end{flushright}
\rule[1ex]{\textwidth}{.1pt}


\begin{questions}
\question
Найдите и упростите P:
\begin{equation*}
\overline{P} = A \cap B \cup \overline{A} \cap \overline{B} \cup A \cap C \cup \overline{B} \cap C
\end{equation*}
Затем найдите элементы множества P, выраженного через множества:
\begin{equation*}
A = \{0, 3, 4, 9\}; 
B = \{1, 3, 4, 7\};
C = \{0, 1, 2, 4, 7, 8, 9\};
I = \{0, 1, 2, 3, 4, 5, 6, 7, 8, 9\}.
\end{equation*}\question
Упростите следующее выражение с учетом того, что $A\subset B \subset C \subset D \subset U; A \neq \O$
\begin{equation*}
A \cap B  \cap \overline{C} \cup \overline{C} \cap D \cup B \cap C \cap D
\end{equation*}

Примечание: U — универсум\question
Дано отношение на множестве $\{1, 2, 3, 4, 5\}$ 
\begin{equation*}
aRb \iff a \geq b^2
\end{equation*}
Напишите обоснованный ответ какими свойствами обладает или не обладает отношение и почему:   
\begin{enumerate} [a)]\setcounter{enumi}{0}
\item рефлексивность
\item антирефлексивность
\item симметричность
\item асимметричность
\item антисимметричность
\item транзитивность
\end{enumerate}

Обоснуйте свой ответ по каждому из приведенных ниже вопросов:
\begin{enumerate} [a)]\setcounter{enumi}{0}
    \item Является ли это отношение отношением эквивалентности?
    \item Является ли это отношение функциональным?
    \item Каким из отношений соответствия (одно-многозначным, много-многозначный и т.д.) оно является?
    \item К каким из отношений порядка (полного, частичного и т.д.) можно отнести данное отношение?
\end{enumerate}


\question
Установите, является ли каждое из перечисленных ниже отношений на А ($R \subseteq A \times A$) отношением эквивалентности (обоснование ответа обязательно). Для каждого отношения эквивалентности постройте классы 
эквивалентности и постройте граф отношения:
\begin{enumerate} [a)]\setcounter{enumi}{0}
\item А - множество целых чисел и отношение $R = \{(a,b)|a + b = 5\}$
\item Пусть A – множество имен. $A = \{ $Алексей, Иван, Петр, Александр, Павел, Андрей$ \}$. Тогда отношение $R $ верно на парах имен, начинающихся с одной и той же буквы, и только на них.
\item На множестве $A = \{1; 2; 3; 4; 5\}$ задано отношение $R = \{(1; 2); (1; 3); (1; 5); (2; 3); (2; 4); (2; 5); (3; 4); (3; 5); (4; 5)\}$
\end{enumerate}\question Составьте полную таблицу истинности, определите, какие переменные являются фиктивными и проверьте, является ли формула тавтологией:
$((P \rightarrow Q) \land (R \rightarrow S) \land \neg (Q \lor S)) \rightarrow \neg (P \lor R)$

\end{questions}
\newpage
%%% begin test
\begin{flushright}
\begin{tabular}{p{2.8in} r l}
%\textbf{\class} & \textbf{ФИО:} & \makebox[2.5in]{\hrulefill}\\
\textbf{\class} & \textbf{ФИО:} &Беспалов Денис Федорович
\\

\textbf{\examdate} &&\\
%\textbf{Time Limit: \timelimit} & Teaching Assistant & \makebox[2in]{\hrulefill}
\end{tabular}\\
\end{flushright}
\rule[1ex]{\textwidth}{.1pt}


\begin{questions}
\question
Найдите и упростите P:
\begin{equation*}
\overline{P} = A \cap \overline{B} \cup \overline{B} \cap C \cup \overline{A} \cap \overline{B} \cup \overline{A} \cap C
\end{equation*}
Затем найдите элементы множества P, выраженного через множества:
\begin{equation*}
A = \{0, 3, 4, 9\}; 
B = \{1, 3, 4, 7\};
C = \{0, 1, 2, 4, 7, 8, 9\};
I = \{0, 1, 2, 3, 4, 5, 6, 7, 8, 9\}.
\end{equation*}\question
Упростите следующее выражение с учетом того, что $A\subset B \subset C \subset D \subset U; A \neq \O$
\begin{equation*}
\overline{B} \cap \overline{C} \cap D \cup \overline{A} \cap \overline{C} \cap D \cup \overline{A} \cap B
\end{equation*}

Примечание: U — универсум\question
Дано отношение на множестве $\{1, 2, 3, 4, 5\}$ 
\begin{equation*}
aRb \iff a \geq b^2
\end{equation*}
Напишите обоснованный ответ какими свойствами обладает или не обладает отношение и почему:   
\begin{enumerate} [a)]\setcounter{enumi}{0}
\item рефлексивность
\item антирефлексивность
\item симметричность
\item асимметричность
\item антисимметричность
\item транзитивность
\end{enumerate}

Обоснуйте свой ответ по каждому из приведенных ниже вопросов:
\begin{enumerate} [a)]\setcounter{enumi}{0}
    \item Является ли это отношение отношением эквивалентности?
    \item Является ли это отношение функциональным?
    \item Каким из отношений соответствия (одно-многозначным, много-многозначный и т.д.) оно является?
    \item К каким из отношений порядка (полного, частичного и т.д.) можно отнести данное отношение?
\end{enumerate}


\question
Установите, является ли каждое из перечисленных ниже отношений на А ($R \subseteq A \times A$) отношением эквивалентности (обоснование ответа обязательно). Для каждого отношения эквивалентности постройте классы 
эквивалентности и постройте граф отношения:
\begin{enumerate} [a)]\setcounter{enumi}{0}
\item $A = \{a, b, c, d, p, t\}$ задано отношение $R = \{(a, a), (b, b), (b, c), (b, d), (c, b), (c, c), (c, d), (d, b), (d, c), (d, d), (p,p), (t,t)\}$
\item $A = \{-10, -9, … , 9, 10\}$ и отношение $R = \{(a,b)|a^{3} = b^{3}\}$

\item $F(x)=x^{2}+1$, где $x \in A = [-2, 4]$ и отношение $R = \{(a,b)|F(a) = F(b)\}$
\end{enumerate}\question Составьте полную таблицу истинности, определите, какие переменные являются фиктивными и проверьте, является ли формула тавтологией:
$(( P \rightarrow Q) \land (Q \rightarrow P)) \rightarrow (P \rightarrow R)$

\end{questions}
\newpage
%%% begin test
\begin{flushright}
\begin{tabular}{p{2.8in} r l}
%\textbf{\class} & \textbf{ФИО:} & \makebox[2.5in]{\hrulefill}\\
\textbf{\class} & \textbf{ФИО:} &Блажков Александр Дмитриевич
\\

\textbf{\examdate} &&\\
%\textbf{Time Limit: \timelimit} & Teaching Assistant & \makebox[2in]{\hrulefill}
\end{tabular}\\
\end{flushright}
\rule[1ex]{\textwidth}{.1pt}


\begin{questions}
\question
Найдите и упростите P:
\begin{equation*}
\overline{P} = A \cap \overline{C} \cup A \cap \overline{B} \cup B \cap \overline{C} \cup A \cap C
\end{equation*}
Затем найдите элементы множества P, выраженного через множества:
\begin{equation*}
A = \{0, 3, 4, 9\}; 
B = \{1, 3, 4, 7\};
C = \{0, 1, 2, 4, 7, 8, 9\};
I = \{0, 1, 2, 3, 4, 5, 6, 7, 8, 9\}.
\end{equation*}\question
Упростите следующее выражение с учетом того, что $A\subset B \subset C \subset D \subset U; A \neq \O$
\begin{equation*}
A \cap  \overline{C} \cup B \cap \overline{D} \cup  \overline{A} \cap C \cap  \overline{D}
\end{equation*}

Примечание: U — универсум\question
Дано отношение на множестве $\{1, 2, 3, 4, 5\}$ 
\begin{equation*}
aRb \iff b > a
\end{equation*}
Напишите обоснованный ответ какими свойствами обладает или не обладает отношение и почему:   
\begin{enumerate} [a)]\setcounter{enumi}{0}
\item рефлексивность
\item антирефлексивность
\item симметричность
\item асимметричность
\item антисимметричность
\item транзитивность
\end{enumerate}

Обоснуйте свой ответ по каждому из приведенных ниже вопросов:
\begin{enumerate} [a)]\setcounter{enumi}{0}
    \item Является ли это отношение отношением эквивалентности?
    \item Является ли это отношение функциональным?
    \item Каким из отношений соответствия (одно-многозначным, много-многозначный и т.д.) оно является?
    \item К каким из отношений порядка (полного, частичного и т.д.) можно отнести данное отношение?
\end{enumerate}

\question
Установите, является ли каждое из перечисленных ниже отношений на А ($R \subseteq A \times A$) отношением эквивалентности (обоснование ответа обязательно). Для каждого отношения эквивалентности 
постройте классы эквивалентности и постройте граф отношения:
\begin{enumerate}[a)]\setcounter{enumi}{0}
\item А - множество целых чисел и отношение $R = \{(a,b)|a + b = 0\}$
\item $A = \{-10, -9, …, 9, 10\}$ и отношение $R = \{(a,b)|a^{3} = b^{3}\}$
\item На множестве $A = \{1; 2; 3\}$ задано отношение $R = \{(1; 1); (2; 2); (3; 3); (2; 1); (1; 2); (2; 3); (3; 2); (3; 1); (1; 3)\}$

\end{enumerate}\question Составьте полную таблицу истинности, определите, какие переменные являются фиктивными и проверьте, является ли формула тавтологией:
$ P \rightarrow (Q \rightarrow ((P \lor Q) \rightarrow (P \land Q)))$

\end{questions}
\newpage
%%% begin test
\begin{flushright}
\begin{tabular}{p{2.8in} r l}
%\textbf{\class} & \textbf{ФИО:} & \makebox[2.5in]{\hrulefill}\\
\textbf{\class} & \textbf{ФИО:} &Васин Владимир Алексеевич
\\

\textbf{\examdate} &&\\
%\textbf{Time Limit: \timelimit} & Teaching Assistant & \makebox[2in]{\hrulefill}
\end{tabular}\\
\end{flushright}
\rule[1ex]{\textwidth}{.1pt}


\begin{questions}
\question
Найдите и упростите P:
\begin{equation*}
\overline{P} = A \cap B \cup \overline{A} \cap \overline{B} \cup A \cap C \cup \overline{B} \cap C
\end{equation*}
Затем найдите элементы множества P, выраженного через множества:
\begin{equation*}
A = \{0, 3, 4, 9\}; 
B = \{1, 3, 4, 7\};
C = \{0, 1, 2, 4, 7, 8, 9\};
I = \{0, 1, 2, 3, 4, 5, 6, 7, 8, 9\}.
\end{equation*}\question
Упростите следующее выражение с учетом того, что $A\subset B \subset C \subset D \subset U; A \neq \O$
\begin{equation*}
A \cap B  \cap \overline{C} \cup \overline{C} \cap D \cup B \cap C \cap D
\end{equation*}

Примечание: U — универсум\question
Дано отношение на множестве $\{1, 2, 3, 4, 5\}$ 
\begin{equation*}
aRb \iff (a+b) \bmod 2 =0
\end{equation*}
Напишите обоснованный ответ какими свойствами обладает или не обладает отношение и почему:   
\begin{enumerate} [a)]\setcounter{enumi}{0}
\item рефлексивность
\item антирефлексивность
\item симметричность
\item асимметричность
\item антисимметричность
\item транзитивность
\end{enumerate}

Обоснуйте свой ответ по каждому из приведенных ниже вопросов:
\begin{enumerate} [a)]\setcounter{enumi}{0}
    \item Является ли это отношение отношением эквивалентности?
    \item Является ли это отношение функциональным?
    \item Каким из отношений соответствия (одно-многозначным, много-многозначный и т.д.) оно является?
    \item К каким из отношений порядка (полного, частичного и т.д.) можно отнести данное отношение?
\end{enumerate}



\question
Установите, является ли каждое из перечисленных ниже отношений на А ($R \subseteq A \times A$) отношением эквивалентности (обоснование ответа обязательно). Для каждого отношения эквивалентности постройте классы 
эквивалентности и постройте граф отношения:
\begin{enumerate} [a)]\setcounter{enumi}{0}
\item $A = \{a, b, c, d, p, t\}$ задано отношение $R = \{(a, a), (b, b), (b, c), (b, d), (c, b), (c, c), (c, d), (d, b), (d, c), (d, d), (p,p), (t,t)\}$
\item $A = \{-10, -9, … , 9, 10\}$ и отношение $R = \{(a,b)|a^{3} = b^{3}\}$

\item $F(x)=x^{2}+1$, где $x \in A = [-2, 4]$ и отношение $R = \{(a,b)|F(a) = F(b)\}$
\end{enumerate}\question Составьте полную таблицу истинности, определите, какие переменные являются фиктивными и проверьте, является ли формула тавтологией:
$ P \rightarrow (Q \rightarrow ((P \lor Q) \rightarrow (P \land Q)))$

\end{questions}
\newpage
%%% begin test
\begin{flushright}
\begin{tabular}{p{2.8in} r l}
%\textbf{\class} & \textbf{ФИО:} & \makebox[2.5in]{\hrulefill}\\
\textbf{\class} & \textbf{ФИО:} &Веркин Дамир Михайлович
\\

\textbf{\examdate} &&\\
%\textbf{Time Limit: \timelimit} & Teaching Assistant & \makebox[2in]{\hrulefill}
\end{tabular}\\
\end{flushright}
\rule[1ex]{\textwidth}{.1pt}


\begin{questions}
\question
Найдите и упростите P:
\begin{equation*}
\overline{P} = A \cap \overline{B} \cup A \cap C \cup B \cap C \cup \overline{A} \cap C
\end{equation*}
Затем найдите элементы множества P, выраженного через множества:
\begin{equation*}
A = \{0, 3, 4, 9\}; 
B = \{1, 3, 4, 7\};
C = \{0, 1, 2, 4, 7, 8, 9\};
I = \{0, 1, 2, 3, 4, 5, 6, 7, 8, 9\}.
\end{equation*}\question
Упростите следующее выражение с учетом того, что $A\subset B \subset C \subset D \subset U; A \neq \O$
\begin{equation*}
A \cap B \cup \overline{A} \cap \overline{C} \cup A \cap C \cup \overline{B} \cap \overline{C}
\end{equation*}

Примечание: U — универсум\question
Дано отношение на множестве $\{1, 2, 3, 4, 5\}$ 
\begin{equation*}
aRb \iff b > a
\end{equation*}
Напишите обоснованный ответ какими свойствами обладает или не обладает отношение и почему:   
\begin{enumerate} [a)]\setcounter{enumi}{0}
\item рефлексивность
\item антирефлексивность
\item симметричность
\item асимметричность
\item антисимметричность
\item транзитивность
\end{enumerate}

Обоснуйте свой ответ по каждому из приведенных ниже вопросов:
\begin{enumerate} [a)]\setcounter{enumi}{0}
    \item Является ли это отношение отношением эквивалентности?
    \item Является ли это отношение функциональным?
    \item Каким из отношений соответствия (одно-многозначным, много-многозначный и т.д.) оно является?
    \item К каким из отношений порядка (полного, частичного и т.д.) можно отнести данное отношение?
\end{enumerate}

\question
Установите, является ли каждое из перечисленных ниже отношений на А ($R \subseteq A \times A$) отношением эквивалентности (обоснование ответа обязательно). Для каждого отношения эквивалентности 
постройте классы эквивалентности и постройте граф отношения:
\begin{enumerate}[a)]\setcounter{enumi}{0}
\item А - множество целых чисел и отношение $R = \{(a,b)|a + b = 0\}$
\item $A = \{-10, -9, …, 9, 10\}$ и отношение $R = \{(a,b)|a^{3} = b^{3}\}$
\item На множестве $A = \{1; 2; 3\}$ задано отношение $R = \{(1; 1); (2; 2); (3; 3); (2; 1); (1; 2); (2; 3); (3; 2); (3; 1); (1; 3)\}$

\end{enumerate}\question Составьте полную таблицу истинности, определите, какие переменные являются фиктивными и проверьте, является ли формула тавтологией:
$(P \rightarrow (Q \rightarrow R)) \rightarrow ((P \rightarrow Q) \rightarrow (P \rightarrow R))$

\end{questions}
\newpage
%%% begin test
\begin{flushright}
\begin{tabular}{p{2.8in} r l}
%\textbf{\class} & \textbf{ФИО:} & \makebox[2.5in]{\hrulefill}\\
\textbf{\class} & \textbf{ФИО:} &Гурман Тимофей Владимирович
\\

\textbf{\examdate} &&\\
%\textbf{Time Limit: \timelimit} & Teaching Assistant & \makebox[2in]{\hrulefill}
\end{tabular}\\
\end{flushright}
\rule[1ex]{\textwidth}{.1pt}


\begin{questions}
\question
Найдите и упростите P:
\begin{equation*}
\overline{P} = A \cap \overline{C} \cup A \cap \overline{B} \cup B \cap \overline{C} \cup A \cap C
\end{equation*}
Затем найдите элементы множества P, выраженного через множества:
\begin{equation*}
A = \{0, 3, 4, 9\}; 
B = \{1, 3, 4, 7\};
C = \{0, 1, 2, 4, 7, 8, 9\};
I = \{0, 1, 2, 3, 4, 5, 6, 7, 8, 9\}.
\end{equation*}\question
Упростите следующее выражение с учетом того, что $A\subset B \subset C \subset D \subset U; A \neq \O$
\begin{equation*}
\overline{A} \cap \overline{C} \cap D \cup \overline{B} \cap \overline{C} \cap D \cup A \cap B
\end{equation*}

Примечание: U — универсум\question
Дано отношение на множестве $\{1, 2, 3, 4, 5\}$ 
\begin{equation*}
aRb \iff b > a
\end{equation*}
Напишите обоснованный ответ какими свойствами обладает или не обладает отношение и почему:   
\begin{enumerate} [a)]\setcounter{enumi}{0}
\item рефлексивность
\item антирефлексивность
\item симметричность
\item асимметричность
\item антисимметричность
\item транзитивность
\end{enumerate}

Обоснуйте свой ответ по каждому из приведенных ниже вопросов:
\begin{enumerate} [a)]\setcounter{enumi}{0}
    \item Является ли это отношение отношением эквивалентности?
    \item Является ли это отношение функциональным?
    \item Каким из отношений соответствия (одно-многозначным, много-многозначный и т.д.) оно является?
    \item К каким из отношений порядка (полного, частичного и т.д.) можно отнести данное отношение?
\end{enumerate}

\question
Установите, является ли каждое из перечисленных ниже отношений на А ($R \subseteq A \times A$) отношением эквивалентности (обоснование ответа обязательно). Для каждого отношения эквивалентности 
постройте классы эквивалентности и постройте граф отношения:
\begin{enumerate}[a)]\setcounter{enumi}{0}
\item А - множество целых чисел и отношение $R = \{(a,b)|a + b = 0\}$
\item $A = \{-10, -9, …, 9, 10\}$ и отношение $R = \{(a,b)|a^{3} = b^{3}\}$
\item На множестве $A = \{1; 2; 3\}$ задано отношение $R = \{(1; 1); (2; 2); (3; 3); (2; 1); (1; 2); (2; 3); (3; 2); (3; 1); (1; 3)\}$

\end{enumerate}\question Составьте полную таблицу истинности, определите, какие переменные являются фиктивными и проверьте, является ли формула тавтологией:

$(P \rightarrow (Q \land R)) \leftrightarrow ((P \rightarrow Q) \land (P \rightarrow R))$

\end{questions}
\newpage
%%% begin test
\begin{flushright}
\begin{tabular}{p{2.8in} r l}
%\textbf{\class} & \textbf{ФИО:} & \makebox[2.5in]{\hrulefill}\\
\textbf{\class} & \textbf{ФИО:} &Евтушенко Иван Дмитриевич
\\

\textbf{\examdate} &&\\
%\textbf{Time Limit: \timelimit} & Teaching Assistant & \makebox[2in]{\hrulefill}
\end{tabular}\\
\end{flushright}
\rule[1ex]{\textwidth}{.1pt}


\begin{questions}
\question
Найдите и упростите P:
\begin{equation*}
\overline{P} = A \cap C \cup \overline{A} \cap \overline{C} \cup \overline{B} \cap C \cup \overline{A} \cap \overline{B}
\end{equation*}
Затем найдите элементы множества P, выраженного через множества:
\begin{equation*}
A = \{0, 3, 4, 9\}; 
B = \{1, 3, 4, 7\};
C = \{0, 1, 2, 4, 7, 8, 9\};
I = \{0, 1, 2, 3, 4, 5, 6, 7, 8, 9\}.
\end{equation*}\question
Упростите следующее выражение с учетом того, что $A\subset B \subset C \subset D \subset U; A \neq \O$
\begin{equation*}
\overline{A} \cap \overline{C} \cap D \cup \overline{B} \cap \overline{C} \cap D \cup A \cap B
\end{equation*}

Примечание: U — универсум\question
Для следующего отношения на множестве $\{1, 2, 3, 4, 5\}$ 
\begin{equation*}
aRb \iff 0 < a-b<2
\end{equation*}
Напишите обоснованный ответ какими свойствами обладает или не обладает отношение и почему:   
\begin{enumerate} [a)]\setcounter{enumi}{0}
\item рефлексивность
\item антирефлексивность
\item симметричность
\item асимметричность
\item антисимметричность
\item транзитивность
\end{enumerate}

Обоснуйте свой ответ по каждому из приведенных ниже вопросов:
\begin{enumerate} [a)]\setcounter{enumi}{0}
    \item Является ли это отношение отношением эквивалентности?
    \item Является ли это отношение функциональным?
    \item Каким из отношений соответствия (одно-многозначным, много-многозначный и т.д.) оно является?
    \item К каким из отношений порядка (полного, частичного и т.д.) можно отнести данное отношение?
\end{enumerate}
\question
Установите, является ли каждое из перечисленных ниже отношений на А ($R \subseteq A \times A$) отношением эквивалентности (обоснование ответа обязательно). Для каждого отношения эквивалентности постройте классы 
эквивалентности и постройте граф отношения:
\begin{enumerate} [a)]\setcounter{enumi}{0}
\item На множестве $A = \{1; 2; 3\}$ задано отношение $R = \{(1; 1); (2; 2); (3; 3); (2; 1); (1; 2); (2; 3); (3; 2); (3; 1); (1; 3)\}$
\item На множестве $A = \{1; 2; 3; 4; 5\}$ задано отношение $R = \{(1; 2); (1; 3); (1; 5); (2; 3); (2; 4); (2; 5); (3; 4); (3; 5); (4; 5)\}$
\item А - множество целых чисел и отношение $R = \{(a,b)|a + b = 0\}$
\end{enumerate}\question Составьте полную таблицу истинности, определите, какие переменные являются фиктивными и проверьте, является ли формула тавтологией:
$((P \rightarrow Q) \lor R) \leftrightarrow (P \rightarrow (Q \lor R))$

\end{questions}
\newpage
%%% begin test
\begin{flushright}
\begin{tabular}{p{2.8in} r l}
%\textbf{\class} & \textbf{ФИО:} & \makebox[2.5in]{\hrulefill}\\
\textbf{\class} & \textbf{ФИО:} &Ершов Александр Романович
\\

\textbf{\examdate} &&\\
%\textbf{Time Limit: \timelimit} & Teaching Assistant & \makebox[2in]{\hrulefill}
\end{tabular}\\
\end{flushright}
\rule[1ex]{\textwidth}{.1pt}


\begin{questions}
\question
Найдите и упростите P:
\begin{equation*}
\overline{P} = A \cap \overline{B} \cup A \cap C \cup B \cap C \cup \overline{A} \cap C
\end{equation*}
Затем найдите элементы множества P, выраженного через множества:
\begin{equation*}
A = \{0, 3, 4, 9\}; 
B = \{1, 3, 4, 7\};
C = \{0, 1, 2, 4, 7, 8, 9\};
I = \{0, 1, 2, 3, 4, 5, 6, 7, 8, 9\}.
\end{equation*}\question
Упростите следующее выражение с учетом того, что $A\subset B \subset C \subset D \subset U; A \neq \O$
\begin{equation*}
A \cap C  \cap D \cup B \cap \overline{C} \cap D \cup B \cap C \cap D
\end{equation*}

Примечание: U — универсум\question
Дано отношение на множестве $\{1, 2, 3, 4, 5\}$ 
\begin{equation*}
aRb \iff (a+b) \bmod 2 =0
\end{equation*}
Напишите обоснованный ответ какими свойствами обладает или не обладает отношение и почему:   
\begin{enumerate} [a)]\setcounter{enumi}{0}
\item рефлексивность
\item антирефлексивность
\item симметричность
\item асимметричность
\item антисимметричность
\item транзитивность
\end{enumerate}

Обоснуйте свой ответ по каждому из приведенных ниже вопросов:
\begin{enumerate} [a)]\setcounter{enumi}{0}
    \item Является ли это отношение отношением эквивалентности?
    \item Является ли это отношение функциональным?
    \item Каким из отношений соответствия (одно-многозначным, много-многозначный и т.д.) оно является?
    \item К каким из отношений порядка (полного, частичного и т.д.) можно отнести данное отношение?
\end{enumerate}



\question
Установите, является ли каждое из перечисленных ниже отношений на А ($R \subseteq A \times A$) отношением эквивалентности (обоснование ответа обязательно). Для каждого отношения эквивалентности 
постройте классы эквивалентности и постройте граф отношения:
\begin{enumerate}[a)]\setcounter{enumi}{0}
\item А - множество целых чисел и отношение $R = \{(a,b)|a + b = 0\}$
\item $A = \{-10, -9, …, 9, 10\}$ и отношение $R = \{(a,b)|a^{3} = b^{3}\}$
\item На множестве $A = \{1; 2; 3\}$ задано отношение $R = \{(1; 1); (2; 2); (3; 3); (2; 1); (1; 2); (2; 3); (3; 2); (3; 1); (1; 3)\}$

\end{enumerate}\question Составьте полную таблицу истинности, определите, какие переменные являются фиктивными и проверьте, является ли формула тавтологией:
$((P \rightarrow Q) \lor R) \leftrightarrow (P \rightarrow (Q \lor R))$

\end{questions}
\newpage
%%% begin test
\begin{flushright}
\begin{tabular}{p{2.8in} r l}
%\textbf{\class} & \textbf{ФИО:} & \makebox[2.5in]{\hrulefill}\\
\textbf{\class} & \textbf{ФИО:} &Захарова Виктория Андреевна
\\

\textbf{\examdate} &&\\
%\textbf{Time Limit: \timelimit} & Teaching Assistant & \makebox[2in]{\hrulefill}
\end{tabular}\\
\end{flushright}
\rule[1ex]{\textwidth}{.1pt}


\begin{questions}
\question
Найдите и упростите P:
\begin{equation*}
\overline{P} = A \cap \overline{C} \cup A \cap \overline{B} \cup B \cap \overline{C} \cup A \cap C
\end{equation*}
Затем найдите элементы множества P, выраженного через множества:
\begin{equation*}
A = \{0, 3, 4, 9\}; 
B = \{1, 3, 4, 7\};
C = \{0, 1, 2, 4, 7, 8, 9\};
I = \{0, 1, 2, 3, 4, 5, 6, 7, 8, 9\}.
\end{equation*}\question
Упростите следующее выражение с учетом того, что $A\subset B \subset C \subset D \subset U; A \neq \O$
\begin{equation*}
A \cap C  \cap D \cup B \cap \overline{C} \cap D \cup B \cap C \cap D
\end{equation*}

Примечание: U — универсум\question
Дано отношение на множестве $\{1, 2, 3, 4, 5\}$ 
\begin{equation*}
aRb \iff b > a
\end{equation*}
Напишите обоснованный ответ какими свойствами обладает или не обладает отношение и почему:   
\begin{enumerate} [a)]\setcounter{enumi}{0}
\item рефлексивность
\item антирефлексивность
\item симметричность
\item асимметричность
\item антисимметричность
\item транзитивность
\end{enumerate}

Обоснуйте свой ответ по каждому из приведенных ниже вопросов:
\begin{enumerate} [a)]\setcounter{enumi}{0}
    \item Является ли это отношение отношением эквивалентности?
    \item Является ли это отношение функциональным?
    \item Каким из отношений соответствия (одно-многозначным, много-многозначный и т.д.) оно является?
    \item К каким из отношений порядка (полного, частичного и т.д.) можно отнести данное отношение?
\end{enumerate}

\question
Установите, является ли каждое из перечисленных ниже отношений на А ($R \subseteq A \times A$) отношением эквивалентности (обоснование ответа обязательно). Для каждого отношения эквивалентности постройте классы 
эквивалентности и постройте граф отношения:
\begin{enumerate} [a)]\setcounter{enumi}{0}
\item На множестве $A = \{1; 2; 3\}$ задано отношение $R = \{(1; 1); (2; 2); (3; 3); (2; 1); (1; 2); (2; 3); (3; 2); (3; 1); (1; 3)\}$
\item На множестве $A = \{1; 2; 3; 4; 5\}$ задано отношение $R = \{(1; 2); (1; 3); (1; 5); (2; 3); (2; 4); (2; 5); (3; 4); (3; 5); (4; 5)\}$
\item А - множество целых чисел и отношение $R = \{(a,b)|a + b = 0\}$
\end{enumerate}\question Составьте полную таблицу истинности, определите, какие переменные являются фиктивными и проверьте, является ли формула тавтологией:
$(P \rightarrow (Q \rightarrow R)) \rightarrow ((P \rightarrow Q) \rightarrow (P \rightarrow R))$

\end{questions}
\newpage
%%% begin test
\begin{flushright}
\begin{tabular}{p{2.8in} r l}
%\textbf{\class} & \textbf{ФИО:} & \makebox[2.5in]{\hrulefill}\\
\textbf{\class} & \textbf{ФИО:} &Иванов Алексей Александрович
\\

\textbf{\examdate} &&\\
%\textbf{Time Limit: \timelimit} & Teaching Assistant & \makebox[2in]{\hrulefill}
\end{tabular}\\
\end{flushright}
\rule[1ex]{\textwidth}{.1pt}


\begin{questions}
\question
Найдите и упростите P:
\begin{equation*}
\overline{P} = A \cap \overline{C} \cup A \cap \overline{B} \cup B \cap \overline{C} \cup A \cap C
\end{equation*}
Затем найдите элементы множества P, выраженного через множества:
\begin{equation*}
A = \{0, 3, 4, 9\}; 
B = \{1, 3, 4, 7\};
C = \{0, 1, 2, 4, 7, 8, 9\};
I = \{0, 1, 2, 3, 4, 5, 6, 7, 8, 9\}.
\end{equation*}\question
Упростите следующее выражение с учетом того, что $A\subset B \subset C \subset D \subset U; A \neq \O$
\begin{equation*}
\overline{A} \cap \overline{C} \cap D \cup \overline{B} \cap \overline{C} \cap D \cup A \cap B
\end{equation*}

Примечание: U — универсум\question
Дано отношение на множестве $\{1, 2, 3, 4, 5\}$ 
\begin{equation*}
aRb \iff a \leq b
\end{equation*}
Напишите обоснованный ответ какими свойствами обладает или не обладает отношение и почему:   
\begin{enumerate} [a)]\setcounter{enumi}{0}
\item рефлексивность
\item антирефлексивность
\item симметричность
\item асимметричность
\item антисимметричность
\item транзитивность
\end{enumerate}

Обоснуйте свой ответ по каждому из приведенных ниже вопросов:
\begin{enumerate} [a)]\setcounter{enumi}{0}
    \item Является ли это отношение отношением эквивалентности?
    \item Является ли это отношение функциональным?
    \item Каким из отношений соответствия (одно-многозначным, много-многозначный и т.д.) оно является?
    \item К каким из отношений порядка (полного, частичного и т.д.) можно отнести данное отношение?
\end{enumerate}


\question
Установите, является ли каждое из перечисленных ниже отношений на А ($R \subseteq A \times A$) отношением эквивалентности (обоснование ответа обязательно). Для каждого отношения эквивалентности постройте классы 
эквивалентности и постройте граф отношения:
\begin{enumerate} [a)]\setcounter{enumi}{0}
\item На множестве $A = \{1; 2; 3\}$ задано отношение $R = \{(1; 1); (2; 2); (3; 3); (2; 1); (1; 2); (2; 3); (3; 2); (3; 1); (1; 3)\}$
\item На множестве $A = \{1; 2; 3; 4; 5\}$ задано отношение $R = \{(1; 2); (1; 3); (1; 5); (2; 3); (2; 4); (2; 5); (3; 4); (3; 5); (4; 5)\}$
\item А - множество целых чисел и отношение $R = \{(a,b)|a + b = 0\}$
\end{enumerate}\question Составьте полную таблицу истинности, определите, какие переменные являются фиктивными и проверьте, является ли формула тавтологией:
$(( P \land \neg Q) \rightarrow (R \land \neg R)) \rightarrow (P \rightarrow Q)$

\end{questions}
\newpage
%%% begin test
\begin{flushright}
\begin{tabular}{p{2.8in} r l}
%\textbf{\class} & \textbf{ФИО:} & \makebox[2.5in]{\hrulefill}\\
\textbf{\class} & \textbf{ФИО:} &Иванов Сергей Андреевич
\\

\textbf{\examdate} &&\\
%\textbf{Time Limit: \timelimit} & Teaching Assistant & \makebox[2in]{\hrulefill}
\end{tabular}\\
\end{flushright}
\rule[1ex]{\textwidth}{.1pt}


\begin{questions}
\question
Найдите и упростите P:
\begin{equation*}
\overline{P} = A \cap B \cup \overline{A} \cap \overline{B} \cup A \cap C \cup \overline{B} \cap C
\end{equation*}
Затем найдите элементы множества P, выраженного через множества:
\begin{equation*}
A = \{0, 3, 4, 9\}; 
B = \{1, 3, 4, 7\};
C = \{0, 1, 2, 4, 7, 8, 9\};
I = \{0, 1, 2, 3, 4, 5, 6, 7, 8, 9\}.
\end{equation*}\question
Упростите следующее выражение с учетом того, что $A\subset B \subset C \subset D \subset U; A \neq \O$
\begin{equation*}
A \cap C  \cap D \cup B \cap \overline{C} \cap D \cup B \cap C \cap D
\end{equation*}

Примечание: U — универсум\question
Дано отношение на множестве $\{1, 2, 3, 4, 5\}$ 
\begin{equation*}
aRb \iff a \leq b
\end{equation*}
Напишите обоснованный ответ какими свойствами обладает или не обладает отношение и почему:   
\begin{enumerate} [a)]\setcounter{enumi}{0}
\item рефлексивность
\item антирефлексивность
\item симметричность
\item асимметричность
\item антисимметричность
\item транзитивность
\end{enumerate}

Обоснуйте свой ответ по каждому из приведенных ниже вопросов:
\begin{enumerate} [a)]\setcounter{enumi}{0}
    \item Является ли это отношение отношением эквивалентности?
    \item Является ли это отношение функциональным?
    \item Каким из отношений соответствия (одно-многозначным, много-многозначный и т.д.) оно является?
    \item К каким из отношений порядка (полного, частичного и т.д.) можно отнести данное отношение?
\end{enumerate}


\question
Установите, является ли каждое из перечисленных ниже отношений на А ($R \subseteq A \times A$) отношением эквивалентности (обоснование ответа обязательно). Для каждого отношения эквивалентности постройте классы 
эквивалентности и постройте граф отношения:
\begin{enumerate} [a)]\setcounter{enumi}{0}
\item $A = \{-10, -9, … , 9, 10\}$ и отношение $R = \{(a,b)|a^{2} = b^{2}\}$
\item $A = \{a, b, c, d, p, t\}$ задано отношение $R = \{(a, a), (b, b), (b, c), (b, d), (c, b), (c, c), (c, d), (d, b), (d, c), (d, d), (p,p), (t,t)\}$
\item Пусть A – множество имен. $A = \{ $Алексей, Иван, Петр, Александр, Павел, Андрей$ \}$. Тогда отношение $R$ верно на парах имен, начинающихся с одной и той же буквы, и только на них.
\end{enumerate}\question Составьте полную таблицу истинности, определите, какие переменные являются фиктивными и проверьте, является ли формула тавтологией:
$(( P \land \neg Q) \rightarrow (R \land \neg R)) \rightarrow (P \rightarrow Q)$

\end{questions}
\newpage
%%% begin test
\begin{flushright}
\begin{tabular}{p{2.8in} r l}
%\textbf{\class} & \textbf{ФИО:} & \makebox[2.5in]{\hrulefill}\\
\textbf{\class} & \textbf{ФИО:} &Казанцев Данил Олегович
\\

\textbf{\examdate} &&\\
%\textbf{Time Limit: \timelimit} & Teaching Assistant & \makebox[2in]{\hrulefill}
\end{tabular}\\
\end{flushright}
\rule[1ex]{\textwidth}{.1pt}


\begin{questions}
\question
Найдите и упростите P:
\begin{equation*}
\overline{P} = \overline{A} \cap B \cup \overline{A} \cap C \cup A \cap \overline{B} \cup \overline{B} \cap C
\end{equation*}
Затем найдите элементы множества P, выраженного через множества:
\begin{equation*}
A = \{0, 3, 4, 9\}; 
B = \{1, 3, 4, 7\};
C = \{0, 1, 2, 4, 7, 8, 9\};
I = \{0, 1, 2, 3, 4, 5, 6, 7, 8, 9\}.
\end{equation*}\question
Упростите следующее выражение с учетом того, что $A\subset B \subset C \subset D \subset U; A \neq \O$
\begin{equation*}
\overline{A} \cap \overline{C} \cap D \cup \overline{B} \cap \overline{C} \cap D \cup A \cap B
\end{equation*}

Примечание: U — универсум\question
Дано отношение на множестве $\{1, 2, 3, 4, 5\}$ 
\begin{equation*}
aRb \iff a \leq b
\end{equation*}
Напишите обоснованный ответ какими свойствами обладает или не обладает отношение и почему:   
\begin{enumerate} [a)]\setcounter{enumi}{0}
\item рефлексивность
\item антирефлексивность
\item симметричность
\item асимметричность
\item антисимметричность
\item транзитивность
\end{enumerate}

Обоснуйте свой ответ по каждому из приведенных ниже вопросов:
\begin{enumerate} [a)]\setcounter{enumi}{0}
    \item Является ли это отношение отношением эквивалентности?
    \item Является ли это отношение функциональным?
    \item Каким из отношений соответствия (одно-многозначным, много-многозначный и т.д.) оно является?
    \item К каким из отношений порядка (полного, частичного и т.д.) можно отнести данное отношение?
\end{enumerate}


\question
Установите, является ли каждое из перечисленных ниже отношений на А ($R \subseteq A \times A$) отношением эквивалентности (обоснование ответа обязательно). Для каждого отношения эквивалентности постройте классы эквивалентности и постройте граф отношения:
\begin{enumerate} [a)]\setcounter{enumi}{0}
\item $F(x)=x^{2}+1$, где $x \in A = [-2, 4]$ и отношение $R = \{(a,b)|F(a) = F(b)\}$
\item А - множество целых чисел и отношение $R = \{(a,b)|a + b = 5\}$
\item На множестве $A = \{1; 2; 3\}$ задано отношение $R = \{(1; 1); (2; 2); (3; 3); (3; 2); (1; 2); (2; 1)\}$

\end{enumerate}\question Составьте полную таблицу истинности, определите, какие переменные являются фиктивными и проверьте, является ли формула тавтологией:
$(P \rightarrow (Q \rightarrow R)) \rightarrow ((P \rightarrow Q) \rightarrow (P \rightarrow R))$

\end{questions}
\newpage
%%% begin test
\begin{flushright}
\begin{tabular}{p{2.8in} r l}
%\textbf{\class} & \textbf{ФИО:} & \makebox[2.5in]{\hrulefill}\\
\textbf{\class} & \textbf{ФИО:} &Климачёва Екатерина Николаевна
\\

\textbf{\examdate} &&\\
%\textbf{Time Limit: \timelimit} & Teaching Assistant & \makebox[2in]{\hrulefill}
\end{tabular}\\
\end{flushright}
\rule[1ex]{\textwidth}{.1pt}


\begin{questions}
\question
Найдите и упростите P:
\begin{equation*}
\overline{P} = A \cap B \cup \overline{A} \cap \overline{B} \cup A \cap C \cup \overline{B} \cap C
\end{equation*}
Затем найдите элементы множества P, выраженного через множества:
\begin{equation*}
A = \{0, 3, 4, 9\}; 
B = \{1, 3, 4, 7\};
C = \{0, 1, 2, 4, 7, 8, 9\};
I = \{0, 1, 2, 3, 4, 5, 6, 7, 8, 9\}.
\end{equation*}\question
Упростите следующее выражение с учетом того, что $A\subset B \subset C \subset D \subset U; A \neq \O$
\begin{equation*}
A \cap C  \cap D \cup B \cap \overline{C} \cap D \cup B \cap C \cap D
\end{equation*}

Примечание: U — универсум\question
Дано отношение на множестве $\{1, 2, 3, 4, 5\}$ 
\begin{equation*}
aRb \iff a \leq b
\end{equation*}
Напишите обоснованный ответ какими свойствами обладает или не обладает отношение и почему:   
\begin{enumerate} [a)]\setcounter{enumi}{0}
\item рефлексивность
\item антирефлексивность
\item симметричность
\item асимметричность
\item антисимметричность
\item транзитивность
\end{enumerate}

Обоснуйте свой ответ по каждому из приведенных ниже вопросов:
\begin{enumerate} [a)]\setcounter{enumi}{0}
    \item Является ли это отношение отношением эквивалентности?
    \item Является ли это отношение функциональным?
    \item Каким из отношений соответствия (одно-многозначным, много-многозначный и т.д.) оно является?
    \item К каким из отношений порядка (полного, частичного и т.д.) можно отнести данное отношение?
\end{enumerate}


\question
Установите, является ли каждое из перечисленных ниже отношений на А ($R \subseteq A \times A$) отношением эквивалентности (обоснование ответа обязательно). Для каждого отношения эквивалентности постройте классы 
эквивалентности и постройте граф отношения:
\begin{enumerate} [a)]\setcounter{enumi}{0}
\item $A = \{-10, -9, … , 9, 10\}$ и отношение $R = \{(a,b)|a^{2} = b^{2}\}$
\item $A = \{a, b, c, d, p, t\}$ задано отношение $R = \{(a, a), (b, b), (b, c), (b, d), (c, b), (c, c), (c, d), (d, b), (d, c), (d, d), (p,p), (t,t)\}$
\item Пусть A – множество имен. $A = \{ $Алексей, Иван, Петр, Александр, Павел, Андрей$ \}$. Тогда отношение $R$ верно на парах имен, начинающихся с одной и той же буквы, и только на них.
\end{enumerate}\question Составьте полную таблицу истинности, определите, какие переменные являются фиктивными и проверьте, является ли формула тавтологией:
$ P \rightarrow (Q \rightarrow ((P \lor Q) \rightarrow (P \land Q)))$

\end{questions}
\newpage
%%% begin test
\begin{flushright}
\begin{tabular}{p{2.8in} r l}
%\textbf{\class} & \textbf{ФИО:} & \makebox[2.5in]{\hrulefill}\\
\textbf{\class} & \textbf{ФИО:} &Кошкин Михаил Сергеевич
\\

\textbf{\examdate} &&\\
%\textbf{Time Limit: \timelimit} & Teaching Assistant & \makebox[2in]{\hrulefill}
\end{tabular}\\
\end{flushright}
\rule[1ex]{\textwidth}{.1pt}


\begin{questions}
\question
Найдите и упростите P:
\begin{equation*}
\overline{P} = A \cap \overline{B} \cup \overline{B} \cap C \cup \overline{A} \cap \overline{B} \cup \overline{A} \cap C
\end{equation*}
Затем найдите элементы множества P, выраженного через множества:
\begin{equation*}
A = \{0, 3, 4, 9\}; 
B = \{1, 3, 4, 7\};
C = \{0, 1, 2, 4, 7, 8, 9\};
I = \{0, 1, 2, 3, 4, 5, 6, 7, 8, 9\}.
\end{equation*}\question
Упростите следующее выражение с учетом того, что $A\subset B \subset C \subset D \subset U; A \neq \O$
\begin{equation*}
\overline{B} \cap \overline{C} \cap D \cup \overline{A} \cap \overline{C} \cap D \cup \overline{A} \cap B
\end{equation*}

Примечание: U — универсум\question
Дано отношение на множестве $\{1, 2, 3, 4, 5\}$ 
\begin{equation*}
aRb \iff a \leq b
\end{equation*}
Напишите обоснованный ответ какими свойствами обладает или не обладает отношение и почему:   
\begin{enumerate} [a)]\setcounter{enumi}{0}
\item рефлексивность
\item антирефлексивность
\item симметричность
\item асимметричность
\item антисимметричность
\item транзитивность
\end{enumerate}

Обоснуйте свой ответ по каждому из приведенных ниже вопросов:
\begin{enumerate} [a)]\setcounter{enumi}{0}
    \item Является ли это отношение отношением эквивалентности?
    \item Является ли это отношение функциональным?
    \item Каким из отношений соответствия (одно-многозначным, много-многозначный и т.д.) оно является?
    \item К каким из отношений порядка (полного, частичного и т.д.) можно отнести данное отношение?
\end{enumerate}


\question
Установите, является ли каждое из перечисленных ниже отношений на А ($R \subseteq A \times A$) отношением эквивалентности (обоснование ответа обязательно). Для каждого отношения эквивалентности постройте классы 
эквивалентности и постройте граф отношения:
\begin{enumerate} [a)]\setcounter{enumi}{0}
\item $A = \{-10, -9, … , 9, 10\}$ и отношение $R = \{(a,b)|a^{2} = b^{2}\}$
\item $A = \{a, b, c, d, p, t\}$ задано отношение $R = \{(a, a), (b, b), (b, c), (b, d), (c, b), (c, c), (c, d), (d, b), (d, c), (d, d), (p,p), (t,t)\}$
\item Пусть A – множество имен. $A = \{ $Алексей, Иван, Петр, Александр, Павел, Андрей$ \}$. Тогда отношение $R$ верно на парах имен, начинающихся с одной и той же буквы, и только на них.
\end{enumerate}\question Составьте полную таблицу истинности, определите, какие переменные являются фиктивными и проверьте, является ли формула тавтологией:
$(P \rightarrow (Q \rightarrow R)) \rightarrow ((P \rightarrow Q) \rightarrow (P \rightarrow R))$

\end{questions}
\newpage
%%% begin test
\begin{flushright}
\begin{tabular}{p{2.8in} r l}
%\textbf{\class} & \textbf{ФИО:} & \makebox[2.5in]{\hrulefill}\\
\textbf{\class} & \textbf{ФИО:} &Красильников Михаил Александрович
\\

\textbf{\examdate} &&\\
%\textbf{Time Limit: \timelimit} & Teaching Assistant & \makebox[2in]{\hrulefill}
\end{tabular}\\
\end{flushright}
\rule[1ex]{\textwidth}{.1pt}


\begin{questions}
\question
Найдите и упростите P:
\begin{equation*}
\overline{P} = A \cap \overline{B} \cup A \cap C \cup B \cap C \cup \overline{A} \cap C
\end{equation*}
Затем найдите элементы множества P, выраженного через множества:
\begin{equation*}
A = \{0, 3, 4, 9\}; 
B = \{1, 3, 4, 7\};
C = \{0, 1, 2, 4, 7, 8, 9\};
I = \{0, 1, 2, 3, 4, 5, 6, 7, 8, 9\}.
\end{equation*}\question
Упростите следующее выражение с учетом того, что $A\subset B \subset C \subset D \subset U; A \neq \O$
\begin{equation*}
A \cap C  \cap D \cup B \cap \overline{C} \cap D \cup B \cap C \cap D
\end{equation*}

Примечание: U — универсум\question
Дано отношение на множестве $\{1, 2, 3, 4, 5\}$ 
\begin{equation*}
aRb \iff a \geq b^2
\end{equation*}
Напишите обоснованный ответ какими свойствами обладает или не обладает отношение и почему:   
\begin{enumerate} [a)]\setcounter{enumi}{0}
\item рефлексивность
\item антирефлексивность
\item симметричность
\item асимметричность
\item антисимметричность
\item транзитивность
\end{enumerate}

Обоснуйте свой ответ по каждому из приведенных ниже вопросов:
\begin{enumerate} [a)]\setcounter{enumi}{0}
    \item Является ли это отношение отношением эквивалентности?
    \item Является ли это отношение функциональным?
    \item Каким из отношений соответствия (одно-многозначным, много-многозначный и т.д.) оно является?
    \item К каким из отношений порядка (полного, частичного и т.д.) можно отнести данное отношение?
\end{enumerate}


\question
Установите, является ли каждое из перечисленных ниже отношений на А ($R \subseteq A \times A$) отношением эквивалентности (обоснование ответа обязательно). Для каждого отношения эквивалентности постройте классы 
эквивалентности и постройте граф отношения:
\begin{enumerate} [a)]\setcounter{enumi}{0}
\item А - множество целых чисел и отношение $R = \{(a,b)|a + b = 5\}$
\item Пусть A – множество имен. $A = \{ $Алексей, Иван, Петр, Александр, Павел, Андрей$ \}$. Тогда отношение $R $ верно на парах имен, начинающихся с одной и той же буквы, и только на них.
\item На множестве $A = \{1; 2; 3; 4; 5\}$ задано отношение $R = \{(1; 2); (1; 3); (1; 5); (2; 3); (2; 4); (2; 5); (3; 4); (3; 5); (4; 5)\}$
\end{enumerate}\question Составьте полную таблицу истинности, определите, какие переменные являются фиктивными и проверьте, является ли формула тавтологией:
$(( P \rightarrow Q) \land (Q \rightarrow P)) \rightarrow (P \rightarrow R)$

\end{questions}
\newpage
%%% begin test
\begin{flushright}
\begin{tabular}{p{2.8in} r l}
%\textbf{\class} & \textbf{ФИО:} & \makebox[2.5in]{\hrulefill}\\
\textbf{\class} & \textbf{ФИО:} &Лавров Дмитрий Антонович
\\

\textbf{\examdate} &&\\
%\textbf{Time Limit: \timelimit} & Teaching Assistant & \makebox[2in]{\hrulefill}
\end{tabular}\\
\end{flushright}
\rule[1ex]{\textwidth}{.1pt}


\begin{questions}
\question
Найдите и упростите P:
\begin{equation*}
\overline{P} = A \cap C \cup \overline{A} \cap \overline{C} \cup \overline{B} \cap C \cup \overline{A} \cap \overline{B}
\end{equation*}
Затем найдите элементы множества P, выраженного через множества:
\begin{equation*}
A = \{0, 3, 4, 9\}; 
B = \{1, 3, 4, 7\};
C = \{0, 1, 2, 4, 7, 8, 9\};
I = \{0, 1, 2, 3, 4, 5, 6, 7, 8, 9\}.
\end{equation*}\question
Упростите следующее выражение с учетом того, что $A\subset B \subset C \subset D \subset U; A \neq \O$
\begin{equation*}
\overline{B} \cap \overline{C} \cap D \cup \overline{A} \cap \overline{C} \cap D \cup \overline{A} \cap B
\end{equation*}

Примечание: U — универсум\question
Дано отношение на множестве $\{1, 2, 3, 4, 5\}$ 
\begin{equation*}
aRb \iff (a+b) \bmod 2 =0
\end{equation*}
Напишите обоснованный ответ какими свойствами обладает или не обладает отношение и почему:   
\begin{enumerate} [a)]\setcounter{enumi}{0}
\item рефлексивность
\item антирефлексивность
\item симметричность
\item асимметричность
\item антисимметричность
\item транзитивность
\end{enumerate}

Обоснуйте свой ответ по каждому из приведенных ниже вопросов:
\begin{enumerate} [a)]\setcounter{enumi}{0}
    \item Является ли это отношение отношением эквивалентности?
    \item Является ли это отношение функциональным?
    \item Каким из отношений соответствия (одно-многозначным, много-многозначный и т.д.) оно является?
    \item К каким из отношений порядка (полного, частичного и т.д.) можно отнести данное отношение?
\end{enumerate}



\question
Установите, является ли каждое из перечисленных ниже отношений на А ($R \subseteq A \times A$) отношением эквивалентности (обоснование ответа обязательно). Для каждого отношения эквивалентности постройте классы 
эквивалентности и постройте граф отношения:
\begin{enumerate} [a)]\setcounter{enumi}{0}
\item А - множество целых чисел и отношение $R = \{(a,b)|a + b = 5\}$
\item Пусть A – множество имен. $A = \{ $Алексей, Иван, Петр, Александр, Павел, Андрей$ \}$. Тогда отношение $R $ верно на парах имен, начинающихся с одной и той же буквы, и только на них.
\item На множестве $A = \{1; 2; 3; 4; 5\}$ задано отношение $R = \{(1; 2); (1; 3); (1; 5); (2; 3); (2; 4); (2; 5); (3; 4); (3; 5); (4; 5)\}$
\end{enumerate}\question Составьте полную таблицу истинности, определите, какие переменные являются фиктивными и проверьте, является ли формула тавтологией:
$ P \rightarrow (Q \rightarrow ((P \lor Q) \rightarrow (P \land Q)))$

\end{questions}
\newpage
%%% begin test
\begin{flushright}
\begin{tabular}{p{2.8in} r l}
%\textbf{\class} & \textbf{ФИО:} & \makebox[2.5in]{\hrulefill}\\
\textbf{\class} & \textbf{ФИО:} &Малыгин Семён Олегович
\\

\textbf{\examdate} &&\\
%\textbf{Time Limit: \timelimit} & Teaching Assistant & \makebox[2in]{\hrulefill}
\end{tabular}\\
\end{flushright}
\rule[1ex]{\textwidth}{.1pt}


\begin{questions}
\question
Найдите и упростите P:
\begin{equation*}
\overline{P} = A \cap \overline{B} \cup \overline{B} \cap C \cup \overline{A} \cap \overline{B} \cup \overline{A} \cap C
\end{equation*}
Затем найдите элементы множества P, выраженного через множества:
\begin{equation*}
A = \{0, 3, 4, 9\}; 
B = \{1, 3, 4, 7\};
C = \{0, 1, 2, 4, 7, 8, 9\};
I = \{0, 1, 2, 3, 4, 5, 6, 7, 8, 9\}.
\end{equation*}\question
Упростите следующее выражение с учетом того, что $A\subset B \subset C \subset D \subset U; A \neq \O$
\begin{equation*}
A \cap B  \cap \overline{C} \cup \overline{C} \cap D \cup B \cap C \cap D
\end{equation*}

Примечание: U — универсум\question
Дано отношение на множестве $\{1, 2, 3, 4, 5\}$ 
\begin{equation*}
aRb \iff |a-b| = 1
\end{equation*}
Напишите обоснованный ответ какими свойствами обладает или не обладает отношение и почему:   
\begin{enumerate} [a)]\setcounter{enumi}{0}
\item рефлексивность
\item антирефлексивность
\item симметричность
\item асимметричность
\item антисимметричность
\item транзитивность
\end{enumerate}

Обоснуйте свой ответ по каждому из приведенных ниже вопросов:
\begin{enumerate} [a)]\setcounter{enumi}{0}
    \item Является ли это отношение отношением эквивалентности?
    \item Является ли это отношение функциональным?
    \item Каким из отношений соответствия (одно-многозначным, много-многозначный и т.д.) оно является?
    \item К каким из отношений порядка (полного, частичного и т.д.) можно отнести данное отношение?
\end{enumerate}

\question
Установите, является ли каждое из перечисленных ниже отношений на А ($R \subseteq A \times A$) отношением эквивалентности (обоснование ответа обязательно). Для каждого отношения эквивалентности постройте классы 
эквивалентности и постройте граф отношения:
\begin{enumerate} [a)]\setcounter{enumi}{0}
\item $A = \{a, b, c, d, p, t\}$ задано отношение $R = \{(a, a), (b, b), (b, c), (b, d), (c, b), (c, c), (c, d), (d, b), (d, c), (d, d), (p,p), (t,t)\}$
\item $A = \{-10, -9, … , 9, 10\}$ и отношение $R = \{(a,b)|a^{3} = b^{3}\}$

\item $F(x)=x^{2}+1$, где $x \in A = [-2, 4]$ и отношение $R = \{(a,b)|F(a) = F(b)\}$
\end{enumerate}\question Составьте полную таблицу истинности, определите, какие переменные являются фиктивными и проверьте, является ли формула тавтологией:
$((P \rightarrow Q) \lor R) \leftrightarrow (P \rightarrow (Q \lor R))$

\end{questions}
\newpage
%%% begin test
\begin{flushright}
\begin{tabular}{p{2.8in} r l}
%\textbf{\class} & \textbf{ФИО:} & \makebox[2.5in]{\hrulefill}\\
\textbf{\class} & \textbf{ФИО:} &Митрофанова Анастасия Александровна
\\

\textbf{\examdate} &&\\
%\textbf{Time Limit: \timelimit} & Teaching Assistant & \makebox[2in]{\hrulefill}
\end{tabular}\\
\end{flushright}
\rule[1ex]{\textwidth}{.1pt}


\begin{questions}
\question
Найдите и упростите P:
\begin{equation*}
\overline{P} = A \cap B \cup \overline{A} \cap \overline{B} \cup A \cap C \cup \overline{B} \cap C
\end{equation*}
Затем найдите элементы множества P, выраженного через множества:
\begin{equation*}
A = \{0, 3, 4, 9\}; 
B = \{1, 3, 4, 7\};
C = \{0, 1, 2, 4, 7, 8, 9\};
I = \{0, 1, 2, 3, 4, 5, 6, 7, 8, 9\}.
\end{equation*}\question
Упростите следующее выражение с учетом того, что $A\subset B \subset C \subset D \subset U; A \neq \O$
\begin{equation*}
A \cap B \cup \overline{A} \cap \overline{C} \cup A \cap C \cup \overline{B} \cap \overline{C}
\end{equation*}

Примечание: U — универсум\question
Дано отношение на множестве $\{1, 2, 3, 4, 5\}$ 
\begin{equation*}
aRb \iff |a-b| = 1
\end{equation*}
Напишите обоснованный ответ какими свойствами обладает или не обладает отношение и почему:   
\begin{enumerate} [a)]\setcounter{enumi}{0}
\item рефлексивность
\item антирефлексивность
\item симметричность
\item асимметричность
\item антисимметричность
\item транзитивность
\end{enumerate}

Обоснуйте свой ответ по каждому из приведенных ниже вопросов:
\begin{enumerate} [a)]\setcounter{enumi}{0}
    \item Является ли это отношение отношением эквивалентности?
    \item Является ли это отношение функциональным?
    \item Каким из отношений соответствия (одно-многозначным, много-многозначный и т.д.) оно является?
    \item К каким из отношений порядка (полного, частичного и т.д.) можно отнести данное отношение?
\end{enumerate}

\question
Установите, является ли каждое из перечисленных ниже отношений на А ($R \subseteq A \times A$) отношением эквивалентности (обоснование ответа обязательно). Для каждого отношения эквивалентности постройте классы 
эквивалентности и постройте граф отношения:
\begin{enumerate} [a)]\setcounter{enumi}{0}
\item $A = \{-10, -9, … , 9, 10\}$ и отношение $R = \{(a,b)|a^{2} = b^{2}\}$
\item $A = \{a, b, c, d, p, t\}$ задано отношение $R = \{(a, a), (b, b), (b, c), (b, d), (c, b), (c, c), (c, d), (d, b), (d, c), (d, d), (p,p), (t,t)\}$
\item Пусть A – множество имен. $A = \{ $Алексей, Иван, Петр, Александр, Павел, Андрей$ \}$. Тогда отношение $R$ верно на парах имен, начинающихся с одной и той же буквы, и только на них.
\end{enumerate}\question Составьте полную таблицу истинности, определите, какие переменные являются фиктивными и проверьте, является ли формула тавтологией:
$(P \rightarrow (Q \rightarrow R)) \rightarrow ((P \rightarrow Q) \rightarrow (P \rightarrow R))$

\end{questions}
\newpage
%%% begin test
\begin{flushright}
\begin{tabular}{p{2.8in} r l}
%\textbf{\class} & \textbf{ФИО:} & \makebox[2.5in]{\hrulefill}\\
\textbf{\class} & \textbf{ФИО:} &Нагорянский Андрей Дмитриевич
\\

\textbf{\examdate} &&\\
%\textbf{Time Limit: \timelimit} & Teaching Assistant & \makebox[2in]{\hrulefill}
\end{tabular}\\
\end{flushright}
\rule[1ex]{\textwidth}{.1pt}


\begin{questions}
\question
Найдите и упростите P:
\begin{equation*}
\overline{P} = A \cap C \cup \overline{A} \cap \overline{C} \cup \overline{B} \cap C \cup \overline{A} \cap \overline{B}
\end{equation*}
Затем найдите элементы множества P, выраженного через множества:
\begin{equation*}
A = \{0, 3, 4, 9\}; 
B = \{1, 3, 4, 7\};
C = \{0, 1, 2, 4, 7, 8, 9\};
I = \{0, 1, 2, 3, 4, 5, 6, 7, 8, 9\}.
\end{equation*}\question
Упростите следующее выражение с учетом того, что $A\subset B \subset C \subset D \subset U; A \neq \O$
\begin{equation*}
\overline{B} \cap \overline{C} \cap D \cup \overline{A} \cap \overline{C} \cap D \cup \overline{A} \cap B
\end{equation*}

Примечание: U — универсум\question
Дано отношение на множестве $\{1, 2, 3, 4, 5\}$ 
\begin{equation*}
aRb \iff a \geq b^2
\end{equation*}
Напишите обоснованный ответ какими свойствами обладает или не обладает отношение и почему:   
\begin{enumerate} [a)]\setcounter{enumi}{0}
\item рефлексивность
\item антирефлексивность
\item симметричность
\item асимметричность
\item антисимметричность
\item транзитивность
\end{enumerate}

Обоснуйте свой ответ по каждому из приведенных ниже вопросов:
\begin{enumerate} [a)]\setcounter{enumi}{0}
    \item Является ли это отношение отношением эквивалентности?
    \item Является ли это отношение функциональным?
    \item Каким из отношений соответствия (одно-многозначным, много-многозначный и т.д.) оно является?
    \item К каким из отношений порядка (полного, частичного и т.д.) можно отнести данное отношение?
\end{enumerate}


\question
Установите, является ли каждое из перечисленных ниже отношений на А ($R \subseteq A \times A$) отношением эквивалентности (обоснование ответа обязательно). Для каждого отношения эквивалентности постройте классы 
эквивалентности и постройте граф отношения:
\begin{enumerate} [a)]\setcounter{enumi}{0}
\item $A = \{-10, -9, … , 9, 10\}$ и отношение $R = \{(a,b)|a^{2} = b^{2}\}$
\item $A = \{a, b, c, d, p, t\}$ задано отношение $R = \{(a, a), (b, b), (b, c), (b, d), (c, b), (c, c), (c, d), (d, b), (d, c), (d, d), (p,p), (t,t)\}$
\item Пусть A – множество имен. $A = \{ $Алексей, Иван, Петр, Александр, Павел, Андрей$ \}$. Тогда отношение $R$ верно на парах имен, начинающихся с одной и той же буквы, и только на них.
\end{enumerate}\question Составьте полную таблицу истинности, определите, какие переменные являются фиктивными и проверьте, является ли формула тавтологией:
$(( P \land \neg Q) \rightarrow (R \land \neg R)) \rightarrow (P \rightarrow Q)$

\end{questions}
\newpage
%%% begin test
\begin{flushright}
\begin{tabular}{p{2.8in} r l}
%\textbf{\class} & \textbf{ФИО:} & \makebox[2.5in]{\hrulefill}\\
\textbf{\class} & \textbf{ФИО:} &Носачёв Виталий Игоревич
\\

\textbf{\examdate} &&\\
%\textbf{Time Limit: \timelimit} & Teaching Assistant & \makebox[2in]{\hrulefill}
\end{tabular}\\
\end{flushright}
\rule[1ex]{\textwidth}{.1pt}


\begin{questions}
\question
Найдите и упростите P:
\begin{equation*}
\overline{P} = A \cap \overline{B} \cup A \cap C \cup B \cap C \cup \overline{A} \cap C
\end{equation*}
Затем найдите элементы множества P, выраженного через множества:
\begin{equation*}
A = \{0, 3, 4, 9\}; 
B = \{1, 3, 4, 7\};
C = \{0, 1, 2, 4, 7, 8, 9\};
I = \{0, 1, 2, 3, 4, 5, 6, 7, 8, 9\}.
\end{equation*}\question
Упростите следующее выражение с учетом того, что $A\subset B \subset C \subset D \subset U; A \neq \O$
\begin{equation*}
A \cap  \overline{C} \cup B \cap \overline{D} \cup  \overline{A} \cap C \cap  \overline{D}
\end{equation*}

Примечание: U — универсум\question
Дано отношение на множестве $\{1, 2, 3, 4, 5\}$ 
\begin{equation*}
aRb \iff (a+b) \bmod 2 =0
\end{equation*}
Напишите обоснованный ответ какими свойствами обладает или не обладает отношение и почему:   
\begin{enumerate} [a)]\setcounter{enumi}{0}
\item рефлексивность
\item антирефлексивность
\item симметричность
\item асимметричность
\item антисимметричность
\item транзитивность
\end{enumerate}

Обоснуйте свой ответ по каждому из приведенных ниже вопросов:
\begin{enumerate} [a)]\setcounter{enumi}{0}
    \item Является ли это отношение отношением эквивалентности?
    \item Является ли это отношение функциональным?
    \item Каким из отношений соответствия (одно-многозначным, много-многозначный и т.д.) оно является?
    \item К каким из отношений порядка (полного, частичного и т.д.) можно отнести данное отношение?
\end{enumerate}



\question
Установите, является ли каждое из перечисленных ниже отношений на А ($R \subseteq A \times A$) отношением эквивалентности (обоснование ответа обязательно). Для каждого отношения эквивалентности постройте классы 
эквивалентности и постройте граф отношения:
\begin{enumerate} [a)]\setcounter{enumi}{0}
\item $A = \{-10, -9, … , 9, 10\}$ и отношение $R = \{(a,b)|a^{2} = b^{2}\}$
\item $A = \{a, b, c, d, p, t\}$ задано отношение $R = \{(a, a), (b, b), (b, c), (b, d), (c, b), (c, c), (c, d), (d, b), (d, c), (d, d), (p,p), (t,t)\}$
\item Пусть A – множество имен. $A = \{ $Алексей, Иван, Петр, Александр, Павел, Андрей$ \}$. Тогда отношение $R$ верно на парах имен, начинающихся с одной и той же буквы, и только на них.
\end{enumerate}\question Составьте полную таблицу истинности, определите, какие переменные являются фиктивными и проверьте, является ли формула тавтологией:
$(P \rightarrow (Q \rightarrow R)) \rightarrow ((P \rightarrow Q) \rightarrow (P \rightarrow R))$

\end{questions}
\newpage
%%% begin test
\begin{flushright}
\begin{tabular}{p{2.8in} r l}
%\textbf{\class} & \textbf{ФИО:} & \makebox[2.5in]{\hrulefill}\\
\textbf{\class} & \textbf{ФИО:} &Степанова Анна Никитовна
\\

\textbf{\examdate} &&\\
%\textbf{Time Limit: \timelimit} & Teaching Assistant & \makebox[2in]{\hrulefill}
\end{tabular}\\
\end{flushright}
\rule[1ex]{\textwidth}{.1pt}


\begin{questions}
\question
Найдите и упростите P:
\begin{equation*}
\overline{P} = A \cap \overline{B} \cup \overline{B} \cap C \cup \overline{A} \cap \overline{B} \cup \overline{A} \cap C
\end{equation*}
Затем найдите элементы множества P, выраженного через множества:
\begin{equation*}
A = \{0, 3, 4, 9\}; 
B = \{1, 3, 4, 7\};
C = \{0, 1, 2, 4, 7, 8, 9\};
I = \{0, 1, 2, 3, 4, 5, 6, 7, 8, 9\}.
\end{equation*}\question
Упростите следующее выражение с учетом того, что $A\subset B \subset C \subset D \subset U; A \neq \O$
\begin{equation*}
A \cap B  \cap \overline{C} \cup \overline{C} \cap D \cup B \cap C \cap D
\end{equation*}

Примечание: U — универсум\question
Для следующего отношения на множестве $\{1, 2, 3, 4, 5\}$ 
\begin{equation*}
aRb \iff 0 < a-b<2
\end{equation*}
Напишите обоснованный ответ какими свойствами обладает или не обладает отношение и почему:   
\begin{enumerate} [a)]\setcounter{enumi}{0}
\item рефлексивность
\item антирефлексивность
\item симметричность
\item асимметричность
\item антисимметричность
\item транзитивность
\end{enumerate}

Обоснуйте свой ответ по каждому из приведенных ниже вопросов:
\begin{enumerate} [a)]\setcounter{enumi}{0}
    \item Является ли это отношение отношением эквивалентности?
    \item Является ли это отношение функциональным?
    \item Каким из отношений соответствия (одно-многозначным, много-многозначный и т.д.) оно является?
    \item К каким из отношений порядка (полного, частичного и т.д.) можно отнести данное отношение?
\end{enumerate}
\question
Установите, является ли каждое из перечисленных ниже отношений на А ($R \subseteq A \times A$) отношением эквивалентности (обоснование ответа обязательно). Для каждого отношения эквивалентности постройте классы 
эквивалентности и постройте граф отношения:
\begin{enumerate} [a)]\setcounter{enumi}{0}
\item Пусть A – множество имен. $A = \{ $Алексей, Иван, Петр, Александр, Павел, Андрей$ \}$. Тогда отношение $R$ верно на парах имен, начинающихся с одной и той же буквы, и только на них.
\item $A = \{-10, -9, … , 9, 10\}$ и отношение $ R = \{(a,b)|a^{2} = b^{2}\}$
\item На множестве $A = \{1; 2; 3\}$ задано отношение $R = \{(1; 1); (2; 2); (3; 3); (3; 2); (1; 2); (2; 1)\}$
\end{enumerate}\question Составьте полную таблицу истинности, определите, какие переменные являются фиктивными и проверьте, является ли формула тавтологией:
$ P \rightarrow (Q \rightarrow ((P \lor Q) \rightarrow (P \land Q)))$

\end{questions}
\newpage
%%% begin test
\begin{flushright}
\begin{tabular}{p{2.8in} r l}
%\textbf{\class} & \textbf{ФИО:} & \makebox[2.5in]{\hrulefill}\\
\textbf{\class} & \textbf{ФИО:} &Титов Даниил Ярославович
\\

\textbf{\examdate} &&\\
%\textbf{Time Limit: \timelimit} & Teaching Assistant & \makebox[2in]{\hrulefill}
\end{tabular}\\
\end{flushright}
\rule[1ex]{\textwidth}{.1pt}


\begin{questions}
\question
Найдите и упростите P:
\begin{equation*}
\overline{P} = A \cap \overline{C} \cup A \cap \overline{B} \cup B \cap \overline{C} \cup A \cap C
\end{equation*}
Затем найдите элементы множества P, выраженного через множества:
\begin{equation*}
A = \{0, 3, 4, 9\}; 
B = \{1, 3, 4, 7\};
C = \{0, 1, 2, 4, 7, 8, 9\};
I = \{0, 1, 2, 3, 4, 5, 6, 7, 8, 9\}.
\end{equation*}\question
Упростите следующее выражение с учетом того, что $A\subset B \subset C \subset D \subset U; A \neq \O$
\begin{equation*}
\overline{A} \cap \overline{B} \cup B \cap \overline{C} \cup \overline{C} \cap D
\end{equation*}

Примечание: U — универсум\question
Дано отношение на множестве $\{1, 2, 3, 4, 5\}$ 
\begin{equation*}
aRb \iff b > a
\end{equation*}
Напишите обоснованный ответ какими свойствами обладает или не обладает отношение и почему:   
\begin{enumerate} [a)]\setcounter{enumi}{0}
\item рефлексивность
\item антирефлексивность
\item симметричность
\item асимметричность
\item антисимметричность
\item транзитивность
\end{enumerate}

Обоснуйте свой ответ по каждому из приведенных ниже вопросов:
\begin{enumerate} [a)]\setcounter{enumi}{0}
    \item Является ли это отношение отношением эквивалентности?
    \item Является ли это отношение функциональным?
    \item Каким из отношений соответствия (одно-многозначным, много-многозначный и т.д.) оно является?
    \item К каким из отношений порядка (полного, частичного и т.д.) можно отнести данное отношение?
\end{enumerate}

\question
Установите, является ли каждое из перечисленных ниже отношений на А ($R \subseteq A \times A$) отношением эквивалентности (обоснование ответа обязательно). Для каждого отношения эквивалентности постройте классы эквивалентности и постройте граф отношения:
\begin{enumerate} [a)]\setcounter{enumi}{0}
\item $F(x)=x^{2}+1$, где $x \in A = [-2, 4]$ и отношение $R = \{(a,b)|F(a) = F(b)\}$
\item А - множество целых чисел и отношение $R = \{(a,b)|a + b = 5\}$
\item На множестве $A = \{1; 2; 3\}$ задано отношение $R = \{(1; 1); (2; 2); (3; 3); (3; 2); (1; 2); (2; 1)\}$

\end{enumerate}\question Составьте полную таблицу истинности, определите, какие переменные являются фиктивными и проверьте, является ли формула тавтологией:
$(( P \land \neg Q) \rightarrow (R \land \neg R)) \rightarrow (P \rightarrow Q)$

\end{questions}
\newpage
%%% begin test
\begin{flushright}
\begin{tabular}{p{2.8in} r l}
%\textbf{\class} & \textbf{ФИО:} & \makebox[2.5in]{\hrulefill}\\
\textbf{\class} & \textbf{ФИО:} &Шацилло Александр Андреевич
\\

\textbf{\examdate} &&\\
%\textbf{Time Limit: \timelimit} & Teaching Assistant & \makebox[2in]{\hrulefill}
\end{tabular}\\
\end{flushright}
\rule[1ex]{\textwidth}{.1pt}


\begin{questions}
\question
Найдите и упростите P:
\begin{equation*}
\overline{P} = \overline{A} \cap B \cup \overline{A} \cap C \cup A \cap \overline{B} \cup \overline{B} \cap C
\end{equation*}
Затем найдите элементы множества P, выраженного через множества:
\begin{equation*}
A = \{0, 3, 4, 9\}; 
B = \{1, 3, 4, 7\};
C = \{0, 1, 2, 4, 7, 8, 9\};
I = \{0, 1, 2, 3, 4, 5, 6, 7, 8, 9\}.
\end{equation*}\question
Упростите следующее выражение с учетом того, что $A\subset B \subset C \subset D \subset U; A \neq \O$
\begin{equation*}
A \cap B  \cap \overline{C} \cup \overline{C} \cap D \cup B \cap C \cap D
\end{equation*}

Примечание: U — универсум\question
Дано отношение на множестве $\{1, 2, 3, 4, 5\}$ 
\begin{equation*}
aRb \iff a \leq b
\end{equation*}
Напишите обоснованный ответ какими свойствами обладает или не обладает отношение и почему:   
\begin{enumerate} [a)]\setcounter{enumi}{0}
\item рефлексивность
\item антирефлексивность
\item симметричность
\item асимметричность
\item антисимметричность
\item транзитивность
\end{enumerate}

Обоснуйте свой ответ по каждому из приведенных ниже вопросов:
\begin{enumerate} [a)]\setcounter{enumi}{0}
    \item Является ли это отношение отношением эквивалентности?
    \item Является ли это отношение функциональным?
    \item Каким из отношений соответствия (одно-многозначным, много-многозначный и т.д.) оно является?
    \item К каким из отношений порядка (полного, частичного и т.д.) можно отнести данное отношение?
\end{enumerate}


\question
Установите, является ли каждое из перечисленных ниже отношений на А ($R \subseteq A \times A$) отношением эквивалентности (обоснование ответа обязательно). Для каждого отношения эквивалентности постройте классы 
эквивалентности и постройте граф отношения:
\begin{enumerate} [a)]\setcounter{enumi}{0}
\item На множестве $A = \{1; 2; 3\}$ задано отношение $R = \{(1; 1); (2; 2); (3; 3); (2; 1); (1; 2); (2; 3); (3; 2); (3; 1); (1; 3)\}$
\item На множестве $A = \{1; 2; 3; 4; 5\}$ задано отношение $R = \{(1; 2); (1; 3); (1; 5); (2; 3); (2; 4); (2; 5); (3; 4); (3; 5); (4; 5)\}$
\item А - множество целых чисел и отношение $R = \{(a,b)|a + b = 0\}$
\end{enumerate}\question Составьте полную таблицу истинности, определите, какие переменные являются фиктивными и проверьте, является ли формула тавтологией:
$((P \rightarrow Q) \land (R \rightarrow S) \land \neg (Q \lor S)) \rightarrow \neg (P \lor R)$

\end{questions}
\newpage
%%% begin test
\begin{flushright}
\begin{tabular}{p{2.8in} r l}
%\textbf{\class} & \textbf{ФИО:} & \makebox[2.5in]{\hrulefill}\\
\textbf{\class} & \textbf{ФИО:} &М3105
\\

\textbf{\examdate} &&\\
%\textbf{Time Limit: \timelimit} & Teaching Assistant & \makebox[2in]{\hrulefill}
\end{tabular}\\
\end{flushright}
\rule[1ex]{\textwidth}{.1pt}


\begin{questions}
\question
Найдите и упростите P:
\begin{equation*}
\overline{P} = A \cap \overline{B} \cup \overline{B} \cap C \cup \overline{A} \cap \overline{B} \cup \overline{A} \cap C
\end{equation*}
Затем найдите элементы множества P, выраженного через множества:
\begin{equation*}
A = \{0, 3, 4, 9\}; 
B = \{1, 3, 4, 7\};
C = \{0, 1, 2, 4, 7, 8, 9\};
I = \{0, 1, 2, 3, 4, 5, 6, 7, 8, 9\}.
\end{equation*}\question
Упростите следующее выражение с учетом того, что $A\subset B \subset C \subset D \subset U; A \neq \O$
\begin{equation*}
\overline{A} \cap \overline{B} \cup B \cap \overline{C} \cup \overline{C} \cap D
\end{equation*}

Примечание: U — универсум\question
Дано отношение на множестве $\{1, 2, 3, 4, 5\}$ 
\begin{equation*}
aRb \iff  \text{НОД}(a,b) =1
\end{equation*}
Напишите обоснованный ответ какими свойствами обладает или не обладает отношение и почему:   
\begin{enumerate} [a)]\setcounter{enumi}{0}
\item рефлексивность
\item антирефлексивность
\item симметричность
\item асимметричность
\item антисимметричность
\item транзитивность
\end{enumerate}

Обоснуйте свой ответ по каждому из приведенных ниже вопросов:
\begin{enumerate} [a)]\setcounter{enumi}{0}
    \item Является ли это отношение отношением эквивалентности?
    \item Является ли это отношение функциональным?
    \item Каким из отношений соответствия (одно-многозначным, много-многозначный и т.д.) оно является?
    \item К каким из отношений порядка (полного, частичного и т.д.) можно отнести данное отношение?
\end{enumerate}


\question
Установите, является ли каждое из перечисленных ниже отношений на А ($R \subseteq A \times A$) отношением эквивалентности (обоснование ответа обязательно). Для каждого отношения эквивалентности постройте классы эквивалентности и постройте граф отношения:
\begin{enumerate} [a)]\setcounter{enumi}{0}
\item $F(x)=x^{2}+1$, где $x \in A = [-2, 4]$ и отношение $R = \{(a,b)|F(a) = F(b)\}$
\item А - множество целых чисел и отношение $R = \{(a,b)|a + b = 5\}$
\item На множестве $A = \{1; 2; 3\}$ задано отношение $R = \{(1; 1); (2; 2); (3; 3); (3; 2); (1; 2); (2; 1)\}$

\end{enumerate}\question Составьте полную таблицу истинности, определите, какие переменные являются фиктивными и проверьте, является ли формула тавтологией:
$(P \rightarrow (Q \rightarrow R)) \rightarrow ((P \rightarrow Q) \rightarrow (P \rightarrow R))$

\end{questions}
\newpage
%%% begin test
\begin{flushright}
\begin{tabular}{p{2.8in} r l}
%\textbf{\class} & \textbf{ФИО:} & \makebox[2.5in]{\hrulefill}\\
\textbf{\class} & \textbf{ФИО:} &Андреев Артём Русланович
\\

\textbf{\examdate} &&\\
%\textbf{Time Limit: \timelimit} & Teaching Assistant & \makebox[2in]{\hrulefill}
\end{tabular}\\
\end{flushright}
\rule[1ex]{\textwidth}{.1pt}


\begin{questions}
\question
Найдите и упростите P:
\begin{equation*}
\overline{P} = A \cap \overline{B} \cup \overline{B} \cap C \cup \overline{A} \cap \overline{B} \cup \overline{A} \cap C
\end{equation*}
Затем найдите элементы множества P, выраженного через множества:
\begin{equation*}
A = \{0, 3, 4, 9\}; 
B = \{1, 3, 4, 7\};
C = \{0, 1, 2, 4, 7, 8, 9\};
I = \{0, 1, 2, 3, 4, 5, 6, 7, 8, 9\}.
\end{equation*}\question
Упростите следующее выражение с учетом того, что $A\subset B \subset C \subset D \subset U; A \neq \O$
\begin{equation*}
A \cap B  \cap \overline{C} \cup \overline{C} \cap D \cup B \cap C \cap D
\end{equation*}

Примечание: U — универсум\question
Дано отношение на множестве $\{1, 2, 3, 4, 5\}$ 
\begin{equation*}
aRb \iff a \geq b^2
\end{equation*}
Напишите обоснованный ответ какими свойствами обладает или не обладает отношение и почему:   
\begin{enumerate} [a)]\setcounter{enumi}{0}
\item рефлексивность
\item антирефлексивность
\item симметричность
\item асимметричность
\item антисимметричность
\item транзитивность
\end{enumerate}

Обоснуйте свой ответ по каждому из приведенных ниже вопросов:
\begin{enumerate} [a)]\setcounter{enumi}{0}
    \item Является ли это отношение отношением эквивалентности?
    \item Является ли это отношение функциональным?
    \item Каким из отношений соответствия (одно-многозначным, много-многозначный и т.д.) оно является?
    \item К каким из отношений порядка (полного, частичного и т.д.) можно отнести данное отношение?
\end{enumerate}


\question
Установите, является ли каждое из перечисленных ниже отношений на А ($R \subseteq A \times A$) отношением эквивалентности (обоснование ответа обязательно). Для каждого отношения эквивалентности постройте классы 
эквивалентности и постройте граф отношения:
\begin{enumerate} [a)]\setcounter{enumi}{0}
\item $A = \{a, b, c, d, p, t\}$ задано отношение $R = \{(a, a), (b, b), (b, c), (b, d), (c, b), (c, c), (c, d), (d, b), (d, c), (d, d), (p,p), (t,t)\}$
\item $A = \{-10, -9, … , 9, 10\}$ и отношение $R = \{(a,b)|a^{3} = b^{3}\}$

\item $F(x)=x^{2}+1$, где $x \in A = [-2, 4]$ и отношение $R = \{(a,b)|F(a) = F(b)\}$
\end{enumerate}\question Составьте полную таблицу истинности, определите, какие переменные являются фиктивными и проверьте, является ли формула тавтологией:

$(P \rightarrow (Q \land R)) \leftrightarrow ((P \rightarrow Q) \land (P \rightarrow R))$

\end{questions}
\newpage
%%% begin test
\begin{flushright}
\begin{tabular}{p{2.8in} r l}
%\textbf{\class} & \textbf{ФИО:} & \makebox[2.5in]{\hrulefill}\\
\textbf{\class} & \textbf{ФИО:} &Астафьев Алексей Владиславович
\\

\textbf{\examdate} &&\\
%\textbf{Time Limit: \timelimit} & Teaching Assistant & \makebox[2in]{\hrulefill}
\end{tabular}\\
\end{flushright}
\rule[1ex]{\textwidth}{.1pt}


\begin{questions}
\question
Найдите и упростите P:
\begin{equation*}
\overline{P} = A \cap \overline{C} \cup A \cap \overline{B} \cup B \cap \overline{C} \cup A \cap C
\end{equation*}
Затем найдите элементы множества P, выраженного через множества:
\begin{equation*}
A = \{0, 3, 4, 9\}; 
B = \{1, 3, 4, 7\};
C = \{0, 1, 2, 4, 7, 8, 9\};
I = \{0, 1, 2, 3, 4, 5, 6, 7, 8, 9\}.
\end{equation*}\question
Упростите следующее выражение с учетом того, что $A\subset B \subset C \subset D \subset U; A \neq \O$
\begin{equation*}
\overline{A} \cap \overline{C} \cap D \cup \overline{B} \cap \overline{C} \cap D \cup A \cap B
\end{equation*}

Примечание: U — универсум\question
Дано отношение на множестве $\{1, 2, 3, 4, 5\}$ 
\begin{equation*}
aRb \iff b > a
\end{equation*}
Напишите обоснованный ответ какими свойствами обладает или не обладает отношение и почему:   
\begin{enumerate} [a)]\setcounter{enumi}{0}
\item рефлексивность
\item антирефлексивность
\item симметричность
\item асимметричность
\item антисимметричность
\item транзитивность
\end{enumerate}

Обоснуйте свой ответ по каждому из приведенных ниже вопросов:
\begin{enumerate} [a)]\setcounter{enumi}{0}
    \item Является ли это отношение отношением эквивалентности?
    \item Является ли это отношение функциональным?
    \item Каким из отношений соответствия (одно-многозначным, много-многозначный и т.д.) оно является?
    \item К каким из отношений порядка (полного, частичного и т.д.) можно отнести данное отношение?
\end{enumerate}

\question
Установите, является ли каждое из перечисленных ниже отношений на А ($R \subseteq A \times A$) отношением эквивалентности (обоснование ответа обязательно). Для каждого отношения эквивалентности постройте классы 
эквивалентности и постройте граф отношения:
\begin{enumerate} [a)]\setcounter{enumi}{0}
\item А - множество целых чисел и отношение $R = \{(a,b)|a + b = 5\}$
\item Пусть A – множество имен. $A = \{ $Алексей, Иван, Петр, Александр, Павел, Андрей$ \}$. Тогда отношение $R $ верно на парах имен, начинающихся с одной и той же буквы, и только на них.
\item На множестве $A = \{1; 2; 3; 4; 5\}$ задано отношение $R = \{(1; 2); (1; 3); (1; 5); (2; 3); (2; 4); (2; 5); (3; 4); (3; 5); (4; 5)\}$
\end{enumerate}\question Составьте полную таблицу истинности, определите, какие переменные являются фиктивными и проверьте, является ли формула тавтологией:
$(P \rightarrow (Q \rightarrow R)) \rightarrow ((P \rightarrow Q) \rightarrow (P \rightarrow R))$

\end{questions}
\newpage
%%% begin test
\begin{flushright}
\begin{tabular}{p{2.8in} r l}
%\textbf{\class} & \textbf{ФИО:} & \makebox[2.5in]{\hrulefill}\\
\textbf{\class} & \textbf{ФИО:} &Атакишиев Давид Вугарович
\\

\textbf{\examdate} &&\\
%\textbf{Time Limit: \timelimit} & Teaching Assistant & \makebox[2in]{\hrulefill}
\end{tabular}\\
\end{flushright}
\rule[1ex]{\textwidth}{.1pt}


\begin{questions}
\question
Найдите и упростите P:
\begin{equation*}
\overline{P} = A \cap \overline{B} \cup \overline{B} \cap C \cup \overline{A} \cap \overline{B} \cup \overline{A} \cap C
\end{equation*}
Затем найдите элементы множества P, выраженного через множества:
\begin{equation*}
A = \{0, 3, 4, 9\}; 
B = \{1, 3, 4, 7\};
C = \{0, 1, 2, 4, 7, 8, 9\};
I = \{0, 1, 2, 3, 4, 5, 6, 7, 8, 9\}.
\end{equation*}\question
Упростите следующее выражение с учетом того, что $A\subset B \subset C \subset D \subset U; A \neq \O$
\begin{equation*}
\overline{A} \cap \overline{B} \cup B \cap \overline{C} \cup \overline{C} \cap D
\end{equation*}

Примечание: U — универсум\question
Дано отношение на множестве $\{1, 2, 3, 4, 5\}$ 
\begin{equation*}
aRb \iff a \geq b^2
\end{equation*}
Напишите обоснованный ответ какими свойствами обладает или не обладает отношение и почему:   
\begin{enumerate} [a)]\setcounter{enumi}{0}
\item рефлексивность
\item антирефлексивность
\item симметричность
\item асимметричность
\item антисимметричность
\item транзитивность
\end{enumerate}

Обоснуйте свой ответ по каждому из приведенных ниже вопросов:
\begin{enumerate} [a)]\setcounter{enumi}{0}
    \item Является ли это отношение отношением эквивалентности?
    \item Является ли это отношение функциональным?
    \item Каким из отношений соответствия (одно-многозначным, много-многозначный и т.д.) оно является?
    \item К каким из отношений порядка (полного, частичного и т.д.) можно отнести данное отношение?
\end{enumerate}


\question
Установите, является ли каждое из перечисленных ниже отношений на А ($R \subseteq A \times A$) отношением эквивалентности (обоснование ответа обязательно). Для каждого отношения эквивалентности постройте классы 
эквивалентности и постройте граф отношения:
\begin{enumerate} [a)]\setcounter{enumi}{0}
\item $A = \{-10, -9, … , 9, 10\}$ и отношение $R = \{(a,b)|a^{2} = b^{2}\}$
\item $A = \{a, b, c, d, p, t\}$ задано отношение $R = \{(a, a), (b, b), (b, c), (b, d), (c, b), (c, c), (c, d), (d, b), (d, c), (d, d), (p,p), (t,t)\}$
\item Пусть A – множество имен. $A = \{ $Алексей, Иван, Петр, Александр, Павел, Андрей$ \}$. Тогда отношение $R$ верно на парах имен, начинающихся с одной и той же буквы, и только на них.
\end{enumerate}\question Составьте полную таблицу истинности, определите, какие переменные являются фиктивными и проверьте, является ли формула тавтологией:
$ P \rightarrow (Q \rightarrow ((P \lor Q) \rightarrow (P \land Q)))$

\end{questions}
\newpage
%%% begin test
\begin{flushright}
\begin{tabular}{p{2.8in} r l}
%\textbf{\class} & \textbf{ФИО:} & \makebox[2.5in]{\hrulefill}\\
\textbf{\class} & \textbf{ФИО:} &Байбуртян Виолетта Артуровна
\\

\textbf{\examdate} &&\\
%\textbf{Time Limit: \timelimit} & Teaching Assistant & \makebox[2in]{\hrulefill}
\end{tabular}\\
\end{flushright}
\rule[1ex]{\textwidth}{.1pt}


\begin{questions}
\question
Найдите и упростите P:
\begin{equation*}
\overline{P} = A \cap C \cup \overline{A} \cap \overline{C} \cup \overline{B} \cap C \cup \overline{A} \cap \overline{B}
\end{equation*}
Затем найдите элементы множества P, выраженного через множества:
\begin{equation*}
A = \{0, 3, 4, 9\}; 
B = \{1, 3, 4, 7\};
C = \{0, 1, 2, 4, 7, 8, 9\};
I = \{0, 1, 2, 3, 4, 5, 6, 7, 8, 9\}.
\end{equation*}\question
Упростите следующее выражение с учетом того, что $A\subset B \subset C \subset D \subset U; A \neq \O$
\begin{equation*}
A \cap B \cup \overline{A} \cap \overline{C} \cup A \cap C \cup \overline{B} \cap \overline{C}
\end{equation*}

Примечание: U — универсум\question
Дано отношение на множестве $\{1, 2, 3, 4, 5\}$ 
\begin{equation*}
aRb \iff (a+b) \bmod 2 =0
\end{equation*}
Напишите обоснованный ответ какими свойствами обладает или не обладает отношение и почему:   
\begin{enumerate} [a)]\setcounter{enumi}{0}
\item рефлексивность
\item антирефлексивность
\item симметричность
\item асимметричность
\item антисимметричность
\item транзитивность
\end{enumerate}

Обоснуйте свой ответ по каждому из приведенных ниже вопросов:
\begin{enumerate} [a)]\setcounter{enumi}{0}
    \item Является ли это отношение отношением эквивалентности?
    \item Является ли это отношение функциональным?
    \item Каким из отношений соответствия (одно-многозначным, много-многозначный и т.д.) оно является?
    \item К каким из отношений порядка (полного, частичного и т.д.) можно отнести данное отношение?
\end{enumerate}



\question
Установите, является ли каждое из перечисленных ниже отношений на А ($R \subseteq A \times A$) отношением эквивалентности (обоснование ответа обязательно). Для каждого отношения эквивалентности постройте классы эквивалентности и постройте граф отношения:
\begin{enumerate} [a)]\setcounter{enumi}{0}
\item $F(x)=x^{2}+1$, где $x \in A = [-2, 4]$ и отношение $R = \{(a,b)|F(a) = F(b)\}$
\item А - множество целых чисел и отношение $R = \{(a,b)|a + b = 5\}$
\item На множестве $A = \{1; 2; 3\}$ задано отношение $R = \{(1; 1); (2; 2); (3; 3); (3; 2); (1; 2); (2; 1)\}$

\end{enumerate}\question Составьте полную таблицу истинности, определите, какие переменные являются фиктивными и проверьте, является ли формула тавтологией:
$(P \rightarrow (Q \rightarrow R)) \rightarrow ((P \rightarrow Q) \rightarrow (P \rightarrow R))$

\end{questions}
\newpage
%%% begin test
\begin{flushright}
\begin{tabular}{p{2.8in} r l}
%\textbf{\class} & \textbf{ФИО:} & \makebox[2.5in]{\hrulefill}\\
\textbf{\class} & \textbf{ФИО:} &Бородин Прохор Алексеевич
\\

\textbf{\examdate} &&\\
%\textbf{Time Limit: \timelimit} & Teaching Assistant & \makebox[2in]{\hrulefill}
\end{tabular}\\
\end{flushright}
\rule[1ex]{\textwidth}{.1pt}


\begin{questions}
\question
Найдите и упростите P:
\begin{equation*}
\overline{P} = A \cap \overline{B} \cup \overline{B} \cap C \cup \overline{A} \cap \overline{B} \cup \overline{A} \cap C
\end{equation*}
Затем найдите элементы множества P, выраженного через множества:
\begin{equation*}
A = \{0, 3, 4, 9\}; 
B = \{1, 3, 4, 7\};
C = \{0, 1, 2, 4, 7, 8, 9\};
I = \{0, 1, 2, 3, 4, 5, 6, 7, 8, 9\}.
\end{equation*}\question
Упростите следующее выражение с учетом того, что $A\subset B \subset C \subset D \subset U; A \neq \O$
\begin{equation*}
\overline{A} \cap \overline{B} \cup B \cap \overline{C} \cup \overline{C} \cap D
\end{equation*}

Примечание: U — универсум\question
Для следующего отношения на множестве $\{1, 2, 3, 4, 5\}$ 
\begin{equation*}
aRb \iff 0 < a-b<2
\end{equation*}
Напишите обоснованный ответ какими свойствами обладает или не обладает отношение и почему:   
\begin{enumerate} [a)]\setcounter{enumi}{0}
\item рефлексивность
\item антирефлексивность
\item симметричность
\item асимметричность
\item антисимметричность
\item транзитивность
\end{enumerate}

Обоснуйте свой ответ по каждому из приведенных ниже вопросов:
\begin{enumerate} [a)]\setcounter{enumi}{0}
    \item Является ли это отношение отношением эквивалентности?
    \item Является ли это отношение функциональным?
    \item Каким из отношений соответствия (одно-многозначным, много-многозначный и т.д.) оно является?
    \item К каким из отношений порядка (полного, частичного и т.д.) можно отнести данное отношение?
\end{enumerate}
\question
Установите, является ли каждое из перечисленных ниже отношений на А ($R \subseteq A \times A$) отношением эквивалентности (обоснование ответа обязательно). Для каждого отношения эквивалентности 
постройте классы эквивалентности и постройте граф отношения:
\begin{enumerate}[a)]\setcounter{enumi}{0}
\item А - множество целых чисел и отношение $R = \{(a,b)|a + b = 0\}$
\item $A = \{-10, -9, …, 9, 10\}$ и отношение $R = \{(a,b)|a^{3} = b^{3}\}$
\item На множестве $A = \{1; 2; 3\}$ задано отношение $R = \{(1; 1); (2; 2); (3; 3); (2; 1); (1; 2); (2; 3); (3; 2); (3; 1); (1; 3)\}$

\end{enumerate}\question Составьте полную таблицу истинности, определите, какие переменные являются фиктивными и проверьте, является ли формула тавтологией:
$ P \rightarrow (Q \rightarrow ((P \lor Q) \rightarrow (P \land Q)))$

\end{questions}
\newpage
%%% begin test
\begin{flushright}
\begin{tabular}{p{2.8in} r l}
%\textbf{\class} & \textbf{ФИО:} & \makebox[2.5in]{\hrulefill}\\
\textbf{\class} & \textbf{ФИО:} &Васильев Артём Сергеевич
\\

\textbf{\examdate} &&\\
%\textbf{Time Limit: \timelimit} & Teaching Assistant & \makebox[2in]{\hrulefill}
\end{tabular}\\
\end{flushright}
\rule[1ex]{\textwidth}{.1pt}


\begin{questions}
\question
Найдите и упростите P:
\begin{equation*}
\overline{P} = B \cap \overline{C} \cup A \cap B \cup \overline{A} \cap C \cup \overline{A} \cap B
\end{equation*}
Затем найдите элементы множества P, выраженного через множества:
\begin{equation*}
A = \{0, 3, 4, 9\}; 
B = \{1, 3, 4, 7\};
C = \{0, 1, 2, 4, 7, 8, 9\};
I = \{0, 1, 2, 3, 4, 5, 6, 7, 8, 9\}.
\end{equation*}\question
Упростите следующее выражение с учетом того, что $A\subset B \subset C \subset D \subset U; A \neq \O$
\begin{equation*}
A \cap B  \cap \overline{C} \cup \overline{C} \cap D \cup B \cap C \cap D
\end{equation*}

Примечание: U — универсум\question
Дано отношение на множестве $\{1, 2, 3, 4, 5\}$ 
\begin{equation*}
aRb \iff (a+b) \bmod 2 =0
\end{equation*}
Напишите обоснованный ответ какими свойствами обладает или не обладает отношение и почему:   
\begin{enumerate} [a)]\setcounter{enumi}{0}
\item рефлексивность
\item антирефлексивность
\item симметричность
\item асимметричность
\item антисимметричность
\item транзитивность
\end{enumerate}

Обоснуйте свой ответ по каждому из приведенных ниже вопросов:
\begin{enumerate} [a)]\setcounter{enumi}{0}
    \item Является ли это отношение отношением эквивалентности?
    \item Является ли это отношение функциональным?
    \item Каким из отношений соответствия (одно-многозначным, много-многозначный и т.д.) оно является?
    \item К каким из отношений порядка (полного, частичного и т.д.) можно отнести данное отношение?
\end{enumerate}



\question
Установите, является ли каждое из перечисленных ниже отношений на А ($R \subseteq A \times A$) отношением эквивалентности (обоснование ответа обязательно). Для каждого отношения эквивалентности постройте классы 
эквивалентности и постройте граф отношения:
\begin{enumerate} [a)]\setcounter{enumi}{0}
\item $A = \{a, b, c, d, p, t\}$ задано отношение $R = \{(a, a), (b, b), (b, c), (b, d), (c, b), (c, c), (c, d), (d, b), (d, c), (d, d), (p,p), (t,t)\}$
\item $A = \{-10, -9, … , 9, 10\}$ и отношение $R = \{(a,b)|a^{3} = b^{3}\}$

\item $F(x)=x^{2}+1$, где $x \in A = [-2, 4]$ и отношение $R = \{(a,b)|F(a) = F(b)\}$
\end{enumerate}\question Составьте полную таблицу истинности, определите, какие переменные являются фиктивными и проверьте, является ли формула тавтологией:
$(P \rightarrow (Q \rightarrow R)) \rightarrow ((P \rightarrow Q) \rightarrow (P \rightarrow R))$

\end{questions}
\newpage
%%% begin test
\begin{flushright}
\begin{tabular}{p{2.8in} r l}
%\textbf{\class} & \textbf{ФИО:} & \makebox[2.5in]{\hrulefill}\\
\textbf{\class} & \textbf{ФИО:} &Верещагин Андрей Алексеевич
\\

\textbf{\examdate} &&\\
%\textbf{Time Limit: \timelimit} & Teaching Assistant & \makebox[2in]{\hrulefill}
\end{tabular}\\
\end{flushright}
\rule[1ex]{\textwidth}{.1pt}


\begin{questions}
\question
Найдите и упростите P:
\begin{equation*}
\overline{P} = \overline{A} \cap B \cup \overline{A} \cap C \cup A \cap \overline{B} \cup \overline{B} \cap C
\end{equation*}
Затем найдите элементы множества P, выраженного через множества:
\begin{equation*}
A = \{0, 3, 4, 9\}; 
B = \{1, 3, 4, 7\};
C = \{0, 1, 2, 4, 7, 8, 9\};
I = \{0, 1, 2, 3, 4, 5, 6, 7, 8, 9\}.
\end{equation*}\question
Упростите следующее выражение с учетом того, что $A\subset B \subset C \subset D \subset U; A \neq \O$
\begin{equation*}
\overline{A} \cap \overline{B} \cup B \cap \overline{C} \cup \overline{C} \cap D
\end{equation*}

Примечание: U — универсум\question
Дано отношение на множестве $\{1, 2, 3, 4, 5\}$ 
\begin{equation*}
aRb \iff a \leq b
\end{equation*}
Напишите обоснованный ответ какими свойствами обладает или не обладает отношение и почему:   
\begin{enumerate} [a)]\setcounter{enumi}{0}
\item рефлексивность
\item антирефлексивность
\item симметричность
\item асимметричность
\item антисимметричность
\item транзитивность
\end{enumerate}

Обоснуйте свой ответ по каждому из приведенных ниже вопросов:
\begin{enumerate} [a)]\setcounter{enumi}{0}
    \item Является ли это отношение отношением эквивалентности?
    \item Является ли это отношение функциональным?
    \item Каким из отношений соответствия (одно-многозначным, много-многозначный и т.д.) оно является?
    \item К каким из отношений порядка (полного, частичного и т.д.) можно отнести данное отношение?
\end{enumerate}


\question
Установите, является ли каждое из перечисленных ниже отношений на А ($R \subseteq A \times A$) отношением эквивалентности (обоснование ответа обязательно). Для каждого отношения эквивалентности постройте классы 
эквивалентности и постройте граф отношения:
\begin{enumerate} [a)]\setcounter{enumi}{0}
\item На множестве $A = \{1; 2; 3\}$ задано отношение $R = \{(1; 1); (2; 2); (3; 3); (2; 1); (1; 2); (2; 3); (3; 2); (3; 1); (1; 3)\}$
\item На множестве $A = \{1; 2; 3; 4; 5\}$ задано отношение $R = \{(1; 2); (1; 3); (1; 5); (2; 3); (2; 4); (2; 5); (3; 4); (3; 5); (4; 5)\}$
\item А - множество целых чисел и отношение $R = \{(a,b)|a + b = 0\}$
\end{enumerate}\question Составьте полную таблицу истинности, определите, какие переменные являются фиктивными и проверьте, является ли формула тавтологией:
$(( P \rightarrow Q) \land (Q \rightarrow P)) \rightarrow (P \rightarrow R)$

\end{questions}
\newpage
%%% begin test
\begin{flushright}
\begin{tabular}{p{2.8in} r l}
%\textbf{\class} & \textbf{ФИО:} & \makebox[2.5in]{\hrulefill}\\
\textbf{\class} & \textbf{ФИО:} &Виллер Полина Андреевна
\\

\textbf{\examdate} &&\\
%\textbf{Time Limit: \timelimit} & Teaching Assistant & \makebox[2in]{\hrulefill}
\end{tabular}\\
\end{flushright}
\rule[1ex]{\textwidth}{.1pt}


\begin{questions}
\question
Найдите и упростите P:
\begin{equation*}
\overline{P} = A \cap B \cup \overline{A} \cap \overline{B} \cup A \cap C \cup \overline{B} \cap C
\end{equation*}
Затем найдите элементы множества P, выраженного через множества:
\begin{equation*}
A = \{0, 3, 4, 9\}; 
B = \{1, 3, 4, 7\};
C = \{0, 1, 2, 4, 7, 8, 9\};
I = \{0, 1, 2, 3, 4, 5, 6, 7, 8, 9\}.
\end{equation*}\question
Упростите следующее выражение с учетом того, что $A\subset B \subset C \subset D \subset U; A \neq \O$
\begin{equation*}
A \cap B \cup \overline{A} \cap \overline{C} \cup A \cap C \cup \overline{B} \cap \overline{C}
\end{equation*}

Примечание: U — универсум\question
Дано отношение на множестве $\{1, 2, 3, 4, 5\}$ 
\begin{equation*}
aRb \iff |a-b| = 1
\end{equation*}
Напишите обоснованный ответ какими свойствами обладает или не обладает отношение и почему:   
\begin{enumerate} [a)]\setcounter{enumi}{0}
\item рефлексивность
\item антирефлексивность
\item симметричность
\item асимметричность
\item антисимметричность
\item транзитивность
\end{enumerate}

Обоснуйте свой ответ по каждому из приведенных ниже вопросов:
\begin{enumerate} [a)]\setcounter{enumi}{0}
    \item Является ли это отношение отношением эквивалентности?
    \item Является ли это отношение функциональным?
    \item Каким из отношений соответствия (одно-многозначным, много-многозначный и т.д.) оно является?
    \item К каким из отношений порядка (полного, частичного и т.д.) можно отнести данное отношение?
\end{enumerate}

\question
Установите, является ли каждое из перечисленных ниже отношений на А ($R \subseteq A \times A$) отношением эквивалентности (обоснование ответа обязательно). Для каждого отношения эквивалентности постройте классы 
эквивалентности и постройте граф отношения:
\begin{enumerate} [a)]\setcounter{enumi}{0}
\item Пусть A – множество имен. $A = \{ $Алексей, Иван, Петр, Александр, Павел, Андрей$ \}$. Тогда отношение $R$ верно на парах имен, начинающихся с одной и той же буквы, и только на них.
\item $A = \{-10, -9, … , 9, 10\}$ и отношение $ R = \{(a,b)|a^{2} = b^{2}\}$
\item На множестве $A = \{1; 2; 3\}$ задано отношение $R = \{(1; 1); (2; 2); (3; 3); (3; 2); (1; 2); (2; 1)\}$
\end{enumerate}\question Составьте полную таблицу истинности, определите, какие переменные являются фиктивными и проверьте, является ли формула тавтологией:

$(P \rightarrow (Q \land R)) \leftrightarrow ((P \rightarrow Q) \land (P \rightarrow R))$

\end{questions}
\newpage
%%% begin test
\begin{flushright}
\begin{tabular}{p{2.8in} r l}
%\textbf{\class} & \textbf{ФИО:} & \makebox[2.5in]{\hrulefill}\\
\textbf{\class} & \textbf{ФИО:} &Елизбарашвили Серго Мамукович
\\

\textbf{\examdate} &&\\
%\textbf{Time Limit: \timelimit} & Teaching Assistant & \makebox[2in]{\hrulefill}
\end{tabular}\\
\end{flushright}
\rule[1ex]{\textwidth}{.1pt}


\begin{questions}
\question
Найдите и упростите P:
\begin{equation*}
\overline{P} = B \cap \overline{C} \cup A \cap B \cup \overline{A} \cap C \cup \overline{A} \cap B
\end{equation*}
Затем найдите элементы множества P, выраженного через множества:
\begin{equation*}
A = \{0, 3, 4, 9\}; 
B = \{1, 3, 4, 7\};
C = \{0, 1, 2, 4, 7, 8, 9\};
I = \{0, 1, 2, 3, 4, 5, 6, 7, 8, 9\}.
\end{equation*}\question
Упростите следующее выражение с учетом того, что $A\subset B \subset C \subset D \subset U; A \neq \O$
\begin{equation*}
A \cap C  \cap D \cup B \cap \overline{C} \cap D \cup B \cap C \cap D
\end{equation*}

Примечание: U — универсум\question
Для следующего отношения на множестве $\{1, 2, 3, 4, 5\}$ 
\begin{equation*}
aRb \iff 0 < a-b<2
\end{equation*}
Напишите обоснованный ответ какими свойствами обладает или не обладает отношение и почему:   
\begin{enumerate} [a)]\setcounter{enumi}{0}
\item рефлексивность
\item антирефлексивность
\item симметричность
\item асимметричность
\item антисимметричность
\item транзитивность
\end{enumerate}

Обоснуйте свой ответ по каждому из приведенных ниже вопросов:
\begin{enumerate} [a)]\setcounter{enumi}{0}
    \item Является ли это отношение отношением эквивалентности?
    \item Является ли это отношение функциональным?
    \item Каким из отношений соответствия (одно-многозначным, много-многозначный и т.д.) оно является?
    \item К каким из отношений порядка (полного, частичного и т.д.) можно отнести данное отношение?
\end{enumerate}
\question
Установите, является ли каждое из перечисленных ниже отношений на А ($R \subseteq A \times A$) отношением эквивалентности (обоснование ответа обязательно). Для каждого отношения эквивалентности постройте классы 
эквивалентности и постройте граф отношения:
\begin{enumerate} [a)]\setcounter{enumi}{0}
\item $A = \{-10, -9, … , 9, 10\}$ и отношение $R = \{(a,b)|a^{2} = b^{2}\}$
\item $A = \{a, b, c, d, p, t\}$ задано отношение $R = \{(a, a), (b, b), (b, c), (b, d), (c, b), (c, c), (c, d), (d, b), (d, c), (d, d), (p,p), (t,t)\}$
\item Пусть A – множество имен. $A = \{ $Алексей, Иван, Петр, Александр, Павел, Андрей$ \}$. Тогда отношение $R$ верно на парах имен, начинающихся с одной и той же буквы, и только на них.
\end{enumerate}\question Составьте полную таблицу истинности, определите, какие переменные являются фиктивными и проверьте, является ли формула тавтологией:
$(( P \land \neg Q) \rightarrow (R \land \neg R)) \rightarrow (P \rightarrow Q)$

\end{questions}
\newpage
%%% begin test
\begin{flushright}
\begin{tabular}{p{2.8in} r l}
%\textbf{\class} & \textbf{ФИО:} & \makebox[2.5in]{\hrulefill}\\
\textbf{\class} & \textbf{ФИО:} &Казанин Михаил Александрович
\\

\textbf{\examdate} &&\\
%\textbf{Time Limit: \timelimit} & Teaching Assistant & \makebox[2in]{\hrulefill}
\end{tabular}\\
\end{flushright}
\rule[1ex]{\textwidth}{.1pt}


\begin{questions}
\question
Найдите и упростите P:
\begin{equation*}
\overline{P} = B \cap \overline{C} \cup A \cap B \cup \overline{A} \cap C \cup \overline{A} \cap B
\end{equation*}
Затем найдите элементы множества P, выраженного через множества:
\begin{equation*}
A = \{0, 3, 4, 9\}; 
B = \{1, 3, 4, 7\};
C = \{0, 1, 2, 4, 7, 8, 9\};
I = \{0, 1, 2, 3, 4, 5, 6, 7, 8, 9\}.
\end{equation*}\question
Упростите следующее выражение с учетом того, что $A\subset B \subset C \subset D \subset U; A \neq \O$
\begin{equation*}
\overline{A} \cap \overline{C} \cap D \cup \overline{B} \cap \overline{C} \cap D \cup A \cap B
\end{equation*}

Примечание: U — универсум\question
Дано отношение на множестве $\{1, 2, 3, 4, 5\}$ 
\begin{equation*}
aRb \iff a \leq b
\end{equation*}
Напишите обоснованный ответ какими свойствами обладает или не обладает отношение и почему:   
\begin{enumerate} [a)]\setcounter{enumi}{0}
\item рефлексивность
\item антирефлексивность
\item симметричность
\item асимметричность
\item антисимметричность
\item транзитивность
\end{enumerate}

Обоснуйте свой ответ по каждому из приведенных ниже вопросов:
\begin{enumerate} [a)]\setcounter{enumi}{0}
    \item Является ли это отношение отношением эквивалентности?
    \item Является ли это отношение функциональным?
    \item Каким из отношений соответствия (одно-многозначным, много-многозначный и т.д.) оно является?
    \item К каким из отношений порядка (полного, частичного и т.д.) можно отнести данное отношение?
\end{enumerate}


\question
Установите, является ли каждое из перечисленных ниже отношений на А ($R \subseteq A \times A$) отношением эквивалентности (обоснование ответа обязательно). Для каждого отношения эквивалентности постройте классы 
эквивалентности и постройте граф отношения:
\begin{enumerate} [a)]\setcounter{enumi}{0}
\item Пусть A – множество имен. $A = \{ $Алексей, Иван, Петр, Александр, Павел, Андрей$ \}$. Тогда отношение $R$ верно на парах имен, начинающихся с одной и той же буквы, и только на них.
\item $A = \{-10, -9, … , 9, 10\}$ и отношение $ R = \{(a,b)|a^{2} = b^{2}\}$
\item На множестве $A = \{1; 2; 3\}$ задано отношение $R = \{(1; 1); (2; 2); (3; 3); (3; 2); (1; 2); (2; 1)\}$
\end{enumerate}\question Составьте полную таблицу истинности, определите, какие переменные являются фиктивными и проверьте, является ли формула тавтологией:
$(P \rightarrow (Q \rightarrow R)) \rightarrow ((P \rightarrow Q) \rightarrow (P \rightarrow R))$

\end{questions}
\newpage
%%% begin test
\begin{flushright}
\begin{tabular}{p{2.8in} r l}
%\textbf{\class} & \textbf{ФИО:} & \makebox[2.5in]{\hrulefill}\\
\textbf{\class} & \textbf{ФИО:} &Ковалева Елизавета Сергеевна
\\

\textbf{\examdate} &&\\
%\textbf{Time Limit: \timelimit} & Teaching Assistant & \makebox[2in]{\hrulefill}
\end{tabular}\\
\end{flushright}
\rule[1ex]{\textwidth}{.1pt}


\begin{questions}
\question
Найдите и упростите P:
\begin{equation*}
\overline{P} = A \cap \overline{C} \cup A \cap \overline{B} \cup B \cap \overline{C} \cup A \cap C
\end{equation*}
Затем найдите элементы множества P, выраженного через множества:
\begin{equation*}
A = \{0, 3, 4, 9\}; 
B = \{1, 3, 4, 7\};
C = \{0, 1, 2, 4, 7, 8, 9\};
I = \{0, 1, 2, 3, 4, 5, 6, 7, 8, 9\}.
\end{equation*}\question
Упростите следующее выражение с учетом того, что $A\subset B \subset C \subset D \subset U; A \neq \O$
\begin{equation*}
A \cap B  \cap \overline{C} \cup \overline{C} \cap D \cup B \cap C \cap D
\end{equation*}

Примечание: U — универсум\question
Дано отношение на множестве $\{1, 2, 3, 4, 5\}$ 
\begin{equation*}
aRb \iff a \leq b
\end{equation*}
Напишите обоснованный ответ какими свойствами обладает или не обладает отношение и почему:   
\begin{enumerate} [a)]\setcounter{enumi}{0}
\item рефлексивность
\item антирефлексивность
\item симметричность
\item асимметричность
\item антисимметричность
\item транзитивность
\end{enumerate}

Обоснуйте свой ответ по каждому из приведенных ниже вопросов:
\begin{enumerate} [a)]\setcounter{enumi}{0}
    \item Является ли это отношение отношением эквивалентности?
    \item Является ли это отношение функциональным?
    \item Каким из отношений соответствия (одно-многозначным, много-многозначный и т.д.) оно является?
    \item К каким из отношений порядка (полного, частичного и т.д.) можно отнести данное отношение?
\end{enumerate}


\question
Установите, является ли каждое из перечисленных ниже отношений на А ($R \subseteq A \times A$) отношением эквивалентности (обоснование ответа обязательно). Для каждого отношения эквивалентности 
постройте классы эквивалентности и постройте граф отношения:
\begin{enumerate}[a)]\setcounter{enumi}{0}
\item А - множество целых чисел и отношение $R = \{(a,b)|a + b = 0\}$
\item $A = \{-10, -9, …, 9, 10\}$ и отношение $R = \{(a,b)|a^{3} = b^{3}\}$
\item На множестве $A = \{1; 2; 3\}$ задано отношение $R = \{(1; 1); (2; 2); (3; 3); (2; 1); (1; 2); (2; 3); (3; 2); (3; 1); (1; 3)\}$

\end{enumerate}\question Составьте полную таблицу истинности, определите, какие переменные являются фиктивными и проверьте, является ли формула тавтологией:
$(( P \rightarrow Q) \land (Q \rightarrow P)) \rightarrow (P \rightarrow R)$

\end{questions}
\newpage
%%% begin test
\begin{flushright}
\begin{tabular}{p{2.8in} r l}
%\textbf{\class} & \textbf{ФИО:} & \makebox[2.5in]{\hrulefill}\\
\textbf{\class} & \textbf{ФИО:} &Курылев Никита Алексеевич
\\

\textbf{\examdate} &&\\
%\textbf{Time Limit: \timelimit} & Teaching Assistant & \makebox[2in]{\hrulefill}
\end{tabular}\\
\end{flushright}
\rule[1ex]{\textwidth}{.1pt}


\begin{questions}
\question
Найдите и упростите P:
\begin{equation*}
\overline{P} = A \cap \overline{B} \cup \overline{B} \cap C \cup \overline{A} \cap \overline{B} \cup \overline{A} \cap C
\end{equation*}
Затем найдите элементы множества P, выраженного через множества:
\begin{equation*}
A = \{0, 3, 4, 9\}; 
B = \{1, 3, 4, 7\};
C = \{0, 1, 2, 4, 7, 8, 9\};
I = \{0, 1, 2, 3, 4, 5, 6, 7, 8, 9\}.
\end{equation*}\question
Упростите следующее выражение с учетом того, что $A\subset B \subset C \subset D \subset U; A \neq \O$
\begin{equation*}
A \cap B \cup \overline{A} \cap \overline{C} \cup A \cap C \cup \overline{B} \cap \overline{C}
\end{equation*}

Примечание: U — универсум\question
Дано отношение на множестве $\{1, 2, 3, 4, 5\}$ 
\begin{equation*}
aRb \iff |a-b| = 1
\end{equation*}
Напишите обоснованный ответ какими свойствами обладает или не обладает отношение и почему:   
\begin{enumerate} [a)]\setcounter{enumi}{0}
\item рефлексивность
\item антирефлексивность
\item симметричность
\item асимметричность
\item антисимметричность
\item транзитивность
\end{enumerate}

Обоснуйте свой ответ по каждому из приведенных ниже вопросов:
\begin{enumerate} [a)]\setcounter{enumi}{0}
    \item Является ли это отношение отношением эквивалентности?
    \item Является ли это отношение функциональным?
    \item Каким из отношений соответствия (одно-многозначным, много-многозначный и т.д.) оно является?
    \item К каким из отношений порядка (полного, частичного и т.д.) можно отнести данное отношение?
\end{enumerate}

\question
Установите, является ли каждое из перечисленных ниже отношений на А ($R \subseteq A \times A$) отношением эквивалентности (обоснование ответа обязательно). Для каждого отношения эквивалентности постройте классы эквивалентности и постройте граф отношения:
\begin{enumerate} [a)]\setcounter{enumi}{0}
\item $F(x)=x^{2}+1$, где $x \in A = [-2, 4]$ и отношение $R = \{(a,b)|F(a) = F(b)\}$
\item А - множество целых чисел и отношение $R = \{(a,b)|a + b = 5\}$
\item На множестве $A = \{1; 2; 3\}$ задано отношение $R = \{(1; 1); (2; 2); (3; 3); (3; 2); (1; 2); (2; 1)\}$

\end{enumerate}\question Составьте полную таблицу истинности, определите, какие переменные являются фиктивными и проверьте, является ли формула тавтологией:
$(( P \rightarrow Q) \land (Q \rightarrow P)) \rightarrow (P \rightarrow R)$

\end{questions}
\newpage
%%% begin test
\begin{flushright}
\begin{tabular}{p{2.8in} r l}
%\textbf{\class} & \textbf{ФИО:} & \makebox[2.5in]{\hrulefill}\\
\textbf{\class} & \textbf{ФИО:} &Ляхов Даниил Евгеньевич
\\

\textbf{\examdate} &&\\
%\textbf{Time Limit: \timelimit} & Teaching Assistant & \makebox[2in]{\hrulefill}
\end{tabular}\\
\end{flushright}
\rule[1ex]{\textwidth}{.1pt}


\begin{questions}
\question
Найдите и упростите P:
\begin{equation*}
\overline{P} = B \cap \overline{C} \cup A \cap B \cup \overline{A} \cap C \cup \overline{A} \cap B
\end{equation*}
Затем найдите элементы множества P, выраженного через множества:
\begin{equation*}
A = \{0, 3, 4, 9\}; 
B = \{1, 3, 4, 7\};
C = \{0, 1, 2, 4, 7, 8, 9\};
I = \{0, 1, 2, 3, 4, 5, 6, 7, 8, 9\}.
\end{equation*}\question
Упростите следующее выражение с учетом того, что $A\subset B \subset C \subset D \subset U; A \neq \O$
\begin{equation*}
\overline{A} \cap \overline{B} \cup B \cap \overline{C} \cup \overline{C} \cap D
\end{equation*}

Примечание: U — универсум\question
Дано отношение на множестве $\{1, 2, 3, 4, 5\}$ 
\begin{equation*}
aRb \iff b > a
\end{equation*}
Напишите обоснованный ответ какими свойствами обладает или не обладает отношение и почему:   
\begin{enumerate} [a)]\setcounter{enumi}{0}
\item рефлексивность
\item антирефлексивность
\item симметричность
\item асимметричность
\item антисимметричность
\item транзитивность
\end{enumerate}

Обоснуйте свой ответ по каждому из приведенных ниже вопросов:
\begin{enumerate} [a)]\setcounter{enumi}{0}
    \item Является ли это отношение отношением эквивалентности?
    \item Является ли это отношение функциональным?
    \item Каким из отношений соответствия (одно-многозначным, много-многозначный и т.д.) оно является?
    \item К каким из отношений порядка (полного, частичного и т.д.) можно отнести данное отношение?
\end{enumerate}

\question
Установите, является ли каждое из перечисленных ниже отношений на А ($R \subseteq A \times A$) отношением эквивалентности (обоснование ответа обязательно). Для каждого отношения эквивалентности постройте классы 
эквивалентности и постройте граф отношения:
\begin{enumerate} [a)]\setcounter{enumi}{0}
\item На множестве $A = \{1; 2; 3\}$ задано отношение $R = \{(1; 1); (2; 2); (3; 3); (2; 1); (1; 2); (2; 3); (3; 2); (3; 1); (1; 3)\}$
\item На множестве $A = \{1; 2; 3; 4; 5\}$ задано отношение $R = \{(1; 2); (1; 3); (1; 5); (2; 3); (2; 4); (2; 5); (3; 4); (3; 5); (4; 5)\}$
\item А - множество целых чисел и отношение $R = \{(a,b)|a + b = 0\}$
\end{enumerate}\question Составьте полную таблицу истинности, определите, какие переменные являются фиктивными и проверьте, является ли формула тавтологией:
$(( P \rightarrow Q) \land (Q \rightarrow P)) \rightarrow (P \rightarrow R)$

\end{questions}
\newpage
%%% begin test
\begin{flushright}
\begin{tabular}{p{2.8in} r l}
%\textbf{\class} & \textbf{ФИО:} & \makebox[2.5in]{\hrulefill}\\
\textbf{\class} & \textbf{ФИО:} &Моторина Евгения Викторовна
\\

\textbf{\examdate} &&\\
%\textbf{Time Limit: \timelimit} & Teaching Assistant & \makebox[2in]{\hrulefill}
\end{tabular}\\
\end{flushright}
\rule[1ex]{\textwidth}{.1pt}


\begin{questions}
\question
Найдите и упростите P:
\begin{equation*}
\overline{P} = A \cap \overline{C} \cup A \cap \overline{B} \cup B \cap \overline{C} \cup A \cap C
\end{equation*}
Затем найдите элементы множества P, выраженного через множества:
\begin{equation*}
A = \{0, 3, 4, 9\}; 
B = \{1, 3, 4, 7\};
C = \{0, 1, 2, 4, 7, 8, 9\};
I = \{0, 1, 2, 3, 4, 5, 6, 7, 8, 9\}.
\end{equation*}\question
Упростите следующее выражение с учетом того, что $A\subset B \subset C \subset D \subset U; A \neq \O$
\begin{equation*}
\overline{A} \cap \overline{B} \cup B \cap \overline{C} \cup \overline{C} \cap D
\end{equation*}

Примечание: U — универсум\question
Дано отношение на множестве $\{1, 2, 3, 4, 5\}$ 
\begin{equation*}
aRb \iff a \leq b
\end{equation*}
Напишите обоснованный ответ какими свойствами обладает или не обладает отношение и почему:   
\begin{enumerate} [a)]\setcounter{enumi}{0}
\item рефлексивность
\item антирефлексивность
\item симметричность
\item асимметричность
\item антисимметричность
\item транзитивность
\end{enumerate}

Обоснуйте свой ответ по каждому из приведенных ниже вопросов:
\begin{enumerate} [a)]\setcounter{enumi}{0}
    \item Является ли это отношение отношением эквивалентности?
    \item Является ли это отношение функциональным?
    \item Каким из отношений соответствия (одно-многозначным, много-многозначный и т.д.) оно является?
    \item К каким из отношений порядка (полного, частичного и т.д.) можно отнести данное отношение?
\end{enumerate}


\question
Установите, является ли каждое из перечисленных ниже отношений на А ($R \subseteq A \times A$) отношением эквивалентности (обоснование ответа обязательно). Для каждого отношения эквивалентности постройте классы 
эквивалентности и постройте граф отношения:
\begin{enumerate} [a)]\setcounter{enumi}{0}
\item $A = \{a, b, c, d, p, t\}$ задано отношение $R = \{(a, a), (b, b), (b, c), (b, d), (c, b), (c, c), (c, d), (d, b), (d, c), (d, d), (p,p), (t,t)\}$
\item $A = \{-10, -9, … , 9, 10\}$ и отношение $R = \{(a,b)|a^{3} = b^{3}\}$

\item $F(x)=x^{2}+1$, где $x \in A = [-2, 4]$ и отношение $R = \{(a,b)|F(a) = F(b)\}$
\end{enumerate}\question Составьте полную таблицу истинности, определите, какие переменные являются фиктивными и проверьте, является ли формула тавтологией:
$(P \rightarrow (Q \rightarrow R)) \rightarrow ((P \rightarrow Q) \rightarrow (P \rightarrow R))$

\end{questions}
\newpage
%%% begin test
\begin{flushright}
\begin{tabular}{p{2.8in} r l}
%\textbf{\class} & \textbf{ФИО:} & \makebox[2.5in]{\hrulefill}\\
\textbf{\class} & \textbf{ФИО:} &Нафиков Айдар Рустемович
\\

\textbf{\examdate} &&\\
%\textbf{Time Limit: \timelimit} & Teaching Assistant & \makebox[2in]{\hrulefill}
\end{tabular}\\
\end{flushright}
\rule[1ex]{\textwidth}{.1pt}


\begin{questions}
\question
Найдите и упростите P:
\begin{equation*}
\overline{P} = A \cap \overline{C} \cup A \cap \overline{B} \cup B \cap \overline{C} \cup A \cap C
\end{equation*}
Затем найдите элементы множества P, выраженного через множества:
\begin{equation*}
A = \{0, 3, 4, 9\}; 
B = \{1, 3, 4, 7\};
C = \{0, 1, 2, 4, 7, 8, 9\};
I = \{0, 1, 2, 3, 4, 5, 6, 7, 8, 9\}.
\end{equation*}\question
Упростите следующее выражение с учетом того, что $A\subset B \subset C \subset D \subset U; A \neq \O$
\begin{equation*}
A \cap B \cup \overline{A} \cap \overline{C} \cup A \cap C \cup \overline{B} \cap \overline{C}
\end{equation*}

Примечание: U — универсум\question
Для следующего отношения на множестве $\{1, 2, 3, 4, 5\}$ 
\begin{equation*}
aRb \iff 0 < a-b<2
\end{equation*}
Напишите обоснованный ответ какими свойствами обладает или не обладает отношение и почему:   
\begin{enumerate} [a)]\setcounter{enumi}{0}
\item рефлексивность
\item антирефлексивность
\item симметричность
\item асимметричность
\item антисимметричность
\item транзитивность
\end{enumerate}

Обоснуйте свой ответ по каждому из приведенных ниже вопросов:
\begin{enumerate} [a)]\setcounter{enumi}{0}
    \item Является ли это отношение отношением эквивалентности?
    \item Является ли это отношение функциональным?
    \item Каким из отношений соответствия (одно-многозначным, много-многозначный и т.д.) оно является?
    \item К каким из отношений порядка (полного, частичного и т.д.) можно отнести данное отношение?
\end{enumerate}
\question
Установите, является ли каждое из перечисленных ниже отношений на А ($R \subseteq A \times A$) отношением эквивалентности (обоснование ответа обязательно). Для каждого отношения эквивалентности постройте классы эквивалентности и постройте граф отношения:
\begin{enumerate} [a)]\setcounter{enumi}{0}
\item $F(x)=x^{2}+1$, где $x \in A = [-2, 4]$ и отношение $R = \{(a,b)|F(a) = F(b)\}$
\item А - множество целых чисел и отношение $R = \{(a,b)|a + b = 5\}$
\item На множестве $A = \{1; 2; 3\}$ задано отношение $R = \{(1; 1); (2; 2); (3; 3); (3; 2); (1; 2); (2; 1)\}$

\end{enumerate}\question Составьте полную таблицу истинности, определите, какие переменные являются фиктивными и проверьте, является ли формула тавтологией:
$((P \rightarrow Q) \land (R \rightarrow S) \land \neg (Q \lor S)) \rightarrow \neg (P \lor R)$

\end{questions}
\newpage
%%% begin test
\begin{flushright}
\begin{tabular}{p{2.8in} r l}
%\textbf{\class} & \textbf{ФИО:} & \makebox[2.5in]{\hrulefill}\\
\textbf{\class} & \textbf{ФИО:} &Олефиренко Егор Витальевич
\\

\textbf{\examdate} &&\\
%\textbf{Time Limit: \timelimit} & Teaching Assistant & \makebox[2in]{\hrulefill}
\end{tabular}\\
\end{flushright}
\rule[1ex]{\textwidth}{.1pt}


\begin{questions}
\question
Найдите и упростите P:
\begin{equation*}
\overline{P} = A \cap \overline{B} \cup \overline{B} \cap C \cup \overline{A} \cap \overline{B} \cup \overline{A} \cap C
\end{equation*}
Затем найдите элементы множества P, выраженного через множества:
\begin{equation*}
A = \{0, 3, 4, 9\}; 
B = \{1, 3, 4, 7\};
C = \{0, 1, 2, 4, 7, 8, 9\};
I = \{0, 1, 2, 3, 4, 5, 6, 7, 8, 9\}.
\end{equation*}\question
Упростите следующее выражение с учетом того, что $A\subset B \subset C \subset D \subset U; A \neq \O$
\begin{equation*}
A \cap  \overline{C} \cup B \cap \overline{D} \cup  \overline{A} \cap C \cap  \overline{D}
\end{equation*}

Примечание: U — универсум\question
Дано отношение на множестве $\{1, 2, 3, 4, 5\}$ 
\begin{equation*}
aRb \iff a \geq b^2
\end{equation*}
Напишите обоснованный ответ какими свойствами обладает или не обладает отношение и почему:   
\begin{enumerate} [a)]\setcounter{enumi}{0}
\item рефлексивность
\item антирефлексивность
\item симметричность
\item асимметричность
\item антисимметричность
\item транзитивность
\end{enumerate}

Обоснуйте свой ответ по каждому из приведенных ниже вопросов:
\begin{enumerate} [a)]\setcounter{enumi}{0}
    \item Является ли это отношение отношением эквивалентности?
    \item Является ли это отношение функциональным?
    \item Каким из отношений соответствия (одно-многозначным, много-многозначный и т.д.) оно является?
    \item К каким из отношений порядка (полного, частичного и т.д.) можно отнести данное отношение?
\end{enumerate}


\question
Установите, является ли каждое из перечисленных ниже отношений на А ($R \subseteq A \times A$) отношением эквивалентности (обоснование ответа обязательно). Для каждого отношения эквивалентности постройте классы 
эквивалентности и постройте граф отношения:
\begin{enumerate} [a)]\setcounter{enumi}{0}
\item $A = \{-10, -9, … , 9, 10\}$ и отношение $R = \{(a,b)|a^{2} = b^{2}\}$
\item $A = \{a, b, c, d, p, t\}$ задано отношение $R = \{(a, a), (b, b), (b, c), (b, d), (c, b), (c, c), (c, d), (d, b), (d, c), (d, d), (p,p), (t,t)\}$
\item Пусть A – множество имен. $A = \{ $Алексей, Иван, Петр, Александр, Павел, Андрей$ \}$. Тогда отношение $R$ верно на парах имен, начинающихся с одной и той же буквы, и только на них.
\end{enumerate}\question Составьте полную таблицу истинности, определите, какие переменные являются фиктивными и проверьте, является ли формула тавтологией:

$(P \rightarrow (Q \land R)) \leftrightarrow ((P \rightarrow Q) \land (P \rightarrow R))$

\end{questions}
\newpage
%%% begin test
\begin{flushright}
\begin{tabular}{p{2.8in} r l}
%\textbf{\class} & \textbf{ФИО:} & \makebox[2.5in]{\hrulefill}\\
\textbf{\class} & \textbf{ФИО:} &Парфенов Никита Николаевич
\\

\textbf{\examdate} &&\\
%\textbf{Time Limit: \timelimit} & Teaching Assistant & \makebox[2in]{\hrulefill}
\end{tabular}\\
\end{flushright}
\rule[1ex]{\textwidth}{.1pt}


\begin{questions}
\question
Найдите и упростите P:
\begin{equation*}
\overline{P} = A \cap \overline{B} \cup \overline{B} \cap C \cup \overline{A} \cap \overline{B} \cup \overline{A} \cap C
\end{equation*}
Затем найдите элементы множества P, выраженного через множества:
\begin{equation*}
A = \{0, 3, 4, 9\}; 
B = \{1, 3, 4, 7\};
C = \{0, 1, 2, 4, 7, 8, 9\};
I = \{0, 1, 2, 3, 4, 5, 6, 7, 8, 9\}.
\end{equation*}\question
Упростите следующее выражение с учетом того, что $A\subset B \subset C \subset D \subset U; A \neq \O$
\begin{equation*}
A \cap C  \cap D \cup B \cap \overline{C} \cap D \cup B \cap C \cap D
\end{equation*}

Примечание: U — универсум\question
Дано отношение на множестве $\{1, 2, 3, 4, 5\}$ 
\begin{equation*}
aRb \iff |a-b| = 1
\end{equation*}
Напишите обоснованный ответ какими свойствами обладает или не обладает отношение и почему:   
\begin{enumerate} [a)]\setcounter{enumi}{0}
\item рефлексивность
\item антирефлексивность
\item симметричность
\item асимметричность
\item антисимметричность
\item транзитивность
\end{enumerate}

Обоснуйте свой ответ по каждому из приведенных ниже вопросов:
\begin{enumerate} [a)]\setcounter{enumi}{0}
    \item Является ли это отношение отношением эквивалентности?
    \item Является ли это отношение функциональным?
    \item Каким из отношений соответствия (одно-многозначным, много-многозначный и т.д.) оно является?
    \item К каким из отношений порядка (полного, частичного и т.д.) можно отнести данное отношение?
\end{enumerate}

\question
Установите, является ли каждое из перечисленных ниже отношений на А ($R \subseteq A \times A$) отношением эквивалентности (обоснование ответа обязательно). Для каждого отношения эквивалентности постройте классы 
эквивалентности и постройте граф отношения:
\begin{enumerate} [a)]\setcounter{enumi}{0}
\item $A = \{a, b, c, d, p, t\}$ задано отношение $R = \{(a, a), (b, b), (b, c), (b, d), (c, b), (c, c), (c, d), (d, b), (d, c), (d, d), (p,p), (t,t)\}$
\item $A = \{-10, -9, … , 9, 10\}$ и отношение $R = \{(a,b)|a^{3} = b^{3}\}$

\item $F(x)=x^{2}+1$, где $x \in A = [-2, 4]$ и отношение $R = \{(a,b)|F(a) = F(b)\}$
\end{enumerate}\question Составьте полную таблицу истинности, определите, какие переменные являются фиктивными и проверьте, является ли формула тавтологией:
$(P \rightarrow (Q \rightarrow R)) \rightarrow ((P \rightarrow Q) \rightarrow (P \rightarrow R))$

\end{questions}
\newpage
%%% begin test
\begin{flushright}
\begin{tabular}{p{2.8in} r l}
%\textbf{\class} & \textbf{ФИО:} & \makebox[2.5in]{\hrulefill}\\
\textbf{\class} & \textbf{ФИО:} &Пасичник Артем Аркадьевич
\\

\textbf{\examdate} &&\\
%\textbf{Time Limit: \timelimit} & Teaching Assistant & \makebox[2in]{\hrulefill}
\end{tabular}\\
\end{flushright}
\rule[1ex]{\textwidth}{.1pt}


\begin{questions}
\question
Найдите и упростите P:
\begin{equation*}
\overline{P} = A \cap B \cup \overline{A} \cap \overline{B} \cup A \cap C \cup \overline{B} \cap C
\end{equation*}
Затем найдите элементы множества P, выраженного через множества:
\begin{equation*}
A = \{0, 3, 4, 9\}; 
B = \{1, 3, 4, 7\};
C = \{0, 1, 2, 4, 7, 8, 9\};
I = \{0, 1, 2, 3, 4, 5, 6, 7, 8, 9\}.
\end{equation*}\question
Упростите следующее выражение с учетом того, что $A\subset B \subset C \subset D \subset U; A \neq \O$
\begin{equation*}
A \cap B \cup \overline{A} \cap \overline{C} \cup A \cap C \cup \overline{B} \cap \overline{C}
\end{equation*}

Примечание: U — универсум\question
Дано отношение на множестве $\{1, 2, 3, 4, 5\}$ 
\begin{equation*}
aRb \iff b > a
\end{equation*}
Напишите обоснованный ответ какими свойствами обладает или не обладает отношение и почему:   
\begin{enumerate} [a)]\setcounter{enumi}{0}
\item рефлексивность
\item антирефлексивность
\item симметричность
\item асимметричность
\item антисимметричность
\item транзитивность
\end{enumerate}

Обоснуйте свой ответ по каждому из приведенных ниже вопросов:
\begin{enumerate} [a)]\setcounter{enumi}{0}
    \item Является ли это отношение отношением эквивалентности?
    \item Является ли это отношение функциональным?
    \item Каким из отношений соответствия (одно-многозначным, много-многозначный и т.д.) оно является?
    \item К каким из отношений порядка (полного, частичного и т.д.) можно отнести данное отношение?
\end{enumerate}

\question
Установите, является ли каждое из перечисленных ниже отношений на А ($R \subseteq A \times A$) отношением эквивалентности (обоснование ответа обязательно). Для каждого отношения эквивалентности постройте классы 
эквивалентности и постройте граф отношения:
\begin{enumerate} [a)]\setcounter{enumi}{0}
\item $A = \{-10, -9, … , 9, 10\}$ и отношение $R = \{(a,b)|a^{2} = b^{2}\}$
\item $A = \{a, b, c, d, p, t\}$ задано отношение $R = \{(a, a), (b, b), (b, c), (b, d), (c, b), (c, c), (c, d), (d, b), (d, c), (d, d), (p,p), (t,t)\}$
\item Пусть A – множество имен. $A = \{ $Алексей, Иван, Петр, Александр, Павел, Андрей$ \}$. Тогда отношение $R$ верно на парах имен, начинающихся с одной и той же буквы, и только на них.
\end{enumerate}\question Составьте полную таблицу истинности, определите, какие переменные являются фиктивными и проверьте, является ли формула тавтологией:
$(( P \rightarrow Q) \land (Q \rightarrow P)) \rightarrow (P \rightarrow R)$

\end{questions}
\newpage
%%% begin test
\begin{flushright}
\begin{tabular}{p{2.8in} r l}
%\textbf{\class} & \textbf{ФИО:} & \makebox[2.5in]{\hrulefill}\\
\textbf{\class} & \textbf{ФИО:} &Писарева Юлия Игоревна
\\

\textbf{\examdate} &&\\
%\textbf{Time Limit: \timelimit} & Teaching Assistant & \makebox[2in]{\hrulefill}
\end{tabular}\\
\end{flushright}
\rule[1ex]{\textwidth}{.1pt}


\begin{questions}
\question
Найдите и упростите P:
\begin{equation*}
\overline{P} = A \cap \overline{B} \cup \overline{B} \cap C \cup \overline{A} \cap \overline{B} \cup \overline{A} \cap C
\end{equation*}
Затем найдите элементы множества P, выраженного через множества:
\begin{equation*}
A = \{0, 3, 4, 9\}; 
B = \{1, 3, 4, 7\};
C = \{0, 1, 2, 4, 7, 8, 9\};
I = \{0, 1, 2, 3, 4, 5, 6, 7, 8, 9\}.
\end{equation*}\question
Упростите следующее выражение с учетом того, что $A\subset B \subset C \subset D \subset U; A \neq \O$
\begin{equation*}
A \cap B \cup \overline{A} \cap \overline{C} \cup A \cap C \cup \overline{B} \cap \overline{C}
\end{equation*}

Примечание: U — универсум\question
Дано отношение на множестве $\{1, 2, 3, 4, 5\}$ 
\begin{equation*}
aRb \iff (a+b) \bmod 2 =0
\end{equation*}
Напишите обоснованный ответ какими свойствами обладает или не обладает отношение и почему:   
\begin{enumerate} [a)]\setcounter{enumi}{0}
\item рефлексивность
\item антирефлексивность
\item симметричность
\item асимметричность
\item антисимметричность
\item транзитивность
\end{enumerate}

Обоснуйте свой ответ по каждому из приведенных ниже вопросов:
\begin{enumerate} [a)]\setcounter{enumi}{0}
    \item Является ли это отношение отношением эквивалентности?
    \item Является ли это отношение функциональным?
    \item Каким из отношений соответствия (одно-многозначным, много-многозначный и т.д.) оно является?
    \item К каким из отношений порядка (полного, частичного и т.д.) можно отнести данное отношение?
\end{enumerate}



\question
Установите, является ли каждое из перечисленных ниже отношений на А ($R \subseteq A \times A$) отношением эквивалентности (обоснование ответа обязательно). Для каждого отношения эквивалентности постройте классы 
эквивалентности и постройте граф отношения:
\begin{enumerate} [a)]\setcounter{enumi}{0}
\item А - множество целых чисел и отношение $R = \{(a,b)|a + b = 5\}$
\item Пусть A – множество имен. $A = \{ $Алексей, Иван, Петр, Александр, Павел, Андрей$ \}$. Тогда отношение $R $ верно на парах имен, начинающихся с одной и той же буквы, и только на них.
\item На множестве $A = \{1; 2; 3; 4; 5\}$ задано отношение $R = \{(1; 2); (1; 3); (1; 5); (2; 3); (2; 4); (2; 5); (3; 4); (3; 5); (4; 5)\}$
\end{enumerate}\question Составьте полную таблицу истинности, определите, какие переменные являются фиктивными и проверьте, является ли формула тавтологией:
$ P \rightarrow (Q \rightarrow ((P \lor Q) \rightarrow (P \land Q)))$

\end{questions}
\newpage
%%% begin test
\begin{flushright}
\begin{tabular}{p{2.8in} r l}
%\textbf{\class} & \textbf{ФИО:} & \makebox[2.5in]{\hrulefill}\\
\textbf{\class} & \textbf{ФИО:} &Преженцов Дмитрий Игоревич
\\

\textbf{\examdate} &&\\
%\textbf{Time Limit: \timelimit} & Teaching Assistant & \makebox[2in]{\hrulefill}
\end{tabular}\\
\end{flushright}
\rule[1ex]{\textwidth}{.1pt}


\begin{questions}
\question
Найдите и упростите P:
\begin{equation*}
\overline{P} = A \cap C \cup \overline{A} \cap \overline{C} \cup \overline{B} \cap C \cup \overline{A} \cap \overline{B}
\end{equation*}
Затем найдите элементы множества P, выраженного через множества:
\begin{equation*}
A = \{0, 3, 4, 9\}; 
B = \{1, 3, 4, 7\};
C = \{0, 1, 2, 4, 7, 8, 9\};
I = \{0, 1, 2, 3, 4, 5, 6, 7, 8, 9\}.
\end{equation*}\question
Упростите следующее выражение с учетом того, что $A\subset B \subset C \subset D \subset U; A \neq \O$
\begin{equation*}
A \cap B  \cap \overline{C} \cup \overline{C} \cap D \cup B \cap C \cap D
\end{equation*}

Примечание: U — универсум\question
Дано отношение на множестве $\{1, 2, 3, 4, 5\}$ 
\begin{equation*}
aRb \iff a \geq b^2
\end{equation*}
Напишите обоснованный ответ какими свойствами обладает или не обладает отношение и почему:   
\begin{enumerate} [a)]\setcounter{enumi}{0}
\item рефлексивность
\item антирефлексивность
\item симметричность
\item асимметричность
\item антисимметричность
\item транзитивность
\end{enumerate}

Обоснуйте свой ответ по каждому из приведенных ниже вопросов:
\begin{enumerate} [a)]\setcounter{enumi}{0}
    \item Является ли это отношение отношением эквивалентности?
    \item Является ли это отношение функциональным?
    \item Каким из отношений соответствия (одно-многозначным, много-многозначный и т.д.) оно является?
    \item К каким из отношений порядка (полного, частичного и т.д.) можно отнести данное отношение?
\end{enumerate}


\question
Установите, является ли каждое из перечисленных ниже отношений на А ($R \subseteq A \times A$) отношением эквивалентности (обоснование ответа обязательно). Для каждого отношения эквивалентности постройте классы 
эквивалентности и постройте граф отношения:
\begin{enumerate} [a)]\setcounter{enumi}{0}
\item $A = \{-10, -9, … , 9, 10\}$ и отношение $R = \{(a,b)|a^{2} = b^{2}\}$
\item $A = \{a, b, c, d, p, t\}$ задано отношение $R = \{(a, a), (b, b), (b, c), (b, d), (c, b), (c, c), (c, d), (d, b), (d, c), (d, d), (p,p), (t,t)\}$
\item Пусть A – множество имен. $A = \{ $Алексей, Иван, Петр, Александр, Павел, Андрей$ \}$. Тогда отношение $R$ верно на парах имен, начинающихся с одной и той же буквы, и только на них.
\end{enumerate}\question Составьте полную таблицу истинности, определите, какие переменные являются фиктивными и проверьте, является ли формула тавтологией:
$ P \rightarrow (Q \rightarrow ((P \lor Q) \rightarrow (P \land Q)))$

\end{questions}
\newpage
%%% begin test
\begin{flushright}
\begin{tabular}{p{2.8in} r l}
%\textbf{\class} & \textbf{ФИО:} & \makebox[2.5in]{\hrulefill}\\
\textbf{\class} & \textbf{ФИО:} &Профе Диана Викторовна
\\

\textbf{\examdate} &&\\
%\textbf{Time Limit: \timelimit} & Teaching Assistant & \makebox[2in]{\hrulefill}
\end{tabular}\\
\end{flushright}
\rule[1ex]{\textwidth}{.1pt}


\begin{questions}
\question
Найдите и упростите P:
\begin{equation*}
\overline{P} = A \cap \overline{B} \cup A \cap C \cup B \cap C \cup \overline{A} \cap C
\end{equation*}
Затем найдите элементы множества P, выраженного через множества:
\begin{equation*}
A = \{0, 3, 4, 9\}; 
B = \{1, 3, 4, 7\};
C = \{0, 1, 2, 4, 7, 8, 9\};
I = \{0, 1, 2, 3, 4, 5, 6, 7, 8, 9\}.
\end{equation*}\question
Упростите следующее выражение с учетом того, что $A\subset B \subset C \subset D \subset U; A \neq \O$
\begin{equation*}
\overline{B} \cap \overline{C} \cap D \cup \overline{A} \cap \overline{C} \cap D \cup \overline{A} \cap B
\end{equation*}

Примечание: U — универсум\question
Для следующего отношения на множестве $\{1, 2, 3, 4, 5\}$ 
\begin{equation*}
aRb \iff 0 < a-b<2
\end{equation*}
Напишите обоснованный ответ какими свойствами обладает или не обладает отношение и почему:   
\begin{enumerate} [a)]\setcounter{enumi}{0}
\item рефлексивность
\item антирефлексивность
\item симметричность
\item асимметричность
\item антисимметричность
\item транзитивность
\end{enumerate}

Обоснуйте свой ответ по каждому из приведенных ниже вопросов:
\begin{enumerate} [a)]\setcounter{enumi}{0}
    \item Является ли это отношение отношением эквивалентности?
    \item Является ли это отношение функциональным?
    \item Каким из отношений соответствия (одно-многозначным, много-многозначный и т.д.) оно является?
    \item К каким из отношений порядка (полного, частичного и т.д.) можно отнести данное отношение?
\end{enumerate}
\question
Установите, является ли каждое из перечисленных ниже отношений на А ($R \subseteq A \times A$) отношением эквивалентности (обоснование ответа обязательно). Для каждого отношения эквивалентности постройте классы 
эквивалентности и постройте граф отношения:
\begin{enumerate} [a)]\setcounter{enumi}{0}
\item $A = \{-10, -9, … , 9, 10\}$ и отношение $R = \{(a,b)|a^{2} = b^{2}\}$
\item $A = \{a, b, c, d, p, t\}$ задано отношение $R = \{(a, a), (b, b), (b, c), (b, d), (c, b), (c, c), (c, d), (d, b), (d, c), (d, d), (p,p), (t,t)\}$
\item Пусть A – множество имен. $A = \{ $Алексей, Иван, Петр, Александр, Павел, Андрей$ \}$. Тогда отношение $R$ верно на парах имен, начинающихся с одной и той же буквы, и только на них.
\end{enumerate}\question Составьте полную таблицу истинности, определите, какие переменные являются фиктивными и проверьте, является ли формула тавтологией:
$((P \rightarrow Q) \land (R \rightarrow S) \land \neg (Q \lor S)) \rightarrow \neg (P \lor R)$

\end{questions}
\newpage
%%% begin test
\begin{flushright}
\begin{tabular}{p{2.8in} r l}
%\textbf{\class} & \textbf{ФИО:} & \makebox[2.5in]{\hrulefill}\\
\textbf{\class} & \textbf{ФИО:} &Раков Максим Александрович
\\

\textbf{\examdate} &&\\
%\textbf{Time Limit: \timelimit} & Teaching Assistant & \makebox[2in]{\hrulefill}
\end{tabular}\\
\end{flushright}
\rule[1ex]{\textwidth}{.1pt}


\begin{questions}
\question
Найдите и упростите P:
\begin{equation*}
\overline{P} = \overline{A} \cap B \cup \overline{A} \cap C \cup A \cap \overline{B} \cup \overline{B} \cap C
\end{equation*}
Затем найдите элементы множества P, выраженного через множества:
\begin{equation*}
A = \{0, 3, 4, 9\}; 
B = \{1, 3, 4, 7\};
C = \{0, 1, 2, 4, 7, 8, 9\};
I = \{0, 1, 2, 3, 4, 5, 6, 7, 8, 9\}.
\end{equation*}\question
Упростите следующее выражение с учетом того, что $A\subset B \subset C \subset D \subset U; A \neq \O$
\begin{equation*}
A \cap B \cup \overline{A} \cap \overline{C} \cup A \cap C \cup \overline{B} \cap \overline{C}
\end{equation*}

Примечание: U — универсум\question
Дано отношение на множестве $\{1, 2, 3, 4, 5\}$ 
\begin{equation*}
aRb \iff a \leq b
\end{equation*}
Напишите обоснованный ответ какими свойствами обладает или не обладает отношение и почему:   
\begin{enumerate} [a)]\setcounter{enumi}{0}
\item рефлексивность
\item антирефлексивность
\item симметричность
\item асимметричность
\item антисимметричность
\item транзитивность
\end{enumerate}

Обоснуйте свой ответ по каждому из приведенных ниже вопросов:
\begin{enumerate} [a)]\setcounter{enumi}{0}
    \item Является ли это отношение отношением эквивалентности?
    \item Является ли это отношение функциональным?
    \item Каким из отношений соответствия (одно-многозначным, много-многозначный и т.д.) оно является?
    \item К каким из отношений порядка (полного, частичного и т.д.) можно отнести данное отношение?
\end{enumerate}


\question
Установите, является ли каждое из перечисленных ниже отношений на А ($R \subseteq A \times A$) отношением эквивалентности (обоснование ответа обязательно). Для каждого отношения эквивалентности постройте классы 
эквивалентности и постройте граф отношения:
\begin{enumerate} [a)]\setcounter{enumi}{0}
\item $A = \{a, b, c, d, p, t\}$ задано отношение $R = \{(a, a), (b, b), (b, c), (b, d), (c, b), (c, c), (c, d), (d, b), (d, c), (d, d), (p,p), (t,t)\}$
\item $A = \{-10, -9, … , 9, 10\}$ и отношение $R = \{(a,b)|a^{3} = b^{3}\}$

\item $F(x)=x^{2}+1$, где $x \in A = [-2, 4]$ и отношение $R = \{(a,b)|F(a) = F(b)\}$
\end{enumerate}\question Составьте полную таблицу истинности, определите, какие переменные являются фиктивными и проверьте, является ли формула тавтологией:
$(( P \rightarrow Q) \land (Q \rightarrow P)) \rightarrow (P \rightarrow R)$

\end{questions}
\newpage
%%% begin test
\begin{flushright}
\begin{tabular}{p{2.8in} r l}
%\textbf{\class} & \textbf{ФИО:} & \makebox[2.5in]{\hrulefill}\\
\textbf{\class} & \textbf{ФИО:} &Рахманкулов Эдгар Ильдарович
\\

\textbf{\examdate} &&\\
%\textbf{Time Limit: \timelimit} & Teaching Assistant & \makebox[2in]{\hrulefill}
\end{tabular}\\
\end{flushright}
\rule[1ex]{\textwidth}{.1pt}


\begin{questions}
\question
Найдите и упростите P:
\begin{equation*}
\overline{P} = \overline{A} \cap B \cup \overline{A} \cap C \cup A \cap \overline{B} \cup \overline{B} \cap C
\end{equation*}
Затем найдите элементы множества P, выраженного через множества:
\begin{equation*}
A = \{0, 3, 4, 9\}; 
B = \{1, 3, 4, 7\};
C = \{0, 1, 2, 4, 7, 8, 9\};
I = \{0, 1, 2, 3, 4, 5, 6, 7, 8, 9\}.
\end{equation*}\question
Упростите следующее выражение с учетом того, что $A\subset B \subset C \subset D \subset U; A \neq \O$
\begin{equation*}
\overline{A} \cap \overline{B} \cup B \cap \overline{C} \cup \overline{C} \cap D
\end{equation*}

Примечание: U — универсум\question
Дано отношение на множестве $\{1, 2, 3, 4, 5\}$ 
\begin{equation*}
aRb \iff (a+b) \bmod 2 =0
\end{equation*}
Напишите обоснованный ответ какими свойствами обладает или не обладает отношение и почему:   
\begin{enumerate} [a)]\setcounter{enumi}{0}
\item рефлексивность
\item антирефлексивность
\item симметричность
\item асимметричность
\item антисимметричность
\item транзитивность
\end{enumerate}

Обоснуйте свой ответ по каждому из приведенных ниже вопросов:
\begin{enumerate} [a)]\setcounter{enumi}{0}
    \item Является ли это отношение отношением эквивалентности?
    \item Является ли это отношение функциональным?
    \item Каким из отношений соответствия (одно-многозначным, много-многозначный и т.д.) оно является?
    \item К каким из отношений порядка (полного, частичного и т.д.) можно отнести данное отношение?
\end{enumerate}



\question
Установите, является ли каждое из перечисленных ниже отношений на А ($R \subseteq A \times A$) отношением эквивалентности (обоснование ответа обязательно). Для каждого отношения эквивалентности постройте классы 
эквивалентности и постройте граф отношения:
\begin{enumerate} [a)]\setcounter{enumi}{0}
\item А - множество целых чисел и отношение $R = \{(a,b)|a + b = 5\}$
\item Пусть A – множество имен. $A = \{ $Алексей, Иван, Петр, Александр, Павел, Андрей$ \}$. Тогда отношение $R $ верно на парах имен, начинающихся с одной и той же буквы, и только на них.
\item На множестве $A = \{1; 2; 3; 4; 5\}$ задано отношение $R = \{(1; 2); (1; 3); (1; 5); (2; 3); (2; 4); (2; 5); (3; 4); (3; 5); (4; 5)\}$
\end{enumerate}\question Составьте полную таблицу истинности, определите, какие переменные являются фиктивными и проверьте, является ли формула тавтологией:
$(( P \land \neg Q) \rightarrow (R \land \neg R)) \rightarrow (P \rightarrow Q)$

\end{questions}
\newpage
%%% begin test
\begin{flushright}
\begin{tabular}{p{2.8in} r l}
%\textbf{\class} & \textbf{ФИО:} & \makebox[2.5in]{\hrulefill}\\
\textbf{\class} & \textbf{ФИО:} &Решетникова Анна Андреевна
\\

\textbf{\examdate} &&\\
%\textbf{Time Limit: \timelimit} & Teaching Assistant & \makebox[2in]{\hrulefill}
\end{tabular}\\
\end{flushright}
\rule[1ex]{\textwidth}{.1pt}


\begin{questions}
\question
Найдите и упростите P:
\begin{equation*}
\overline{P} = \overline{A} \cap B \cup \overline{A} \cap C \cup A \cap \overline{B} \cup \overline{B} \cap C
\end{equation*}
Затем найдите элементы множества P, выраженного через множества:
\begin{equation*}
A = \{0, 3, 4, 9\}; 
B = \{1, 3, 4, 7\};
C = \{0, 1, 2, 4, 7, 8, 9\};
I = \{0, 1, 2, 3, 4, 5, 6, 7, 8, 9\}.
\end{equation*}\question
Упростите следующее выражение с учетом того, что $A\subset B \subset C \subset D \subset U; A \neq \O$
\begin{equation*}
A \cap B \cup \overline{A} \cap \overline{C} \cup A \cap C \cup \overline{B} \cap \overline{C}
\end{equation*}

Примечание: U — универсум\question
Дано отношение на множестве $\{1, 2, 3, 4, 5\}$ 
\begin{equation*}
aRb \iff b > a
\end{equation*}
Напишите обоснованный ответ какими свойствами обладает или не обладает отношение и почему:   
\begin{enumerate} [a)]\setcounter{enumi}{0}
\item рефлексивность
\item антирефлексивность
\item симметричность
\item асимметричность
\item антисимметричность
\item транзитивность
\end{enumerate}

Обоснуйте свой ответ по каждому из приведенных ниже вопросов:
\begin{enumerate} [a)]\setcounter{enumi}{0}
    \item Является ли это отношение отношением эквивалентности?
    \item Является ли это отношение функциональным?
    \item Каким из отношений соответствия (одно-многозначным, много-многозначный и т.д.) оно является?
    \item К каким из отношений порядка (полного, частичного и т.д.) можно отнести данное отношение?
\end{enumerate}

\question
Установите, является ли каждое из перечисленных ниже отношений на А ($R \subseteq A \times A$) отношением эквивалентности (обоснование ответа обязательно). Для каждого отношения эквивалентности 
постройте классы эквивалентности и постройте граф отношения:
\begin{enumerate}[a)]\setcounter{enumi}{0}
\item А - множество целых чисел и отношение $R = \{(a,b)|a + b = 0\}$
\item $A = \{-10, -9, …, 9, 10\}$ и отношение $R = \{(a,b)|a^{3} = b^{3}\}$
\item На множестве $A = \{1; 2; 3\}$ задано отношение $R = \{(1; 1); (2; 2); (3; 3); (2; 1); (1; 2); (2; 3); (3; 2); (3; 1); (1; 3)\}$

\end{enumerate}\question Составьте полную таблицу истинности, определите, какие переменные являются фиктивными и проверьте, является ли формула тавтологией:
$(P \rightarrow (Q \rightarrow R)) \rightarrow ((P \rightarrow Q) \rightarrow (P \rightarrow R))$

\end{questions}
\newpage
%%% begin test
\begin{flushright}
\begin{tabular}{p{2.8in} r l}
%\textbf{\class} & \textbf{ФИО:} & \makebox[2.5in]{\hrulefill}\\
\textbf{\class} & \textbf{ФИО:} &Романов Никита Романович
\\

\textbf{\examdate} &&\\
%\textbf{Time Limit: \timelimit} & Teaching Assistant & \makebox[2in]{\hrulefill}
\end{tabular}\\
\end{flushright}
\rule[1ex]{\textwidth}{.1pt}


\begin{questions}
\question
Найдите и упростите P:
\begin{equation*}
\overline{P} = \overline{A} \cap B \cup \overline{A} \cap C \cup A \cap \overline{B} \cup \overline{B} \cap C
\end{equation*}
Затем найдите элементы множества P, выраженного через множества:
\begin{equation*}
A = \{0, 3, 4, 9\}; 
B = \{1, 3, 4, 7\};
C = \{0, 1, 2, 4, 7, 8, 9\};
I = \{0, 1, 2, 3, 4, 5, 6, 7, 8, 9\}.
\end{equation*}\question
Упростите следующее выражение с учетом того, что $A\subset B \subset C \subset D \subset U; A \neq \O$
\begin{equation*}
A \cap B  \cap \overline{C} \cup \overline{C} \cap D \cup B \cap C \cap D
\end{equation*}

Примечание: U — универсум\question
Для следующего отношения на множестве $\{1, 2, 3, 4, 5\}$ 
\begin{equation*}
aRb \iff 0 < a-b<2
\end{equation*}
Напишите обоснованный ответ какими свойствами обладает или не обладает отношение и почему:   
\begin{enumerate} [a)]\setcounter{enumi}{0}
\item рефлексивность
\item антирефлексивность
\item симметричность
\item асимметричность
\item антисимметричность
\item транзитивность
\end{enumerate}

Обоснуйте свой ответ по каждому из приведенных ниже вопросов:
\begin{enumerate} [a)]\setcounter{enumi}{0}
    \item Является ли это отношение отношением эквивалентности?
    \item Является ли это отношение функциональным?
    \item Каким из отношений соответствия (одно-многозначным, много-многозначный и т.д.) оно является?
    \item К каким из отношений порядка (полного, частичного и т.д.) можно отнести данное отношение?
\end{enumerate}
\question
Установите, является ли каждое из перечисленных ниже отношений на А ($R \subseteq A \times A$) отношением эквивалентности (обоснование ответа обязательно). Для каждого отношения эквивалентности постройте классы 
эквивалентности и постройте граф отношения:
\begin{enumerate} [a)]\setcounter{enumi}{0}
\item На множестве $A = \{1; 2; 3\}$ задано отношение $R = \{(1; 1); (2; 2); (3; 3); (2; 1); (1; 2); (2; 3); (3; 2); (3; 1); (1; 3)\}$
\item На множестве $A = \{1; 2; 3; 4; 5\}$ задано отношение $R = \{(1; 2); (1; 3); (1; 5); (2; 3); (2; 4); (2; 5); (3; 4); (3; 5); (4; 5)\}$
\item А - множество целых чисел и отношение $R = \{(a,b)|a + b = 0\}$
\end{enumerate}\question Составьте полную таблицу истинности, определите, какие переменные являются фиктивными и проверьте, является ли формула тавтологией:
$((P \rightarrow Q) \lor R) \leftrightarrow (P \rightarrow (Q \lor R))$

\end{questions}
\newpage
%%% begin test
\begin{flushright}
\begin{tabular}{p{2.8in} r l}
%\textbf{\class} & \textbf{ФИО:} & \makebox[2.5in]{\hrulefill}\\
\textbf{\class} & \textbf{ФИО:} &Юрпалов Сергей Николаевич
\\

\textbf{\examdate} &&\\
%\textbf{Time Limit: \timelimit} & Teaching Assistant & \makebox[2in]{\hrulefill}
\end{tabular}\\
\end{flushright}
\rule[1ex]{\textwidth}{.1pt}


\begin{questions}
\question
Найдите и упростите P:
\begin{equation*}
\overline{P} = A \cap \overline{B} \cup \overline{B} \cap C \cup \overline{A} \cap \overline{B} \cup \overline{A} \cap C
\end{equation*}
Затем найдите элементы множества P, выраженного через множества:
\begin{equation*}
A = \{0, 3, 4, 9\}; 
B = \{1, 3, 4, 7\};
C = \{0, 1, 2, 4, 7, 8, 9\};
I = \{0, 1, 2, 3, 4, 5, 6, 7, 8, 9\}.
\end{equation*}\question
Упростите следующее выражение с учетом того, что $A\subset B \subset C \subset D \subset U; A \neq \O$
\begin{equation*}
A \cap  \overline{C} \cup B \cap \overline{D} \cup  \overline{A} \cap C \cap  \overline{D}
\end{equation*}

Примечание: U — универсум\question
Дано отношение на множестве $\{1, 2, 3, 4, 5\}$ 
\begin{equation*}
aRb \iff a \geq b^2
\end{equation*}
Напишите обоснованный ответ какими свойствами обладает или не обладает отношение и почему:   
\begin{enumerate} [a)]\setcounter{enumi}{0}
\item рефлексивность
\item антирефлексивность
\item симметричность
\item асимметричность
\item антисимметричность
\item транзитивность
\end{enumerate}

Обоснуйте свой ответ по каждому из приведенных ниже вопросов:
\begin{enumerate} [a)]\setcounter{enumi}{0}
    \item Является ли это отношение отношением эквивалентности?
    \item Является ли это отношение функциональным?
    \item Каким из отношений соответствия (одно-многозначным, много-многозначный и т.д.) оно является?
    \item К каким из отношений порядка (полного, частичного и т.д.) можно отнести данное отношение?
\end{enumerate}


\question
Установите, является ли каждое из перечисленных ниже отношений на А ($R \subseteq A \times A$) отношением эквивалентности (обоснование ответа обязательно). Для каждого отношения эквивалентности постройте классы 
эквивалентности и постройте граф отношения:
\begin{enumerate} [a)]\setcounter{enumi}{0}
\item Пусть A – множество имен. $A = \{ $Алексей, Иван, Петр, Александр, Павел, Андрей$ \}$. Тогда отношение $R$ верно на парах имен, начинающихся с одной и той же буквы, и только на них.
\item $A = \{-10, -9, … , 9, 10\}$ и отношение $ R = \{(a,b)|a^{2} = b^{2}\}$
\item На множестве $A = \{1; 2; 3\}$ задано отношение $R = \{(1; 1); (2; 2); (3; 3); (3; 2); (1; 2); (2; 1)\}$
\end{enumerate}\question Составьте полную таблицу истинности, определите, какие переменные являются фиктивными и проверьте, является ли формула тавтологией:
$((P \rightarrow Q) \land (R \rightarrow S) \land \neg (Q \lor S)) \rightarrow \neg (P \lor R)$

\end{questions}
\newpage
%%% begin test
\begin{flushright}
\begin{tabular}{p{2.8in} r l}
%\textbf{\class} & \textbf{ФИО:} & \makebox[2.5in]{\hrulefill}\\
\textbf{\class} & \textbf{ФИО:} &Якуничева Олеся Сергеевна
\\

\textbf{\examdate} &&\\
%\textbf{Time Limit: \timelimit} & Teaching Assistant & \makebox[2in]{\hrulefill}
\end{tabular}\\
\end{flushright}
\rule[1ex]{\textwidth}{.1pt}


\begin{questions}
\question
Найдите и упростите P:
\begin{equation*}
\overline{P} = A \cap \overline{B} \cup \overline{B} \cap C \cup \overline{A} \cap \overline{B} \cup \overline{A} \cap C
\end{equation*}
Затем найдите элементы множества P, выраженного через множества:
\begin{equation*}
A = \{0, 3, 4, 9\}; 
B = \{1, 3, 4, 7\};
C = \{0, 1, 2, 4, 7, 8, 9\};
I = \{0, 1, 2, 3, 4, 5, 6, 7, 8, 9\}.
\end{equation*}\question
Упростите следующее выражение с учетом того, что $A\subset B \subset C \subset D \subset U; A \neq \O$
\begin{equation*}
A \cap C  \cap D \cup B \cap \overline{C} \cap D \cup B \cap C \cap D
\end{equation*}

Примечание: U — универсум\question
Дано отношение на множестве $\{1, 2, 3, 4, 5\}$ 
\begin{equation*}
aRb \iff a \geq b^2
\end{equation*}
Напишите обоснованный ответ какими свойствами обладает или не обладает отношение и почему:   
\begin{enumerate} [a)]\setcounter{enumi}{0}
\item рефлексивность
\item антирефлексивность
\item симметричность
\item асимметричность
\item антисимметричность
\item транзитивность
\end{enumerate}

Обоснуйте свой ответ по каждому из приведенных ниже вопросов:
\begin{enumerate} [a)]\setcounter{enumi}{0}
    \item Является ли это отношение отношением эквивалентности?
    \item Является ли это отношение функциональным?
    \item Каким из отношений соответствия (одно-многозначным, много-многозначный и т.д.) оно является?
    \item К каким из отношений порядка (полного, частичного и т.д.) можно отнести данное отношение?
\end{enumerate}


\question
Установите, является ли каждое из перечисленных ниже отношений на А ($R \subseteq A \times A$) отношением эквивалентности (обоснование ответа обязательно). Для каждого отношения эквивалентности постройте классы 
эквивалентности и постройте граф отношения:
\begin{enumerate} [a)]\setcounter{enumi}{0}
\item Пусть A – множество имен. $A = \{ $Алексей, Иван, Петр, Александр, Павел, Андрей$ \}$. Тогда отношение $R$ верно на парах имен, начинающихся с одной и той же буквы, и только на них.
\item $A = \{-10, -9, … , 9, 10\}$ и отношение $ R = \{(a,b)|a^{2} = b^{2}\}$
\item На множестве $A = \{1; 2; 3\}$ задано отношение $R = \{(1; 1); (2; 2); (3; 3); (3; 2); (1; 2); (2; 1)\}$
\end{enumerate}\question Составьте полную таблицу истинности, определите, какие переменные являются фиктивными и проверьте, является ли формула тавтологией:

$(P \rightarrow (Q \land R)) \leftrightarrow ((P \rightarrow Q) \land (P \rightarrow R))$

\end{questions}
\newpage
%%% begin test
\begin{flushright}
\begin{tabular}{p{2.8in} r l}
%\textbf{\class} & \textbf{ФИО:} & \makebox[2.5in]{\hrulefill}\\
\textbf{\class} & \textbf{ФИО:} &М3106
\\

\textbf{\examdate} &&\\
%\textbf{Time Limit: \timelimit} & Teaching Assistant & \makebox[2in]{\hrulefill}
\end{tabular}\\
\end{flushright}
\rule[1ex]{\textwidth}{.1pt}


\begin{questions}
\question
Найдите и упростите P:
\begin{equation*}
\overline{P} = A \cap \overline{C} \cup A \cap \overline{B} \cup B \cap \overline{C} \cup A \cap C
\end{equation*}
Затем найдите элементы множества P, выраженного через множества:
\begin{equation*}
A = \{0, 3, 4, 9\}; 
B = \{1, 3, 4, 7\};
C = \{0, 1, 2, 4, 7, 8, 9\};
I = \{0, 1, 2, 3, 4, 5, 6, 7, 8, 9\}.
\end{equation*}\question
Упростите следующее выражение с учетом того, что $A\subset B \subset C \subset D \subset U; A \neq \O$
\begin{equation*}
A \cap  \overline{C} \cup B \cap \overline{D} \cup  \overline{A} \cap C \cap  \overline{D}
\end{equation*}

Примечание: U — универсум\question
Дано отношение на множестве $\{1, 2, 3, 4, 5\}$ 
\begin{equation*}
aRb \iff a \leq b
\end{equation*}
Напишите обоснованный ответ какими свойствами обладает или не обладает отношение и почему:   
\begin{enumerate} [a)]\setcounter{enumi}{0}
\item рефлексивность
\item антирефлексивность
\item симметричность
\item асимметричность
\item антисимметричность
\item транзитивность
\end{enumerate}

Обоснуйте свой ответ по каждому из приведенных ниже вопросов:
\begin{enumerate} [a)]\setcounter{enumi}{0}
    \item Является ли это отношение отношением эквивалентности?
    \item Является ли это отношение функциональным?
    \item Каким из отношений соответствия (одно-многозначным, много-многозначный и т.д.) оно является?
    \item К каким из отношений порядка (полного, частичного и т.д.) можно отнести данное отношение?
\end{enumerate}


\question
Установите, является ли каждое из перечисленных ниже отношений на А ($R \subseteq A \times A$) отношением эквивалентности (обоснование ответа обязательно). Для каждого отношения эквивалентности постройте классы 
эквивалентности и постройте граф отношения:
\begin{enumerate} [a)]\setcounter{enumi}{0}
\item А - множество целых чисел и отношение $R = \{(a,b)|a + b = 5\}$
\item Пусть A – множество имен. $A = \{ $Алексей, Иван, Петр, Александр, Павел, Андрей$ \}$. Тогда отношение $R $ верно на парах имен, начинающихся с одной и той же буквы, и только на них.
\item На множестве $A = \{1; 2; 3; 4; 5\}$ задано отношение $R = \{(1; 2); (1; 3); (1; 5); (2; 3); (2; 4); (2; 5); (3; 4); (3; 5); (4; 5)\}$
\end{enumerate}\question Составьте полную таблицу истинности, определите, какие переменные являются фиктивными и проверьте, является ли формула тавтологией:
$(P \rightarrow (Q \rightarrow R)) \rightarrow ((P \rightarrow Q) \rightarrow (P \rightarrow R))$

\end{questions}
\newpage
%%% begin test
\begin{flushright}
\begin{tabular}{p{2.8in} r l}
%\textbf{\class} & \textbf{ФИО:} & \makebox[2.5in]{\hrulefill}\\
\textbf{\class} & \textbf{ФИО:} &Антипин Илья Дмитриевич
\\

\textbf{\examdate} &&\\
%\textbf{Time Limit: \timelimit} & Teaching Assistant & \makebox[2in]{\hrulefill}
\end{tabular}\\
\end{flushright}
\rule[1ex]{\textwidth}{.1pt}


\begin{questions}
\question
Найдите и упростите P:
\begin{equation*}
\overline{P} = A \cap B \cup \overline{A} \cap \overline{B} \cup A \cap C \cup \overline{B} \cap C
\end{equation*}
Затем найдите элементы множества P, выраженного через множества:
\begin{equation*}
A = \{0, 3, 4, 9\}; 
B = \{1, 3, 4, 7\};
C = \{0, 1, 2, 4, 7, 8, 9\};
I = \{0, 1, 2, 3, 4, 5, 6, 7, 8, 9\}.
\end{equation*}\question
Упростите следующее выражение с учетом того, что $A\subset B \subset C \subset D \subset U; A \neq \O$
\begin{equation*}
A \cap  \overline{C} \cup B \cap \overline{D} \cup  \overline{A} \cap C \cap  \overline{D}
\end{equation*}

Примечание: U — универсум\question
Дано отношение на множестве $\{1, 2, 3, 4, 5\}$ 
\begin{equation*}
aRb \iff |a-b| = 1
\end{equation*}
Напишите обоснованный ответ какими свойствами обладает или не обладает отношение и почему:   
\begin{enumerate} [a)]\setcounter{enumi}{0}
\item рефлексивность
\item антирефлексивность
\item симметричность
\item асимметричность
\item антисимметричность
\item транзитивность
\end{enumerate}

Обоснуйте свой ответ по каждому из приведенных ниже вопросов:
\begin{enumerate} [a)]\setcounter{enumi}{0}
    \item Является ли это отношение отношением эквивалентности?
    \item Является ли это отношение функциональным?
    \item Каким из отношений соответствия (одно-многозначным, много-многозначный и т.д.) оно является?
    \item К каким из отношений порядка (полного, частичного и т.д.) можно отнести данное отношение?
\end{enumerate}

\question
Установите, является ли каждое из перечисленных ниже отношений на А ($R \subseteq A \times A$) отношением эквивалентности (обоснование ответа обязательно). Для каждого отношения эквивалентности постройте классы 
эквивалентности и постройте граф отношения:
\begin{enumerate} [a)]\setcounter{enumi}{0}
\item А - множество целых чисел и отношение $R = \{(a,b)|a + b = 5\}$
\item Пусть A – множество имен. $A = \{ $Алексей, Иван, Петр, Александр, Павел, Андрей$ \}$. Тогда отношение $R $ верно на парах имен, начинающихся с одной и той же буквы, и только на них.
\item На множестве $A = \{1; 2; 3; 4; 5\}$ задано отношение $R = \{(1; 2); (1; 3); (1; 5); (2; 3); (2; 4); (2; 5); (3; 4); (3; 5); (4; 5)\}$
\end{enumerate}\question Составьте полную таблицу истинности, определите, какие переменные являются фиктивными и проверьте, является ли формула тавтологией:
$((P \rightarrow Q) \land (R \rightarrow S) \land \neg (Q \lor S)) \rightarrow \neg (P \lor R)$

\end{questions}
\newpage
%%% begin test
\begin{flushright}
\begin{tabular}{p{2.8in} r l}
%\textbf{\class} & \textbf{ФИО:} & \makebox[2.5in]{\hrulefill}\\
\textbf{\class} & \textbf{ФИО:} &Белянин Николай Романович
\\

\textbf{\examdate} &&\\
%\textbf{Time Limit: \timelimit} & Teaching Assistant & \makebox[2in]{\hrulefill}
\end{tabular}\\
\end{flushright}
\rule[1ex]{\textwidth}{.1pt}


\begin{questions}
\question
Найдите и упростите P:
\begin{equation*}
\overline{P} = A \cap B \cup \overline{A} \cap \overline{B} \cup A \cap C \cup \overline{B} \cap C
\end{equation*}
Затем найдите элементы множества P, выраженного через множества:
\begin{equation*}
A = \{0, 3, 4, 9\}; 
B = \{1, 3, 4, 7\};
C = \{0, 1, 2, 4, 7, 8, 9\};
I = \{0, 1, 2, 3, 4, 5, 6, 7, 8, 9\}.
\end{equation*}\question
Упростите следующее выражение с учетом того, что $A\subset B \subset C \subset D \subset U; A \neq \O$
\begin{equation*}
\overline{A} \cap \overline{B} \cup B \cap \overline{C} \cup \overline{C} \cap D
\end{equation*}

Примечание: U — универсум\question
Дано отношение на множестве $\{1, 2, 3, 4, 5\}$ 
\begin{equation*}
aRb \iff  \text{НОД}(a,b) =1
\end{equation*}
Напишите обоснованный ответ какими свойствами обладает или не обладает отношение и почему:   
\begin{enumerate} [a)]\setcounter{enumi}{0}
\item рефлексивность
\item антирефлексивность
\item симметричность
\item асимметричность
\item антисимметричность
\item транзитивность
\end{enumerate}

Обоснуйте свой ответ по каждому из приведенных ниже вопросов:
\begin{enumerate} [a)]\setcounter{enumi}{0}
    \item Является ли это отношение отношением эквивалентности?
    \item Является ли это отношение функциональным?
    \item Каким из отношений соответствия (одно-многозначным, много-многозначный и т.д.) оно является?
    \item К каким из отношений порядка (полного, частичного и т.д.) можно отнести данное отношение?
\end{enumerate}


\question
Установите, является ли каждое из перечисленных ниже отношений на А ($R \subseteq A \times A$) отношением эквивалентности (обоснование ответа обязательно). Для каждого отношения эквивалентности постройте классы 
эквивалентности и постройте граф отношения:
\begin{enumerate} [a)]\setcounter{enumi}{0}
\item $A = \{a, b, c, d, p, t\}$ задано отношение $R = \{(a, a), (b, b), (b, c), (b, d), (c, b), (c, c), (c, d), (d, b), (d, c), (d, d), (p,p), (t,t)\}$
\item $A = \{-10, -9, … , 9, 10\}$ и отношение $R = \{(a,b)|a^{3} = b^{3}\}$

\item $F(x)=x^{2}+1$, где $x \in A = [-2, 4]$ и отношение $R = \{(a,b)|F(a) = F(b)\}$
\end{enumerate}\question Составьте полную таблицу истинности, определите, какие переменные являются фиктивными и проверьте, является ли формула тавтологией:
$((P \rightarrow Q) \land (R \rightarrow S) \land \neg (Q \lor S)) \rightarrow \neg (P \lor R)$

\end{questions}
\newpage
%%% begin test
\begin{flushright}
\begin{tabular}{p{2.8in} r l}
%\textbf{\class} & \textbf{ФИО:} & \makebox[2.5in]{\hrulefill}\\
\textbf{\class} & \textbf{ФИО:} &Богатырева Дарья Дмитриевна
\\

\textbf{\examdate} &&\\
%\textbf{Time Limit: \timelimit} & Teaching Assistant & \makebox[2in]{\hrulefill}
\end{tabular}\\
\end{flushright}
\rule[1ex]{\textwidth}{.1pt}


\begin{questions}
\question
Найдите и упростите P:
\begin{equation*}
\overline{P} = \overline{A} \cap B \cup \overline{A} \cap C \cup A \cap \overline{B} \cup \overline{B} \cap C
\end{equation*}
Затем найдите элементы множества P, выраженного через множества:
\begin{equation*}
A = \{0, 3, 4, 9\}; 
B = \{1, 3, 4, 7\};
C = \{0, 1, 2, 4, 7, 8, 9\};
I = \{0, 1, 2, 3, 4, 5, 6, 7, 8, 9\}.
\end{equation*}\question
Упростите следующее выражение с учетом того, что $A\subset B \subset C \subset D \subset U; A \neq \O$
\begin{equation*}
A \cap B  \cap \overline{C} \cup \overline{C} \cap D \cup B \cap C \cap D
\end{equation*}

Примечание: U — универсум\question
Дано отношение на множестве $\{1, 2, 3, 4, 5\}$ 
\begin{equation*}
aRb \iff a \leq b
\end{equation*}
Напишите обоснованный ответ какими свойствами обладает или не обладает отношение и почему:   
\begin{enumerate} [a)]\setcounter{enumi}{0}
\item рефлексивность
\item антирефлексивность
\item симметричность
\item асимметричность
\item антисимметричность
\item транзитивность
\end{enumerate}

Обоснуйте свой ответ по каждому из приведенных ниже вопросов:
\begin{enumerate} [a)]\setcounter{enumi}{0}
    \item Является ли это отношение отношением эквивалентности?
    \item Является ли это отношение функциональным?
    \item Каким из отношений соответствия (одно-многозначным, много-многозначный и т.д.) оно является?
    \item К каким из отношений порядка (полного, частичного и т.д.) можно отнести данное отношение?
\end{enumerate}


\question
Установите, является ли каждое из перечисленных ниже отношений на А ($R \subseteq A \times A$) отношением эквивалентности (обоснование ответа обязательно). Для каждого отношения эквивалентности постройте классы 
эквивалентности и постройте граф отношения:
\begin{enumerate} [a)]\setcounter{enumi}{0}
\item Пусть A – множество имен. $A = \{ $Алексей, Иван, Петр, Александр, Павел, Андрей$ \}$. Тогда отношение $R$ верно на парах имен, начинающихся с одной и той же буквы, и только на них.
\item $A = \{-10, -9, … , 9, 10\}$ и отношение $ R = \{(a,b)|a^{2} = b^{2}\}$
\item На множестве $A = \{1; 2; 3\}$ задано отношение $R = \{(1; 1); (2; 2); (3; 3); (3; 2); (1; 2); (2; 1)\}$
\end{enumerate}\question Составьте полную таблицу истинности, определите, какие переменные являются фиктивными и проверьте, является ли формула тавтологией:
$(P \rightarrow (Q \rightarrow R)) \rightarrow ((P \rightarrow Q) \rightarrow (P \rightarrow R))$

\end{questions}
\newpage
%%% begin test
\begin{flushright}
\begin{tabular}{p{2.8in} r l}
%\textbf{\class} & \textbf{ФИО:} & \makebox[2.5in]{\hrulefill}\\
\textbf{\class} & \textbf{ФИО:} &Бурдужел Никита Евгеньевич
\\

\textbf{\examdate} &&\\
%\textbf{Time Limit: \timelimit} & Teaching Assistant & \makebox[2in]{\hrulefill}
\end{tabular}\\
\end{flushright}
\rule[1ex]{\textwidth}{.1pt}


\begin{questions}
\question
Найдите и упростите P:
\begin{equation*}
\overline{P} = B \cap \overline{C} \cup A \cap B \cup \overline{A} \cap C \cup \overline{A} \cap B
\end{equation*}
Затем найдите элементы множества P, выраженного через множества:
\begin{equation*}
A = \{0, 3, 4, 9\}; 
B = \{1, 3, 4, 7\};
C = \{0, 1, 2, 4, 7, 8, 9\};
I = \{0, 1, 2, 3, 4, 5, 6, 7, 8, 9\}.
\end{equation*}\question
Упростите следующее выражение с учетом того, что $A\subset B \subset C \subset D \subset U; A \neq \O$
\begin{equation*}
A \cap C  \cap D \cup B \cap \overline{C} \cap D \cup B \cap C \cap D
\end{equation*}

Примечание: U — универсум\question
Дано отношение на множестве $\{1, 2, 3, 4, 5\}$ 
\begin{equation*}
aRb \iff (a+b) \bmod 2 =0
\end{equation*}
Напишите обоснованный ответ какими свойствами обладает или не обладает отношение и почему:   
\begin{enumerate} [a)]\setcounter{enumi}{0}
\item рефлексивность
\item антирефлексивность
\item симметричность
\item асимметричность
\item антисимметричность
\item транзитивность
\end{enumerate}

Обоснуйте свой ответ по каждому из приведенных ниже вопросов:
\begin{enumerate} [a)]\setcounter{enumi}{0}
    \item Является ли это отношение отношением эквивалентности?
    \item Является ли это отношение функциональным?
    \item Каким из отношений соответствия (одно-многозначным, много-многозначный и т.д.) оно является?
    \item К каким из отношений порядка (полного, частичного и т.д.) можно отнести данное отношение?
\end{enumerate}



\question
Установите, является ли каждое из перечисленных ниже отношений на А ($R \subseteq A \times A$) отношением эквивалентности (обоснование ответа обязательно). Для каждого отношения эквивалентности постройте классы эквивалентности и постройте граф отношения:
\begin{enumerate} [a)]\setcounter{enumi}{0}
\item $F(x)=x^{2}+1$, где $x \in A = [-2, 4]$ и отношение $R = \{(a,b)|F(a) = F(b)\}$
\item А - множество целых чисел и отношение $R = \{(a,b)|a + b = 5\}$
\item На множестве $A = \{1; 2; 3\}$ задано отношение $R = \{(1; 1); (2; 2); (3; 3); (3; 2); (1; 2); (2; 1)\}$

\end{enumerate}\question Составьте полную таблицу истинности, определите, какие переменные являются фиктивными и проверьте, является ли формула тавтологией:
$(P \rightarrow (Q \rightarrow R)) \rightarrow ((P \rightarrow Q) \rightarrow (P \rightarrow R))$

\end{questions}
\newpage
%%% begin test
\begin{flushright}
\begin{tabular}{p{2.8in} r l}
%\textbf{\class} & \textbf{ФИО:} & \makebox[2.5in]{\hrulefill}\\
\textbf{\class} & \textbf{ФИО:} &Вениченко Даниил Владимирович
\\

\textbf{\examdate} &&\\
%\textbf{Time Limit: \timelimit} & Teaching Assistant & \makebox[2in]{\hrulefill}
\end{tabular}\\
\end{flushright}
\rule[1ex]{\textwidth}{.1pt}


\begin{questions}
\question
Найдите и упростите P:
\begin{equation*}
\overline{P} = A \cap B \cup \overline{A} \cap \overline{B} \cup A \cap C \cup \overline{B} \cap C
\end{equation*}
Затем найдите элементы множества P, выраженного через множества:
\begin{equation*}
A = \{0, 3, 4, 9\}; 
B = \{1, 3, 4, 7\};
C = \{0, 1, 2, 4, 7, 8, 9\};
I = \{0, 1, 2, 3, 4, 5, 6, 7, 8, 9\}.
\end{equation*}\question
Упростите следующее выражение с учетом того, что $A\subset B \subset C \subset D \subset U; A \neq \O$
\begin{equation*}
A \cap B \cup \overline{A} \cap \overline{C} \cup A \cap C \cup \overline{B} \cap \overline{C}
\end{equation*}

Примечание: U — универсум\question
Дано отношение на множестве $\{1, 2, 3, 4, 5\}$ 
\begin{equation*}
aRb \iff  \text{НОД}(a,b) =1
\end{equation*}
Напишите обоснованный ответ какими свойствами обладает или не обладает отношение и почему:   
\begin{enumerate} [a)]\setcounter{enumi}{0}
\item рефлексивность
\item антирефлексивность
\item симметричность
\item асимметричность
\item антисимметричность
\item транзитивность
\end{enumerate}

Обоснуйте свой ответ по каждому из приведенных ниже вопросов:
\begin{enumerate} [a)]\setcounter{enumi}{0}
    \item Является ли это отношение отношением эквивалентности?
    \item Является ли это отношение функциональным?
    \item Каким из отношений соответствия (одно-многозначным, много-многозначный и т.д.) оно является?
    \item К каким из отношений порядка (полного, частичного и т.д.) можно отнести данное отношение?
\end{enumerate}


\question
Установите, является ли каждое из перечисленных ниже отношений на А ($R \subseteq A \times A$) отношением эквивалентности (обоснование ответа обязательно). Для каждого отношения эквивалентности постройте классы эквивалентности и постройте граф отношения:
\begin{enumerate} [a)]\setcounter{enumi}{0}
\item $F(x)=x^{2}+1$, где $x \in A = [-2, 4]$ и отношение $R = \{(a,b)|F(a) = F(b)\}$
\item А - множество целых чисел и отношение $R = \{(a,b)|a + b = 5\}$
\item На множестве $A = \{1; 2; 3\}$ задано отношение $R = \{(1; 1); (2; 2); (3; 3); (3; 2); (1; 2); (2; 1)\}$

\end{enumerate}\question Составьте полную таблицу истинности, определите, какие переменные являются фиктивными и проверьте, является ли формула тавтологией:
$ P \rightarrow (Q \rightarrow ((P \lor Q) \rightarrow (P \land Q)))$

\end{questions}
\newpage
%%% begin test
\begin{flushright}
\begin{tabular}{p{2.8in} r l}
%\textbf{\class} & \textbf{ФИО:} & \makebox[2.5in]{\hrulefill}\\
\textbf{\class} & \textbf{ФИО:} &Волков Глеб Романович
\\

\textbf{\examdate} &&\\
%\textbf{Time Limit: \timelimit} & Teaching Assistant & \makebox[2in]{\hrulefill}
\end{tabular}\\
\end{flushright}
\rule[1ex]{\textwidth}{.1pt}


\begin{questions}
\question
Найдите и упростите P:
\begin{equation*}
\overline{P} = A \cap \overline{C} \cup A \cap \overline{B} \cup B \cap \overline{C} \cup A \cap C
\end{equation*}
Затем найдите элементы множества P, выраженного через множества:
\begin{equation*}
A = \{0, 3, 4, 9\}; 
B = \{1, 3, 4, 7\};
C = \{0, 1, 2, 4, 7, 8, 9\};
I = \{0, 1, 2, 3, 4, 5, 6, 7, 8, 9\}.
\end{equation*}\question
Упростите следующее выражение с учетом того, что $A\subset B \subset C \subset D \subset U; A \neq \O$
\begin{equation*}
\overline{B} \cap \overline{C} \cap D \cup \overline{A} \cap \overline{C} \cap D \cup \overline{A} \cap B
\end{equation*}

Примечание: U — универсум\question
Дано отношение на множестве $\{1, 2, 3, 4, 5\}$ 
\begin{equation*}
aRb \iff a \leq b
\end{equation*}
Напишите обоснованный ответ какими свойствами обладает или не обладает отношение и почему:   
\begin{enumerate} [a)]\setcounter{enumi}{0}
\item рефлексивность
\item антирефлексивность
\item симметричность
\item асимметричность
\item антисимметричность
\item транзитивность
\end{enumerate}

Обоснуйте свой ответ по каждому из приведенных ниже вопросов:
\begin{enumerate} [a)]\setcounter{enumi}{0}
    \item Является ли это отношение отношением эквивалентности?
    \item Является ли это отношение функциональным?
    \item Каким из отношений соответствия (одно-многозначным, много-многозначный и т.д.) оно является?
    \item К каким из отношений порядка (полного, частичного и т.д.) можно отнести данное отношение?
\end{enumerate}


\question
Установите, является ли каждое из перечисленных ниже отношений на А ($R \subseteq A \times A$) отношением эквивалентности (обоснование ответа обязательно). Для каждого отношения эквивалентности постройте классы 
эквивалентности и постройте граф отношения:
\begin{enumerate} [a)]\setcounter{enumi}{0}
\item Пусть A – множество имен. $A = \{ $Алексей, Иван, Петр, Александр, Павел, Андрей$ \}$. Тогда отношение $R$ верно на парах имен, начинающихся с одной и той же буквы, и только на них.
\item $A = \{-10, -9, … , 9, 10\}$ и отношение $ R = \{(a,b)|a^{2} = b^{2}\}$
\item На множестве $A = \{1; 2; 3\}$ задано отношение $R = \{(1; 1); (2; 2); (3; 3); (3; 2); (1; 2); (2; 1)\}$
\end{enumerate}\question Составьте полную таблицу истинности, определите, какие переменные являются фиктивными и проверьте, является ли формула тавтологией:
$(P \rightarrow (Q \rightarrow R)) \rightarrow ((P \rightarrow Q) \rightarrow (P \rightarrow R))$

\end{questions}
\newpage
%%% begin test
\begin{flushright}
\begin{tabular}{p{2.8in} r l}
%\textbf{\class} & \textbf{ФИО:} & \makebox[2.5in]{\hrulefill}\\
\textbf{\class} & \textbf{ФИО:} &Гарманов Кирилл Николаевич
\\

\textbf{\examdate} &&\\
%\textbf{Time Limit: \timelimit} & Teaching Assistant & \makebox[2in]{\hrulefill}
\end{tabular}\\
\end{flushright}
\rule[1ex]{\textwidth}{.1pt}


\begin{questions}
\question
Найдите и упростите P:
\begin{equation*}
\overline{P} = A \cap C \cup \overline{A} \cap \overline{C} \cup \overline{B} \cap C \cup \overline{A} \cap \overline{B}
\end{equation*}
Затем найдите элементы множества P, выраженного через множества:
\begin{equation*}
A = \{0, 3, 4, 9\}; 
B = \{1, 3, 4, 7\};
C = \{0, 1, 2, 4, 7, 8, 9\};
I = \{0, 1, 2, 3, 4, 5, 6, 7, 8, 9\}.
\end{equation*}\question
Упростите следующее выражение с учетом того, что $A\subset B \subset C \subset D \subset U; A \neq \O$
\begin{equation*}
\overline{A} \cap \overline{C} \cap D \cup \overline{B} \cap \overline{C} \cap D \cup A \cap B
\end{equation*}

Примечание: U — универсум\question
Дано отношение на множестве $\{1, 2, 3, 4, 5\}$ 
\begin{equation*}
aRb \iff (a+b) \bmod 2 =0
\end{equation*}
Напишите обоснованный ответ какими свойствами обладает или не обладает отношение и почему:   
\begin{enumerate} [a)]\setcounter{enumi}{0}
\item рефлексивность
\item антирефлексивность
\item симметричность
\item асимметричность
\item антисимметричность
\item транзитивность
\end{enumerate}

Обоснуйте свой ответ по каждому из приведенных ниже вопросов:
\begin{enumerate} [a)]\setcounter{enumi}{0}
    \item Является ли это отношение отношением эквивалентности?
    \item Является ли это отношение функциональным?
    \item Каким из отношений соответствия (одно-многозначным, много-многозначный и т.д.) оно является?
    \item К каким из отношений порядка (полного, частичного и т.д.) можно отнести данное отношение?
\end{enumerate}



\question
Установите, является ли каждое из перечисленных ниже отношений на А ($R \subseteq A \times A$) отношением эквивалентности (обоснование ответа обязательно). Для каждого отношения эквивалентности постройте классы 
эквивалентности и постройте граф отношения:
\begin{enumerate} [a)]\setcounter{enumi}{0}
\item $A = \{-10, -9, … , 9, 10\}$ и отношение $R = \{(a,b)|a^{2} = b^{2}\}$
\item $A = \{a, b, c, d, p, t\}$ задано отношение $R = \{(a, a), (b, b), (b, c), (b, d), (c, b), (c, c), (c, d), (d, b), (d, c), (d, d), (p,p), (t,t)\}$
\item Пусть A – множество имен. $A = \{ $Алексей, Иван, Петр, Александр, Павел, Андрей$ \}$. Тогда отношение $R$ верно на парах имен, начинающихся с одной и той же буквы, и только на них.
\end{enumerate}\question Составьте полную таблицу истинности, определите, какие переменные являются фиктивными и проверьте, является ли формула тавтологией:
$(P \rightarrow (Q \rightarrow R)) \rightarrow ((P \rightarrow Q) \rightarrow (P \rightarrow R))$

\end{questions}
\newpage
%%% begin test
\begin{flushright}
\begin{tabular}{p{2.8in} r l}
%\textbf{\class} & \textbf{ФИО:} & \makebox[2.5in]{\hrulefill}\\
\textbf{\class} & \textbf{ФИО:} &Еремченко Владимир Алексеевич
\\

\textbf{\examdate} &&\\
%\textbf{Time Limit: \timelimit} & Teaching Assistant & \makebox[2in]{\hrulefill}
\end{tabular}\\
\end{flushright}
\rule[1ex]{\textwidth}{.1pt}


\begin{questions}
\question
Найдите и упростите P:
\begin{equation*}
\overline{P} = A \cap C \cup \overline{A} \cap \overline{C} \cup \overline{B} \cap C \cup \overline{A} \cap \overline{B}
\end{equation*}
Затем найдите элементы множества P, выраженного через множества:
\begin{equation*}
A = \{0, 3, 4, 9\}; 
B = \{1, 3, 4, 7\};
C = \{0, 1, 2, 4, 7, 8, 9\};
I = \{0, 1, 2, 3, 4, 5, 6, 7, 8, 9\}.
\end{equation*}\question
Упростите следующее выражение с учетом того, что $A\subset B \subset C \subset D \subset U; A \neq \O$
\begin{equation*}
A \cap B \cup \overline{A} \cap \overline{C} \cup A \cap C \cup \overline{B} \cap \overline{C}
\end{equation*}

Примечание: U — универсум\question
Дано отношение на множестве $\{1, 2, 3, 4, 5\}$ 
\begin{equation*}
aRb \iff |a-b| = 1
\end{equation*}
Напишите обоснованный ответ какими свойствами обладает или не обладает отношение и почему:   
\begin{enumerate} [a)]\setcounter{enumi}{0}
\item рефлексивность
\item антирефлексивность
\item симметричность
\item асимметричность
\item антисимметричность
\item транзитивность
\end{enumerate}

Обоснуйте свой ответ по каждому из приведенных ниже вопросов:
\begin{enumerate} [a)]\setcounter{enumi}{0}
    \item Является ли это отношение отношением эквивалентности?
    \item Является ли это отношение функциональным?
    \item Каким из отношений соответствия (одно-многозначным, много-многозначный и т.д.) оно является?
    \item К каким из отношений порядка (полного, частичного и т.д.) можно отнести данное отношение?
\end{enumerate}

\question
Установите, является ли каждое из перечисленных ниже отношений на А ($R \subseteq A \times A$) отношением эквивалентности (обоснование ответа обязательно). Для каждого отношения эквивалентности постройте классы 
эквивалентности и постройте граф отношения:
\begin{enumerate} [a)]\setcounter{enumi}{0}
\item А - множество целых чисел и отношение $R = \{(a,b)|a + b = 5\}$
\item Пусть A – множество имен. $A = \{ $Алексей, Иван, Петр, Александр, Павел, Андрей$ \}$. Тогда отношение $R $ верно на парах имен, начинающихся с одной и той же буквы, и только на них.
\item На множестве $A = \{1; 2; 3; 4; 5\}$ задано отношение $R = \{(1; 2); (1; 3); (1; 5); (2; 3); (2; 4); (2; 5); (3; 4); (3; 5); (4; 5)\}$
\end{enumerate}\question Составьте полную таблицу истинности, определите, какие переменные являются фиктивными и проверьте, является ли формула тавтологией:
$ P \rightarrow (Q \rightarrow ((P \lor Q) \rightarrow (P \land Q)))$

\end{questions}
\newpage
%%% begin test
\begin{flushright}
\begin{tabular}{p{2.8in} r l}
%\textbf{\class} & \textbf{ФИО:} & \makebox[2.5in]{\hrulefill}\\
\textbf{\class} & \textbf{ФИО:} &Либченко Михаил Вадимович
\\

\textbf{\examdate} &&\\
%\textbf{Time Limit: \timelimit} & Teaching Assistant & \makebox[2in]{\hrulefill}
\end{tabular}\\
\end{flushright}
\rule[1ex]{\textwidth}{.1pt}


\begin{questions}
\question
Найдите и упростите P:
\begin{equation*}
\overline{P} = B \cap \overline{C} \cup A \cap B \cup \overline{A} \cap C \cup \overline{A} \cap B
\end{equation*}
Затем найдите элементы множества P, выраженного через множества:
\begin{equation*}
A = \{0, 3, 4, 9\}; 
B = \{1, 3, 4, 7\};
C = \{0, 1, 2, 4, 7, 8, 9\};
I = \{0, 1, 2, 3, 4, 5, 6, 7, 8, 9\}.
\end{equation*}\question
Упростите следующее выражение с учетом того, что $A\subset B \subset C \subset D \subset U; A \neq \O$
\begin{equation*}
\overline{A} \cap \overline{B} \cup B \cap \overline{C} \cup \overline{C} \cap D
\end{equation*}

Примечание: U — универсум\question
Дано отношение на множестве $\{1, 2, 3, 4, 5\}$ 
\begin{equation*}
aRb \iff (a+b) \bmod 2 =0
\end{equation*}
Напишите обоснованный ответ какими свойствами обладает или не обладает отношение и почему:   
\begin{enumerate} [a)]\setcounter{enumi}{0}
\item рефлексивность
\item антирефлексивность
\item симметричность
\item асимметричность
\item антисимметричность
\item транзитивность
\end{enumerate}

Обоснуйте свой ответ по каждому из приведенных ниже вопросов:
\begin{enumerate} [a)]\setcounter{enumi}{0}
    \item Является ли это отношение отношением эквивалентности?
    \item Является ли это отношение функциональным?
    \item Каким из отношений соответствия (одно-многозначным, много-многозначный и т.д.) оно является?
    \item К каким из отношений порядка (полного, частичного и т.д.) можно отнести данное отношение?
\end{enumerate}



\question
Установите, является ли каждое из перечисленных ниже отношений на А ($R \subseteq A \times A$) отношением эквивалентности (обоснование ответа обязательно). Для каждого отношения эквивалентности постройте классы 
эквивалентности и постройте граф отношения:
\begin{enumerate} [a)]\setcounter{enumi}{0}
\item $A = \{-10, -9, … , 9, 10\}$ и отношение $R = \{(a,b)|a^{2} = b^{2}\}$
\item $A = \{a, b, c, d, p, t\}$ задано отношение $R = \{(a, a), (b, b), (b, c), (b, d), (c, b), (c, c), (c, d), (d, b), (d, c), (d, d), (p,p), (t,t)\}$
\item Пусть A – множество имен. $A = \{ $Алексей, Иван, Петр, Александр, Павел, Андрей$ \}$. Тогда отношение $R$ верно на парах имен, начинающихся с одной и той же буквы, и только на них.
\end{enumerate}\question Составьте полную таблицу истинности, определите, какие переменные являются фиктивными и проверьте, является ли формула тавтологией:
$(P \rightarrow (Q \rightarrow R)) \rightarrow ((P \rightarrow Q) \rightarrow (P \rightarrow R))$

\end{questions}
\newpage
%%% begin test
\begin{flushright}
\begin{tabular}{p{2.8in} r l}
%\textbf{\class} & \textbf{ФИО:} & \makebox[2.5in]{\hrulefill}\\
\textbf{\class} & \textbf{ФИО:} &Макаров Михаил Андреевич
\\

\textbf{\examdate} &&\\
%\textbf{Time Limit: \timelimit} & Teaching Assistant & \makebox[2in]{\hrulefill}
\end{tabular}\\
\end{flushright}
\rule[1ex]{\textwidth}{.1pt}


\begin{questions}
\question
Найдите и упростите P:
\begin{equation*}
\overline{P} = A \cap \overline{B} \cup \overline{B} \cap C \cup \overline{A} \cap \overline{B} \cup \overline{A} \cap C
\end{equation*}
Затем найдите элементы множества P, выраженного через множества:
\begin{equation*}
A = \{0, 3, 4, 9\}; 
B = \{1, 3, 4, 7\};
C = \{0, 1, 2, 4, 7, 8, 9\};
I = \{0, 1, 2, 3, 4, 5, 6, 7, 8, 9\}.
\end{equation*}\question
Упростите следующее выражение с учетом того, что $A\subset B \subset C \subset D \subset U; A \neq \O$
\begin{equation*}
\overline{A} \cap \overline{C} \cap D \cup \overline{B} \cap \overline{C} \cap D \cup A \cap B
\end{equation*}

Примечание: U — универсум\question
Для следующего отношения на множестве $\{1, 2, 3, 4, 5\}$ 
\begin{equation*}
aRb \iff 0 < a-b<2
\end{equation*}
Напишите обоснованный ответ какими свойствами обладает или не обладает отношение и почему:   
\begin{enumerate} [a)]\setcounter{enumi}{0}
\item рефлексивность
\item антирефлексивность
\item симметричность
\item асимметричность
\item антисимметричность
\item транзитивность
\end{enumerate}

Обоснуйте свой ответ по каждому из приведенных ниже вопросов:
\begin{enumerate} [a)]\setcounter{enumi}{0}
    \item Является ли это отношение отношением эквивалентности?
    \item Является ли это отношение функциональным?
    \item Каким из отношений соответствия (одно-многозначным, много-многозначный и т.д.) оно является?
    \item К каким из отношений порядка (полного, частичного и т.д.) можно отнести данное отношение?
\end{enumerate}
\question
Установите, является ли каждое из перечисленных ниже отношений на А ($R \subseteq A \times A$) отношением эквивалентности (обоснование ответа обязательно). Для каждого отношения эквивалентности постройте классы эквивалентности и постройте граф отношения:
\begin{enumerate} [a)]\setcounter{enumi}{0}
\item $F(x)=x^{2}+1$, где $x \in A = [-2, 4]$ и отношение $R = \{(a,b)|F(a) = F(b)\}$
\item А - множество целых чисел и отношение $R = \{(a,b)|a + b = 5\}$
\item На множестве $A = \{1; 2; 3\}$ задано отношение $R = \{(1; 1); (2; 2); (3; 3); (3; 2); (1; 2); (2; 1)\}$

\end{enumerate}\question Составьте полную таблицу истинности, определите, какие переменные являются фиктивными и проверьте, является ли формула тавтологией:
$((P \rightarrow Q) \lor R) \leftrightarrow (P \rightarrow (Q \lor R))$

\end{questions}
\newpage
%%% begin test
\begin{flushright}
\begin{tabular}{p{2.8in} r l}
%\textbf{\class} & \textbf{ФИО:} & \makebox[2.5in]{\hrulefill}\\
\textbf{\class} & \textbf{ФИО:} &Мирошниченко Александр Андреевич
\\

\textbf{\examdate} &&\\
%\textbf{Time Limit: \timelimit} & Teaching Assistant & \makebox[2in]{\hrulefill}
\end{tabular}\\
\end{flushright}
\rule[1ex]{\textwidth}{.1pt}


\begin{questions}
\question
Найдите и упростите P:
\begin{equation*}
\overline{P} = \overline{A} \cap B \cup \overline{A} \cap C \cup A \cap \overline{B} \cup \overline{B} \cap C
\end{equation*}
Затем найдите элементы множества P, выраженного через множества:
\begin{equation*}
A = \{0, 3, 4, 9\}; 
B = \{1, 3, 4, 7\};
C = \{0, 1, 2, 4, 7, 8, 9\};
I = \{0, 1, 2, 3, 4, 5, 6, 7, 8, 9\}.
\end{equation*}\question
Упростите следующее выражение с учетом того, что $A\subset B \subset C \subset D \subset U; A \neq \O$
\begin{equation*}
A \cap B \cup \overline{A} \cap \overline{C} \cup A \cap C \cup \overline{B} \cap \overline{C}
\end{equation*}

Примечание: U — универсум\question
Дано отношение на множестве $\{1, 2, 3, 4, 5\}$ 
\begin{equation*}
aRb \iff (a+b) \bmod 2 =0
\end{equation*}
Напишите обоснованный ответ какими свойствами обладает или не обладает отношение и почему:   
\begin{enumerate} [a)]\setcounter{enumi}{0}
\item рефлексивность
\item антирефлексивность
\item симметричность
\item асимметричность
\item антисимметричность
\item транзитивность
\end{enumerate}

Обоснуйте свой ответ по каждому из приведенных ниже вопросов:
\begin{enumerate} [a)]\setcounter{enumi}{0}
    \item Является ли это отношение отношением эквивалентности?
    \item Является ли это отношение функциональным?
    \item Каким из отношений соответствия (одно-многозначным, много-многозначный и т.д.) оно является?
    \item К каким из отношений порядка (полного, частичного и т.д.) можно отнести данное отношение?
\end{enumerate}



\question
Установите, является ли каждое из перечисленных ниже отношений на А ($R \subseteq A \times A$) отношением эквивалентности (обоснование ответа обязательно). Для каждого отношения эквивалентности постройте классы 
эквивалентности и постройте граф отношения:
\begin{enumerate} [a)]\setcounter{enumi}{0}
\item Пусть A – множество имен. $A = \{ $Алексей, Иван, Петр, Александр, Павел, Андрей$ \}$. Тогда отношение $R$ верно на парах имен, начинающихся с одной и той же буквы, и только на них.
\item $A = \{-10, -9, … , 9, 10\}$ и отношение $ R = \{(a,b)|a^{2} = b^{2}\}$
\item На множестве $A = \{1; 2; 3\}$ задано отношение $R = \{(1; 1); (2; 2); (3; 3); (3; 2); (1; 2); (2; 1)\}$
\end{enumerate}\question Составьте полную таблицу истинности, определите, какие переменные являются фиктивными и проверьте, является ли формула тавтологией:
$((P \rightarrow Q) \land (R \rightarrow S) \land \neg (Q \lor S)) \rightarrow \neg (P \lor R)$

\end{questions}
\newpage
%%% begin test
\begin{flushright}
\begin{tabular}{p{2.8in} r l}
%\textbf{\class} & \textbf{ФИО:} & \makebox[2.5in]{\hrulefill}\\
\textbf{\class} & \textbf{ФИО:} &Овсянникова Ольга Андреевна
\\

\textbf{\examdate} &&\\
%\textbf{Time Limit: \timelimit} & Teaching Assistant & \makebox[2in]{\hrulefill}
\end{tabular}\\
\end{flushright}
\rule[1ex]{\textwidth}{.1pt}


\begin{questions}
\question
Найдите и упростите P:
\begin{equation*}
\overline{P} = \overline{A} \cap B \cup \overline{A} \cap C \cup A \cap \overline{B} \cup \overline{B} \cap C
\end{equation*}
Затем найдите элементы множества P, выраженного через множества:
\begin{equation*}
A = \{0, 3, 4, 9\}; 
B = \{1, 3, 4, 7\};
C = \{0, 1, 2, 4, 7, 8, 9\};
I = \{0, 1, 2, 3, 4, 5, 6, 7, 8, 9\}.
\end{equation*}\question
Упростите следующее выражение с учетом того, что $A\subset B \subset C \subset D \subset U; A \neq \O$
\begin{equation*}
\overline{A} \cap \overline{C} \cap D \cup \overline{B} \cap \overline{C} \cap D \cup A \cap B
\end{equation*}

Примечание: U — универсум\question
Для следующего отношения на множестве $\{1, 2, 3, 4, 5\}$ 
\begin{equation*}
aRb \iff 0 < a-b<2
\end{equation*}
Напишите обоснованный ответ какими свойствами обладает или не обладает отношение и почему:   
\begin{enumerate} [a)]\setcounter{enumi}{0}
\item рефлексивность
\item антирефлексивность
\item симметричность
\item асимметричность
\item антисимметричность
\item транзитивность
\end{enumerate}

Обоснуйте свой ответ по каждому из приведенных ниже вопросов:
\begin{enumerate} [a)]\setcounter{enumi}{0}
    \item Является ли это отношение отношением эквивалентности?
    \item Является ли это отношение функциональным?
    \item Каким из отношений соответствия (одно-многозначным, много-многозначный и т.д.) оно является?
    \item К каким из отношений порядка (полного, частичного и т.д.) можно отнести данное отношение?
\end{enumerate}
\question
Установите, является ли каждое из перечисленных ниже отношений на А ($R \subseteq A \times A$) отношением эквивалентности (обоснование ответа обязательно). Для каждого отношения эквивалентности 
постройте классы эквивалентности и постройте граф отношения:
\begin{enumerate}[a)]\setcounter{enumi}{0}
\item А - множество целых чисел и отношение $R = \{(a,b)|a + b = 0\}$
\item $A = \{-10, -9, …, 9, 10\}$ и отношение $R = \{(a,b)|a^{3} = b^{3}\}$
\item На множестве $A = \{1; 2; 3\}$ задано отношение $R = \{(1; 1); (2; 2); (3; 3); (2; 1); (1; 2); (2; 3); (3; 2); (3; 1); (1; 3)\}$

\end{enumerate}\question Составьте полную таблицу истинности, определите, какие переменные являются фиктивными и проверьте, является ли формула тавтологией:
$ P \rightarrow (Q \rightarrow ((P \lor Q) \rightarrow (P \land Q)))$

\end{questions}
\newpage
%%% begin test
\begin{flushright}
\begin{tabular}{p{2.8in} r l}
%\textbf{\class} & \textbf{ФИО:} & \makebox[2.5in]{\hrulefill}\\
\textbf{\class} & \textbf{ФИО:} &Одинец Виктор Алексеевич
\\

\textbf{\examdate} &&\\
%\textbf{Time Limit: \timelimit} & Teaching Assistant & \makebox[2in]{\hrulefill}
\end{tabular}\\
\end{flushright}
\rule[1ex]{\textwidth}{.1pt}


\begin{questions}
\question
Найдите и упростите P:
\begin{equation*}
\overline{P} = \overline{A} \cap B \cup \overline{A} \cap C \cup A \cap \overline{B} \cup \overline{B} \cap C
\end{equation*}
Затем найдите элементы множества P, выраженного через множества:
\begin{equation*}
A = \{0, 3, 4, 9\}; 
B = \{1, 3, 4, 7\};
C = \{0, 1, 2, 4, 7, 8, 9\};
I = \{0, 1, 2, 3, 4, 5, 6, 7, 8, 9\}.
\end{equation*}\question
Упростите следующее выражение с учетом того, что $A\subset B \subset C \subset D \subset U; A \neq \O$
\begin{equation*}
\overline{A} \cap \overline{C} \cap D \cup \overline{B} \cap \overline{C} \cap D \cup A \cap B
\end{equation*}

Примечание: U — универсум\question
Дано отношение на множестве $\{1, 2, 3, 4, 5\}$ 
\begin{equation*}
aRb \iff |a-b| = 1
\end{equation*}
Напишите обоснованный ответ какими свойствами обладает или не обладает отношение и почему:   
\begin{enumerate} [a)]\setcounter{enumi}{0}
\item рефлексивность
\item антирефлексивность
\item симметричность
\item асимметричность
\item антисимметричность
\item транзитивность
\end{enumerate}

Обоснуйте свой ответ по каждому из приведенных ниже вопросов:
\begin{enumerate} [a)]\setcounter{enumi}{0}
    \item Является ли это отношение отношением эквивалентности?
    \item Является ли это отношение функциональным?
    \item Каким из отношений соответствия (одно-многозначным, много-многозначный и т.д.) оно является?
    \item К каким из отношений порядка (полного, частичного и т.д.) можно отнести данное отношение?
\end{enumerate}

\question
Установите, является ли каждое из перечисленных ниже отношений на А ($R \subseteq A \times A$) отношением эквивалентности (обоснование ответа обязательно). Для каждого отношения эквивалентности постройте классы 
эквивалентности и постройте граф отношения:
\begin{enumerate} [a)]\setcounter{enumi}{0}
\item На множестве $A = \{1; 2; 3\}$ задано отношение $R = \{(1; 1); (2; 2); (3; 3); (2; 1); (1; 2); (2; 3); (3; 2); (3; 1); (1; 3)\}$
\item На множестве $A = \{1; 2; 3; 4; 5\}$ задано отношение $R = \{(1; 2); (1; 3); (1; 5); (2; 3); (2; 4); (2; 5); (3; 4); (3; 5); (4; 5)\}$
\item А - множество целых чисел и отношение $R = \{(a,b)|a + b = 0\}$
\end{enumerate}\question Составьте полную таблицу истинности, определите, какие переменные являются фиктивными и проверьте, является ли формула тавтологией:
$((P \rightarrow Q) \land (R \rightarrow S) \land \neg (Q \lor S)) \rightarrow \neg (P \lor R)$

\end{questions}
\newpage
%%% begin test
\begin{flushright}
\begin{tabular}{p{2.8in} r l}
%\textbf{\class} & \textbf{ФИО:} & \makebox[2.5in]{\hrulefill}\\
\textbf{\class} & \textbf{ФИО:} &Орлов Андрей Николаевич
\\

\textbf{\examdate} &&\\
%\textbf{Time Limit: \timelimit} & Teaching Assistant & \makebox[2in]{\hrulefill}
\end{tabular}\\
\end{flushright}
\rule[1ex]{\textwidth}{.1pt}


\begin{questions}
\question
Найдите и упростите P:
\begin{equation*}
\overline{P} = A \cap \overline{B} \cup \overline{B} \cap C \cup \overline{A} \cap \overline{B} \cup \overline{A} \cap C
\end{equation*}
Затем найдите элементы множества P, выраженного через множества:
\begin{equation*}
A = \{0, 3, 4, 9\}; 
B = \{1, 3, 4, 7\};
C = \{0, 1, 2, 4, 7, 8, 9\};
I = \{0, 1, 2, 3, 4, 5, 6, 7, 8, 9\}.
\end{equation*}\question
Упростите следующее выражение с учетом того, что $A\subset B \subset C \subset D \subset U; A \neq \O$
\begin{equation*}
\overline{B} \cap \overline{C} \cap D \cup \overline{A} \cap \overline{C} \cap D \cup \overline{A} \cap B
\end{equation*}

Примечание: U — универсум\question
Для следующего отношения на множестве $\{1, 2, 3, 4, 5\}$ 
\begin{equation*}
aRb \iff 0 < a-b<2
\end{equation*}
Напишите обоснованный ответ какими свойствами обладает или не обладает отношение и почему:   
\begin{enumerate} [a)]\setcounter{enumi}{0}
\item рефлексивность
\item антирефлексивность
\item симметричность
\item асимметричность
\item антисимметричность
\item транзитивность
\end{enumerate}

Обоснуйте свой ответ по каждому из приведенных ниже вопросов:
\begin{enumerate} [a)]\setcounter{enumi}{0}
    \item Является ли это отношение отношением эквивалентности?
    \item Является ли это отношение функциональным?
    \item Каким из отношений соответствия (одно-многозначным, много-многозначный и т.д.) оно является?
    \item К каким из отношений порядка (полного, частичного и т.д.) можно отнести данное отношение?
\end{enumerate}
\question
Установите, является ли каждое из перечисленных ниже отношений на А ($R \subseteq A \times A$) отношением эквивалентности (обоснование ответа обязательно). Для каждого отношения эквивалентности постройте классы 
эквивалентности и постройте граф отношения:
\begin{enumerate} [a)]\setcounter{enumi}{0}
\item А - множество целых чисел и отношение $R = \{(a,b)|a + b = 5\}$
\item Пусть A – множество имен. $A = \{ $Алексей, Иван, Петр, Александр, Павел, Андрей$ \}$. Тогда отношение $R $ верно на парах имен, начинающихся с одной и той же буквы, и только на них.
\item На множестве $A = \{1; 2; 3; 4; 5\}$ задано отношение $R = \{(1; 2); (1; 3); (1; 5); (2; 3); (2; 4); (2; 5); (3; 4); (3; 5); (4; 5)\}$
\end{enumerate}\question Составьте полную таблицу истинности, определите, какие переменные являются фиктивными и проверьте, является ли формула тавтологией:

$(P \rightarrow (Q \land R)) \leftrightarrow ((P \rightarrow Q) \land (P \rightarrow R))$

\end{questions}
\newpage
%%% begin test
\begin{flushright}
\begin{tabular}{p{2.8in} r l}
%\textbf{\class} & \textbf{ФИО:} & \makebox[2.5in]{\hrulefill}\\
\textbf{\class} & \textbf{ФИО:} &Папенко Татьяна Сергеевна
\\

\textbf{\examdate} &&\\
%\textbf{Time Limit: \timelimit} & Teaching Assistant & \makebox[2in]{\hrulefill}
\end{tabular}\\
\end{flushright}
\rule[1ex]{\textwidth}{.1pt}


\begin{questions}
\question
Найдите и упростите P:
\begin{equation*}
\overline{P} = A \cap \overline{C} \cup A \cap \overline{B} \cup B \cap \overline{C} \cup A \cap C
\end{equation*}
Затем найдите элементы множества P, выраженного через множества:
\begin{equation*}
A = \{0, 3, 4, 9\}; 
B = \{1, 3, 4, 7\};
C = \{0, 1, 2, 4, 7, 8, 9\};
I = \{0, 1, 2, 3, 4, 5, 6, 7, 8, 9\}.
\end{equation*}\question
Упростите следующее выражение с учетом того, что $A\subset B \subset C \subset D \subset U; A \neq \O$
\begin{equation*}
A \cap  \overline{C} \cup B \cap \overline{D} \cup  \overline{A} \cap C \cap  \overline{D}
\end{equation*}

Примечание: U — универсум\question
Дано отношение на множестве $\{1, 2, 3, 4, 5\}$ 
\begin{equation*}
aRb \iff (a+b) \bmod 2 =0
\end{equation*}
Напишите обоснованный ответ какими свойствами обладает или не обладает отношение и почему:   
\begin{enumerate} [a)]\setcounter{enumi}{0}
\item рефлексивность
\item антирефлексивность
\item симметричность
\item асимметричность
\item антисимметричность
\item транзитивность
\end{enumerate}

Обоснуйте свой ответ по каждому из приведенных ниже вопросов:
\begin{enumerate} [a)]\setcounter{enumi}{0}
    \item Является ли это отношение отношением эквивалентности?
    \item Является ли это отношение функциональным?
    \item Каким из отношений соответствия (одно-многозначным, много-многозначный и т.д.) оно является?
    \item К каким из отношений порядка (полного, частичного и т.д.) можно отнести данное отношение?
\end{enumerate}



\question
Установите, является ли каждое из перечисленных ниже отношений на А ($R \subseteq A \times A$) отношением эквивалентности (обоснование ответа обязательно). Для каждого отношения эквивалентности постройте классы 
эквивалентности и постройте граф отношения:
\begin{enumerate} [a)]\setcounter{enumi}{0}
\item На множестве $A = \{1; 2; 3\}$ задано отношение $R = \{(1; 1); (2; 2); (3; 3); (2; 1); (1; 2); (2; 3); (3; 2); (3; 1); (1; 3)\}$
\item На множестве $A = \{1; 2; 3; 4; 5\}$ задано отношение $R = \{(1; 2); (1; 3); (1; 5); (2; 3); (2; 4); (2; 5); (3; 4); (3; 5); (4; 5)\}$
\item А - множество целых чисел и отношение $R = \{(a,b)|a + b = 0\}$
\end{enumerate}\question Составьте полную таблицу истинности, определите, какие переменные являются фиктивными и проверьте, является ли формула тавтологией:
$(( P \rightarrow Q) \land (Q \rightarrow P)) \rightarrow (P \rightarrow R)$

\end{questions}
\newpage
%%% begin test
\begin{flushright}
\begin{tabular}{p{2.8in} r l}
%\textbf{\class} & \textbf{ФИО:} & \makebox[2.5in]{\hrulefill}\\
\textbf{\class} & \textbf{ФИО:} &Певцов Дмитрий Валерьевич
\\

\textbf{\examdate} &&\\
%\textbf{Time Limit: \timelimit} & Teaching Assistant & \makebox[2in]{\hrulefill}
\end{tabular}\\
\end{flushright}
\rule[1ex]{\textwidth}{.1pt}


\begin{questions}
\question
Найдите и упростите P:
\begin{equation*}
\overline{P} = B \cap \overline{C} \cup A \cap B \cup \overline{A} \cap C \cup \overline{A} \cap B
\end{equation*}
Затем найдите элементы множества P, выраженного через множества:
\begin{equation*}
A = \{0, 3, 4, 9\}; 
B = \{1, 3, 4, 7\};
C = \{0, 1, 2, 4, 7, 8, 9\};
I = \{0, 1, 2, 3, 4, 5, 6, 7, 8, 9\}.
\end{equation*}\question
Упростите следующее выражение с учетом того, что $A\subset B \subset C \subset D \subset U; A \neq \O$
\begin{equation*}
A \cap  \overline{C} \cup B \cap \overline{D} \cup  \overline{A} \cap C \cap  \overline{D}
\end{equation*}

Примечание: U — универсум\question
Дано отношение на множестве $\{1, 2, 3, 4, 5\}$ 
\begin{equation*}
aRb \iff (a+b) \bmod 2 =0
\end{equation*}
Напишите обоснованный ответ какими свойствами обладает или не обладает отношение и почему:   
\begin{enumerate} [a)]\setcounter{enumi}{0}
\item рефлексивность
\item антирефлексивность
\item симметричность
\item асимметричность
\item антисимметричность
\item транзитивность
\end{enumerate}

Обоснуйте свой ответ по каждому из приведенных ниже вопросов:
\begin{enumerate} [a)]\setcounter{enumi}{0}
    \item Является ли это отношение отношением эквивалентности?
    \item Является ли это отношение функциональным?
    \item Каким из отношений соответствия (одно-многозначным, много-многозначный и т.д.) оно является?
    \item К каким из отношений порядка (полного, частичного и т.д.) можно отнести данное отношение?
\end{enumerate}



\question
Установите, является ли каждое из перечисленных ниже отношений на А ($R \subseteq A \times A$) отношением эквивалентности (обоснование ответа обязательно). Для каждого отношения эквивалентности постройте классы 
эквивалентности и постройте граф отношения:
\begin{enumerate} [a)]\setcounter{enumi}{0}
\item На множестве $A = \{1; 2; 3\}$ задано отношение $R = \{(1; 1); (2; 2); (3; 3); (2; 1); (1; 2); (2; 3); (3; 2); (3; 1); (1; 3)\}$
\item На множестве $A = \{1; 2; 3; 4; 5\}$ задано отношение $R = \{(1; 2); (1; 3); (1; 5); (2; 3); (2; 4); (2; 5); (3; 4); (3; 5); (4; 5)\}$
\item А - множество целых чисел и отношение $R = \{(a,b)|a + b = 0\}$
\end{enumerate}\question Составьте полную таблицу истинности, определите, какие переменные являются фиктивными и проверьте, является ли формула тавтологией:
$ P \rightarrow (Q \rightarrow ((P \lor Q) \rightarrow (P \land Q)))$

\end{questions}
\newpage
%%% begin test
\begin{flushright}
\begin{tabular}{p{2.8in} r l}
%\textbf{\class} & \textbf{ФИО:} & \makebox[2.5in]{\hrulefill}\\
\textbf{\class} & \textbf{ФИО:} &Ржевская Александра Валентиновна
\\

\textbf{\examdate} &&\\
%\textbf{Time Limit: \timelimit} & Teaching Assistant & \makebox[2in]{\hrulefill}
\end{tabular}\\
\end{flushright}
\rule[1ex]{\textwidth}{.1pt}


\begin{questions}
\question
Найдите и упростите P:
\begin{equation*}
\overline{P} = B \cap \overline{C} \cup A \cap B \cup \overline{A} \cap C \cup \overline{A} \cap B
\end{equation*}
Затем найдите элементы множества P, выраженного через множества:
\begin{equation*}
A = \{0, 3, 4, 9\}; 
B = \{1, 3, 4, 7\};
C = \{0, 1, 2, 4, 7, 8, 9\};
I = \{0, 1, 2, 3, 4, 5, 6, 7, 8, 9\}.
\end{equation*}\question
Упростите следующее выражение с учетом того, что $A\subset B \subset C \subset D \subset U; A \neq \O$
\begin{equation*}
A \cap B  \cap \overline{C} \cup \overline{C} \cap D \cup B \cap C \cap D
\end{equation*}

Примечание: U — универсум\question
Дано отношение на множестве $\{1, 2, 3, 4, 5\}$ 
\begin{equation*}
aRb \iff a \leq b
\end{equation*}
Напишите обоснованный ответ какими свойствами обладает или не обладает отношение и почему:   
\begin{enumerate} [a)]\setcounter{enumi}{0}
\item рефлексивность
\item антирефлексивность
\item симметричность
\item асимметричность
\item антисимметричность
\item транзитивность
\end{enumerate}

Обоснуйте свой ответ по каждому из приведенных ниже вопросов:
\begin{enumerate} [a)]\setcounter{enumi}{0}
    \item Является ли это отношение отношением эквивалентности?
    \item Является ли это отношение функциональным?
    \item Каким из отношений соответствия (одно-многозначным, много-многозначный и т.д.) оно является?
    \item К каким из отношений порядка (полного, частичного и т.д.) можно отнести данное отношение?
\end{enumerate}


\question
Установите, является ли каждое из перечисленных ниже отношений на А ($R \subseteq A \times A$) отношением эквивалентности (обоснование ответа обязательно). Для каждого отношения эквивалентности 
постройте классы эквивалентности и постройте граф отношения:
\begin{enumerate}[a)]\setcounter{enumi}{0}
\item А - множество целых чисел и отношение $R = \{(a,b)|a + b = 0\}$
\item $A = \{-10, -9, …, 9, 10\}$ и отношение $R = \{(a,b)|a^{3} = b^{3}\}$
\item На множестве $A = \{1; 2; 3\}$ задано отношение $R = \{(1; 1); (2; 2); (3; 3); (2; 1); (1; 2); (2; 3); (3; 2); (3; 1); (1; 3)\}$

\end{enumerate}\question Составьте полную таблицу истинности, определите, какие переменные являются фиктивными и проверьте, является ли формула тавтологией:
$(( P \rightarrow Q) \land (Q \rightarrow P)) \rightarrow (P \rightarrow R)$

\end{questions}
\newpage
%%% begin test
\begin{flushright}
\begin{tabular}{p{2.8in} r l}
%\textbf{\class} & \textbf{ФИО:} & \makebox[2.5in]{\hrulefill}\\
\textbf{\class} & \textbf{ФИО:} &Саратовцев Эдгар Юрьевич
\\

\textbf{\examdate} &&\\
%\textbf{Time Limit: \timelimit} & Teaching Assistant & \makebox[2in]{\hrulefill}
\end{tabular}\\
\end{flushright}
\rule[1ex]{\textwidth}{.1pt}


\begin{questions}
\question
Найдите и упростите P:
\begin{equation*}
\overline{P} = A \cap C \cup \overline{A} \cap \overline{C} \cup \overline{B} \cap C \cup \overline{A} \cap \overline{B}
\end{equation*}
Затем найдите элементы множества P, выраженного через множества:
\begin{equation*}
A = \{0, 3, 4, 9\}; 
B = \{1, 3, 4, 7\};
C = \{0, 1, 2, 4, 7, 8, 9\};
I = \{0, 1, 2, 3, 4, 5, 6, 7, 8, 9\}.
\end{equation*}\question
Упростите следующее выражение с учетом того, что $A\subset B \subset C \subset D \subset U; A \neq \O$
\begin{equation*}
A \cap C  \cap D \cup B \cap \overline{C} \cap D \cup B \cap C \cap D
\end{equation*}

Примечание: U — универсум\question
Дано отношение на множестве $\{1, 2, 3, 4, 5\}$ 
\begin{equation*}
aRb \iff a \geq b^2
\end{equation*}
Напишите обоснованный ответ какими свойствами обладает или не обладает отношение и почему:   
\begin{enumerate} [a)]\setcounter{enumi}{0}
\item рефлексивность
\item антирефлексивность
\item симметричность
\item асимметричность
\item антисимметричность
\item транзитивность
\end{enumerate}

Обоснуйте свой ответ по каждому из приведенных ниже вопросов:
\begin{enumerate} [a)]\setcounter{enumi}{0}
    \item Является ли это отношение отношением эквивалентности?
    \item Является ли это отношение функциональным?
    \item Каким из отношений соответствия (одно-многозначным, много-многозначный и т.д.) оно является?
    \item К каким из отношений порядка (полного, частичного и т.д.) можно отнести данное отношение?
\end{enumerate}


\question
Установите, является ли каждое из перечисленных ниже отношений на А ($R \subseteq A \times A$) отношением эквивалентности (обоснование ответа обязательно). Для каждого отношения эквивалентности постройте классы эквивалентности и постройте граф отношения:
\begin{enumerate} [a)]\setcounter{enumi}{0}
\item $F(x)=x^{2}+1$, где $x \in A = [-2, 4]$ и отношение $R = \{(a,b)|F(a) = F(b)\}$
\item А - множество целых чисел и отношение $R = \{(a,b)|a + b = 5\}$
\item На множестве $A = \{1; 2; 3\}$ задано отношение $R = \{(1; 1); (2; 2); (3; 3); (3; 2); (1; 2); (2; 1)\}$

\end{enumerate}\question Составьте полную таблицу истинности, определите, какие переменные являются фиктивными и проверьте, является ли формула тавтологией:
$(P \rightarrow (Q \rightarrow R)) \rightarrow ((P \rightarrow Q) \rightarrow (P \rightarrow R))$

\end{questions}
\newpage
%%% begin test
\begin{flushright}
\begin{tabular}{p{2.8in} r l}
%\textbf{\class} & \textbf{ФИО:} & \makebox[2.5in]{\hrulefill}\\
\textbf{\class} & \textbf{ФИО:} &Свистунов Александр Андреевич
\\

\textbf{\examdate} &&\\
%\textbf{Time Limit: \timelimit} & Teaching Assistant & \makebox[2in]{\hrulefill}
\end{tabular}\\
\end{flushright}
\rule[1ex]{\textwidth}{.1pt}


\begin{questions}
\question
Найдите и упростите P:
\begin{equation*}
\overline{P} = A \cap \overline{B} \cup A \cap C \cup B \cap C \cup \overline{A} \cap C
\end{equation*}
Затем найдите элементы множества P, выраженного через множества:
\begin{equation*}
A = \{0, 3, 4, 9\}; 
B = \{1, 3, 4, 7\};
C = \{0, 1, 2, 4, 7, 8, 9\};
I = \{0, 1, 2, 3, 4, 5, 6, 7, 8, 9\}.
\end{equation*}\question
Упростите следующее выражение с учетом того, что $A\subset B \subset C \subset D \subset U; A \neq \O$
\begin{equation*}
A \cap  \overline{C} \cup B \cap \overline{D} \cup  \overline{A} \cap C \cap  \overline{D}
\end{equation*}

Примечание: U — универсум\question
Дано отношение на множестве $\{1, 2, 3, 4, 5\}$ 
\begin{equation*}
aRb \iff a \leq b
\end{equation*}
Напишите обоснованный ответ какими свойствами обладает или не обладает отношение и почему:   
\begin{enumerate} [a)]\setcounter{enumi}{0}
\item рефлексивность
\item антирефлексивность
\item симметричность
\item асимметричность
\item антисимметричность
\item транзитивность
\end{enumerate}

Обоснуйте свой ответ по каждому из приведенных ниже вопросов:
\begin{enumerate} [a)]\setcounter{enumi}{0}
    \item Является ли это отношение отношением эквивалентности?
    \item Является ли это отношение функциональным?
    \item Каким из отношений соответствия (одно-многозначным, много-многозначный и т.д.) оно является?
    \item К каким из отношений порядка (полного, частичного и т.д.) можно отнести данное отношение?
\end{enumerate}


\question
Установите, является ли каждое из перечисленных ниже отношений на А ($R \subseteq A \times A$) отношением эквивалентности (обоснование ответа обязательно). Для каждого отношения эквивалентности 
постройте классы эквивалентности и постройте граф отношения:
\begin{enumerate}[a)]\setcounter{enumi}{0}
\item А - множество целых чисел и отношение $R = \{(a,b)|a + b = 0\}$
\item $A = \{-10, -9, …, 9, 10\}$ и отношение $R = \{(a,b)|a^{3} = b^{3}\}$
\item На множестве $A = \{1; 2; 3\}$ задано отношение $R = \{(1; 1); (2; 2); (3; 3); (2; 1); (1; 2); (2; 3); (3; 2); (3; 1); (1; 3)\}$

\end{enumerate}\question Составьте полную таблицу истинности, определите, какие переменные являются фиктивными и проверьте, является ли формула тавтологией:

$(P \rightarrow (Q \land R)) \leftrightarrow ((P \rightarrow Q) \land (P \rightarrow R))$

\end{questions}
\newpage
%%% begin test
\begin{flushright}
\begin{tabular}{p{2.8in} r l}
%\textbf{\class} & \textbf{ФИО:} & \makebox[2.5in]{\hrulefill}\\
\textbf{\class} & \textbf{ФИО:} &Смирнов Иван Сергеевич
\\

\textbf{\examdate} &&\\
%\textbf{Time Limit: \timelimit} & Teaching Assistant & \makebox[2in]{\hrulefill}
\end{tabular}\\
\end{flushright}
\rule[1ex]{\textwidth}{.1pt}


\begin{questions}
\question
Найдите и упростите P:
\begin{equation*}
\overline{P} = A \cap \overline{B} \cup \overline{B} \cap C \cup \overline{A} \cap \overline{B} \cup \overline{A} \cap C
\end{equation*}
Затем найдите элементы множества P, выраженного через множества:
\begin{equation*}
A = \{0, 3, 4, 9\}; 
B = \{1, 3, 4, 7\};
C = \{0, 1, 2, 4, 7, 8, 9\};
I = \{0, 1, 2, 3, 4, 5, 6, 7, 8, 9\}.
\end{equation*}\question
Упростите следующее выражение с учетом того, что $A\subset B \subset C \subset D \subset U; A \neq \O$
\begin{equation*}
A \cap B \cup \overline{A} \cap \overline{C} \cup A \cap C \cup \overline{B} \cap \overline{C}
\end{equation*}

Примечание: U — универсум\question
Дано отношение на множестве $\{1, 2, 3, 4, 5\}$ 
\begin{equation*}
aRb \iff (a+b) \bmod 2 =0
\end{equation*}
Напишите обоснованный ответ какими свойствами обладает или не обладает отношение и почему:   
\begin{enumerate} [a)]\setcounter{enumi}{0}
\item рефлексивность
\item антирефлексивность
\item симметричность
\item асимметричность
\item антисимметричность
\item транзитивность
\end{enumerate}

Обоснуйте свой ответ по каждому из приведенных ниже вопросов:
\begin{enumerate} [a)]\setcounter{enumi}{0}
    \item Является ли это отношение отношением эквивалентности?
    \item Является ли это отношение функциональным?
    \item Каким из отношений соответствия (одно-многозначным, много-многозначный и т.д.) оно является?
    \item К каким из отношений порядка (полного, частичного и т.д.) можно отнести данное отношение?
\end{enumerate}



\question
Установите, является ли каждое из перечисленных ниже отношений на А ($R \subseteq A \times A$) отношением эквивалентности (обоснование ответа обязательно). Для каждого отношения эквивалентности постройте классы 
эквивалентности и постройте граф отношения:
\begin{enumerate} [a)]\setcounter{enumi}{0}
\item $A = \{a, b, c, d, p, t\}$ задано отношение $R = \{(a, a), (b, b), (b, c), (b, d), (c, b), (c, c), (c, d), (d, b), (d, c), (d, d), (p,p), (t,t)\}$
\item $A = \{-10, -9, … , 9, 10\}$ и отношение $R = \{(a,b)|a^{3} = b^{3}\}$

\item $F(x)=x^{2}+1$, где $x \in A = [-2, 4]$ и отношение $R = \{(a,b)|F(a) = F(b)\}$
\end{enumerate}\question Составьте полную таблицу истинности, определите, какие переменные являются фиктивными и проверьте, является ли формула тавтологией:
$((P \rightarrow Q) \land (R \rightarrow S) \land \neg (Q \lor S)) \rightarrow \neg (P \lor R)$

\end{questions}
\newpage
%%% begin test
\begin{flushright}
\begin{tabular}{p{2.8in} r l}
%\textbf{\class} & \textbf{ФИО:} & \makebox[2.5in]{\hrulefill}\\
\textbf{\class} & \textbf{ФИО:} &Суслов Михаил Анатольевич
\\

\textbf{\examdate} &&\\
%\textbf{Time Limit: \timelimit} & Teaching Assistant & \makebox[2in]{\hrulefill}
\end{tabular}\\
\end{flushright}
\rule[1ex]{\textwidth}{.1pt}


\begin{questions}
\question
Найдите и упростите P:
\begin{equation*}
\overline{P} = A \cap \overline{B} \cup A \cap C \cup B \cap C \cup \overline{A} \cap C
\end{equation*}
Затем найдите элементы множества P, выраженного через множества:
\begin{equation*}
A = \{0, 3, 4, 9\}; 
B = \{1, 3, 4, 7\};
C = \{0, 1, 2, 4, 7, 8, 9\};
I = \{0, 1, 2, 3, 4, 5, 6, 7, 8, 9\}.
\end{equation*}\question
Упростите следующее выражение с учетом того, что $A\subset B \subset C \subset D \subset U; A \neq \O$
\begin{equation*}
A \cap B \cup \overline{A} \cap \overline{C} \cup A \cap C \cup \overline{B} \cap \overline{C}
\end{equation*}

Примечание: U — универсум\question
Дано отношение на множестве $\{1, 2, 3, 4, 5\}$ 
\begin{equation*}
aRb \iff (a+b) \bmod 2 =0
\end{equation*}
Напишите обоснованный ответ какими свойствами обладает или не обладает отношение и почему:   
\begin{enumerate} [a)]\setcounter{enumi}{0}
\item рефлексивность
\item антирефлексивность
\item симметричность
\item асимметричность
\item антисимметричность
\item транзитивность
\end{enumerate}

Обоснуйте свой ответ по каждому из приведенных ниже вопросов:
\begin{enumerate} [a)]\setcounter{enumi}{0}
    \item Является ли это отношение отношением эквивалентности?
    \item Является ли это отношение функциональным?
    \item Каким из отношений соответствия (одно-многозначным, много-многозначный и т.д.) оно является?
    \item К каким из отношений порядка (полного, частичного и т.д.) можно отнести данное отношение?
\end{enumerate}



\question
Установите, является ли каждое из перечисленных ниже отношений на А ($R \subseteq A \times A$) отношением эквивалентности (обоснование ответа обязательно). Для каждого отношения эквивалентности постройте классы 
эквивалентности и постройте граф отношения:
\begin{enumerate} [a)]\setcounter{enumi}{0}
\item А - множество целых чисел и отношение $R = \{(a,b)|a + b = 5\}$
\item Пусть A – множество имен. $A = \{ $Алексей, Иван, Петр, Александр, Павел, Андрей$ \}$. Тогда отношение $R $ верно на парах имен, начинающихся с одной и той же буквы, и только на них.
\item На множестве $A = \{1; 2; 3; 4; 5\}$ задано отношение $R = \{(1; 2); (1; 3); (1; 5); (2; 3); (2; 4); (2; 5); (3; 4); (3; 5); (4; 5)\}$
\end{enumerate}\question Составьте полную таблицу истинности, определите, какие переменные являются фиктивными и проверьте, является ли формула тавтологией:
$ P \rightarrow (Q \rightarrow ((P \lor Q) \rightarrow (P \land Q)))$

\end{questions}
\newpage
%%% begin test
\begin{flushright}
\begin{tabular}{p{2.8in} r l}
%\textbf{\class} & \textbf{ФИО:} & \makebox[2.5in]{\hrulefill}\\
\textbf{\class} & \textbf{ФИО:} &Файзуллин Тагир Русланович
\\

\textbf{\examdate} &&\\
%\textbf{Time Limit: \timelimit} & Teaching Assistant & \makebox[2in]{\hrulefill}
\end{tabular}\\
\end{flushright}
\rule[1ex]{\textwidth}{.1pt}


\begin{questions}
\question
Найдите и упростите P:
\begin{equation*}
\overline{P} = \overline{A} \cap B \cup \overline{A} \cap C \cup A \cap \overline{B} \cup \overline{B} \cap C
\end{equation*}
Затем найдите элементы множества P, выраженного через множества:
\begin{equation*}
A = \{0, 3, 4, 9\}; 
B = \{1, 3, 4, 7\};
C = \{0, 1, 2, 4, 7, 8, 9\};
I = \{0, 1, 2, 3, 4, 5, 6, 7, 8, 9\}.
\end{equation*}\question
Упростите следующее выражение с учетом того, что $A\subset B \subset C \subset D \subset U; A \neq \O$
\begin{equation*}
A \cap  \overline{C} \cup B \cap \overline{D} \cup  \overline{A} \cap C \cap  \overline{D}
\end{equation*}

Примечание: U — универсум\question
Дано отношение на множестве $\{1, 2, 3, 4, 5\}$ 
\begin{equation*}
aRb \iff a \geq b^2
\end{equation*}
Напишите обоснованный ответ какими свойствами обладает или не обладает отношение и почему:   
\begin{enumerate} [a)]\setcounter{enumi}{0}
\item рефлексивность
\item антирефлексивность
\item симметричность
\item асимметричность
\item антисимметричность
\item транзитивность
\end{enumerate}

Обоснуйте свой ответ по каждому из приведенных ниже вопросов:
\begin{enumerate} [a)]\setcounter{enumi}{0}
    \item Является ли это отношение отношением эквивалентности?
    \item Является ли это отношение функциональным?
    \item Каким из отношений соответствия (одно-многозначным, много-многозначный и т.д.) оно является?
    \item К каким из отношений порядка (полного, частичного и т.д.) можно отнести данное отношение?
\end{enumerate}


\question
Установите, является ли каждое из перечисленных ниже отношений на А ($R \subseteq A \times A$) отношением эквивалентности (обоснование ответа обязательно). Для каждого отношения эквивалентности постройте классы эквивалентности и постройте граф отношения:
\begin{enumerate} [a)]\setcounter{enumi}{0}
\item $F(x)=x^{2}+1$, где $x \in A = [-2, 4]$ и отношение $R = \{(a,b)|F(a) = F(b)\}$
\item А - множество целых чисел и отношение $R = \{(a,b)|a + b = 5\}$
\item На множестве $A = \{1; 2; 3\}$ задано отношение $R = \{(1; 1); (2; 2); (3; 3); (3; 2); (1; 2); (2; 1)\}$

\end{enumerate}\question Составьте полную таблицу истинности, определите, какие переменные являются фиктивными и проверьте, является ли формула тавтологией:
$(( P \rightarrow Q) \land (Q \rightarrow P)) \rightarrow (P \rightarrow R)$

\end{questions}
\newpage
%%% begin test
\begin{flushright}
\begin{tabular}{p{2.8in} r l}
%\textbf{\class} & \textbf{ФИО:} & \makebox[2.5in]{\hrulefill}\\
\textbf{\class} & \textbf{ФИО:} &Ханин Виктор Александрович
\\

\textbf{\examdate} &&\\
%\textbf{Time Limit: \timelimit} & Teaching Assistant & \makebox[2in]{\hrulefill}
\end{tabular}\\
\end{flushright}
\rule[1ex]{\textwidth}{.1pt}


\begin{questions}
\question
Найдите и упростите P:
\begin{equation*}
\overline{P} = A \cap C \cup \overline{A} \cap \overline{C} \cup \overline{B} \cap C \cup \overline{A} \cap \overline{B}
\end{equation*}
Затем найдите элементы множества P, выраженного через множества:
\begin{equation*}
A = \{0, 3, 4, 9\}; 
B = \{1, 3, 4, 7\};
C = \{0, 1, 2, 4, 7, 8, 9\};
I = \{0, 1, 2, 3, 4, 5, 6, 7, 8, 9\}.
\end{equation*}\question
Упростите следующее выражение с учетом того, что $A\subset B \subset C \subset D \subset U; A \neq \O$
\begin{equation*}
A \cap C  \cap D \cup B \cap \overline{C} \cap D \cup B \cap C \cap D
\end{equation*}

Примечание: U — универсум\question
Дано отношение на множестве $\{1, 2, 3, 4, 5\}$ 
\begin{equation*}
aRb \iff a \leq b
\end{equation*}
Напишите обоснованный ответ какими свойствами обладает или не обладает отношение и почему:   
\begin{enumerate} [a)]\setcounter{enumi}{0}
\item рефлексивность
\item антирефлексивность
\item симметричность
\item асимметричность
\item антисимметричность
\item транзитивность
\end{enumerate}

Обоснуйте свой ответ по каждому из приведенных ниже вопросов:
\begin{enumerate} [a)]\setcounter{enumi}{0}
    \item Является ли это отношение отношением эквивалентности?
    \item Является ли это отношение функциональным?
    \item Каким из отношений соответствия (одно-многозначным, много-многозначный и т.д.) оно является?
    \item К каким из отношений порядка (полного, частичного и т.д.) можно отнести данное отношение?
\end{enumerate}


\question
Установите, является ли каждое из перечисленных ниже отношений на А ($R \subseteq A \times A$) отношением эквивалентности (обоснование ответа обязательно). Для каждого отношения эквивалентности постройте классы 
эквивалентности и постройте граф отношения:
\begin{enumerate} [a)]\setcounter{enumi}{0}
\item На множестве $A = \{1; 2; 3\}$ задано отношение $R = \{(1; 1); (2; 2); (3; 3); (2; 1); (1; 2); (2; 3); (3; 2); (3; 1); (1; 3)\}$
\item На множестве $A = \{1; 2; 3; 4; 5\}$ задано отношение $R = \{(1; 2); (1; 3); (1; 5); (2; 3); (2; 4); (2; 5); (3; 4); (3; 5); (4; 5)\}$
\item А - множество целых чисел и отношение $R = \{(a,b)|a + b = 0\}$
\end{enumerate}\question Составьте полную таблицу истинности, определите, какие переменные являются фиктивными и проверьте, является ли формула тавтологией:
$((P \rightarrow Q) \land (R \rightarrow S) \land \neg (Q \lor S)) \rightarrow \neg (P \lor R)$

\end{questions}
\newpage
%%% begin test
\begin{flushright}
\begin{tabular}{p{2.8in} r l}
%\textbf{\class} & \textbf{ФИО:} & \makebox[2.5in]{\hrulefill}\\
\textbf{\class} & \textbf{ФИО:} &Цыганков Андрей Петрович
\\

\textbf{\examdate} &&\\
%\textbf{Time Limit: \timelimit} & Teaching Assistant & \makebox[2in]{\hrulefill}
\end{tabular}\\
\end{flushright}
\rule[1ex]{\textwidth}{.1pt}


\begin{questions}
\question
Найдите и упростите P:
\begin{equation*}
\overline{P} = A \cap \overline{B} \cup A \cap C \cup B \cap C \cup \overline{A} \cap C
\end{equation*}
Затем найдите элементы множества P, выраженного через множества:
\begin{equation*}
A = \{0, 3, 4, 9\}; 
B = \{1, 3, 4, 7\};
C = \{0, 1, 2, 4, 7, 8, 9\};
I = \{0, 1, 2, 3, 4, 5, 6, 7, 8, 9\}.
\end{equation*}\question
Упростите следующее выражение с учетом того, что $A\subset B \subset C \subset D \subset U; A \neq \O$
\begin{equation*}
\overline{B} \cap \overline{C} \cap D \cup \overline{A} \cap \overline{C} \cap D \cup \overline{A} \cap B
\end{equation*}

Примечание: U — универсум\question
Дано отношение на множестве $\{1, 2, 3, 4, 5\}$ 
\begin{equation*}
aRb \iff b > a
\end{equation*}
Напишите обоснованный ответ какими свойствами обладает или не обладает отношение и почему:   
\begin{enumerate} [a)]\setcounter{enumi}{0}
\item рефлексивность
\item антирефлексивность
\item симметричность
\item асимметричность
\item антисимметричность
\item транзитивность
\end{enumerate}

Обоснуйте свой ответ по каждому из приведенных ниже вопросов:
\begin{enumerate} [a)]\setcounter{enumi}{0}
    \item Является ли это отношение отношением эквивалентности?
    \item Является ли это отношение функциональным?
    \item Каким из отношений соответствия (одно-многозначным, много-многозначный и т.д.) оно является?
    \item К каким из отношений порядка (полного, частичного и т.д.) можно отнести данное отношение?
\end{enumerate}

\question
Установите, является ли каждое из перечисленных ниже отношений на А ($R \subseteq A \times A$) отношением эквивалентности (обоснование ответа обязательно). Для каждого отношения эквивалентности постройте классы эквивалентности и постройте граф отношения:
\begin{enumerate} [a)]\setcounter{enumi}{0}
\item $F(x)=x^{2}+1$, где $x \in A = [-2, 4]$ и отношение $R = \{(a,b)|F(a) = F(b)\}$
\item А - множество целых чисел и отношение $R = \{(a,b)|a + b = 5\}$
\item На множестве $A = \{1; 2; 3\}$ задано отношение $R = \{(1; 1); (2; 2); (3; 3); (3; 2); (1; 2); (2; 1)\}$

\end{enumerate}\question Составьте полную таблицу истинности, определите, какие переменные являются фиктивными и проверьте, является ли формула тавтологией:
$(( P \land \neg Q) \rightarrow (R \land \neg R)) \rightarrow (P \rightarrow Q)$

\end{questions}
\newpage
%%% begin test
\begin{flushright}
\begin{tabular}{p{2.8in} r l}
%\textbf{\class} & \textbf{ФИО:} & \makebox[2.5in]{\hrulefill}\\
\textbf{\class} & \textbf{ФИО:} &Чернышев Матвей Александрович
\\

\textbf{\examdate} &&\\
%\textbf{Time Limit: \timelimit} & Teaching Assistant & \makebox[2in]{\hrulefill}
\end{tabular}\\
\end{flushright}
\rule[1ex]{\textwidth}{.1pt}


\begin{questions}
\question
Найдите и упростите P:
\begin{equation*}
\overline{P} = A \cap \overline{B} \cup A \cap C \cup B \cap C \cup \overline{A} \cap C
\end{equation*}
Затем найдите элементы множества P, выраженного через множества:
\begin{equation*}
A = \{0, 3, 4, 9\}; 
B = \{1, 3, 4, 7\};
C = \{0, 1, 2, 4, 7, 8, 9\};
I = \{0, 1, 2, 3, 4, 5, 6, 7, 8, 9\}.
\end{equation*}\question
Упростите следующее выражение с учетом того, что $A\subset B \subset C \subset D \subset U; A \neq \O$
\begin{equation*}
A \cap B  \cap \overline{C} \cup \overline{C} \cap D \cup B \cap C \cap D
\end{equation*}

Примечание: U — универсум\question
Для следующего отношения на множестве $\{1, 2, 3, 4, 5\}$ 
\begin{equation*}
aRb \iff 0 < a-b<2
\end{equation*}
Напишите обоснованный ответ какими свойствами обладает или не обладает отношение и почему:   
\begin{enumerate} [a)]\setcounter{enumi}{0}
\item рефлексивность
\item антирефлексивность
\item симметричность
\item асимметричность
\item антисимметричность
\item транзитивность
\end{enumerate}

Обоснуйте свой ответ по каждому из приведенных ниже вопросов:
\begin{enumerate} [a)]\setcounter{enumi}{0}
    \item Является ли это отношение отношением эквивалентности?
    \item Является ли это отношение функциональным?
    \item Каким из отношений соответствия (одно-многозначным, много-многозначный и т.д.) оно является?
    \item К каким из отношений порядка (полного, частичного и т.д.) можно отнести данное отношение?
\end{enumerate}
\question
Установите, является ли каждое из перечисленных ниже отношений на А ($R \subseteq A \times A$) отношением эквивалентности (обоснование ответа обязательно). Для каждого отношения эквивалентности 
постройте классы эквивалентности и постройте граф отношения:
\begin{enumerate}[a)]\setcounter{enumi}{0}
\item А - множество целых чисел и отношение $R = \{(a,b)|a + b = 0\}$
\item $A = \{-10, -9, …, 9, 10\}$ и отношение $R = \{(a,b)|a^{3} = b^{3}\}$
\item На множестве $A = \{1; 2; 3\}$ задано отношение $R = \{(1; 1); (2; 2); (3; 3); (2; 1); (1; 2); (2; 3); (3; 2); (3; 1); (1; 3)\}$

\end{enumerate}\question Составьте полную таблицу истинности, определите, какие переменные являются фиктивными и проверьте, является ли формула тавтологией:
$ P \rightarrow (Q \rightarrow ((P \lor Q) \rightarrow (P \land Q)))$

\end{questions}
\newpage
%%% begin test
\begin{flushright}
\begin{tabular}{p{2.8in} r l}
%\textbf{\class} & \textbf{ФИО:} & \makebox[2.5in]{\hrulefill}\\
\textbf{\class} & \textbf{ФИО:} &Шевчук Даниил Александрович
\\

\textbf{\examdate} &&\\
%\textbf{Time Limit: \timelimit} & Teaching Assistant & \makebox[2in]{\hrulefill}
\end{tabular}\\
\end{flushright}
\rule[1ex]{\textwidth}{.1pt}


\begin{questions}
\question
Найдите и упростите P:
\begin{equation*}
\overline{P} = A \cap \overline{C} \cup A \cap \overline{B} \cup B \cap \overline{C} \cup A \cap C
\end{equation*}
Затем найдите элементы множества P, выраженного через множества:
\begin{equation*}
A = \{0, 3, 4, 9\}; 
B = \{1, 3, 4, 7\};
C = \{0, 1, 2, 4, 7, 8, 9\};
I = \{0, 1, 2, 3, 4, 5, 6, 7, 8, 9\}.
\end{equation*}\question
Упростите следующее выражение с учетом того, что $A\subset B \subset C \subset D \subset U; A \neq \O$
\begin{equation*}
\overline{B} \cap \overline{C} \cap D \cup \overline{A} \cap \overline{C} \cap D \cup \overline{A} \cap B
\end{equation*}

Примечание: U — универсум\question
Дано отношение на множестве $\{1, 2, 3, 4, 5\}$ 
\begin{equation*}
aRb \iff |a-b| = 1
\end{equation*}
Напишите обоснованный ответ какими свойствами обладает или не обладает отношение и почему:   
\begin{enumerate} [a)]\setcounter{enumi}{0}
\item рефлексивность
\item антирефлексивность
\item симметричность
\item асимметричность
\item антисимметричность
\item транзитивность
\end{enumerate}

Обоснуйте свой ответ по каждому из приведенных ниже вопросов:
\begin{enumerate} [a)]\setcounter{enumi}{0}
    \item Является ли это отношение отношением эквивалентности?
    \item Является ли это отношение функциональным?
    \item Каким из отношений соответствия (одно-многозначным, много-многозначный и т.д.) оно является?
    \item К каким из отношений порядка (полного, частичного и т.д.) можно отнести данное отношение?
\end{enumerate}

\question
Установите, является ли каждое из перечисленных ниже отношений на А ($R \subseteq A \times A$) отношением эквивалентности (обоснование ответа обязательно). Для каждого отношения эквивалентности 
постройте классы эквивалентности и постройте граф отношения:
\begin{enumerate}[a)]\setcounter{enumi}{0}
\item А - множество целых чисел и отношение $R = \{(a,b)|a + b = 0\}$
\item $A = \{-10, -9, …, 9, 10\}$ и отношение $R = \{(a,b)|a^{3} = b^{3}\}$
\item На множестве $A = \{1; 2; 3\}$ задано отношение $R = \{(1; 1); (2; 2); (3; 3); (2; 1); (1; 2); (2; 3); (3; 2); (3; 1); (1; 3)\}$

\end{enumerate}\question Составьте полную таблицу истинности, определите, какие переменные являются фиктивными и проверьте, является ли формула тавтологией:
$ P \rightarrow (Q \rightarrow ((P \lor Q) \rightarrow (P \land Q)))$

\end{questions}
\newpage
%%% begin test
\begin{flushright}
\begin{tabular}{p{2.8in} r l}
%\textbf{\class} & \textbf{ФИО:} & \makebox[2.5in]{\hrulefill}\\
\textbf{\class} & \textbf{ФИО:} &Шеин Максим Андреевич
\\

\textbf{\examdate} &&\\
%\textbf{Time Limit: \timelimit} & Teaching Assistant & \makebox[2in]{\hrulefill}
\end{tabular}\\
\end{flushright}
\rule[1ex]{\textwidth}{.1pt}


\begin{questions}
\question
Найдите и упростите P:
\begin{equation*}
\overline{P} = B \cap \overline{C} \cup A \cap B \cup \overline{A} \cap C \cup \overline{A} \cap B
\end{equation*}
Затем найдите элементы множества P, выраженного через множества:
\begin{equation*}
A = \{0, 3, 4, 9\}; 
B = \{1, 3, 4, 7\};
C = \{0, 1, 2, 4, 7, 8, 9\};
I = \{0, 1, 2, 3, 4, 5, 6, 7, 8, 9\}.
\end{equation*}\question
Упростите следующее выражение с учетом того, что $A\subset B \subset C \subset D \subset U; A \neq \O$
\begin{equation*}
A \cap  \overline{C} \cup B \cap \overline{D} \cup  \overline{A} \cap C \cap  \overline{D}
\end{equation*}

Примечание: U — универсум\question
Дано отношение на множестве $\{1, 2, 3, 4, 5\}$ 
\begin{equation*}
aRb \iff (a+b) \bmod 2 =0
\end{equation*}
Напишите обоснованный ответ какими свойствами обладает или не обладает отношение и почему:   
\begin{enumerate} [a)]\setcounter{enumi}{0}
\item рефлексивность
\item антирефлексивность
\item симметричность
\item асимметричность
\item антисимметричность
\item транзитивность
\end{enumerate}

Обоснуйте свой ответ по каждому из приведенных ниже вопросов:
\begin{enumerate} [a)]\setcounter{enumi}{0}
    \item Является ли это отношение отношением эквивалентности?
    \item Является ли это отношение функциональным?
    \item Каким из отношений соответствия (одно-многозначным, много-многозначный и т.д.) оно является?
    \item К каким из отношений порядка (полного, частичного и т.д.) можно отнести данное отношение?
\end{enumerate}



\question
Установите, является ли каждое из перечисленных ниже отношений на А ($R \subseteq A \times A$) отношением эквивалентности (обоснование ответа обязательно). Для каждого отношения эквивалентности постройте классы 
эквивалентности и постройте граф отношения:
\begin{enumerate} [a)]\setcounter{enumi}{0}
\item На множестве $A = \{1; 2; 3\}$ задано отношение $R = \{(1; 1); (2; 2); (3; 3); (2; 1); (1; 2); (2; 3); (3; 2); (3; 1); (1; 3)\}$
\item На множестве $A = \{1; 2; 3; 4; 5\}$ задано отношение $R = \{(1; 2); (1; 3); (1; 5); (2; 3); (2; 4); (2; 5); (3; 4); (3; 5); (4; 5)\}$
\item А - множество целых чисел и отношение $R = \{(a,b)|a + b = 0\}$
\end{enumerate}\question Составьте полную таблицу истинности, определите, какие переменные являются фиктивными и проверьте, является ли формула тавтологией:
$((P \rightarrow Q) \land (R \rightarrow S) \land \neg (Q \lor S)) \rightarrow \neg (P \lor R)$

\end{questions}
\newpage
%%% begin test
\begin{flushright}
\begin{tabular}{p{2.8in} r l}
%\textbf{\class} & \textbf{ФИО:} & \makebox[2.5in]{\hrulefill}\\
\textbf{\class} & \textbf{ФИО:} &М3107
\\

\textbf{\examdate} &&\\
%\textbf{Time Limit: \timelimit} & Teaching Assistant & \makebox[2in]{\hrulefill}
\end{tabular}\\
\end{flushright}
\rule[1ex]{\textwidth}{.1pt}


\begin{questions}
\question
Найдите и упростите P:
\begin{equation*}
\overline{P} = A \cap C \cup \overline{A} \cap \overline{C} \cup \overline{B} \cap C \cup \overline{A} \cap \overline{B}
\end{equation*}
Затем найдите элементы множества P, выраженного через множества:
\begin{equation*}
A = \{0, 3, 4, 9\}; 
B = \{1, 3, 4, 7\};
C = \{0, 1, 2, 4, 7, 8, 9\};
I = \{0, 1, 2, 3, 4, 5, 6, 7, 8, 9\}.
\end{equation*}\question
Упростите следующее выражение с учетом того, что $A\subset B \subset C \subset D \subset U; A \neq \O$
\begin{equation*}
\overline{A} \cap \overline{C} \cap D \cup \overline{B} \cap \overline{C} \cap D \cup A \cap B
\end{equation*}

Примечание: U — универсум\question
Дано отношение на множестве $\{1, 2, 3, 4, 5\}$ 
\begin{equation*}
aRb \iff  \text{НОД}(a,b) =1
\end{equation*}
Напишите обоснованный ответ какими свойствами обладает или не обладает отношение и почему:   
\begin{enumerate} [a)]\setcounter{enumi}{0}
\item рефлексивность
\item антирефлексивность
\item симметричность
\item асимметричность
\item антисимметричность
\item транзитивность
\end{enumerate}

Обоснуйте свой ответ по каждому из приведенных ниже вопросов:
\begin{enumerate} [a)]\setcounter{enumi}{0}
    \item Является ли это отношение отношением эквивалентности?
    \item Является ли это отношение функциональным?
    \item Каким из отношений соответствия (одно-многозначным, много-многозначный и т.д.) оно является?
    \item К каким из отношений порядка (полного, частичного и т.д.) можно отнести данное отношение?
\end{enumerate}


\question
Установите, является ли каждое из перечисленных ниже отношений на А ($R \subseteq A \times A$) отношением эквивалентности (обоснование ответа обязательно). Для каждого отношения эквивалентности постройте классы эквивалентности и постройте граф отношения:
\begin{enumerate} [a)]\setcounter{enumi}{0}
\item $F(x)=x^{2}+1$, где $x \in A = [-2, 4]$ и отношение $R = \{(a,b)|F(a) = F(b)\}$
\item А - множество целых чисел и отношение $R = \{(a,b)|a + b = 5\}$
\item На множестве $A = \{1; 2; 3\}$ задано отношение $R = \{(1; 1); (2; 2); (3; 3); (3; 2); (1; 2); (2; 1)\}$

\end{enumerate}\question Составьте полную таблицу истинности, определите, какие переменные являются фиктивными и проверьте, является ли формула тавтологией:
$ P \rightarrow (Q \rightarrow ((P \lor Q) \rightarrow (P \land Q)))$

\end{questions}
\newpage
%%% begin test
\begin{flushright}
\begin{tabular}{p{2.8in} r l}
%\textbf{\class} & \textbf{ФИО:} & \makebox[2.5in]{\hrulefill}\\
\textbf{\class} & \textbf{ФИО:} &Абаева Василиса Борисовна
\\

\textbf{\examdate} &&\\
%\textbf{Time Limit: \timelimit} & Teaching Assistant & \makebox[2in]{\hrulefill}
\end{tabular}\\
\end{flushright}
\rule[1ex]{\textwidth}{.1pt}


\begin{questions}
\question
Найдите и упростите P:
\begin{equation*}
\overline{P} = \overline{A} \cap B \cup \overline{A} \cap C \cup A \cap \overline{B} \cup \overline{B} \cap C
\end{equation*}
Затем найдите элементы множества P, выраженного через множества:
\begin{equation*}
A = \{0, 3, 4, 9\}; 
B = \{1, 3, 4, 7\};
C = \{0, 1, 2, 4, 7, 8, 9\};
I = \{0, 1, 2, 3, 4, 5, 6, 7, 8, 9\}.
\end{equation*}\question
Упростите следующее выражение с учетом того, что $A\subset B \subset C \subset D \subset U; A \neq \O$
\begin{equation*}
\overline{A} \cap \overline{C} \cap D \cup \overline{B} \cap \overline{C} \cap D \cup A \cap B
\end{equation*}

Примечание: U — универсум\question
Дано отношение на множестве $\{1, 2, 3, 4, 5\}$ 
\begin{equation*}
aRb \iff |a-b| = 1
\end{equation*}
Напишите обоснованный ответ какими свойствами обладает или не обладает отношение и почему:   
\begin{enumerate} [a)]\setcounter{enumi}{0}
\item рефлексивность
\item антирефлексивность
\item симметричность
\item асимметричность
\item антисимметричность
\item транзитивность
\end{enumerate}

Обоснуйте свой ответ по каждому из приведенных ниже вопросов:
\begin{enumerate} [a)]\setcounter{enumi}{0}
    \item Является ли это отношение отношением эквивалентности?
    \item Является ли это отношение функциональным?
    \item Каким из отношений соответствия (одно-многозначным, много-многозначный и т.д.) оно является?
    \item К каким из отношений порядка (полного, частичного и т.д.) можно отнести данное отношение?
\end{enumerate}

\question
Установите, является ли каждое из перечисленных ниже отношений на А ($R \subseteq A \times A$) отношением эквивалентности (обоснование ответа обязательно). Для каждого отношения эквивалентности 
постройте классы эквивалентности и постройте граф отношения:
\begin{enumerate}[a)]\setcounter{enumi}{0}
\item А - множество целых чисел и отношение $R = \{(a,b)|a + b = 0\}$
\item $A = \{-10, -9, …, 9, 10\}$ и отношение $R = \{(a,b)|a^{3} = b^{3}\}$
\item На множестве $A = \{1; 2; 3\}$ задано отношение $R = \{(1; 1); (2; 2); (3; 3); (2; 1); (1; 2); (2; 3); (3; 2); (3; 1); (1; 3)\}$

\end{enumerate}\question Составьте полную таблицу истинности, определите, какие переменные являются фиктивными и проверьте, является ли формула тавтологией:
$(( P \land \neg Q) \rightarrow (R \land \neg R)) \rightarrow (P \rightarrow Q)$

\end{questions}
\newpage
%%% begin test
\begin{flushright}
\begin{tabular}{p{2.8in} r l}
%\textbf{\class} & \textbf{ФИО:} & \makebox[2.5in]{\hrulefill}\\
\textbf{\class} & \textbf{ФИО:} &Ананьин Николай Николаевич
\\

\textbf{\examdate} &&\\
%\textbf{Time Limit: \timelimit} & Teaching Assistant & \makebox[2in]{\hrulefill}
\end{tabular}\\
\end{flushright}
\rule[1ex]{\textwidth}{.1pt}


\begin{questions}
\question
Найдите и упростите P:
\begin{equation*}
\overline{P} = A \cap \overline{C} \cup A \cap \overline{B} \cup B \cap \overline{C} \cup A \cap C
\end{equation*}
Затем найдите элементы множества P, выраженного через множества:
\begin{equation*}
A = \{0, 3, 4, 9\}; 
B = \{1, 3, 4, 7\};
C = \{0, 1, 2, 4, 7, 8, 9\};
I = \{0, 1, 2, 3, 4, 5, 6, 7, 8, 9\}.
\end{equation*}\question
Упростите следующее выражение с учетом того, что $A\subset B \subset C \subset D \subset U; A \neq \O$
\begin{equation*}
\overline{A} \cap \overline{B} \cup B \cap \overline{C} \cup \overline{C} \cap D
\end{equation*}

Примечание: U — универсум\question
Для следующего отношения на множестве $\{1, 2, 3, 4, 5\}$ 
\begin{equation*}
aRb \iff 0 < a-b<2
\end{equation*}
Напишите обоснованный ответ какими свойствами обладает или не обладает отношение и почему:   
\begin{enumerate} [a)]\setcounter{enumi}{0}
\item рефлексивность
\item антирефлексивность
\item симметричность
\item асимметричность
\item антисимметричность
\item транзитивность
\end{enumerate}

Обоснуйте свой ответ по каждому из приведенных ниже вопросов:
\begin{enumerate} [a)]\setcounter{enumi}{0}
    \item Является ли это отношение отношением эквивалентности?
    \item Является ли это отношение функциональным?
    \item Каким из отношений соответствия (одно-многозначным, много-многозначный и т.д.) оно является?
    \item К каким из отношений порядка (полного, частичного и т.д.) можно отнести данное отношение?
\end{enumerate}
\question
Установите, является ли каждое из перечисленных ниже отношений на А ($R \subseteq A \times A$) отношением эквивалентности (обоснование ответа обязательно). Для каждого отношения эквивалентности постройте классы эквивалентности и постройте граф отношения:
\begin{enumerate} [a)]\setcounter{enumi}{0}
\item $F(x)=x^{2}+1$, где $x \in A = [-2, 4]$ и отношение $R = \{(a,b)|F(a) = F(b)\}$
\item А - множество целых чисел и отношение $R = \{(a,b)|a + b = 5\}$
\item На множестве $A = \{1; 2; 3\}$ задано отношение $R = \{(1; 1); (2; 2); (3; 3); (3; 2); (1; 2); (2; 1)\}$

\end{enumerate}\question Составьте полную таблицу истинности, определите, какие переменные являются фиктивными и проверьте, является ли формула тавтологией:

$(P \rightarrow (Q \land R)) \leftrightarrow ((P \rightarrow Q) \land (P \rightarrow R))$

\end{questions}
\newpage
%%% begin test
\begin{flushright}
\begin{tabular}{p{2.8in} r l}
%\textbf{\class} & \textbf{ФИО:} & \makebox[2.5in]{\hrulefill}\\
\textbf{\class} & \textbf{ФИО:} &Бабичев Мирон Олегович
\\

\textbf{\examdate} &&\\
%\textbf{Time Limit: \timelimit} & Teaching Assistant & \makebox[2in]{\hrulefill}
\end{tabular}\\
\end{flushright}
\rule[1ex]{\textwidth}{.1pt}


\begin{questions}
\question
Найдите и упростите P:
\begin{equation*}
\overline{P} = A \cap \overline{C} \cup A \cap \overline{B} \cup B \cap \overline{C} \cup A \cap C
\end{equation*}
Затем найдите элементы множества P, выраженного через множества:
\begin{equation*}
A = \{0, 3, 4, 9\}; 
B = \{1, 3, 4, 7\};
C = \{0, 1, 2, 4, 7, 8, 9\};
I = \{0, 1, 2, 3, 4, 5, 6, 7, 8, 9\}.
\end{equation*}\question
Упростите следующее выражение с учетом того, что $A\subset B \subset C \subset D \subset U; A \neq \O$
\begin{equation*}
A \cap B \cup \overline{A} \cap \overline{C} \cup A \cap C \cup \overline{B} \cap \overline{C}
\end{equation*}

Примечание: U — универсум\question
Дано отношение на множестве $\{1, 2, 3, 4, 5\}$ 
\begin{equation*}
aRb \iff b > a
\end{equation*}
Напишите обоснованный ответ какими свойствами обладает или не обладает отношение и почему:   
\begin{enumerate} [a)]\setcounter{enumi}{0}
\item рефлексивность
\item антирефлексивность
\item симметричность
\item асимметричность
\item антисимметричность
\item транзитивность
\end{enumerate}

Обоснуйте свой ответ по каждому из приведенных ниже вопросов:
\begin{enumerate} [a)]\setcounter{enumi}{0}
    \item Является ли это отношение отношением эквивалентности?
    \item Является ли это отношение функциональным?
    \item Каким из отношений соответствия (одно-многозначным, много-многозначный и т.д.) оно является?
    \item К каким из отношений порядка (полного, частичного и т.д.) можно отнести данное отношение?
\end{enumerate}

\question
Установите, является ли каждое из перечисленных ниже отношений на А ($R \subseteq A \times A$) отношением эквивалентности (обоснование ответа обязательно). Для каждого отношения эквивалентности постройте классы эквивалентности и постройте граф отношения:
\begin{enumerate} [a)]\setcounter{enumi}{0}
\item $F(x)=x^{2}+1$, где $x \in A = [-2, 4]$ и отношение $R = \{(a,b)|F(a) = F(b)\}$
\item А - множество целых чисел и отношение $R = \{(a,b)|a + b = 5\}$
\item На множестве $A = \{1; 2; 3\}$ задано отношение $R = \{(1; 1); (2; 2); (3; 3); (3; 2); (1; 2); (2; 1)\}$

\end{enumerate}\question Составьте полную таблицу истинности, определите, какие переменные являются фиктивными и проверьте, является ли формула тавтологией:
$(( P \land \neg Q) \rightarrow (R \land \neg R)) \rightarrow (P \rightarrow Q)$

\end{questions}
\newpage
%%% begin test
\begin{flushright}
\begin{tabular}{p{2.8in} r l}
%\textbf{\class} & \textbf{ФИО:} & \makebox[2.5in]{\hrulefill}\\
\textbf{\class} & \textbf{ФИО:} &Данилин Александр Константинович
\\

\textbf{\examdate} &&\\
%\textbf{Time Limit: \timelimit} & Teaching Assistant & \makebox[2in]{\hrulefill}
\end{tabular}\\
\end{flushright}
\rule[1ex]{\textwidth}{.1pt}


\begin{questions}
\question
Найдите и упростите P:
\begin{equation*}
\overline{P} = A \cap B \cup \overline{A} \cap \overline{B} \cup A \cap C \cup \overline{B} \cap C
\end{equation*}
Затем найдите элементы множества P, выраженного через множества:
\begin{equation*}
A = \{0, 3, 4, 9\}; 
B = \{1, 3, 4, 7\};
C = \{0, 1, 2, 4, 7, 8, 9\};
I = \{0, 1, 2, 3, 4, 5, 6, 7, 8, 9\}.
\end{equation*}\question
Упростите следующее выражение с учетом того, что $A\subset B \subset C \subset D \subset U; A \neq \O$
\begin{equation*}
\overline{B} \cap \overline{C} \cap D \cup \overline{A} \cap \overline{C} \cap D \cup \overline{A} \cap B
\end{equation*}

Примечание: U — универсум\question
Для следующего отношения на множестве $\{1, 2, 3, 4, 5\}$ 
\begin{equation*}
aRb \iff 0 < a-b<2
\end{equation*}
Напишите обоснованный ответ какими свойствами обладает или не обладает отношение и почему:   
\begin{enumerate} [a)]\setcounter{enumi}{0}
\item рефлексивность
\item антирефлексивность
\item симметричность
\item асимметричность
\item антисимметричность
\item транзитивность
\end{enumerate}

Обоснуйте свой ответ по каждому из приведенных ниже вопросов:
\begin{enumerate} [a)]\setcounter{enumi}{0}
    \item Является ли это отношение отношением эквивалентности?
    \item Является ли это отношение функциональным?
    \item Каким из отношений соответствия (одно-многозначным, много-многозначный и т.д.) оно является?
    \item К каким из отношений порядка (полного, частичного и т.д.) можно отнести данное отношение?
\end{enumerate}
\question
Установите, является ли каждое из перечисленных ниже отношений на А ($R \subseteq A \times A$) отношением эквивалентности (обоснование ответа обязательно). Для каждого отношения эквивалентности постройте классы 
эквивалентности и постройте граф отношения:
\begin{enumerate} [a)]\setcounter{enumi}{0}
\item На множестве $A = \{1; 2; 3\}$ задано отношение $R = \{(1; 1); (2; 2); (3; 3); (2; 1); (1; 2); (2; 3); (3; 2); (3; 1); (1; 3)\}$
\item На множестве $A = \{1; 2; 3; 4; 5\}$ задано отношение $R = \{(1; 2); (1; 3); (1; 5); (2; 3); (2; 4); (2; 5); (3; 4); (3; 5); (4; 5)\}$
\item А - множество целых чисел и отношение $R = \{(a,b)|a + b = 0\}$
\end{enumerate}\question Составьте полную таблицу истинности, определите, какие переменные являются фиктивными и проверьте, является ли формула тавтологией:
$((P \rightarrow Q) \land (R \rightarrow S) \land \neg (Q \lor S)) \rightarrow \neg (P \lor R)$

\end{questions}
\newpage
%%% begin test
\begin{flushright}
\begin{tabular}{p{2.8in} r l}
%\textbf{\class} & \textbf{ФИО:} & \makebox[2.5in]{\hrulefill}\\
\textbf{\class} & \textbf{ФИО:} &Евтюхов Дмитрий Вадимович
\\

\textbf{\examdate} &&\\
%\textbf{Time Limit: \timelimit} & Teaching Assistant & \makebox[2in]{\hrulefill}
\end{tabular}\\
\end{flushright}
\rule[1ex]{\textwidth}{.1pt}


\begin{questions}
\question
Найдите и упростите P:
\begin{equation*}
\overline{P} = B \cap \overline{C} \cup A \cap B \cup \overline{A} \cap C \cup \overline{A} \cap B
\end{equation*}
Затем найдите элементы множества P, выраженного через множества:
\begin{equation*}
A = \{0, 3, 4, 9\}; 
B = \{1, 3, 4, 7\};
C = \{0, 1, 2, 4, 7, 8, 9\};
I = \{0, 1, 2, 3, 4, 5, 6, 7, 8, 9\}.
\end{equation*}\question
Упростите следующее выражение с учетом того, что $A\subset B \subset C \subset D \subset U; A \neq \O$
\begin{equation*}
A \cap  \overline{C} \cup B \cap \overline{D} \cup  \overline{A} \cap C \cap  \overline{D}
\end{equation*}

Примечание: U — универсум\question
Дано отношение на множестве $\{1, 2, 3, 4, 5\}$ 
\begin{equation*}
aRb \iff b > a
\end{equation*}
Напишите обоснованный ответ какими свойствами обладает или не обладает отношение и почему:   
\begin{enumerate} [a)]\setcounter{enumi}{0}
\item рефлексивность
\item антирефлексивность
\item симметричность
\item асимметричность
\item антисимметричность
\item транзитивность
\end{enumerate}

Обоснуйте свой ответ по каждому из приведенных ниже вопросов:
\begin{enumerate} [a)]\setcounter{enumi}{0}
    \item Является ли это отношение отношением эквивалентности?
    \item Является ли это отношение функциональным?
    \item Каким из отношений соответствия (одно-многозначным, много-многозначный и т.д.) оно является?
    \item К каким из отношений порядка (полного, частичного и т.д.) можно отнести данное отношение?
\end{enumerate}

\question
Установите, является ли каждое из перечисленных ниже отношений на А ($R \subseteq A \times A$) отношением эквивалентности (обоснование ответа обязательно). Для каждого отношения эквивалентности постройте классы 
эквивалентности и постройте граф отношения:
\begin{enumerate} [a)]\setcounter{enumi}{0}
\item $A = \{-10, -9, … , 9, 10\}$ и отношение $R = \{(a,b)|a^{2} = b^{2}\}$
\item $A = \{a, b, c, d, p, t\}$ задано отношение $R = \{(a, a), (b, b), (b, c), (b, d), (c, b), (c, c), (c, d), (d, b), (d, c), (d, d), (p,p), (t,t)\}$
\item Пусть A – множество имен. $A = \{ $Алексей, Иван, Петр, Александр, Павел, Андрей$ \}$. Тогда отношение $R$ верно на парах имен, начинающихся с одной и той же буквы, и только на них.
\end{enumerate}\question Составьте полную таблицу истинности, определите, какие переменные являются фиктивными и проверьте, является ли формула тавтологией:
$(( P \rightarrow Q) \land (Q \rightarrow P)) \rightarrow (P \rightarrow R)$

\end{questions}
\newpage
%%% begin test
\begin{flushright}
\begin{tabular}{p{2.8in} r l}
%\textbf{\class} & \textbf{ФИО:} & \makebox[2.5in]{\hrulefill}\\
\textbf{\class} & \textbf{ФИО:} &Жарков Григорий Алексеевич
\\

\textbf{\examdate} &&\\
%\textbf{Time Limit: \timelimit} & Teaching Assistant & \makebox[2in]{\hrulefill}
\end{tabular}\\
\end{flushright}
\rule[1ex]{\textwidth}{.1pt}


\begin{questions}
\question
Найдите и упростите P:
\begin{equation*}
\overline{P} = A \cap C \cup \overline{A} \cap \overline{C} \cup \overline{B} \cap C \cup \overline{A} \cap \overline{B}
\end{equation*}
Затем найдите элементы множества P, выраженного через множества:
\begin{equation*}
A = \{0, 3, 4, 9\}; 
B = \{1, 3, 4, 7\};
C = \{0, 1, 2, 4, 7, 8, 9\};
I = \{0, 1, 2, 3, 4, 5, 6, 7, 8, 9\}.
\end{equation*}\question
Упростите следующее выражение с учетом того, что $A\subset B \subset C \subset D \subset U; A \neq \O$
\begin{equation*}
A \cap B \cup \overline{A} \cap \overline{C} \cup A \cap C \cup \overline{B} \cap \overline{C}
\end{equation*}

Примечание: U — универсум\question
Дано отношение на множестве $\{1, 2, 3, 4, 5\}$ 
\begin{equation*}
aRb \iff (a+b) \bmod 2 =0
\end{equation*}
Напишите обоснованный ответ какими свойствами обладает или не обладает отношение и почему:   
\begin{enumerate} [a)]\setcounter{enumi}{0}
\item рефлексивность
\item антирефлексивность
\item симметричность
\item асимметричность
\item антисимметричность
\item транзитивность
\end{enumerate}

Обоснуйте свой ответ по каждому из приведенных ниже вопросов:
\begin{enumerate} [a)]\setcounter{enumi}{0}
    \item Является ли это отношение отношением эквивалентности?
    \item Является ли это отношение функциональным?
    \item Каким из отношений соответствия (одно-многозначным, много-многозначный и т.д.) оно является?
    \item К каким из отношений порядка (полного, частичного и т.д.) можно отнести данное отношение?
\end{enumerate}



\question
Установите, является ли каждое из перечисленных ниже отношений на А ($R \subseteq A \times A$) отношением эквивалентности (обоснование ответа обязательно). Для каждого отношения эквивалентности постройте классы 
эквивалентности и постройте граф отношения:
\begin{enumerate} [a)]\setcounter{enumi}{0}
\item $A = \{-10, -9, … , 9, 10\}$ и отношение $R = \{(a,b)|a^{2} = b^{2}\}$
\item $A = \{a, b, c, d, p, t\}$ задано отношение $R = \{(a, a), (b, b), (b, c), (b, d), (c, b), (c, c), (c, d), (d, b), (d, c), (d, d), (p,p), (t,t)\}$
\item Пусть A – множество имен. $A = \{ $Алексей, Иван, Петр, Александр, Павел, Андрей$ \}$. Тогда отношение $R$ верно на парах имен, начинающихся с одной и той же буквы, и только на них.
\end{enumerate}\question Составьте полную таблицу истинности, определите, какие переменные являются фиктивными и проверьте, является ли формула тавтологией:

$(P \rightarrow (Q \land R)) \leftrightarrow ((P \rightarrow Q) \land (P \rightarrow R))$

\end{questions}
\newpage
%%% begin test
\begin{flushright}
\begin{tabular}{p{2.8in} r l}
%\textbf{\class} & \textbf{ФИО:} & \makebox[2.5in]{\hrulefill}\\
\textbf{\class} & \textbf{ФИО:} &Зубов Данила Андреевич
\\

\textbf{\examdate} &&\\
%\textbf{Time Limit: \timelimit} & Teaching Assistant & \makebox[2in]{\hrulefill}
\end{tabular}\\
\end{flushright}
\rule[1ex]{\textwidth}{.1pt}


\begin{questions}
\question
Найдите и упростите P:
\begin{equation*}
\overline{P} = A \cap \overline{C} \cup A \cap \overline{B} \cup B \cap \overline{C} \cup A \cap C
\end{equation*}
Затем найдите элементы множества P, выраженного через множества:
\begin{equation*}
A = \{0, 3, 4, 9\}; 
B = \{1, 3, 4, 7\};
C = \{0, 1, 2, 4, 7, 8, 9\};
I = \{0, 1, 2, 3, 4, 5, 6, 7, 8, 9\}.
\end{equation*}\question
Упростите следующее выражение с учетом того, что $A\subset B \subset C \subset D \subset U; A \neq \O$
\begin{equation*}
\overline{A} \cap \overline{C} \cap D \cup \overline{B} \cap \overline{C} \cap D \cup A \cap B
\end{equation*}

Примечание: U — универсум\question
Дано отношение на множестве $\{1, 2, 3, 4, 5\}$ 
\begin{equation*}
aRb \iff b > a
\end{equation*}
Напишите обоснованный ответ какими свойствами обладает или не обладает отношение и почему:   
\begin{enumerate} [a)]\setcounter{enumi}{0}
\item рефлексивность
\item антирефлексивность
\item симметричность
\item асимметричность
\item антисимметричность
\item транзитивность
\end{enumerate}

Обоснуйте свой ответ по каждому из приведенных ниже вопросов:
\begin{enumerate} [a)]\setcounter{enumi}{0}
    \item Является ли это отношение отношением эквивалентности?
    \item Является ли это отношение функциональным?
    \item Каким из отношений соответствия (одно-многозначным, много-многозначный и т.д.) оно является?
    \item К каким из отношений порядка (полного, частичного и т.д.) можно отнести данное отношение?
\end{enumerate}

\question
Установите, является ли каждое из перечисленных ниже отношений на А ($R \subseteq A \times A$) отношением эквивалентности (обоснование ответа обязательно). Для каждого отношения эквивалентности 
постройте классы эквивалентности и постройте граф отношения:
\begin{enumerate}[a)]\setcounter{enumi}{0}
\item А - множество целых чисел и отношение $R = \{(a,b)|a + b = 0\}$
\item $A = \{-10, -9, …, 9, 10\}$ и отношение $R = \{(a,b)|a^{3} = b^{3}\}$
\item На множестве $A = \{1; 2; 3\}$ задано отношение $R = \{(1; 1); (2; 2); (3; 3); (2; 1); (1; 2); (2; 3); (3; 2); (3; 1); (1; 3)\}$

\end{enumerate}\question Составьте полную таблицу истинности, определите, какие переменные являются фиктивными и проверьте, является ли формула тавтологией:
$((P \rightarrow Q) \lor R) \leftrightarrow (P \rightarrow (Q \lor R))$

\end{questions}
\newpage
%%% begin test
\begin{flushright}
\begin{tabular}{p{2.8in} r l}
%\textbf{\class} & \textbf{ФИО:} & \makebox[2.5in]{\hrulefill}\\
\textbf{\class} & \textbf{ФИО:} &Казанцев Даниил Владимирович
\\

\textbf{\examdate} &&\\
%\textbf{Time Limit: \timelimit} & Teaching Assistant & \makebox[2in]{\hrulefill}
\end{tabular}\\
\end{flushright}
\rule[1ex]{\textwidth}{.1pt}


\begin{questions}
\question
Найдите и упростите P:
\begin{equation*}
\overline{P} = \overline{A} \cap B \cup \overline{A} \cap C \cup A \cap \overline{B} \cup \overline{B} \cap C
\end{equation*}
Затем найдите элементы множества P, выраженного через множества:
\begin{equation*}
A = \{0, 3, 4, 9\}; 
B = \{1, 3, 4, 7\};
C = \{0, 1, 2, 4, 7, 8, 9\};
I = \{0, 1, 2, 3, 4, 5, 6, 7, 8, 9\}.
\end{equation*}\question
Упростите следующее выражение с учетом того, что $A\subset B \subset C \subset D \subset U; A \neq \O$
\begin{equation*}
\overline{B} \cap \overline{C} \cap D \cup \overline{A} \cap \overline{C} \cap D \cup \overline{A} \cap B
\end{equation*}

Примечание: U — универсум\question
Для следующего отношения на множестве $\{1, 2, 3, 4, 5\}$ 
\begin{equation*}
aRb \iff 0 < a-b<2
\end{equation*}
Напишите обоснованный ответ какими свойствами обладает или не обладает отношение и почему:   
\begin{enumerate} [a)]\setcounter{enumi}{0}
\item рефлексивность
\item антирефлексивность
\item симметричность
\item асимметричность
\item антисимметричность
\item транзитивность
\end{enumerate}

Обоснуйте свой ответ по каждому из приведенных ниже вопросов:
\begin{enumerate} [a)]\setcounter{enumi}{0}
    \item Является ли это отношение отношением эквивалентности?
    \item Является ли это отношение функциональным?
    \item Каким из отношений соответствия (одно-многозначным, много-многозначный и т.д.) оно является?
    \item К каким из отношений порядка (полного, частичного и т.д.) можно отнести данное отношение?
\end{enumerate}
\question
Установите, является ли каждое из перечисленных ниже отношений на А ($R \subseteq A \times A$) отношением эквивалентности (обоснование ответа обязательно). Для каждого отношения эквивалентности постройте классы 
эквивалентности и постройте граф отношения:
\begin{enumerate} [a)]\setcounter{enumi}{0}
\item Пусть A – множество имен. $A = \{ $Алексей, Иван, Петр, Александр, Павел, Андрей$ \}$. Тогда отношение $R$ верно на парах имен, начинающихся с одной и той же буквы, и только на них.
\item $A = \{-10, -9, … , 9, 10\}$ и отношение $ R = \{(a,b)|a^{2} = b^{2}\}$
\item На множестве $A = \{1; 2; 3\}$ задано отношение $R = \{(1; 1); (2; 2); (3; 3); (3; 2); (1; 2); (2; 1)\}$
\end{enumerate}\question Составьте полную таблицу истинности, определите, какие переменные являются фиктивными и проверьте, является ли формула тавтологией:
$(P \rightarrow (Q \rightarrow R)) \rightarrow ((P \rightarrow Q) \rightarrow (P \rightarrow R))$

\end{questions}
\newpage
%%% begin test
\begin{flushright}
\begin{tabular}{p{2.8in} r l}
%\textbf{\class} & \textbf{ФИО:} & \makebox[2.5in]{\hrulefill}\\
\textbf{\class} & \textbf{ФИО:} &Карепин Денис Дмитриевич
\\

\textbf{\examdate} &&\\
%\textbf{Time Limit: \timelimit} & Teaching Assistant & \makebox[2in]{\hrulefill}
\end{tabular}\\
\end{flushright}
\rule[1ex]{\textwidth}{.1pt}


\begin{questions}
\question
Найдите и упростите P:
\begin{equation*}
\overline{P} = B \cap \overline{C} \cup A \cap B \cup \overline{A} \cap C \cup \overline{A} \cap B
\end{equation*}
Затем найдите элементы множества P, выраженного через множества:
\begin{equation*}
A = \{0, 3, 4, 9\}; 
B = \{1, 3, 4, 7\};
C = \{0, 1, 2, 4, 7, 8, 9\};
I = \{0, 1, 2, 3, 4, 5, 6, 7, 8, 9\}.
\end{equation*}\question
Упростите следующее выражение с учетом того, что $A\subset B \subset C \subset D \subset U; A \neq \O$
\begin{equation*}
\overline{A} \cap \overline{C} \cap D \cup \overline{B} \cap \overline{C} \cap D \cup A \cap B
\end{equation*}

Примечание: U — универсум\question
Для следующего отношения на множестве $\{1, 2, 3, 4, 5\}$ 
\begin{equation*}
aRb \iff 0 < a-b<2
\end{equation*}
Напишите обоснованный ответ какими свойствами обладает или не обладает отношение и почему:   
\begin{enumerate} [a)]\setcounter{enumi}{0}
\item рефлексивность
\item антирефлексивность
\item симметричность
\item асимметричность
\item антисимметричность
\item транзитивность
\end{enumerate}

Обоснуйте свой ответ по каждому из приведенных ниже вопросов:
\begin{enumerate} [a)]\setcounter{enumi}{0}
    \item Является ли это отношение отношением эквивалентности?
    \item Является ли это отношение функциональным?
    \item Каким из отношений соответствия (одно-многозначным, много-многозначный и т.д.) оно является?
    \item К каким из отношений порядка (полного, частичного и т.д.) можно отнести данное отношение?
\end{enumerate}
\question
Установите, является ли каждое из перечисленных ниже отношений на А ($R \subseteq A \times A$) отношением эквивалентности (обоснование ответа обязательно). Для каждого отношения эквивалентности постройте классы 
эквивалентности и постройте граф отношения:
\begin{enumerate} [a)]\setcounter{enumi}{0}
\item $A = \{-10, -9, … , 9, 10\}$ и отношение $R = \{(a,b)|a^{2} = b^{2}\}$
\item $A = \{a, b, c, d, p, t\}$ задано отношение $R = \{(a, a), (b, b), (b, c), (b, d), (c, b), (c, c), (c, d), (d, b), (d, c), (d, d), (p,p), (t,t)\}$
\item Пусть A – множество имен. $A = \{ $Алексей, Иван, Петр, Александр, Павел, Андрей$ \}$. Тогда отношение $R$ верно на парах имен, начинающихся с одной и той же буквы, и только на них.
\end{enumerate}\question Составьте полную таблицу истинности, определите, какие переменные являются фиктивными и проверьте, является ли формула тавтологией:
$((P \rightarrow Q) \lor R) \leftrightarrow (P \rightarrow (Q \lor R))$

\end{questions}
\newpage
%%% begin test
\begin{flushright}
\begin{tabular}{p{2.8in} r l}
%\textbf{\class} & \textbf{ФИО:} & \makebox[2.5in]{\hrulefill}\\
\textbf{\class} & \textbf{ФИО:} &Киреев Кирилл Сергеевич
\\

\textbf{\examdate} &&\\
%\textbf{Time Limit: \timelimit} & Teaching Assistant & \makebox[2in]{\hrulefill}
\end{tabular}\\
\end{flushright}
\rule[1ex]{\textwidth}{.1pt}


\begin{questions}
\question
Найдите и упростите P:
\begin{equation*}
\overline{P} = A \cap C \cup \overline{A} \cap \overline{C} \cup \overline{B} \cap C \cup \overline{A} \cap \overline{B}
\end{equation*}
Затем найдите элементы множества P, выраженного через множества:
\begin{equation*}
A = \{0, 3, 4, 9\}; 
B = \{1, 3, 4, 7\};
C = \{0, 1, 2, 4, 7, 8, 9\};
I = \{0, 1, 2, 3, 4, 5, 6, 7, 8, 9\}.
\end{equation*}\question
Упростите следующее выражение с учетом того, что $A\subset B \subset C \subset D \subset U; A \neq \O$
\begin{equation*}
\overline{B} \cap \overline{C} \cap D \cup \overline{A} \cap \overline{C} \cap D \cup \overline{A} \cap B
\end{equation*}

Примечание: U — универсум\question
Дано отношение на множестве $\{1, 2, 3, 4, 5\}$ 
\begin{equation*}
aRb \iff |a-b| = 1
\end{equation*}
Напишите обоснованный ответ какими свойствами обладает или не обладает отношение и почему:   
\begin{enumerate} [a)]\setcounter{enumi}{0}
\item рефлексивность
\item антирефлексивность
\item симметричность
\item асимметричность
\item антисимметричность
\item транзитивность
\end{enumerate}

Обоснуйте свой ответ по каждому из приведенных ниже вопросов:
\begin{enumerate} [a)]\setcounter{enumi}{0}
    \item Является ли это отношение отношением эквивалентности?
    \item Является ли это отношение функциональным?
    \item Каким из отношений соответствия (одно-многозначным, много-многозначный и т.д.) оно является?
    \item К каким из отношений порядка (полного, частичного и т.д.) можно отнести данное отношение?
\end{enumerate}

\question
Установите, является ли каждое из перечисленных ниже отношений на А ($R \subseteq A \times A$) отношением эквивалентности (обоснование ответа обязательно). Для каждого отношения эквивалентности постройте классы 
эквивалентности и постройте граф отношения:
\begin{enumerate} [a)]\setcounter{enumi}{0}
\item А - множество целых чисел и отношение $R = \{(a,b)|a + b = 5\}$
\item Пусть A – множество имен. $A = \{ $Алексей, Иван, Петр, Александр, Павел, Андрей$ \}$. Тогда отношение $R $ верно на парах имен, начинающихся с одной и той же буквы, и только на них.
\item На множестве $A = \{1; 2; 3; 4; 5\}$ задано отношение $R = \{(1; 2); (1; 3); (1; 5); (2; 3); (2; 4); (2; 5); (3; 4); (3; 5); (4; 5)\}$
\end{enumerate}\question Составьте полную таблицу истинности, определите, какие переменные являются фиктивными и проверьте, является ли формула тавтологией:
$((P \rightarrow Q) \land (R \rightarrow S) \land \neg (Q \lor S)) \rightarrow \neg (P \lor R)$

\end{questions}
\newpage
%%% begin test
\begin{flushright}
\begin{tabular}{p{2.8in} r l}
%\textbf{\class} & \textbf{ФИО:} & \makebox[2.5in]{\hrulefill}\\
\textbf{\class} & \textbf{ФИО:} &Кирьяков Юрий Вадимович
\\

\textbf{\examdate} &&\\
%\textbf{Time Limit: \timelimit} & Teaching Assistant & \makebox[2in]{\hrulefill}
\end{tabular}\\
\end{flushright}
\rule[1ex]{\textwidth}{.1pt}


\begin{questions}
\question
Найдите и упростите P:
\begin{equation*}
\overline{P} = A \cap \overline{B} \cup \overline{B} \cap C \cup \overline{A} \cap \overline{B} \cup \overline{A} \cap C
\end{equation*}
Затем найдите элементы множества P, выраженного через множества:
\begin{equation*}
A = \{0, 3, 4, 9\}; 
B = \{1, 3, 4, 7\};
C = \{0, 1, 2, 4, 7, 8, 9\};
I = \{0, 1, 2, 3, 4, 5, 6, 7, 8, 9\}.
\end{equation*}\question
Упростите следующее выражение с учетом того, что $A\subset B \subset C \subset D \subset U; A \neq \O$
\begin{equation*}
\overline{B} \cap \overline{C} \cap D \cup \overline{A} \cap \overline{C} \cap D \cup \overline{A} \cap B
\end{equation*}

Примечание: U — универсум\question
Дано отношение на множестве $\{1, 2, 3, 4, 5\}$ 
\begin{equation*}
aRb \iff  \text{НОД}(a,b) =1
\end{equation*}
Напишите обоснованный ответ какими свойствами обладает или не обладает отношение и почему:   
\begin{enumerate} [a)]\setcounter{enumi}{0}
\item рефлексивность
\item антирефлексивность
\item симметричность
\item асимметричность
\item антисимметричность
\item транзитивность
\end{enumerate}

Обоснуйте свой ответ по каждому из приведенных ниже вопросов:
\begin{enumerate} [a)]\setcounter{enumi}{0}
    \item Является ли это отношение отношением эквивалентности?
    \item Является ли это отношение функциональным?
    \item Каким из отношений соответствия (одно-многозначным, много-многозначный и т.д.) оно является?
    \item К каким из отношений порядка (полного, частичного и т.д.) можно отнести данное отношение?
\end{enumerate}


\question
Установите, является ли каждое из перечисленных ниже отношений на А ($R \subseteq A \times A$) отношением эквивалентности (обоснование ответа обязательно). Для каждого отношения эквивалентности постройте классы эквивалентности и постройте граф отношения:
\begin{enumerate} [a)]\setcounter{enumi}{0}
\item $F(x)=x^{2}+1$, где $x \in A = [-2, 4]$ и отношение $R = \{(a,b)|F(a) = F(b)\}$
\item А - множество целых чисел и отношение $R = \{(a,b)|a + b = 5\}$
\item На множестве $A = \{1; 2; 3\}$ задано отношение $R = \{(1; 1); (2; 2); (3; 3); (3; 2); (1; 2); (2; 1)\}$

\end{enumerate}\question Составьте полную таблицу истинности, определите, какие переменные являются фиктивными и проверьте, является ли формула тавтологией:
$((P \rightarrow Q) \land (R \rightarrow S) \land \neg (Q \lor S)) \rightarrow \neg (P \lor R)$

\end{questions}
\newpage
%%% begin test
\begin{flushright}
\begin{tabular}{p{2.8in} r l}
%\textbf{\class} & \textbf{ФИО:} & \makebox[2.5in]{\hrulefill}\\
\textbf{\class} & \textbf{ФИО:} &Кононенко Александр Александрович
\\

\textbf{\examdate} &&\\
%\textbf{Time Limit: \timelimit} & Teaching Assistant & \makebox[2in]{\hrulefill}
\end{tabular}\\
\end{flushright}
\rule[1ex]{\textwidth}{.1pt}


\begin{questions}
\question
Найдите и упростите P:
\begin{equation*}
\overline{P} = A \cap \overline{B} \cup \overline{B} \cap C \cup \overline{A} \cap \overline{B} \cup \overline{A} \cap C
\end{equation*}
Затем найдите элементы множества P, выраженного через множества:
\begin{equation*}
A = \{0, 3, 4, 9\}; 
B = \{1, 3, 4, 7\};
C = \{0, 1, 2, 4, 7, 8, 9\};
I = \{0, 1, 2, 3, 4, 5, 6, 7, 8, 9\}.
\end{equation*}\question
Упростите следующее выражение с учетом того, что $A\subset B \subset C \subset D \subset U; A \neq \O$
\begin{equation*}
A \cap C  \cap D \cup B \cap \overline{C} \cap D \cup B \cap C \cap D
\end{equation*}

Примечание: U — универсум\question
Дано отношение на множестве $\{1, 2, 3, 4, 5\}$ 
\begin{equation*}
aRb \iff a \leq b
\end{equation*}
Напишите обоснованный ответ какими свойствами обладает или не обладает отношение и почему:   
\begin{enumerate} [a)]\setcounter{enumi}{0}
\item рефлексивность
\item антирефлексивность
\item симметричность
\item асимметричность
\item антисимметричность
\item транзитивность
\end{enumerate}

Обоснуйте свой ответ по каждому из приведенных ниже вопросов:
\begin{enumerate} [a)]\setcounter{enumi}{0}
    \item Является ли это отношение отношением эквивалентности?
    \item Является ли это отношение функциональным?
    \item Каким из отношений соответствия (одно-многозначным, много-многозначный и т.д.) оно является?
    \item К каким из отношений порядка (полного, частичного и т.д.) можно отнести данное отношение?
\end{enumerate}


\question
Установите, является ли каждое из перечисленных ниже отношений на А ($R \subseteq A \times A$) отношением эквивалентности (обоснование ответа обязательно). Для каждого отношения эквивалентности постройте классы 
эквивалентности и постройте граф отношения:
\begin{enumerate} [a)]\setcounter{enumi}{0}
\item На множестве $A = \{1; 2; 3\}$ задано отношение $R = \{(1; 1); (2; 2); (3; 3); (2; 1); (1; 2); (2; 3); (3; 2); (3; 1); (1; 3)\}$
\item На множестве $A = \{1; 2; 3; 4; 5\}$ задано отношение $R = \{(1; 2); (1; 3); (1; 5); (2; 3); (2; 4); (2; 5); (3; 4); (3; 5); (4; 5)\}$
\item А - множество целых чисел и отношение $R = \{(a,b)|a + b = 0\}$
\end{enumerate}\question Составьте полную таблицу истинности, определите, какие переменные являются фиктивными и проверьте, является ли формула тавтологией:
$(P \rightarrow (Q \rightarrow R)) \rightarrow ((P \rightarrow Q) \rightarrow (P \rightarrow R))$

\end{questions}
\newpage
%%% begin test
\begin{flushright}
\begin{tabular}{p{2.8in} r l}
%\textbf{\class} & \textbf{ФИО:} & \makebox[2.5in]{\hrulefill}\\
\textbf{\class} & \textbf{ФИО:} &Кочубеев Николай Сергеевич
\\

\textbf{\examdate} &&\\
%\textbf{Time Limit: \timelimit} & Teaching Assistant & \makebox[2in]{\hrulefill}
\end{tabular}\\
\end{flushright}
\rule[1ex]{\textwidth}{.1pt}


\begin{questions}
\question
Найдите и упростите P:
\begin{equation*}
\overline{P} = A \cap \overline{B} \cup \overline{B} \cap C \cup \overline{A} \cap \overline{B} \cup \overline{A} \cap C
\end{equation*}
Затем найдите элементы множества P, выраженного через множества:
\begin{equation*}
A = \{0, 3, 4, 9\}; 
B = \{1, 3, 4, 7\};
C = \{0, 1, 2, 4, 7, 8, 9\};
I = \{0, 1, 2, 3, 4, 5, 6, 7, 8, 9\}.
\end{equation*}\question
Упростите следующее выражение с учетом того, что $A\subset B \subset C \subset D \subset U; A \neq \O$
\begin{equation*}
\overline{A} \cap \overline{C} \cap D \cup \overline{B} \cap \overline{C} \cap D \cup A \cap B
\end{equation*}

Примечание: U — универсум\question
Дано отношение на множестве $\{1, 2, 3, 4, 5\}$ 
\begin{equation*}
aRb \iff b > a
\end{equation*}
Напишите обоснованный ответ какими свойствами обладает или не обладает отношение и почему:   
\begin{enumerate} [a)]\setcounter{enumi}{0}
\item рефлексивность
\item антирефлексивность
\item симметричность
\item асимметричность
\item антисимметричность
\item транзитивность
\end{enumerate}

Обоснуйте свой ответ по каждому из приведенных ниже вопросов:
\begin{enumerate} [a)]\setcounter{enumi}{0}
    \item Является ли это отношение отношением эквивалентности?
    \item Является ли это отношение функциональным?
    \item Каким из отношений соответствия (одно-многозначным, много-многозначный и т.д.) оно является?
    \item К каким из отношений порядка (полного, частичного и т.д.) можно отнести данное отношение?
\end{enumerate}

\question
Установите, является ли каждое из перечисленных ниже отношений на А ($R \subseteq A \times A$) отношением эквивалентности (обоснование ответа обязательно). Для каждого отношения эквивалентности 
постройте классы эквивалентности и постройте граф отношения:
\begin{enumerate}[a)]\setcounter{enumi}{0}
\item А - множество целых чисел и отношение $R = \{(a,b)|a + b = 0\}$
\item $A = \{-10, -9, …, 9, 10\}$ и отношение $R = \{(a,b)|a^{3} = b^{3}\}$
\item На множестве $A = \{1; 2; 3\}$ задано отношение $R = \{(1; 1); (2; 2); (3; 3); (2; 1); (1; 2); (2; 3); (3; 2); (3; 1); (1; 3)\}$

\end{enumerate}\question Составьте полную таблицу истинности, определите, какие переменные являются фиктивными и проверьте, является ли формула тавтологией:
$((P \rightarrow Q) \lor R) \leftrightarrow (P \rightarrow (Q \lor R))$

\end{questions}
\newpage
%%% begin test
\begin{flushright}
\begin{tabular}{p{2.8in} r l}
%\textbf{\class} & \textbf{ФИО:} & \makebox[2.5in]{\hrulefill}\\
\textbf{\class} & \textbf{ФИО:} &Кузнецов Павел Григорьевич
\\

\textbf{\examdate} &&\\
%\textbf{Time Limit: \timelimit} & Teaching Assistant & \makebox[2in]{\hrulefill}
\end{tabular}\\
\end{flushright}
\rule[1ex]{\textwidth}{.1pt}


\begin{questions}
\question
Найдите и упростите P:
\begin{equation*}
\overline{P} = A \cap \overline{C} \cup A \cap \overline{B} \cup B \cap \overline{C} \cup A \cap C
\end{equation*}
Затем найдите элементы множества P, выраженного через множества:
\begin{equation*}
A = \{0, 3, 4, 9\}; 
B = \{1, 3, 4, 7\};
C = \{0, 1, 2, 4, 7, 8, 9\};
I = \{0, 1, 2, 3, 4, 5, 6, 7, 8, 9\}.
\end{equation*}\question
Упростите следующее выражение с учетом того, что $A\subset B \subset C \subset D \subset U; A \neq \O$
\begin{equation*}
\overline{A} \cap \overline{C} \cap D \cup \overline{B} \cap \overline{C} \cap D \cup A \cap B
\end{equation*}

Примечание: U — универсум\question
Дано отношение на множестве $\{1, 2, 3, 4, 5\}$ 
\begin{equation*}
aRb \iff a \geq b^2
\end{equation*}
Напишите обоснованный ответ какими свойствами обладает или не обладает отношение и почему:   
\begin{enumerate} [a)]\setcounter{enumi}{0}
\item рефлексивность
\item антирефлексивность
\item симметричность
\item асимметричность
\item антисимметричность
\item транзитивность
\end{enumerate}

Обоснуйте свой ответ по каждому из приведенных ниже вопросов:
\begin{enumerate} [a)]\setcounter{enumi}{0}
    \item Является ли это отношение отношением эквивалентности?
    \item Является ли это отношение функциональным?
    \item Каким из отношений соответствия (одно-многозначным, много-многозначный и т.д.) оно является?
    \item К каким из отношений порядка (полного, частичного и т.д.) можно отнести данное отношение?
\end{enumerate}


\question
Установите, является ли каждое из перечисленных ниже отношений на А ($R \subseteq A \times A$) отношением эквивалентности (обоснование ответа обязательно). Для каждого отношения эквивалентности постройте классы 
эквивалентности и постройте граф отношения:
\begin{enumerate} [a)]\setcounter{enumi}{0}
\item А - множество целых чисел и отношение $R = \{(a,b)|a + b = 5\}$
\item Пусть A – множество имен. $A = \{ $Алексей, Иван, Петр, Александр, Павел, Андрей$ \}$. Тогда отношение $R $ верно на парах имен, начинающихся с одной и той же буквы, и только на них.
\item На множестве $A = \{1; 2; 3; 4; 5\}$ задано отношение $R = \{(1; 2); (1; 3); (1; 5); (2; 3); (2; 4); (2; 5); (3; 4); (3; 5); (4; 5)\}$
\end{enumerate}\question Составьте полную таблицу истинности, определите, какие переменные являются фиктивными и проверьте, является ли формула тавтологией:

$(P \rightarrow (Q \land R)) \leftrightarrow ((P \rightarrow Q) \land (P \rightarrow R))$

\end{questions}
\newpage
%%% begin test
\begin{flushright}
\begin{tabular}{p{2.8in} r l}
%\textbf{\class} & \textbf{ФИО:} & \makebox[2.5in]{\hrulefill}\\
\textbf{\class} & \textbf{ФИО:} &Никонов Илья Владимирович
\\

\textbf{\examdate} &&\\
%\textbf{Time Limit: \timelimit} & Teaching Assistant & \makebox[2in]{\hrulefill}
\end{tabular}\\
\end{flushright}
\rule[1ex]{\textwidth}{.1pt}


\begin{questions}
\question
Найдите и упростите P:
\begin{equation*}
\overline{P} = A \cap C \cup \overline{A} \cap \overline{C} \cup \overline{B} \cap C \cup \overline{A} \cap \overline{B}
\end{equation*}
Затем найдите элементы множества P, выраженного через множества:
\begin{equation*}
A = \{0, 3, 4, 9\}; 
B = \{1, 3, 4, 7\};
C = \{0, 1, 2, 4, 7, 8, 9\};
I = \{0, 1, 2, 3, 4, 5, 6, 7, 8, 9\}.
\end{equation*}\question
Упростите следующее выражение с учетом того, что $A\subset B \subset C \subset D \subset U; A \neq \O$
\begin{equation*}
A \cap B  \cap \overline{C} \cup \overline{C} \cap D \cup B \cap C \cap D
\end{equation*}

Примечание: U — универсум\question
Дано отношение на множестве $\{1, 2, 3, 4, 5\}$ 
\begin{equation*}
aRb \iff (a+b) \bmod 2 =0
\end{equation*}
Напишите обоснованный ответ какими свойствами обладает или не обладает отношение и почему:   
\begin{enumerate} [a)]\setcounter{enumi}{0}
\item рефлексивность
\item антирефлексивность
\item симметричность
\item асимметричность
\item антисимметричность
\item транзитивность
\end{enumerate}

Обоснуйте свой ответ по каждому из приведенных ниже вопросов:
\begin{enumerate} [a)]\setcounter{enumi}{0}
    \item Является ли это отношение отношением эквивалентности?
    \item Является ли это отношение функциональным?
    \item Каким из отношений соответствия (одно-многозначным, много-многозначный и т.д.) оно является?
    \item К каким из отношений порядка (полного, частичного и т.д.) можно отнести данное отношение?
\end{enumerate}



\question
Установите, является ли каждое из перечисленных ниже отношений на А ($R \subseteq A \times A$) отношением эквивалентности (обоснование ответа обязательно). Для каждого отношения эквивалентности постройте классы 
эквивалентности и постройте граф отношения:
\begin{enumerate} [a)]\setcounter{enumi}{0}
\item $A = \{a, b, c, d, p, t\}$ задано отношение $R = \{(a, a), (b, b), (b, c), (b, d), (c, b), (c, c), (c, d), (d, b), (d, c), (d, d), (p,p), (t,t)\}$
\item $A = \{-10, -9, … , 9, 10\}$ и отношение $R = \{(a,b)|a^{3} = b^{3}\}$

\item $F(x)=x^{2}+1$, где $x \in A = [-2, 4]$ и отношение $R = \{(a,b)|F(a) = F(b)\}$
\end{enumerate}\question Составьте полную таблицу истинности, определите, какие переменные являются фиктивными и проверьте, является ли формула тавтологией:
$ P \rightarrow (Q \rightarrow ((P \lor Q) \rightarrow (P \land Q)))$

\end{questions}
\newpage
%%% begin test
\begin{flushright}
\begin{tabular}{p{2.8in} r l}
%\textbf{\class} & \textbf{ФИО:} & \makebox[2.5in]{\hrulefill}\\
\textbf{\class} & \textbf{ФИО:} &Новгородов Артем Николаевич
\\

\textbf{\examdate} &&\\
%\textbf{Time Limit: \timelimit} & Teaching Assistant & \makebox[2in]{\hrulefill}
\end{tabular}\\
\end{flushright}
\rule[1ex]{\textwidth}{.1pt}


\begin{questions}
\question
Найдите и упростите P:
\begin{equation*}
\overline{P} = A \cap \overline{C} \cup A \cap \overline{B} \cup B \cap \overline{C} \cup A \cap C
\end{equation*}
Затем найдите элементы множества P, выраженного через множества:
\begin{equation*}
A = \{0, 3, 4, 9\}; 
B = \{1, 3, 4, 7\};
C = \{0, 1, 2, 4, 7, 8, 9\};
I = \{0, 1, 2, 3, 4, 5, 6, 7, 8, 9\}.
\end{equation*}\question
Упростите следующее выражение с учетом того, что $A\subset B \subset C \subset D \subset U; A \neq \O$
\begin{equation*}
A \cap C  \cap D \cup B \cap \overline{C} \cap D \cup B \cap C \cap D
\end{equation*}

Примечание: U — универсум\question
Дано отношение на множестве $\{1, 2, 3, 4, 5\}$ 
\begin{equation*}
aRb \iff b > a
\end{equation*}
Напишите обоснованный ответ какими свойствами обладает или не обладает отношение и почему:   
\begin{enumerate} [a)]\setcounter{enumi}{0}
\item рефлексивность
\item антирефлексивность
\item симметричность
\item асимметричность
\item антисимметричность
\item транзитивность
\end{enumerate}

Обоснуйте свой ответ по каждому из приведенных ниже вопросов:
\begin{enumerate} [a)]\setcounter{enumi}{0}
    \item Является ли это отношение отношением эквивалентности?
    \item Является ли это отношение функциональным?
    \item Каким из отношений соответствия (одно-многозначным, много-многозначный и т.д.) оно является?
    \item К каким из отношений порядка (полного, частичного и т.д.) можно отнести данное отношение?
\end{enumerate}

\question
Установите, является ли каждое из перечисленных ниже отношений на А ($R \subseteq A \times A$) отношением эквивалентности (обоснование ответа обязательно). Для каждого отношения эквивалентности 
постройте классы эквивалентности и постройте граф отношения:
\begin{enumerate}[a)]\setcounter{enumi}{0}
\item А - множество целых чисел и отношение $R = \{(a,b)|a + b = 0\}$
\item $A = \{-10, -9, …, 9, 10\}$ и отношение $R = \{(a,b)|a^{3} = b^{3}\}$
\item На множестве $A = \{1; 2; 3\}$ задано отношение $R = \{(1; 1); (2; 2); (3; 3); (2; 1); (1; 2); (2; 3); (3; 2); (3; 1); (1; 3)\}$

\end{enumerate}\question Составьте полную таблицу истинности, определите, какие переменные являются фиктивными и проверьте, является ли формула тавтологией:

$(P \rightarrow (Q \land R)) \leftrightarrow ((P \rightarrow Q) \land (P \rightarrow R))$

\end{questions}
\newpage
%%% begin test
\begin{flushright}
\begin{tabular}{p{2.8in} r l}
%\textbf{\class} & \textbf{ФИО:} & \makebox[2.5in]{\hrulefill}\\
\textbf{\class} & \textbf{ФИО:} &Пезин Максим Сергеевич
\\

\textbf{\examdate} &&\\
%\textbf{Time Limit: \timelimit} & Teaching Assistant & \makebox[2in]{\hrulefill}
\end{tabular}\\
\end{flushright}
\rule[1ex]{\textwidth}{.1pt}


\begin{questions}
\question
Найдите и упростите P:
\begin{equation*}
\overline{P} = \overline{A} \cap B \cup \overline{A} \cap C \cup A \cap \overline{B} \cup \overline{B} \cap C
\end{equation*}
Затем найдите элементы множества P, выраженного через множества:
\begin{equation*}
A = \{0, 3, 4, 9\}; 
B = \{1, 3, 4, 7\};
C = \{0, 1, 2, 4, 7, 8, 9\};
I = \{0, 1, 2, 3, 4, 5, 6, 7, 8, 9\}.
\end{equation*}\question
Упростите следующее выражение с учетом того, что $A\subset B \subset C \subset D \subset U; A \neq \O$
\begin{equation*}
\overline{A} \cap \overline{B} \cup B \cap \overline{C} \cup \overline{C} \cap D
\end{equation*}

Примечание: U — универсум\question
Для следующего отношения на множестве $\{1, 2, 3, 4, 5\}$ 
\begin{equation*}
aRb \iff 0 < a-b<2
\end{equation*}
Напишите обоснованный ответ какими свойствами обладает или не обладает отношение и почему:   
\begin{enumerate} [a)]\setcounter{enumi}{0}
\item рефлексивность
\item антирефлексивность
\item симметричность
\item асимметричность
\item антисимметричность
\item транзитивность
\end{enumerate}

Обоснуйте свой ответ по каждому из приведенных ниже вопросов:
\begin{enumerate} [a)]\setcounter{enumi}{0}
    \item Является ли это отношение отношением эквивалентности?
    \item Является ли это отношение функциональным?
    \item Каким из отношений соответствия (одно-многозначным, много-многозначный и т.д.) оно является?
    \item К каким из отношений порядка (полного, частичного и т.д.) можно отнести данное отношение?
\end{enumerate}
\question
Установите, является ли каждое из перечисленных ниже отношений на А ($R \subseteq A \times A$) отношением эквивалентности (обоснование ответа обязательно). Для каждого отношения эквивалентности 
постройте классы эквивалентности и постройте граф отношения:
\begin{enumerate}[a)]\setcounter{enumi}{0}
\item А - множество целых чисел и отношение $R = \{(a,b)|a + b = 0\}$
\item $A = \{-10, -9, …, 9, 10\}$ и отношение $R = \{(a,b)|a^{3} = b^{3}\}$
\item На множестве $A = \{1; 2; 3\}$ задано отношение $R = \{(1; 1); (2; 2); (3; 3); (2; 1); (1; 2); (2; 3); (3; 2); (3; 1); (1; 3)\}$

\end{enumerate}\question Составьте полную таблицу истинности, определите, какие переменные являются фиктивными и проверьте, является ли формула тавтологией:
$ P \rightarrow (Q \rightarrow ((P \lor Q) \rightarrow (P \land Q)))$

\end{questions}
\newpage
%%% begin test
\begin{flushright}
\begin{tabular}{p{2.8in} r l}
%\textbf{\class} & \textbf{ФИО:} & \makebox[2.5in]{\hrulefill}\\
\textbf{\class} & \textbf{ФИО:} &Петкевич Андрей Алексеевич
\\

\textbf{\examdate} &&\\
%\textbf{Time Limit: \timelimit} & Teaching Assistant & \makebox[2in]{\hrulefill}
\end{tabular}\\
\end{flushright}
\rule[1ex]{\textwidth}{.1pt}


\begin{questions}
\question
Найдите и упростите P:
\begin{equation*}
\overline{P} = \overline{A} \cap B \cup \overline{A} \cap C \cup A \cap \overline{B} \cup \overline{B} \cap C
\end{equation*}
Затем найдите элементы множества P, выраженного через множества:
\begin{equation*}
A = \{0, 3, 4, 9\}; 
B = \{1, 3, 4, 7\};
C = \{0, 1, 2, 4, 7, 8, 9\};
I = \{0, 1, 2, 3, 4, 5, 6, 7, 8, 9\}.
\end{equation*}\question
Упростите следующее выражение с учетом того, что $A\subset B \subset C \subset D \subset U; A \neq \O$
\begin{equation*}
\overline{A} \cap \overline{B} \cup B \cap \overline{C} \cup \overline{C} \cap D
\end{equation*}

Примечание: U — универсум\question
Дано отношение на множестве $\{1, 2, 3, 4, 5\}$ 
\begin{equation*}
aRb \iff a \geq b^2
\end{equation*}
Напишите обоснованный ответ какими свойствами обладает или не обладает отношение и почему:   
\begin{enumerate} [a)]\setcounter{enumi}{0}
\item рефлексивность
\item антирефлексивность
\item симметричность
\item асимметричность
\item антисимметричность
\item транзитивность
\end{enumerate}

Обоснуйте свой ответ по каждому из приведенных ниже вопросов:
\begin{enumerate} [a)]\setcounter{enumi}{0}
    \item Является ли это отношение отношением эквивалентности?
    \item Является ли это отношение функциональным?
    \item Каким из отношений соответствия (одно-многозначным, много-многозначный и т.д.) оно является?
    \item К каким из отношений порядка (полного, частичного и т.д.) можно отнести данное отношение?
\end{enumerate}


\question
Установите, является ли каждое из перечисленных ниже отношений на А ($R \subseteq A \times A$) отношением эквивалентности (обоснование ответа обязательно). Для каждого отношения эквивалентности постройте классы 
эквивалентности и постройте граф отношения:
\begin{enumerate} [a)]\setcounter{enumi}{0}
\item На множестве $A = \{1; 2; 3\}$ задано отношение $R = \{(1; 1); (2; 2); (3; 3); (2; 1); (1; 2); (2; 3); (3; 2); (3; 1); (1; 3)\}$
\item На множестве $A = \{1; 2; 3; 4; 5\}$ задано отношение $R = \{(1; 2); (1; 3); (1; 5); (2; 3); (2; 4); (2; 5); (3; 4); (3; 5); (4; 5)\}$
\item А - множество целых чисел и отношение $R = \{(a,b)|a + b = 0\}$
\end{enumerate}\question Составьте полную таблицу истинности, определите, какие переменные являются фиктивными и проверьте, является ли формула тавтологией:
$(P \rightarrow (Q \rightarrow R)) \rightarrow ((P \rightarrow Q) \rightarrow (P \rightarrow R))$

\end{questions}
\newpage
%%% begin test
\begin{flushright}
\begin{tabular}{p{2.8in} r l}
%\textbf{\class} & \textbf{ФИО:} & \makebox[2.5in]{\hrulefill}\\
\textbf{\class} & \textbf{ФИО:} &Познянский Кирилл Олегович
\\

\textbf{\examdate} &&\\
%\textbf{Time Limit: \timelimit} & Teaching Assistant & \makebox[2in]{\hrulefill}
\end{tabular}\\
\end{flushright}
\rule[1ex]{\textwidth}{.1pt}


\begin{questions}
\question
Найдите и упростите P:
\begin{equation*}
\overline{P} = B \cap \overline{C} \cup A \cap B \cup \overline{A} \cap C \cup \overline{A} \cap B
\end{equation*}
Затем найдите элементы множества P, выраженного через множества:
\begin{equation*}
A = \{0, 3, 4, 9\}; 
B = \{1, 3, 4, 7\};
C = \{0, 1, 2, 4, 7, 8, 9\};
I = \{0, 1, 2, 3, 4, 5, 6, 7, 8, 9\}.
\end{equation*}\question
Упростите следующее выражение с учетом того, что $A\subset B \subset C \subset D \subset U; A \neq \O$
\begin{equation*}
\overline{B} \cap \overline{C} \cap D \cup \overline{A} \cap \overline{C} \cap D \cup \overline{A} \cap B
\end{equation*}

Примечание: U — универсум\question
Дано отношение на множестве $\{1, 2, 3, 4, 5\}$ 
\begin{equation*}
aRb \iff (a+b) \bmod 2 =0
\end{equation*}
Напишите обоснованный ответ какими свойствами обладает или не обладает отношение и почему:   
\begin{enumerate} [a)]\setcounter{enumi}{0}
\item рефлексивность
\item антирефлексивность
\item симметричность
\item асимметричность
\item антисимметричность
\item транзитивность
\end{enumerate}

Обоснуйте свой ответ по каждому из приведенных ниже вопросов:
\begin{enumerate} [a)]\setcounter{enumi}{0}
    \item Является ли это отношение отношением эквивалентности?
    \item Является ли это отношение функциональным?
    \item Каким из отношений соответствия (одно-многозначным, много-многозначный и т.д.) оно является?
    \item К каким из отношений порядка (полного, частичного и т.д.) можно отнести данное отношение?
\end{enumerate}



\question
Установите, является ли каждое из перечисленных ниже отношений на А ($R \subseteq A \times A$) отношением эквивалентности (обоснование ответа обязательно). Для каждого отношения эквивалентности 
постройте классы эквивалентности и постройте граф отношения:
\begin{enumerate}[a)]\setcounter{enumi}{0}
\item А - множество целых чисел и отношение $R = \{(a,b)|a + b = 0\}$
\item $A = \{-10, -9, …, 9, 10\}$ и отношение $R = \{(a,b)|a^{3} = b^{3}\}$
\item На множестве $A = \{1; 2; 3\}$ задано отношение $R = \{(1; 1); (2; 2); (3; 3); (2; 1); (1; 2); (2; 3); (3; 2); (3; 1); (1; 3)\}$

\end{enumerate}\question Составьте полную таблицу истинности, определите, какие переменные являются фиктивными и проверьте, является ли формула тавтологией:
$ P \rightarrow (Q \rightarrow ((P \lor Q) \rightarrow (P \land Q)))$

\end{questions}
\newpage
%%% begin test
\begin{flushright}
\begin{tabular}{p{2.8in} r l}
%\textbf{\class} & \textbf{ФИО:} & \makebox[2.5in]{\hrulefill}\\
\textbf{\class} & \textbf{ФИО:} &Савоськин Максим Евгеньевич
\\

\textbf{\examdate} &&\\
%\textbf{Time Limit: \timelimit} & Teaching Assistant & \makebox[2in]{\hrulefill}
\end{tabular}\\
\end{flushright}
\rule[1ex]{\textwidth}{.1pt}


\begin{questions}
\question
Найдите и упростите P:
\begin{equation*}
\overline{P} = A \cap C \cup \overline{A} \cap \overline{C} \cup \overline{B} \cap C \cup \overline{A} \cap \overline{B}
\end{equation*}
Затем найдите элементы множества P, выраженного через множества:
\begin{equation*}
A = \{0, 3, 4, 9\}; 
B = \{1, 3, 4, 7\};
C = \{0, 1, 2, 4, 7, 8, 9\};
I = \{0, 1, 2, 3, 4, 5, 6, 7, 8, 9\}.
\end{equation*}\question
Упростите следующее выражение с учетом того, что $A\subset B \subset C \subset D \subset U; A \neq \O$
\begin{equation*}
A \cap C  \cap D \cup B \cap \overline{C} \cap D \cup B \cap C \cap D
\end{equation*}

Примечание: U — универсум\question
Для следующего отношения на множестве $\{1, 2, 3, 4, 5\}$ 
\begin{equation*}
aRb \iff 0 < a-b<2
\end{equation*}
Напишите обоснованный ответ какими свойствами обладает или не обладает отношение и почему:   
\begin{enumerate} [a)]\setcounter{enumi}{0}
\item рефлексивность
\item антирефлексивность
\item симметричность
\item асимметричность
\item антисимметричность
\item транзитивность
\end{enumerate}

Обоснуйте свой ответ по каждому из приведенных ниже вопросов:
\begin{enumerate} [a)]\setcounter{enumi}{0}
    \item Является ли это отношение отношением эквивалентности?
    \item Является ли это отношение функциональным?
    \item Каким из отношений соответствия (одно-многозначным, много-многозначный и т.д.) оно является?
    \item К каким из отношений порядка (полного, частичного и т.д.) можно отнести данное отношение?
\end{enumerate}
\question
Установите, является ли каждое из перечисленных ниже отношений на А ($R \subseteq A \times A$) отношением эквивалентности (обоснование ответа обязательно). Для каждого отношения эквивалентности постройте классы эквивалентности и постройте граф отношения:
\begin{enumerate} [a)]\setcounter{enumi}{0}
\item $F(x)=x^{2}+1$, где $x \in A = [-2, 4]$ и отношение $R = \{(a,b)|F(a) = F(b)\}$
\item А - множество целых чисел и отношение $R = \{(a,b)|a + b = 5\}$
\item На множестве $A = \{1; 2; 3\}$ задано отношение $R = \{(1; 1); (2; 2); (3; 3); (3; 2); (1; 2); (2; 1)\}$

\end{enumerate}\question Составьте полную таблицу истинности, определите, какие переменные являются фиктивными и проверьте, является ли формула тавтологией:
$(( P \land \neg Q) \rightarrow (R \land \neg R)) \rightarrow (P \rightarrow Q)$

\end{questions}
\newpage
%%% begin test
\begin{flushright}
\begin{tabular}{p{2.8in} r l}
%\textbf{\class} & \textbf{ФИО:} & \makebox[2.5in]{\hrulefill}\\
\textbf{\class} & \textbf{ФИО:} &Сидорцов Владимир Сергеевич
\\

\textbf{\examdate} &&\\
%\textbf{Time Limit: \timelimit} & Teaching Assistant & \makebox[2in]{\hrulefill}
\end{tabular}\\
\end{flushright}
\rule[1ex]{\textwidth}{.1pt}


\begin{questions}
\question
Найдите и упростите P:
\begin{equation*}
\overline{P} = A \cap \overline{C} \cup A \cap \overline{B} \cup B \cap \overline{C} \cup A \cap C
\end{equation*}
Затем найдите элементы множества P, выраженного через множества:
\begin{equation*}
A = \{0, 3, 4, 9\}; 
B = \{1, 3, 4, 7\};
C = \{0, 1, 2, 4, 7, 8, 9\};
I = \{0, 1, 2, 3, 4, 5, 6, 7, 8, 9\}.
\end{equation*}\question
Упростите следующее выражение с учетом того, что $A\subset B \subset C \subset D \subset U; A \neq \O$
\begin{equation*}
\overline{A} \cap \overline{C} \cap D \cup \overline{B} \cap \overline{C} \cap D \cup A \cap B
\end{equation*}

Примечание: U — универсум\question
Дано отношение на множестве $\{1, 2, 3, 4, 5\}$ 
\begin{equation*}
aRb \iff  \text{НОД}(a,b) =1
\end{equation*}
Напишите обоснованный ответ какими свойствами обладает или не обладает отношение и почему:   
\begin{enumerate} [a)]\setcounter{enumi}{0}
\item рефлексивность
\item антирефлексивность
\item симметричность
\item асимметричность
\item антисимметричность
\item транзитивность
\end{enumerate}

Обоснуйте свой ответ по каждому из приведенных ниже вопросов:
\begin{enumerate} [a)]\setcounter{enumi}{0}
    \item Является ли это отношение отношением эквивалентности?
    \item Является ли это отношение функциональным?
    \item Каким из отношений соответствия (одно-многозначным, много-многозначный и т.д.) оно является?
    \item К каким из отношений порядка (полного, частичного и т.д.) можно отнести данное отношение?
\end{enumerate}


\question
Установите, является ли каждое из перечисленных ниже отношений на А ($R \subseteq A \times A$) отношением эквивалентности (обоснование ответа обязательно). Для каждого отношения эквивалентности постройте классы эквивалентности и постройте граф отношения:
\begin{enumerate} [a)]\setcounter{enumi}{0}
\item $F(x)=x^{2}+1$, где $x \in A = [-2, 4]$ и отношение $R = \{(a,b)|F(a) = F(b)\}$
\item А - множество целых чисел и отношение $R = \{(a,b)|a + b = 5\}$
\item На множестве $A = \{1; 2; 3\}$ задано отношение $R = \{(1; 1); (2; 2); (3; 3); (3; 2); (1; 2); (2; 1)\}$

\end{enumerate}\question Составьте полную таблицу истинности, определите, какие переменные являются фиктивными и проверьте, является ли формула тавтологией:
$ P \rightarrow (Q \rightarrow ((P \lor Q) \rightarrow (P \land Q)))$

\end{questions}
\newpage
%%% begin test
\begin{flushright}
\begin{tabular}{p{2.8in} r l}
%\textbf{\class} & \textbf{ФИО:} & \makebox[2.5in]{\hrulefill}\\
\textbf{\class} & \textbf{ФИО:} &Темников Алексей Николаевич
\\

\textbf{\examdate} &&\\
%\textbf{Time Limit: \timelimit} & Teaching Assistant & \makebox[2in]{\hrulefill}
\end{tabular}\\
\end{flushright}
\rule[1ex]{\textwidth}{.1pt}


\begin{questions}
\question
Найдите и упростите P:
\begin{equation*}
\overline{P} = A \cap \overline{B} \cup A \cap C \cup B \cap C \cup \overline{A} \cap C
\end{equation*}
Затем найдите элементы множества P, выраженного через множества:
\begin{equation*}
A = \{0, 3, 4, 9\}; 
B = \{1, 3, 4, 7\};
C = \{0, 1, 2, 4, 7, 8, 9\};
I = \{0, 1, 2, 3, 4, 5, 6, 7, 8, 9\}.
\end{equation*}\question
Упростите следующее выражение с учетом того, что $A\subset B \subset C \subset D \subset U; A \neq \O$
\begin{equation*}
A \cap B  \cap \overline{C} \cup \overline{C} \cap D \cup B \cap C \cap D
\end{equation*}

Примечание: U — универсум\question
Для следующего отношения на множестве $\{1, 2, 3, 4, 5\}$ 
\begin{equation*}
aRb \iff 0 < a-b<2
\end{equation*}
Напишите обоснованный ответ какими свойствами обладает или не обладает отношение и почему:   
\begin{enumerate} [a)]\setcounter{enumi}{0}
\item рефлексивность
\item антирефлексивность
\item симметричность
\item асимметричность
\item антисимметричность
\item транзитивность
\end{enumerate}

Обоснуйте свой ответ по каждому из приведенных ниже вопросов:
\begin{enumerate} [a)]\setcounter{enumi}{0}
    \item Является ли это отношение отношением эквивалентности?
    \item Является ли это отношение функциональным?
    \item Каким из отношений соответствия (одно-многозначным, много-многозначный и т.д.) оно является?
    \item К каким из отношений порядка (полного, частичного и т.д.) можно отнести данное отношение?
\end{enumerate}
\question
Установите, является ли каждое из перечисленных ниже отношений на А ($R \subseteq A \times A$) отношением эквивалентности (обоснование ответа обязательно). Для каждого отношения эквивалентности постройте классы 
эквивалентности и постройте граф отношения:
\begin{enumerate} [a)]\setcounter{enumi}{0}
\item Пусть A – множество имен. $A = \{ $Алексей, Иван, Петр, Александр, Павел, Андрей$ \}$. Тогда отношение $R$ верно на парах имен, начинающихся с одной и той же буквы, и только на них.
\item $A = \{-10, -9, … , 9, 10\}$ и отношение $ R = \{(a,b)|a^{2} = b^{2}\}$
\item На множестве $A = \{1; 2; 3\}$ задано отношение $R = \{(1; 1); (2; 2); (3; 3); (3; 2); (1; 2); (2; 1)\}$
\end{enumerate}\question Составьте полную таблицу истинности, определите, какие переменные являются фиктивными и проверьте, является ли формула тавтологией:
$(( P \rightarrow Q) \land (Q \rightarrow P)) \rightarrow (P \rightarrow R)$

\end{questions}
\newpage
%%% begin test
\begin{flushright}
\begin{tabular}{p{2.8in} r l}
%\textbf{\class} & \textbf{ФИО:} & \makebox[2.5in]{\hrulefill}\\
\textbf{\class} & \textbf{ФИО:} &Толмачев Сергей Евгеньевич
\\

\textbf{\examdate} &&\\
%\textbf{Time Limit: \timelimit} & Teaching Assistant & \makebox[2in]{\hrulefill}
\end{tabular}\\
\end{flushright}
\rule[1ex]{\textwidth}{.1pt}


\begin{questions}
\question
Найдите и упростите P:
\begin{equation*}
\overline{P} = A \cap \overline{B} \cup A \cap C \cup B \cap C \cup \overline{A} \cap C
\end{equation*}
Затем найдите элементы множества P, выраженного через множества:
\begin{equation*}
A = \{0, 3, 4, 9\}; 
B = \{1, 3, 4, 7\};
C = \{0, 1, 2, 4, 7, 8, 9\};
I = \{0, 1, 2, 3, 4, 5, 6, 7, 8, 9\}.
\end{equation*}\question
Упростите следующее выражение с учетом того, что $A\subset B \subset C \subset D \subset U; A \neq \O$
\begin{equation*}
A \cap  \overline{C} \cup B \cap \overline{D} \cup  \overline{A} \cap C \cap  \overline{D}
\end{equation*}

Примечание: U — универсум\question
Дано отношение на множестве $\{1, 2, 3, 4, 5\}$ 
\begin{equation*}
aRb \iff a \geq b^2
\end{equation*}
Напишите обоснованный ответ какими свойствами обладает или не обладает отношение и почему:   
\begin{enumerate} [a)]\setcounter{enumi}{0}
\item рефлексивность
\item антирефлексивность
\item симметричность
\item асимметричность
\item антисимметричность
\item транзитивность
\end{enumerate}

Обоснуйте свой ответ по каждому из приведенных ниже вопросов:
\begin{enumerate} [a)]\setcounter{enumi}{0}
    \item Является ли это отношение отношением эквивалентности?
    \item Является ли это отношение функциональным?
    \item Каким из отношений соответствия (одно-многозначным, много-многозначный и т.д.) оно является?
    \item К каким из отношений порядка (полного, частичного и т.д.) можно отнести данное отношение?
\end{enumerate}


\question
Установите, является ли каждое из перечисленных ниже отношений на А ($R \subseteq A \times A$) отношением эквивалентности (обоснование ответа обязательно). Для каждого отношения эквивалентности постройте классы эквивалентности и постройте граф отношения:
\begin{enumerate} [a)]\setcounter{enumi}{0}
\item $F(x)=x^{2}+1$, где $x \in A = [-2, 4]$ и отношение $R = \{(a,b)|F(a) = F(b)\}$
\item А - множество целых чисел и отношение $R = \{(a,b)|a + b = 5\}$
\item На множестве $A = \{1; 2; 3\}$ задано отношение $R = \{(1; 1); (2; 2); (3; 3); (3; 2); (1; 2); (2; 1)\}$

\end{enumerate}\question Составьте полную таблицу истинности, определите, какие переменные являются фиктивными и проверьте, является ли формула тавтологией:
$(P \rightarrow (Q \rightarrow R)) \rightarrow ((P \rightarrow Q) \rightarrow (P \rightarrow R))$

\end{questions}
\newpage
%%% begin test
\begin{flushright}
\begin{tabular}{p{2.8in} r l}
%\textbf{\class} & \textbf{ФИО:} & \makebox[2.5in]{\hrulefill}\\
\textbf{\class} & \textbf{ФИО:} &Хабаров Сергей Михайлович
\\

\textbf{\examdate} &&\\
%\textbf{Time Limit: \timelimit} & Teaching Assistant & \makebox[2in]{\hrulefill}
\end{tabular}\\
\end{flushright}
\rule[1ex]{\textwidth}{.1pt}


\begin{questions}
\question
Найдите и упростите P:
\begin{equation*}
\overline{P} = \overline{A} \cap B \cup \overline{A} \cap C \cup A \cap \overline{B} \cup \overline{B} \cap C
\end{equation*}
Затем найдите элементы множества P, выраженного через множества:
\begin{equation*}
A = \{0, 3, 4, 9\}; 
B = \{1, 3, 4, 7\};
C = \{0, 1, 2, 4, 7, 8, 9\};
I = \{0, 1, 2, 3, 4, 5, 6, 7, 8, 9\}.
\end{equation*}\question
Упростите следующее выражение с учетом того, что $A\subset B \subset C \subset D \subset U; A \neq \O$
\begin{equation*}
\overline{B} \cap \overline{C} \cap D \cup \overline{A} \cap \overline{C} \cap D \cup \overline{A} \cap B
\end{equation*}

Примечание: U — универсум\question
Дано отношение на множестве $\{1, 2, 3, 4, 5\}$ 
\begin{equation*}
aRb \iff (a+b) \bmod 2 =0
\end{equation*}
Напишите обоснованный ответ какими свойствами обладает или не обладает отношение и почему:   
\begin{enumerate} [a)]\setcounter{enumi}{0}
\item рефлексивность
\item антирефлексивность
\item симметричность
\item асимметричность
\item антисимметричность
\item транзитивность
\end{enumerate}

Обоснуйте свой ответ по каждому из приведенных ниже вопросов:
\begin{enumerate} [a)]\setcounter{enumi}{0}
    \item Является ли это отношение отношением эквивалентности?
    \item Является ли это отношение функциональным?
    \item Каким из отношений соответствия (одно-многозначным, много-многозначный и т.д.) оно является?
    \item К каким из отношений порядка (полного, частичного и т.д.) можно отнести данное отношение?
\end{enumerate}



\question
Установите, является ли каждое из перечисленных ниже отношений на А ($R \subseteq A \times A$) отношением эквивалентности (обоснование ответа обязательно). Для каждого отношения эквивалентности постройте классы 
эквивалентности и постройте граф отношения:
\begin{enumerate} [a)]\setcounter{enumi}{0}
\item На множестве $A = \{1; 2; 3\}$ задано отношение $R = \{(1; 1); (2; 2); (3; 3); (2; 1); (1; 2); (2; 3); (3; 2); (3; 1); (1; 3)\}$
\item На множестве $A = \{1; 2; 3; 4; 5\}$ задано отношение $R = \{(1; 2); (1; 3); (1; 5); (2; 3); (2; 4); (2; 5); (3; 4); (3; 5); (4; 5)\}$
\item А - множество целых чисел и отношение $R = \{(a,b)|a + b = 0\}$
\end{enumerate}\question Составьте полную таблицу истинности, определите, какие переменные являются фиктивными и проверьте, является ли формула тавтологией:
$(P \rightarrow (Q \rightarrow R)) \rightarrow ((P \rightarrow Q) \rightarrow (P \rightarrow R))$

\end{questions}
\newpage
%%% begin test
\begin{flushright}
\begin{tabular}{p{2.8in} r l}
%\textbf{\class} & \textbf{ФИО:} & \makebox[2.5in]{\hrulefill}\\
\textbf{\class} & \textbf{ФИО:} &Хромин Сергей Константинович
\\

\textbf{\examdate} &&\\
%\textbf{Time Limit: \timelimit} & Teaching Assistant & \makebox[2in]{\hrulefill}
\end{tabular}\\
\end{flushright}
\rule[1ex]{\textwidth}{.1pt}


\begin{questions}
\question
Найдите и упростите P:
\begin{equation*}
\overline{P} = A \cap B \cup \overline{A} \cap \overline{B} \cup A \cap C \cup \overline{B} \cap C
\end{equation*}
Затем найдите элементы множества P, выраженного через множества:
\begin{equation*}
A = \{0, 3, 4, 9\}; 
B = \{1, 3, 4, 7\};
C = \{0, 1, 2, 4, 7, 8, 9\};
I = \{0, 1, 2, 3, 4, 5, 6, 7, 8, 9\}.
\end{equation*}\question
Упростите следующее выражение с учетом того, что $A\subset B \subset C \subset D \subset U; A \neq \O$
\begin{equation*}
A \cap B  \cap \overline{C} \cup \overline{C} \cap D \cup B \cap C \cap D
\end{equation*}

Примечание: U — универсум\question
Дано отношение на множестве $\{1, 2, 3, 4, 5\}$ 
\begin{equation*}
aRb \iff |a-b| = 1
\end{equation*}
Напишите обоснованный ответ какими свойствами обладает или не обладает отношение и почему:   
\begin{enumerate} [a)]\setcounter{enumi}{0}
\item рефлексивность
\item антирефлексивность
\item симметричность
\item асимметричность
\item антисимметричность
\item транзитивность
\end{enumerate}

Обоснуйте свой ответ по каждому из приведенных ниже вопросов:
\begin{enumerate} [a)]\setcounter{enumi}{0}
    \item Является ли это отношение отношением эквивалентности?
    \item Является ли это отношение функциональным?
    \item Каким из отношений соответствия (одно-многозначным, много-многозначный и т.д.) оно является?
    \item К каким из отношений порядка (полного, частичного и т.д.) можно отнести данное отношение?
\end{enumerate}

\question
Установите, является ли каждое из перечисленных ниже отношений на А ($R \subseteq A \times A$) отношением эквивалентности (обоснование ответа обязательно). Для каждого отношения эквивалентности постройте классы 
эквивалентности и постройте граф отношения:
\begin{enumerate} [a)]\setcounter{enumi}{0}
\item На множестве $A = \{1; 2; 3\}$ задано отношение $R = \{(1; 1); (2; 2); (3; 3); (2; 1); (1; 2); (2; 3); (3; 2); (3; 1); (1; 3)\}$
\item На множестве $A = \{1; 2; 3; 4; 5\}$ задано отношение $R = \{(1; 2); (1; 3); (1; 5); (2; 3); (2; 4); (2; 5); (3; 4); (3; 5); (4; 5)\}$
\item А - множество целых чисел и отношение $R = \{(a,b)|a + b = 0\}$
\end{enumerate}\question Составьте полную таблицу истинности, определите, какие переменные являются фиктивными и проверьте, является ли формула тавтологией:
$ P \rightarrow (Q \rightarrow ((P \lor Q) \rightarrow (P \land Q)))$

\end{questions}
\newpage
%%% begin test
\begin{flushright}
\begin{tabular}{p{2.8in} r l}
%\textbf{\class} & \textbf{ФИО:} & \makebox[2.5in]{\hrulefill}\\
\textbf{\class} & \textbf{ФИО:} &Шаров Михаил Олегович
\\

\textbf{\examdate} &&\\
%\textbf{Time Limit: \timelimit} & Teaching Assistant & \makebox[2in]{\hrulefill}
\end{tabular}\\
\end{flushright}
\rule[1ex]{\textwidth}{.1pt}


\begin{questions}
\question
Найдите и упростите P:
\begin{equation*}
\overline{P} = A \cap B \cup \overline{A} \cap \overline{B} \cup A \cap C \cup \overline{B} \cap C
\end{equation*}
Затем найдите элементы множества P, выраженного через множества:
\begin{equation*}
A = \{0, 3, 4, 9\}; 
B = \{1, 3, 4, 7\};
C = \{0, 1, 2, 4, 7, 8, 9\};
I = \{0, 1, 2, 3, 4, 5, 6, 7, 8, 9\}.
\end{equation*}\question
Упростите следующее выражение с учетом того, что $A\subset B \subset C \subset D \subset U; A \neq \O$
\begin{equation*}
A \cap  \overline{C} \cup B \cap \overline{D} \cup  \overline{A} \cap C \cap  \overline{D}
\end{equation*}

Примечание: U — универсум\question
Дано отношение на множестве $\{1, 2, 3, 4, 5\}$ 
\begin{equation*}
aRb \iff  \text{НОД}(a,b) =1
\end{equation*}
Напишите обоснованный ответ какими свойствами обладает или не обладает отношение и почему:   
\begin{enumerate} [a)]\setcounter{enumi}{0}
\item рефлексивность
\item антирефлексивность
\item симметричность
\item асимметричность
\item антисимметричность
\item транзитивность
\end{enumerate}

Обоснуйте свой ответ по каждому из приведенных ниже вопросов:
\begin{enumerate} [a)]\setcounter{enumi}{0}
    \item Является ли это отношение отношением эквивалентности?
    \item Является ли это отношение функциональным?
    \item Каким из отношений соответствия (одно-многозначным, много-многозначный и т.д.) оно является?
    \item К каким из отношений порядка (полного, частичного и т.д.) можно отнести данное отношение?
\end{enumerate}


\question
Установите, является ли каждое из перечисленных ниже отношений на А ($R \subseteq A \times A$) отношением эквивалентности (обоснование ответа обязательно). Для каждого отношения эквивалентности постройте классы 
эквивалентности и постройте граф отношения:
\begin{enumerate} [a)]\setcounter{enumi}{0}
\item На множестве $A = \{1; 2; 3\}$ задано отношение $R = \{(1; 1); (2; 2); (3; 3); (2; 1); (1; 2); (2; 3); (3; 2); (3; 1); (1; 3)\}$
\item На множестве $A = \{1; 2; 3; 4; 5\}$ задано отношение $R = \{(1; 2); (1; 3); (1; 5); (2; 3); (2; 4); (2; 5); (3; 4); (3; 5); (4; 5)\}$
\item А - множество целых чисел и отношение $R = \{(a,b)|a + b = 0\}$
\end{enumerate}\question Составьте полную таблицу истинности, определите, какие переменные являются фиктивными и проверьте, является ли формула тавтологией:
$ P \rightarrow (Q \rightarrow ((P \lor Q) \rightarrow (P \land Q)))$

\end{questions}
\newpage
%%% begin test
\begin{flushright}
\begin{tabular}{p{2.8in} r l}
%\textbf{\class} & \textbf{ФИО:} & \makebox[2.5in]{\hrulefill}\\
\textbf{\class} & \textbf{ФИО:} &Шуваев Федор Васильевич
\\

\textbf{\examdate} &&\\
%\textbf{Time Limit: \timelimit} & Teaching Assistant & \makebox[2in]{\hrulefill}
\end{tabular}\\
\end{flushright}
\rule[1ex]{\textwidth}{.1pt}


\begin{questions}
\question
Найдите и упростите P:
\begin{equation*}
\overline{P} = A \cap \overline{B} \cup A \cap C \cup B \cap C \cup \overline{A} \cap C
\end{equation*}
Затем найдите элементы множества P, выраженного через множества:
\begin{equation*}
A = \{0, 3, 4, 9\}; 
B = \{1, 3, 4, 7\};
C = \{0, 1, 2, 4, 7, 8, 9\};
I = \{0, 1, 2, 3, 4, 5, 6, 7, 8, 9\}.
\end{equation*}\question
Упростите следующее выражение с учетом того, что $A\subset B \subset C \subset D \subset U; A \neq \O$
\begin{equation*}
A \cap C  \cap D \cup B \cap \overline{C} \cap D \cup B \cap C \cap D
\end{equation*}

Примечание: U — универсум\question
Дано отношение на множестве $\{1, 2, 3, 4, 5\}$ 
\begin{equation*}
aRb \iff (a+b) \bmod 2 =0
\end{equation*}
Напишите обоснованный ответ какими свойствами обладает или не обладает отношение и почему:   
\begin{enumerate} [a)]\setcounter{enumi}{0}
\item рефлексивность
\item антирефлексивность
\item симметричность
\item асимметричность
\item антисимметричность
\item транзитивность
\end{enumerate}

Обоснуйте свой ответ по каждому из приведенных ниже вопросов:
\begin{enumerate} [a)]\setcounter{enumi}{0}
    \item Является ли это отношение отношением эквивалентности?
    \item Является ли это отношение функциональным?
    \item Каким из отношений соответствия (одно-многозначным, много-многозначный и т.д.) оно является?
    \item К каким из отношений порядка (полного, частичного и т.д.) можно отнести данное отношение?
\end{enumerate}



\question
Установите, является ли каждое из перечисленных ниже отношений на А ($R \subseteq A \times A$) отношением эквивалентности (обоснование ответа обязательно). Для каждого отношения эквивалентности постройте классы 
эквивалентности и постройте граф отношения:
\begin{enumerate} [a)]\setcounter{enumi}{0}
\item Пусть A – множество имен. $A = \{ $Алексей, Иван, Петр, Александр, Павел, Андрей$ \}$. Тогда отношение $R$ верно на парах имен, начинающихся с одной и той же буквы, и только на них.
\item $A = \{-10, -9, … , 9, 10\}$ и отношение $ R = \{(a,b)|a^{2} = b^{2}\}$
\item На множестве $A = \{1; 2; 3\}$ задано отношение $R = \{(1; 1); (2; 2); (3; 3); (3; 2); (1; 2); (2; 1)\}$
\end{enumerate}\question Составьте полную таблицу истинности, определите, какие переменные являются фиктивными и проверьте, является ли формула тавтологией:
$ P \rightarrow (Q \rightarrow ((P \lor Q) \rightarrow (P \land Q)))$

\end{questions}
\newpage
%%% begin test
\begin{flushright}
\begin{tabular}{p{2.8in} r l}
%\textbf{\class} & \textbf{ФИО:} & \makebox[2.5in]{\hrulefill}\\
\textbf{\class} & \textbf{ФИО:} &М3108
\\

\textbf{\examdate} &&\\
%\textbf{Time Limit: \timelimit} & Teaching Assistant & \makebox[2in]{\hrulefill}
\end{tabular}\\
\end{flushright}
\rule[1ex]{\textwidth}{.1pt}


\begin{questions}
\question
Найдите и упростите P:
\begin{equation*}
\overline{P} = B \cap \overline{C} \cup A \cap B \cup \overline{A} \cap C \cup \overline{A} \cap B
\end{equation*}
Затем найдите элементы множества P, выраженного через множества:
\begin{equation*}
A = \{0, 3, 4, 9\}; 
B = \{1, 3, 4, 7\};
C = \{0, 1, 2, 4, 7, 8, 9\};
I = \{0, 1, 2, 3, 4, 5, 6, 7, 8, 9\}.
\end{equation*}\question
Упростите следующее выражение с учетом того, что $A\subset B \subset C \subset D \subset U; A \neq \O$
\begin{equation*}
\overline{A} \cap \overline{C} \cap D \cup \overline{B} \cap \overline{C} \cap D \cup A \cap B
\end{equation*}

Примечание: U — универсум\question
Дано отношение на множестве $\{1, 2, 3, 4, 5\}$ 
\begin{equation*}
aRb \iff a \leq b
\end{equation*}
Напишите обоснованный ответ какими свойствами обладает или не обладает отношение и почему:   
\begin{enumerate} [a)]\setcounter{enumi}{0}
\item рефлексивность
\item антирефлексивность
\item симметричность
\item асимметричность
\item антисимметричность
\item транзитивность
\end{enumerate}

Обоснуйте свой ответ по каждому из приведенных ниже вопросов:
\begin{enumerate} [a)]\setcounter{enumi}{0}
    \item Является ли это отношение отношением эквивалентности?
    \item Является ли это отношение функциональным?
    \item Каким из отношений соответствия (одно-многозначным, много-многозначный и т.д.) оно является?
    \item К каким из отношений порядка (полного, частичного и т.д.) можно отнести данное отношение?
\end{enumerate}


\question
Установите, является ли каждое из перечисленных ниже отношений на А ($R \subseteq A \times A$) отношением эквивалентности (обоснование ответа обязательно). Для каждого отношения эквивалентности постройте классы 
эквивалентности и постройте граф отношения:
\begin{enumerate} [a)]\setcounter{enumi}{0}
\item На множестве $A = \{1; 2; 3\}$ задано отношение $R = \{(1; 1); (2; 2); (3; 3); (2; 1); (1; 2); (2; 3); (3; 2); (3; 1); (1; 3)\}$
\item На множестве $A = \{1; 2; 3; 4; 5\}$ задано отношение $R = \{(1; 2); (1; 3); (1; 5); (2; 3); (2; 4); (2; 5); (3; 4); (3; 5); (4; 5)\}$
\item А - множество целых чисел и отношение $R = \{(a,b)|a + b = 0\}$
\end{enumerate}\question Составьте полную таблицу истинности, определите, какие переменные являются фиктивными и проверьте, является ли формула тавтологией:
$(( P \land \neg Q) \rightarrow (R \land \neg R)) \rightarrow (P \rightarrow Q)$

\end{questions}
\newpage
%%% begin test
\begin{flushright}
\begin{tabular}{p{2.8in} r l}
%\textbf{\class} & \textbf{ФИО:} & \makebox[2.5in]{\hrulefill}\\
\textbf{\class} & \textbf{ФИО:} &Андреев Михаил Дмитриевич
\\

\textbf{\examdate} &&\\
%\textbf{Time Limit: \timelimit} & Teaching Assistant & \makebox[2in]{\hrulefill}
\end{tabular}\\
\end{flushright}
\rule[1ex]{\textwidth}{.1pt}


\begin{questions}
\question
Найдите и упростите P:
\begin{equation*}
\overline{P} = \overline{A} \cap B \cup \overline{A} \cap C \cup A \cap \overline{B} \cup \overline{B} \cap C
\end{equation*}
Затем найдите элементы множества P, выраженного через множества:
\begin{equation*}
A = \{0, 3, 4, 9\}; 
B = \{1, 3, 4, 7\};
C = \{0, 1, 2, 4, 7, 8, 9\};
I = \{0, 1, 2, 3, 4, 5, 6, 7, 8, 9\}.
\end{equation*}\question
Упростите следующее выражение с учетом того, что $A\subset B \subset C \subset D \subset U; A \neq \O$
\begin{equation*}
A \cap B  \cap \overline{C} \cup \overline{C} \cap D \cup B \cap C \cap D
\end{equation*}

Примечание: U — универсум\question
Дано отношение на множестве $\{1, 2, 3, 4, 5\}$ 
\begin{equation*}
aRb \iff a \geq b^2
\end{equation*}
Напишите обоснованный ответ какими свойствами обладает или не обладает отношение и почему:   
\begin{enumerate} [a)]\setcounter{enumi}{0}
\item рефлексивность
\item антирефлексивность
\item симметричность
\item асимметричность
\item антисимметричность
\item транзитивность
\end{enumerate}

Обоснуйте свой ответ по каждому из приведенных ниже вопросов:
\begin{enumerate} [a)]\setcounter{enumi}{0}
    \item Является ли это отношение отношением эквивалентности?
    \item Является ли это отношение функциональным?
    \item Каким из отношений соответствия (одно-многозначным, много-многозначный и т.д.) оно является?
    \item К каким из отношений порядка (полного, частичного и т.д.) можно отнести данное отношение?
\end{enumerate}


\question
Установите, является ли каждое из перечисленных ниже отношений на А ($R \subseteq A \times A$) отношением эквивалентности (обоснование ответа обязательно). Для каждого отношения эквивалентности постройте классы 
эквивалентности и постройте граф отношения:
\begin{enumerate} [a)]\setcounter{enumi}{0}
\item На множестве $A = \{1; 2; 3\}$ задано отношение $R = \{(1; 1); (2; 2); (3; 3); (2; 1); (1; 2); (2; 3); (3; 2); (3; 1); (1; 3)\}$
\item На множестве $A = \{1; 2; 3; 4; 5\}$ задано отношение $R = \{(1; 2); (1; 3); (1; 5); (2; 3); (2; 4); (2; 5); (3; 4); (3; 5); (4; 5)\}$
\item А - множество целых чисел и отношение $R = \{(a,b)|a + b = 0\}$
\end{enumerate}\question Составьте полную таблицу истинности, определите, какие переменные являются фиктивными и проверьте, является ли формула тавтологией:
$((P \rightarrow Q) \land (R \rightarrow S) \land \neg (Q \lor S)) \rightarrow \neg (P \lor R)$

\end{questions}
\newpage
%%% begin test
\begin{flushright}
\begin{tabular}{p{2.8in} r l}
%\textbf{\class} & \textbf{ФИО:} & \makebox[2.5in]{\hrulefill}\\
\textbf{\class} & \textbf{ФИО:} &Багомедов Багомед Тимурович
\\

\textbf{\examdate} &&\\
%\textbf{Time Limit: \timelimit} & Teaching Assistant & \makebox[2in]{\hrulefill}
\end{tabular}\\
\end{flushright}
\rule[1ex]{\textwidth}{.1pt}


\begin{questions}
\question
Найдите и упростите P:
\begin{equation*}
\overline{P} = A \cap \overline{C} \cup A \cap \overline{B} \cup B \cap \overline{C} \cup A \cap C
\end{equation*}
Затем найдите элементы множества P, выраженного через множества:
\begin{equation*}
A = \{0, 3, 4, 9\}; 
B = \{1, 3, 4, 7\};
C = \{0, 1, 2, 4, 7, 8, 9\};
I = \{0, 1, 2, 3, 4, 5, 6, 7, 8, 9\}.
\end{equation*}\question
Упростите следующее выражение с учетом того, что $A\subset B \subset C \subset D \subset U; A \neq \O$
\begin{equation*}
\overline{A} \cap \overline{C} \cap D \cup \overline{B} \cap \overline{C} \cap D \cup A \cap B
\end{equation*}

Примечание: U — универсум\question
Дано отношение на множестве $\{1, 2, 3, 4, 5\}$ 
\begin{equation*}
aRb \iff  \text{НОД}(a,b) =1
\end{equation*}
Напишите обоснованный ответ какими свойствами обладает или не обладает отношение и почему:   
\begin{enumerate} [a)]\setcounter{enumi}{0}
\item рефлексивность
\item антирефлексивность
\item симметричность
\item асимметричность
\item антисимметричность
\item транзитивность
\end{enumerate}

Обоснуйте свой ответ по каждому из приведенных ниже вопросов:
\begin{enumerate} [a)]\setcounter{enumi}{0}
    \item Является ли это отношение отношением эквивалентности?
    \item Является ли это отношение функциональным?
    \item Каким из отношений соответствия (одно-многозначным, много-многозначный и т.д.) оно является?
    \item К каким из отношений порядка (полного, частичного и т.д.) можно отнести данное отношение?
\end{enumerate}


\question
Установите, является ли каждое из перечисленных ниже отношений на А ($R \subseteq A \times A$) отношением эквивалентности (обоснование ответа обязательно). Для каждого отношения эквивалентности 
постройте классы эквивалентности и постройте граф отношения:
\begin{enumerate}[a)]\setcounter{enumi}{0}
\item А - множество целых чисел и отношение $R = \{(a,b)|a + b = 0\}$
\item $A = \{-10, -9, …, 9, 10\}$ и отношение $R = \{(a,b)|a^{3} = b^{3}\}$
\item На множестве $A = \{1; 2; 3\}$ задано отношение $R = \{(1; 1); (2; 2); (3; 3); (2; 1); (1; 2); (2; 3); (3; 2); (3; 1); (1; 3)\}$

\end{enumerate}\question Составьте полную таблицу истинности, определите, какие переменные являются фиктивными и проверьте, является ли формула тавтологией:
$(( P \rightarrow Q) \land (Q \rightarrow P)) \rightarrow (P \rightarrow R)$

\end{questions}
\newpage
%%% begin test
\begin{flushright}
\begin{tabular}{p{2.8in} r l}
%\textbf{\class} & \textbf{ФИО:} & \makebox[2.5in]{\hrulefill}\\
\textbf{\class} & \textbf{ФИО:} &Гришин Леонид Владимирович
\\

\textbf{\examdate} &&\\
%\textbf{Time Limit: \timelimit} & Teaching Assistant & \makebox[2in]{\hrulefill}
\end{tabular}\\
\end{flushright}
\rule[1ex]{\textwidth}{.1pt}


\begin{questions}
\question
Найдите и упростите P:
\begin{equation*}
\overline{P} = \overline{A} \cap B \cup \overline{A} \cap C \cup A \cap \overline{B} \cup \overline{B} \cap C
\end{equation*}
Затем найдите элементы множества P, выраженного через множества:
\begin{equation*}
A = \{0, 3, 4, 9\}; 
B = \{1, 3, 4, 7\};
C = \{0, 1, 2, 4, 7, 8, 9\};
I = \{0, 1, 2, 3, 4, 5, 6, 7, 8, 9\}.
\end{equation*}\question
Упростите следующее выражение с учетом того, что $A\subset B \subset C \subset D \subset U; A \neq \O$
\begin{equation*}
\overline{A} \cap \overline{C} \cap D \cup \overline{B} \cap \overline{C} \cap D \cup A \cap B
\end{equation*}

Примечание: U — универсум\question
Дано отношение на множестве $\{1, 2, 3, 4, 5\}$ 
\begin{equation*}
aRb \iff a \leq b
\end{equation*}
Напишите обоснованный ответ какими свойствами обладает или не обладает отношение и почему:   
\begin{enumerate} [a)]\setcounter{enumi}{0}
\item рефлексивность
\item антирефлексивность
\item симметричность
\item асимметричность
\item антисимметричность
\item транзитивность
\end{enumerate}

Обоснуйте свой ответ по каждому из приведенных ниже вопросов:
\begin{enumerate} [a)]\setcounter{enumi}{0}
    \item Является ли это отношение отношением эквивалентности?
    \item Является ли это отношение функциональным?
    \item Каким из отношений соответствия (одно-многозначным, много-многозначный и т.д.) оно является?
    \item К каким из отношений порядка (полного, частичного и т.д.) можно отнести данное отношение?
\end{enumerate}


\question
Установите, является ли каждое из перечисленных ниже отношений на А ($R \subseteq A \times A$) отношением эквивалентности (обоснование ответа обязательно). Для каждого отношения эквивалентности 
постройте классы эквивалентности и постройте граф отношения:
\begin{enumerate}[a)]\setcounter{enumi}{0}
\item А - множество целых чисел и отношение $R = \{(a,b)|a + b = 0\}$
\item $A = \{-10, -9, …, 9, 10\}$ и отношение $R = \{(a,b)|a^{3} = b^{3}\}$
\item На множестве $A = \{1; 2; 3\}$ задано отношение $R = \{(1; 1); (2; 2); (3; 3); (2; 1); (1; 2); (2; 3); (3; 2); (3; 1); (1; 3)\}$

\end{enumerate}\question Составьте полную таблицу истинности, определите, какие переменные являются фиктивными и проверьте, является ли формула тавтологией:
$(( P \rightarrow Q) \land (Q \rightarrow P)) \rightarrow (P \rightarrow R)$

\end{questions}
\newpage
%%% begin test
\begin{flushright}
\begin{tabular}{p{2.8in} r l}
%\textbf{\class} & \textbf{ФИО:} & \makebox[2.5in]{\hrulefill}\\
\textbf{\class} & \textbf{ФИО:} &Дольник Даниил Владиславович
\\

\textbf{\examdate} &&\\
%\textbf{Time Limit: \timelimit} & Teaching Assistant & \makebox[2in]{\hrulefill}
\end{tabular}\\
\end{flushright}
\rule[1ex]{\textwidth}{.1pt}


\begin{questions}
\question
Найдите и упростите P:
\begin{equation*}
\overline{P} = A \cap \overline{B} \cup A \cap C \cup B \cap C \cup \overline{A} \cap C
\end{equation*}
Затем найдите элементы множества P, выраженного через множества:
\begin{equation*}
A = \{0, 3, 4, 9\}; 
B = \{1, 3, 4, 7\};
C = \{0, 1, 2, 4, 7, 8, 9\};
I = \{0, 1, 2, 3, 4, 5, 6, 7, 8, 9\}.
\end{equation*}\question
Упростите следующее выражение с учетом того, что $A\subset B \subset C \subset D \subset U; A \neq \O$
\begin{equation*}
\overline{A} \cap \overline{C} \cap D \cup \overline{B} \cap \overline{C} \cap D \cup A \cap B
\end{equation*}

Примечание: U — универсум\question
Дано отношение на множестве $\{1, 2, 3, 4, 5\}$ 
\begin{equation*}
aRb \iff (a+b) \bmod 2 =0
\end{equation*}
Напишите обоснованный ответ какими свойствами обладает или не обладает отношение и почему:   
\begin{enumerate} [a)]\setcounter{enumi}{0}
\item рефлексивность
\item антирефлексивность
\item симметричность
\item асимметричность
\item антисимметричность
\item транзитивность
\end{enumerate}

Обоснуйте свой ответ по каждому из приведенных ниже вопросов:
\begin{enumerate} [a)]\setcounter{enumi}{0}
    \item Является ли это отношение отношением эквивалентности?
    \item Является ли это отношение функциональным?
    \item Каким из отношений соответствия (одно-многозначным, много-многозначный и т.д.) оно является?
    \item К каким из отношений порядка (полного, частичного и т.д.) можно отнести данное отношение?
\end{enumerate}



\question
Установите, является ли каждое из перечисленных ниже отношений на А ($R \subseteq A \times A$) отношением эквивалентности (обоснование ответа обязательно). Для каждого отношения эквивалентности постройте классы 
эквивалентности и постройте граф отношения:
\begin{enumerate} [a)]\setcounter{enumi}{0}
\item Пусть A – множество имен. $A = \{ $Алексей, Иван, Петр, Александр, Павел, Андрей$ \}$. Тогда отношение $R$ верно на парах имен, начинающихся с одной и той же буквы, и только на них.
\item $A = \{-10, -9, … , 9, 10\}$ и отношение $ R = \{(a,b)|a^{2} = b^{2}\}$
\item На множестве $A = \{1; 2; 3\}$ задано отношение $R = \{(1; 1); (2; 2); (3; 3); (3; 2); (1; 2); (2; 1)\}$
\end{enumerate}\question Составьте полную таблицу истинности, определите, какие переменные являются фиктивными и проверьте, является ли формула тавтологией:
$(P \rightarrow (Q \rightarrow R)) \rightarrow ((P \rightarrow Q) \rightarrow (P \rightarrow R))$

\end{questions}
\newpage
%%% begin test
\begin{flushright}
\begin{tabular}{p{2.8in} r l}
%\textbf{\class} & \textbf{ФИО:} & \makebox[2.5in]{\hrulefill}\\
\textbf{\class} & \textbf{ФИО:} &Ермаков Никита Альбертович
\\

\textbf{\examdate} &&\\
%\textbf{Time Limit: \timelimit} & Teaching Assistant & \makebox[2in]{\hrulefill}
\end{tabular}\\
\end{flushright}
\rule[1ex]{\textwidth}{.1pt}


\begin{questions}
\question
Найдите и упростите P:
\begin{equation*}
\overline{P} = A \cap B \cup \overline{A} \cap \overline{B} \cup A \cap C \cup \overline{B} \cap C
\end{equation*}
Затем найдите элементы множества P, выраженного через множества:
\begin{equation*}
A = \{0, 3, 4, 9\}; 
B = \{1, 3, 4, 7\};
C = \{0, 1, 2, 4, 7, 8, 9\};
I = \{0, 1, 2, 3, 4, 5, 6, 7, 8, 9\}.
\end{equation*}\question
Упростите следующее выражение с учетом того, что $A\subset B \subset C \subset D \subset U; A \neq \O$
\begin{equation*}
\overline{A} \cap \overline{B} \cup B \cap \overline{C} \cup \overline{C} \cap D
\end{equation*}

Примечание: U — универсум\question
Дано отношение на множестве $\{1, 2, 3, 4, 5\}$ 
\begin{equation*}
aRb \iff a \geq b^2
\end{equation*}
Напишите обоснованный ответ какими свойствами обладает или не обладает отношение и почему:   
\begin{enumerate} [a)]\setcounter{enumi}{0}
\item рефлексивность
\item антирефлексивность
\item симметричность
\item асимметричность
\item антисимметричность
\item транзитивность
\end{enumerate}

Обоснуйте свой ответ по каждому из приведенных ниже вопросов:
\begin{enumerate} [a)]\setcounter{enumi}{0}
    \item Является ли это отношение отношением эквивалентности?
    \item Является ли это отношение функциональным?
    \item Каким из отношений соответствия (одно-многозначным, много-многозначный и т.д.) оно является?
    \item К каким из отношений порядка (полного, частичного и т.д.) можно отнести данное отношение?
\end{enumerate}


\question
Установите, является ли каждое из перечисленных ниже отношений на А ($R \subseteq A \times A$) отношением эквивалентности (обоснование ответа обязательно). Для каждого отношения эквивалентности постройте классы 
эквивалентности и постройте граф отношения:
\begin{enumerate} [a)]\setcounter{enumi}{0}
\item На множестве $A = \{1; 2; 3\}$ задано отношение $R = \{(1; 1); (2; 2); (3; 3); (2; 1); (1; 2); (2; 3); (3; 2); (3; 1); (1; 3)\}$
\item На множестве $A = \{1; 2; 3; 4; 5\}$ задано отношение $R = \{(1; 2); (1; 3); (1; 5); (2; 3); (2; 4); (2; 5); (3; 4); (3; 5); (4; 5)\}$
\item А - множество целых чисел и отношение $R = \{(a,b)|a + b = 0\}$
\end{enumerate}\question Составьте полную таблицу истинности, определите, какие переменные являются фиктивными и проверьте, является ли формула тавтологией:
$(( P \rightarrow Q) \land (Q \rightarrow P)) \rightarrow (P \rightarrow R)$

\end{questions}
\newpage
%%% begin test
\begin{flushright}
\begin{tabular}{p{2.8in} r l}
%\textbf{\class} & \textbf{ФИО:} & \makebox[2.5in]{\hrulefill}\\
\textbf{\class} & \textbf{ФИО:} &Зайцев Кирилл Дмитриевич
\\

\textbf{\examdate} &&\\
%\textbf{Time Limit: \timelimit} & Teaching Assistant & \makebox[2in]{\hrulefill}
\end{tabular}\\
\end{flushright}
\rule[1ex]{\textwidth}{.1pt}


\begin{questions}
\question
Найдите и упростите P:
\begin{equation*}
\overline{P} = A \cap \overline{B} \cup A \cap C \cup B \cap C \cup \overline{A} \cap C
\end{equation*}
Затем найдите элементы множества P, выраженного через множества:
\begin{equation*}
A = \{0, 3, 4, 9\}; 
B = \{1, 3, 4, 7\};
C = \{0, 1, 2, 4, 7, 8, 9\};
I = \{0, 1, 2, 3, 4, 5, 6, 7, 8, 9\}.
\end{equation*}\question
Упростите следующее выражение с учетом того, что $A\subset B \subset C \subset D \subset U; A \neq \O$
\begin{equation*}
\overline{A} \cap \overline{B} \cup B \cap \overline{C} \cup \overline{C} \cap D
\end{equation*}

Примечание: U — универсум\question
Для следующего отношения на множестве $\{1, 2, 3, 4, 5\}$ 
\begin{equation*}
aRb \iff 0 < a-b<2
\end{equation*}
Напишите обоснованный ответ какими свойствами обладает или не обладает отношение и почему:   
\begin{enumerate} [a)]\setcounter{enumi}{0}
\item рефлексивность
\item антирефлексивность
\item симметричность
\item асимметричность
\item антисимметричность
\item транзитивность
\end{enumerate}

Обоснуйте свой ответ по каждому из приведенных ниже вопросов:
\begin{enumerate} [a)]\setcounter{enumi}{0}
    \item Является ли это отношение отношением эквивалентности?
    \item Является ли это отношение функциональным?
    \item Каким из отношений соответствия (одно-многозначным, много-многозначный и т.д.) оно является?
    \item К каким из отношений порядка (полного, частичного и т.д.) можно отнести данное отношение?
\end{enumerate}
\question
Установите, является ли каждое из перечисленных ниже отношений на А ($R \subseteq A \times A$) отношением эквивалентности (обоснование ответа обязательно). Для каждого отношения эквивалентности постройте классы эквивалентности и постройте граф отношения:
\begin{enumerate} [a)]\setcounter{enumi}{0}
\item $F(x)=x^{2}+1$, где $x \in A = [-2, 4]$ и отношение $R = \{(a,b)|F(a) = F(b)\}$
\item А - множество целых чисел и отношение $R = \{(a,b)|a + b = 5\}$
\item На множестве $A = \{1; 2; 3\}$ задано отношение $R = \{(1; 1); (2; 2); (3; 3); (3; 2); (1; 2); (2; 1)\}$

\end{enumerate}\question Составьте полную таблицу истинности, определите, какие переменные являются фиктивными и проверьте, является ли формула тавтологией:
$((P \rightarrow Q) \lor R) \leftrightarrow (P \rightarrow (Q \lor R))$

\end{questions}
\newpage
%%% begin test
\begin{flushright}
\begin{tabular}{p{2.8in} r l}
%\textbf{\class} & \textbf{ФИО:} & \makebox[2.5in]{\hrulefill}\\
\textbf{\class} & \textbf{ФИО:} &Иванов Максим Игоревич
\\

\textbf{\examdate} &&\\
%\textbf{Time Limit: \timelimit} & Teaching Assistant & \makebox[2in]{\hrulefill}
\end{tabular}\\
\end{flushright}
\rule[1ex]{\textwidth}{.1pt}


\begin{questions}
\question
Найдите и упростите P:
\begin{equation*}
\overline{P} = A \cap C \cup \overline{A} \cap \overline{C} \cup \overline{B} \cap C \cup \overline{A} \cap \overline{B}
\end{equation*}
Затем найдите элементы множества P, выраженного через множества:
\begin{equation*}
A = \{0, 3, 4, 9\}; 
B = \{1, 3, 4, 7\};
C = \{0, 1, 2, 4, 7, 8, 9\};
I = \{0, 1, 2, 3, 4, 5, 6, 7, 8, 9\}.
\end{equation*}\question
Упростите следующее выражение с учетом того, что $A\subset B \subset C \subset D \subset U; A \neq \O$
\begin{equation*}
A \cap B  \cap \overline{C} \cup \overline{C} \cap D \cup B \cap C \cap D
\end{equation*}

Примечание: U — универсум\question
Дано отношение на множестве $\{1, 2, 3, 4, 5\}$ 
\begin{equation*}
aRb \iff  \text{НОД}(a,b) =1
\end{equation*}
Напишите обоснованный ответ какими свойствами обладает или не обладает отношение и почему:   
\begin{enumerate} [a)]\setcounter{enumi}{0}
\item рефлексивность
\item антирефлексивность
\item симметричность
\item асимметричность
\item антисимметричность
\item транзитивность
\end{enumerate}

Обоснуйте свой ответ по каждому из приведенных ниже вопросов:
\begin{enumerate} [a)]\setcounter{enumi}{0}
    \item Является ли это отношение отношением эквивалентности?
    \item Является ли это отношение функциональным?
    \item Каким из отношений соответствия (одно-многозначным, много-многозначный и т.д.) оно является?
    \item К каким из отношений порядка (полного, частичного и т.д.) можно отнести данное отношение?
\end{enumerate}


\question
Установите, является ли каждое из перечисленных ниже отношений на А ($R \subseteq A \times A$) отношением эквивалентности (обоснование ответа обязательно). Для каждого отношения эквивалентности постройте классы 
эквивалентности и постройте граф отношения:
\begin{enumerate} [a)]\setcounter{enumi}{0}
\item $A = \{a, b, c, d, p, t\}$ задано отношение $R = \{(a, a), (b, b), (b, c), (b, d), (c, b), (c, c), (c, d), (d, b), (d, c), (d, d), (p,p), (t,t)\}$
\item $A = \{-10, -9, … , 9, 10\}$ и отношение $R = \{(a,b)|a^{3} = b^{3}\}$

\item $F(x)=x^{2}+1$, где $x \in A = [-2, 4]$ и отношение $R = \{(a,b)|F(a) = F(b)\}$
\end{enumerate}\question Составьте полную таблицу истинности, определите, какие переменные являются фиктивными и проверьте, является ли формула тавтологией:
$(P \rightarrow (Q \rightarrow R)) \rightarrow ((P \rightarrow Q) \rightarrow (P \rightarrow R))$

\end{questions}
\newpage
%%% begin test
\begin{flushright}
\begin{tabular}{p{2.8in} r l}
%\textbf{\class} & \textbf{ФИО:} & \makebox[2.5in]{\hrulefill}\\
\textbf{\class} & \textbf{ФИО:} &Капитонов Максим Александрович
\\

\textbf{\examdate} &&\\
%\textbf{Time Limit: \timelimit} & Teaching Assistant & \makebox[2in]{\hrulefill}
\end{tabular}\\
\end{flushright}
\rule[1ex]{\textwidth}{.1pt}


\begin{questions}
\question
Найдите и упростите P:
\begin{equation*}
\overline{P} = A \cap \overline{C} \cup A \cap \overline{B} \cup B \cap \overline{C} \cup A \cap C
\end{equation*}
Затем найдите элементы множества P, выраженного через множества:
\begin{equation*}
A = \{0, 3, 4, 9\}; 
B = \{1, 3, 4, 7\};
C = \{0, 1, 2, 4, 7, 8, 9\};
I = \{0, 1, 2, 3, 4, 5, 6, 7, 8, 9\}.
\end{equation*}\question
Упростите следующее выражение с учетом того, что $A\subset B \subset C \subset D \subset U; A \neq \O$
\begin{equation*}
A \cap C  \cap D \cup B \cap \overline{C} \cap D \cup B \cap C \cap D
\end{equation*}

Примечание: U — универсум\question
Дано отношение на множестве $\{1, 2, 3, 4, 5\}$ 
\begin{equation*}
aRb \iff a \leq b
\end{equation*}
Напишите обоснованный ответ какими свойствами обладает или не обладает отношение и почему:   
\begin{enumerate} [a)]\setcounter{enumi}{0}
\item рефлексивность
\item антирефлексивность
\item симметричность
\item асимметричность
\item антисимметричность
\item транзитивность
\end{enumerate}

Обоснуйте свой ответ по каждому из приведенных ниже вопросов:
\begin{enumerate} [a)]\setcounter{enumi}{0}
    \item Является ли это отношение отношением эквивалентности?
    \item Является ли это отношение функциональным?
    \item Каким из отношений соответствия (одно-многозначным, много-многозначный и т.д.) оно является?
    \item К каким из отношений порядка (полного, частичного и т.д.) можно отнести данное отношение?
\end{enumerate}


\question
Установите, является ли каждое из перечисленных ниже отношений на А ($R \subseteq A \times A$) отношением эквивалентности (обоснование ответа обязательно). Для каждого отношения эквивалентности постройте классы 
эквивалентности и постройте граф отношения:
\begin{enumerate} [a)]\setcounter{enumi}{0}
\item Пусть A – множество имен. $A = \{ $Алексей, Иван, Петр, Александр, Павел, Андрей$ \}$. Тогда отношение $R$ верно на парах имен, начинающихся с одной и той же буквы, и только на них.
\item $A = \{-10, -9, … , 9, 10\}$ и отношение $ R = \{(a,b)|a^{2} = b^{2}\}$
\item На множестве $A = \{1; 2; 3\}$ задано отношение $R = \{(1; 1); (2; 2); (3; 3); (3; 2); (1; 2); (2; 1)\}$
\end{enumerate}\question Составьте полную таблицу истинности, определите, какие переменные являются фиктивными и проверьте, является ли формула тавтологией:
$((P \rightarrow Q) \lor R) \leftrightarrow (P \rightarrow (Q \lor R))$

\end{questions}
\newpage
%%% begin test
\begin{flushright}
\begin{tabular}{p{2.8in} r l}
%\textbf{\class} & \textbf{ФИО:} & \makebox[2.5in]{\hrulefill}\\
\textbf{\class} & \textbf{ФИО:} &Кудашев Искандер Эдуардович
\\

\textbf{\examdate} &&\\
%\textbf{Time Limit: \timelimit} & Teaching Assistant & \makebox[2in]{\hrulefill}
\end{tabular}\\
\end{flushright}
\rule[1ex]{\textwidth}{.1pt}


\begin{questions}
\question
Найдите и упростите P:
\begin{equation*}
\overline{P} = A \cap B \cup \overline{A} \cap \overline{B} \cup A \cap C \cup \overline{B} \cap C
\end{equation*}
Затем найдите элементы множества P, выраженного через множества:
\begin{equation*}
A = \{0, 3, 4, 9\}; 
B = \{1, 3, 4, 7\};
C = \{0, 1, 2, 4, 7, 8, 9\};
I = \{0, 1, 2, 3, 4, 5, 6, 7, 8, 9\}.
\end{equation*}\question
Упростите следующее выражение с учетом того, что $A\subset B \subset C \subset D \subset U; A \neq \O$
\begin{equation*}
\overline{A} \cap \overline{B} \cup B \cap \overline{C} \cup \overline{C} \cap D
\end{equation*}

Примечание: U — универсум\question
Дано отношение на множестве $\{1, 2, 3, 4, 5\}$ 
\begin{equation*}
aRb \iff (a+b) \bmod 2 =0
\end{equation*}
Напишите обоснованный ответ какими свойствами обладает или не обладает отношение и почему:   
\begin{enumerate} [a)]\setcounter{enumi}{0}
\item рефлексивность
\item антирефлексивность
\item симметричность
\item асимметричность
\item антисимметричность
\item транзитивность
\end{enumerate}

Обоснуйте свой ответ по каждому из приведенных ниже вопросов:
\begin{enumerate} [a)]\setcounter{enumi}{0}
    \item Является ли это отношение отношением эквивалентности?
    \item Является ли это отношение функциональным?
    \item Каким из отношений соответствия (одно-многозначным, много-многозначный и т.д.) оно является?
    \item К каким из отношений порядка (полного, частичного и т.д.) можно отнести данное отношение?
\end{enumerate}



\question
Установите, является ли каждое из перечисленных ниже отношений на А ($R \subseteq A \times A$) отношением эквивалентности (обоснование ответа обязательно). Для каждого отношения эквивалентности постройте классы 
эквивалентности и постройте граф отношения:
\begin{enumerate} [a)]\setcounter{enumi}{0}
\item $A = \{a, b, c, d, p, t\}$ задано отношение $R = \{(a, a), (b, b), (b, c), (b, d), (c, b), (c, c), (c, d), (d, b), (d, c), (d, d), (p,p), (t,t)\}$
\item $A = \{-10, -9, … , 9, 10\}$ и отношение $R = \{(a,b)|a^{3} = b^{3}\}$

\item $F(x)=x^{2}+1$, где $x \in A = [-2, 4]$ и отношение $R = \{(a,b)|F(a) = F(b)\}$
\end{enumerate}\question Составьте полную таблицу истинности, определите, какие переменные являются фиктивными и проверьте, является ли формула тавтологией:
$(( P \rightarrow Q) \land (Q \rightarrow P)) \rightarrow (P \rightarrow R)$

\end{questions}
\newpage
%%% begin test
\begin{flushright}
\begin{tabular}{p{2.8in} r l}
%\textbf{\class} & \textbf{ФИО:} & \makebox[2.5in]{\hrulefill}\\
\textbf{\class} & \textbf{ФИО:} &Кузнецова Алика Анатольевна
\\

\textbf{\examdate} &&\\
%\textbf{Time Limit: \timelimit} & Teaching Assistant & \makebox[2in]{\hrulefill}
\end{tabular}\\
\end{flushright}
\rule[1ex]{\textwidth}{.1pt}


\begin{questions}
\question
Найдите и упростите P:
\begin{equation*}
\overline{P} = A \cap \overline{B} \cup A \cap C \cup B \cap C \cup \overline{A} \cap C
\end{equation*}
Затем найдите элементы множества P, выраженного через множества:
\begin{equation*}
A = \{0, 3, 4, 9\}; 
B = \{1, 3, 4, 7\};
C = \{0, 1, 2, 4, 7, 8, 9\};
I = \{0, 1, 2, 3, 4, 5, 6, 7, 8, 9\}.
\end{equation*}\question
Упростите следующее выражение с учетом того, что $A\subset B \subset C \subset D \subset U; A \neq \O$
\begin{equation*}
A \cap B \cup \overline{A} \cap \overline{C} \cup A \cap C \cup \overline{B} \cap \overline{C}
\end{equation*}

Примечание: U — универсум\question
Дано отношение на множестве $\{1, 2, 3, 4, 5\}$ 
\begin{equation*}
aRb \iff b > a
\end{equation*}
Напишите обоснованный ответ какими свойствами обладает или не обладает отношение и почему:   
\begin{enumerate} [a)]\setcounter{enumi}{0}
\item рефлексивность
\item антирефлексивность
\item симметричность
\item асимметричность
\item антисимметричность
\item транзитивность
\end{enumerate}

Обоснуйте свой ответ по каждому из приведенных ниже вопросов:
\begin{enumerate} [a)]\setcounter{enumi}{0}
    \item Является ли это отношение отношением эквивалентности?
    \item Является ли это отношение функциональным?
    \item Каким из отношений соответствия (одно-многозначным, много-многозначный и т.д.) оно является?
    \item К каким из отношений порядка (полного, частичного и т.д.) можно отнести данное отношение?
\end{enumerate}

\question
Установите, является ли каждое из перечисленных ниже отношений на А ($R \subseteq A \times A$) отношением эквивалентности (обоснование ответа обязательно). Для каждого отношения эквивалентности постройте классы 
эквивалентности и постройте граф отношения:
\begin{enumerate} [a)]\setcounter{enumi}{0}
\item А - множество целых чисел и отношение $R = \{(a,b)|a + b = 5\}$
\item Пусть A – множество имен. $A = \{ $Алексей, Иван, Петр, Александр, Павел, Андрей$ \}$. Тогда отношение $R $ верно на парах имен, начинающихся с одной и той же буквы, и только на них.
\item На множестве $A = \{1; 2; 3; 4; 5\}$ задано отношение $R = \{(1; 2); (1; 3); (1; 5); (2; 3); (2; 4); (2; 5); (3; 4); (3; 5); (4; 5)\}$
\end{enumerate}\question Составьте полную таблицу истинности, определите, какие переменные являются фиктивными и проверьте, является ли формула тавтологией:
$(( P \rightarrow Q) \land (Q \rightarrow P)) \rightarrow (P \rightarrow R)$

\end{questions}
\newpage
%%% begin test
\begin{flushright}
\begin{tabular}{p{2.8in} r l}
%\textbf{\class} & \textbf{ФИО:} & \makebox[2.5in]{\hrulefill}\\
\textbf{\class} & \textbf{ФИО:} &Ласточкин Максим Александрович
\\

\textbf{\examdate} &&\\
%\textbf{Time Limit: \timelimit} & Teaching Assistant & \makebox[2in]{\hrulefill}
\end{tabular}\\
\end{flushright}
\rule[1ex]{\textwidth}{.1pt}


\begin{questions}
\question
Найдите и упростите P:
\begin{equation*}
\overline{P} = A \cap \overline{C} \cup A \cap \overline{B} \cup B \cap \overline{C} \cup A \cap C
\end{equation*}
Затем найдите элементы множества P, выраженного через множества:
\begin{equation*}
A = \{0, 3, 4, 9\}; 
B = \{1, 3, 4, 7\};
C = \{0, 1, 2, 4, 7, 8, 9\};
I = \{0, 1, 2, 3, 4, 5, 6, 7, 8, 9\}.
\end{equation*}\question
Упростите следующее выражение с учетом того, что $A\subset B \subset C \subset D \subset U; A \neq \O$
\begin{equation*}
\overline{B} \cap \overline{C} \cap D \cup \overline{A} \cap \overline{C} \cap D \cup \overline{A} \cap B
\end{equation*}

Примечание: U — универсум\question
Дано отношение на множестве $\{1, 2, 3, 4, 5\}$ 
\begin{equation*}
aRb \iff a \leq b
\end{equation*}
Напишите обоснованный ответ какими свойствами обладает или не обладает отношение и почему:   
\begin{enumerate} [a)]\setcounter{enumi}{0}
\item рефлексивность
\item антирефлексивность
\item симметричность
\item асимметричность
\item антисимметричность
\item транзитивность
\end{enumerate}

Обоснуйте свой ответ по каждому из приведенных ниже вопросов:
\begin{enumerate} [a)]\setcounter{enumi}{0}
    \item Является ли это отношение отношением эквивалентности?
    \item Является ли это отношение функциональным?
    \item Каким из отношений соответствия (одно-многозначным, много-многозначный и т.д.) оно является?
    \item К каким из отношений порядка (полного, частичного и т.д.) можно отнести данное отношение?
\end{enumerate}


\question
Установите, является ли каждое из перечисленных ниже отношений на А ($R \subseteq A \times A$) отношением эквивалентности (обоснование ответа обязательно). Для каждого отношения эквивалентности постройте классы 
эквивалентности и постройте граф отношения:
\begin{enumerate} [a)]\setcounter{enumi}{0}
\item Пусть A – множество имен. $A = \{ $Алексей, Иван, Петр, Александр, Павел, Андрей$ \}$. Тогда отношение $R$ верно на парах имен, начинающихся с одной и той же буквы, и только на них.
\item $A = \{-10, -9, … , 9, 10\}$ и отношение $ R = \{(a,b)|a^{2} = b^{2}\}$
\item На множестве $A = \{1; 2; 3\}$ задано отношение $R = \{(1; 1); (2; 2); (3; 3); (3; 2); (1; 2); (2; 1)\}$
\end{enumerate}\question Составьте полную таблицу истинности, определите, какие переменные являются фиктивными и проверьте, является ли формула тавтологией:
$((P \rightarrow Q) \land (R \rightarrow S) \land \neg (Q \lor S)) \rightarrow \neg (P \lor R)$

\end{questions}
\newpage
%%% begin test
\begin{flushright}
\begin{tabular}{p{2.8in} r l}
%\textbf{\class} & \textbf{ФИО:} & \makebox[2.5in]{\hrulefill}\\
\textbf{\class} & \textbf{ФИО:} &Лукинский Даниил Валерьевич
\\

\textbf{\examdate} &&\\
%\textbf{Time Limit: \timelimit} & Teaching Assistant & \makebox[2in]{\hrulefill}
\end{tabular}\\
\end{flushright}
\rule[1ex]{\textwidth}{.1pt}


\begin{questions}
\question
Найдите и упростите P:
\begin{equation*}
\overline{P} = \overline{A} \cap B \cup \overline{A} \cap C \cup A \cap \overline{B} \cup \overline{B} \cap C
\end{equation*}
Затем найдите элементы множества P, выраженного через множества:
\begin{equation*}
A = \{0, 3, 4, 9\}; 
B = \{1, 3, 4, 7\};
C = \{0, 1, 2, 4, 7, 8, 9\};
I = \{0, 1, 2, 3, 4, 5, 6, 7, 8, 9\}.
\end{equation*}\question
Упростите следующее выражение с учетом того, что $A\subset B \subset C \subset D \subset U; A \neq \O$
\begin{equation*}
A \cap C  \cap D \cup B \cap \overline{C} \cap D \cup B \cap C \cap D
\end{equation*}

Примечание: U — универсум\question
Дано отношение на множестве $\{1, 2, 3, 4, 5\}$ 
\begin{equation*}
aRb \iff a \geq b^2
\end{equation*}
Напишите обоснованный ответ какими свойствами обладает или не обладает отношение и почему:   
\begin{enumerate} [a)]\setcounter{enumi}{0}
\item рефлексивность
\item антирефлексивность
\item симметричность
\item асимметричность
\item антисимметричность
\item транзитивность
\end{enumerate}

Обоснуйте свой ответ по каждому из приведенных ниже вопросов:
\begin{enumerate} [a)]\setcounter{enumi}{0}
    \item Является ли это отношение отношением эквивалентности?
    \item Является ли это отношение функциональным?
    \item Каким из отношений соответствия (одно-многозначным, много-многозначный и т.д.) оно является?
    \item К каким из отношений порядка (полного, частичного и т.д.) можно отнести данное отношение?
\end{enumerate}


\question
Установите, является ли каждое из перечисленных ниже отношений на А ($R \subseteq A \times A$) отношением эквивалентности (обоснование ответа обязательно). Для каждого отношения эквивалентности постройте классы 
эквивалентности и постройте граф отношения:
\begin{enumerate} [a)]\setcounter{enumi}{0}
\item $A = \{a, b, c, d, p, t\}$ задано отношение $R = \{(a, a), (b, b), (b, c), (b, d), (c, b), (c, c), (c, d), (d, b), (d, c), (d, d), (p,p), (t,t)\}$
\item $A = \{-10, -9, … , 9, 10\}$ и отношение $R = \{(a,b)|a^{3} = b^{3}\}$

\item $F(x)=x^{2}+1$, где $x \in A = [-2, 4]$ и отношение $R = \{(a,b)|F(a) = F(b)\}$
\end{enumerate}\question Составьте полную таблицу истинности, определите, какие переменные являются фиктивными и проверьте, является ли формула тавтологией:
$((P \rightarrow Q) \land (R \rightarrow S) \land \neg (Q \lor S)) \rightarrow \neg (P \lor R)$

\end{questions}
\newpage
%%% begin test
\begin{flushright}
\begin{tabular}{p{2.8in} r l}
%\textbf{\class} & \textbf{ФИО:} & \makebox[2.5in]{\hrulefill}\\
\textbf{\class} & \textbf{ФИО:} &Мокрищев Николай Павлович
\\

\textbf{\examdate} &&\\
%\textbf{Time Limit: \timelimit} & Teaching Assistant & \makebox[2in]{\hrulefill}
\end{tabular}\\
\end{flushright}
\rule[1ex]{\textwidth}{.1pt}


\begin{questions}
\question
Найдите и упростите P:
\begin{equation*}
\overline{P} = \overline{A} \cap B \cup \overline{A} \cap C \cup A \cap \overline{B} \cup \overline{B} \cap C
\end{equation*}
Затем найдите элементы множества P, выраженного через множества:
\begin{equation*}
A = \{0, 3, 4, 9\}; 
B = \{1, 3, 4, 7\};
C = \{0, 1, 2, 4, 7, 8, 9\};
I = \{0, 1, 2, 3, 4, 5, 6, 7, 8, 9\}.
\end{equation*}\question
Упростите следующее выражение с учетом того, что $A\subset B \subset C \subset D \subset U; A \neq \O$
\begin{equation*}
\overline{B} \cap \overline{C} \cap D \cup \overline{A} \cap \overline{C} \cap D \cup \overline{A} \cap B
\end{equation*}

Примечание: U — универсум\question
Дано отношение на множестве $\{1, 2, 3, 4, 5\}$ 
\begin{equation*}
aRb \iff a \leq b
\end{equation*}
Напишите обоснованный ответ какими свойствами обладает или не обладает отношение и почему:   
\begin{enumerate} [a)]\setcounter{enumi}{0}
\item рефлексивность
\item антирефлексивность
\item симметричность
\item асимметричность
\item антисимметричность
\item транзитивность
\end{enumerate}

Обоснуйте свой ответ по каждому из приведенных ниже вопросов:
\begin{enumerate} [a)]\setcounter{enumi}{0}
    \item Является ли это отношение отношением эквивалентности?
    \item Является ли это отношение функциональным?
    \item Каким из отношений соответствия (одно-многозначным, много-многозначный и т.д.) оно является?
    \item К каким из отношений порядка (полного, частичного и т.д.) можно отнести данное отношение?
\end{enumerate}


\question
Установите, является ли каждое из перечисленных ниже отношений на А ($R \subseteq A \times A$) отношением эквивалентности (обоснование ответа обязательно). Для каждого отношения эквивалентности постройте классы эквивалентности и постройте граф отношения:
\begin{enumerate} [a)]\setcounter{enumi}{0}
\item $F(x)=x^{2}+1$, где $x \in A = [-2, 4]$ и отношение $R = \{(a,b)|F(a) = F(b)\}$
\item А - множество целых чисел и отношение $R = \{(a,b)|a + b = 5\}$
\item На множестве $A = \{1; 2; 3\}$ задано отношение $R = \{(1; 1); (2; 2); (3; 3); (3; 2); (1; 2); (2; 1)\}$

\end{enumerate}\question Составьте полную таблицу истинности, определите, какие переменные являются фиктивными и проверьте, является ли формула тавтологией:
$(P \rightarrow (Q \rightarrow R)) \rightarrow ((P \rightarrow Q) \rightarrow (P \rightarrow R))$

\end{questions}
\newpage
%%% begin test
\begin{flushright}
\begin{tabular}{p{2.8in} r l}
%\textbf{\class} & \textbf{ФИО:} & \makebox[2.5in]{\hrulefill}\\
\textbf{\class} & \textbf{ФИО:} &Мухин Арсений Игоревич
\\

\textbf{\examdate} &&\\
%\textbf{Time Limit: \timelimit} & Teaching Assistant & \makebox[2in]{\hrulefill}
\end{tabular}\\
\end{flushright}
\rule[1ex]{\textwidth}{.1pt}


\begin{questions}
\question
Найдите и упростите P:
\begin{equation*}
\overline{P} = A \cap \overline{B} \cup \overline{B} \cap C \cup \overline{A} \cap \overline{B} \cup \overline{A} \cap C
\end{equation*}
Затем найдите элементы множества P, выраженного через множества:
\begin{equation*}
A = \{0, 3, 4, 9\}; 
B = \{1, 3, 4, 7\};
C = \{0, 1, 2, 4, 7, 8, 9\};
I = \{0, 1, 2, 3, 4, 5, 6, 7, 8, 9\}.
\end{equation*}\question
Упростите следующее выражение с учетом того, что $A\subset B \subset C \subset D \subset U; A \neq \O$
\begin{equation*}
\overline{A} \cap \overline{B} \cup B \cap \overline{C} \cup \overline{C} \cap D
\end{equation*}

Примечание: U — универсум\question
Дано отношение на множестве $\{1, 2, 3, 4, 5\}$ 
\begin{equation*}
aRb \iff (a+b) \bmod 2 =0
\end{equation*}
Напишите обоснованный ответ какими свойствами обладает или не обладает отношение и почему:   
\begin{enumerate} [a)]\setcounter{enumi}{0}
\item рефлексивность
\item антирефлексивность
\item симметричность
\item асимметричность
\item антисимметричность
\item транзитивность
\end{enumerate}

Обоснуйте свой ответ по каждому из приведенных ниже вопросов:
\begin{enumerate} [a)]\setcounter{enumi}{0}
    \item Является ли это отношение отношением эквивалентности?
    \item Является ли это отношение функциональным?
    \item Каким из отношений соответствия (одно-многозначным, много-многозначный и т.д.) оно является?
    \item К каким из отношений порядка (полного, частичного и т.д.) можно отнести данное отношение?
\end{enumerate}



\question
Установите, является ли каждое из перечисленных ниже отношений на А ($R \subseteq A \times A$) отношением эквивалентности (обоснование ответа обязательно). Для каждого отношения эквивалентности постройте классы 
эквивалентности и постройте граф отношения:
\begin{enumerate} [a)]\setcounter{enumi}{0}
\item $A = \{-10, -9, … , 9, 10\}$ и отношение $R = \{(a,b)|a^{2} = b^{2}\}$
\item $A = \{a, b, c, d, p, t\}$ задано отношение $R = \{(a, a), (b, b), (b, c), (b, d), (c, b), (c, c), (c, d), (d, b), (d, c), (d, d), (p,p), (t,t)\}$
\item Пусть A – множество имен. $A = \{ $Алексей, Иван, Петр, Александр, Павел, Андрей$ \}$. Тогда отношение $R$ верно на парах имен, начинающихся с одной и той же буквы, и только на них.
\end{enumerate}\question Составьте полную таблицу истинности, определите, какие переменные являются фиктивными и проверьте, является ли формула тавтологией:
$(P \rightarrow (Q \rightarrow R)) \rightarrow ((P \rightarrow Q) \rightarrow (P \rightarrow R))$

\end{questions}
\newpage
%%% begin test
\begin{flushright}
\begin{tabular}{p{2.8in} r l}
%\textbf{\class} & \textbf{ФИО:} & \makebox[2.5in]{\hrulefill}\\
\textbf{\class} & \textbf{ФИО:} &Овчаров Никита Андреевич
\\

\textbf{\examdate} &&\\
%\textbf{Time Limit: \timelimit} & Teaching Assistant & \makebox[2in]{\hrulefill}
\end{tabular}\\
\end{flushright}
\rule[1ex]{\textwidth}{.1pt}


\begin{questions}
\question
Найдите и упростите P:
\begin{equation*}
\overline{P} = A \cap C \cup \overline{A} \cap \overline{C} \cup \overline{B} \cap C \cup \overline{A} \cap \overline{B}
\end{equation*}
Затем найдите элементы множества P, выраженного через множества:
\begin{equation*}
A = \{0, 3, 4, 9\}; 
B = \{1, 3, 4, 7\};
C = \{0, 1, 2, 4, 7, 8, 9\};
I = \{0, 1, 2, 3, 4, 5, 6, 7, 8, 9\}.
\end{equation*}\question
Упростите следующее выражение с учетом того, что $A\subset B \subset C \subset D \subset U; A \neq \O$
\begin{equation*}
\overline{B} \cap \overline{C} \cap D \cup \overline{A} \cap \overline{C} \cap D \cup \overline{A} \cap B
\end{equation*}

Примечание: U — универсум\question
Дано отношение на множестве $\{1, 2, 3, 4, 5\}$ 
\begin{equation*}
aRb \iff b > a
\end{equation*}
Напишите обоснованный ответ какими свойствами обладает или не обладает отношение и почему:   
\begin{enumerate} [a)]\setcounter{enumi}{0}
\item рефлексивность
\item антирефлексивность
\item симметричность
\item асимметричность
\item антисимметричность
\item транзитивность
\end{enumerate}

Обоснуйте свой ответ по каждому из приведенных ниже вопросов:
\begin{enumerate} [a)]\setcounter{enumi}{0}
    \item Является ли это отношение отношением эквивалентности?
    \item Является ли это отношение функциональным?
    \item Каким из отношений соответствия (одно-многозначным, много-многозначный и т.д.) оно является?
    \item К каким из отношений порядка (полного, частичного и т.д.) можно отнести данное отношение?
\end{enumerate}

\question
Установите, является ли каждое из перечисленных ниже отношений на А ($R \subseteq A \times A$) отношением эквивалентности (обоснование ответа обязательно). Для каждого отношения эквивалентности постройте классы 
эквивалентности и постройте граф отношения:
\begin{enumerate} [a)]\setcounter{enumi}{0}
\item $A = \{-10, -9, … , 9, 10\}$ и отношение $R = \{(a,b)|a^{2} = b^{2}\}$
\item $A = \{a, b, c, d, p, t\}$ задано отношение $R = \{(a, a), (b, b), (b, c), (b, d), (c, b), (c, c), (c, d), (d, b), (d, c), (d, d), (p,p), (t,t)\}$
\item Пусть A – множество имен. $A = \{ $Алексей, Иван, Петр, Александр, Павел, Андрей$ \}$. Тогда отношение $R$ верно на парах имен, начинающихся с одной и той же буквы, и только на них.
\end{enumerate}\question Составьте полную таблицу истинности, определите, какие переменные являются фиктивными и проверьте, является ли формула тавтологией:

$(P \rightarrow (Q \land R)) \leftrightarrow ((P \rightarrow Q) \land (P \rightarrow R))$

\end{questions}
\newpage
%%% begin test
\begin{flushright}
\begin{tabular}{p{2.8in} r l}
%\textbf{\class} & \textbf{ФИО:} & \makebox[2.5in]{\hrulefill}\\
\textbf{\class} & \textbf{ФИО:} &Парамонов Арсений Сергеевич
\\

\textbf{\examdate} &&\\
%\textbf{Time Limit: \timelimit} & Teaching Assistant & \makebox[2in]{\hrulefill}
\end{tabular}\\
\end{flushright}
\rule[1ex]{\textwidth}{.1pt}


\begin{questions}
\question
Найдите и упростите P:
\begin{equation*}
\overline{P} = B \cap \overline{C} \cup A \cap B \cup \overline{A} \cap C \cup \overline{A} \cap B
\end{equation*}
Затем найдите элементы множества P, выраженного через множества:
\begin{equation*}
A = \{0, 3, 4, 9\}; 
B = \{1, 3, 4, 7\};
C = \{0, 1, 2, 4, 7, 8, 9\};
I = \{0, 1, 2, 3, 4, 5, 6, 7, 8, 9\}.
\end{equation*}\question
Упростите следующее выражение с учетом того, что $A\subset B \subset C \subset D \subset U; A \neq \O$
\begin{equation*}
\overline{A} \cap \overline{C} \cap D \cup \overline{B} \cap \overline{C} \cap D \cup A \cap B
\end{equation*}

Примечание: U — универсум\question
Дано отношение на множестве $\{1, 2, 3, 4, 5\}$ 
\begin{equation*}
aRb \iff b > a
\end{equation*}
Напишите обоснованный ответ какими свойствами обладает или не обладает отношение и почему:   
\begin{enumerate} [a)]\setcounter{enumi}{0}
\item рефлексивность
\item антирефлексивность
\item симметричность
\item асимметричность
\item антисимметричность
\item транзитивность
\end{enumerate}

Обоснуйте свой ответ по каждому из приведенных ниже вопросов:
\begin{enumerate} [a)]\setcounter{enumi}{0}
    \item Является ли это отношение отношением эквивалентности?
    \item Является ли это отношение функциональным?
    \item Каким из отношений соответствия (одно-многозначным, много-многозначный и т.д.) оно является?
    \item К каким из отношений порядка (полного, частичного и т.д.) можно отнести данное отношение?
\end{enumerate}

\question
Установите, является ли каждое из перечисленных ниже отношений на А ($R \subseteq A \times A$) отношением эквивалентности (обоснование ответа обязательно). Для каждого отношения эквивалентности постройте классы 
эквивалентности и постройте граф отношения:
\begin{enumerate} [a)]\setcounter{enumi}{0}
\item $A = \{-10, -9, … , 9, 10\}$ и отношение $R = \{(a,b)|a^{2} = b^{2}\}$
\item $A = \{a, b, c, d, p, t\}$ задано отношение $R = \{(a, a), (b, b), (b, c), (b, d), (c, b), (c, c), (c, d), (d, b), (d, c), (d, d), (p,p), (t,t)\}$
\item Пусть A – множество имен. $A = \{ $Алексей, Иван, Петр, Александр, Павел, Андрей$ \}$. Тогда отношение $R$ верно на парах имен, начинающихся с одной и той же буквы, и только на них.
\end{enumerate}\question Составьте полную таблицу истинности, определите, какие переменные являются фиктивными и проверьте, является ли формула тавтологией:
$((P \rightarrow Q) \lor R) \leftrightarrow (P \rightarrow (Q \lor R))$

\end{questions}
\newpage
%%% begin test
\begin{flushright}
\begin{tabular}{p{2.8in} r l}
%\textbf{\class} & \textbf{ФИО:} & \makebox[2.5in]{\hrulefill}\\
\textbf{\class} & \textbf{ФИО:} &Саркисов Никита Дмитриевич
\\

\textbf{\examdate} &&\\
%\textbf{Time Limit: \timelimit} & Teaching Assistant & \makebox[2in]{\hrulefill}
\end{tabular}\\
\end{flushright}
\rule[1ex]{\textwidth}{.1pt}


\begin{questions}
\question
Найдите и упростите P:
\begin{equation*}
\overline{P} = A \cap \overline{C} \cup A \cap \overline{B} \cup B \cap \overline{C} \cup A \cap C
\end{equation*}
Затем найдите элементы множества P, выраженного через множества:
\begin{equation*}
A = \{0, 3, 4, 9\}; 
B = \{1, 3, 4, 7\};
C = \{0, 1, 2, 4, 7, 8, 9\};
I = \{0, 1, 2, 3, 4, 5, 6, 7, 8, 9\}.
\end{equation*}\question
Упростите следующее выражение с учетом того, что $A\subset B \subset C \subset D \subset U; A \neq \O$
\begin{equation*}
\overline{B} \cap \overline{C} \cap D \cup \overline{A} \cap \overline{C} \cap D \cup \overline{A} \cap B
\end{equation*}

Примечание: U — универсум\question
Дано отношение на множестве $\{1, 2, 3, 4, 5\}$ 
\begin{equation*}
aRb \iff a \leq b
\end{equation*}
Напишите обоснованный ответ какими свойствами обладает или не обладает отношение и почему:   
\begin{enumerate} [a)]\setcounter{enumi}{0}
\item рефлексивность
\item антирефлексивность
\item симметричность
\item асимметричность
\item антисимметричность
\item транзитивность
\end{enumerate}

Обоснуйте свой ответ по каждому из приведенных ниже вопросов:
\begin{enumerate} [a)]\setcounter{enumi}{0}
    \item Является ли это отношение отношением эквивалентности?
    \item Является ли это отношение функциональным?
    \item Каким из отношений соответствия (одно-многозначным, много-многозначный и т.д.) оно является?
    \item К каким из отношений порядка (полного, частичного и т.д.) можно отнести данное отношение?
\end{enumerate}


\question
Установите, является ли каждое из перечисленных ниже отношений на А ($R \subseteq A \times A$) отношением эквивалентности (обоснование ответа обязательно). Для каждого отношения эквивалентности постройте классы 
эквивалентности и постройте граф отношения:
\begin{enumerate} [a)]\setcounter{enumi}{0}
\item $A = \{a, b, c, d, p, t\}$ задано отношение $R = \{(a, a), (b, b), (b, c), (b, d), (c, b), (c, c), (c, d), (d, b), (d, c), (d, d), (p,p), (t,t)\}$
\item $A = \{-10, -9, … , 9, 10\}$ и отношение $R = \{(a,b)|a^{3} = b^{3}\}$

\item $F(x)=x^{2}+1$, где $x \in A = [-2, 4]$ и отношение $R = \{(a,b)|F(a) = F(b)\}$
\end{enumerate}\question Составьте полную таблицу истинности, определите, какие переменные являются фиктивными и проверьте, является ли формула тавтологией:
$(( P \rightarrow Q) \land (Q \rightarrow P)) \rightarrow (P \rightarrow R)$

\end{questions}
\newpage
%%% begin test
\begin{flushright}
\begin{tabular}{p{2.8in} r l}
%\textbf{\class} & \textbf{ФИО:} & \makebox[2.5in]{\hrulefill}\\
\textbf{\class} & \textbf{ФИО:} &Суетин Иван Михайлович
\\

\textbf{\examdate} &&\\
%\textbf{Time Limit: \timelimit} & Teaching Assistant & \makebox[2in]{\hrulefill}
\end{tabular}\\
\end{flushright}
\rule[1ex]{\textwidth}{.1pt}


\begin{questions}
\question
Найдите и упростите P:
\begin{equation*}
\overline{P} = A \cap B \cup \overline{A} \cap \overline{B} \cup A \cap C \cup \overline{B} \cap C
\end{equation*}
Затем найдите элементы множества P, выраженного через множества:
\begin{equation*}
A = \{0, 3, 4, 9\}; 
B = \{1, 3, 4, 7\};
C = \{0, 1, 2, 4, 7, 8, 9\};
I = \{0, 1, 2, 3, 4, 5, 6, 7, 8, 9\}.
\end{equation*}\question
Упростите следующее выражение с учетом того, что $A\subset B \subset C \subset D \subset U; A \neq \O$
\begin{equation*}
A \cap  \overline{C} \cup B \cap \overline{D} \cup  \overline{A} \cap C \cap  \overline{D}
\end{equation*}

Примечание: U — универсум\question
Дано отношение на множестве $\{1, 2, 3, 4, 5\}$ 
\begin{equation*}
aRb \iff  \text{НОД}(a,b) =1
\end{equation*}
Напишите обоснованный ответ какими свойствами обладает или не обладает отношение и почему:   
\begin{enumerate} [a)]\setcounter{enumi}{0}
\item рефлексивность
\item антирефлексивность
\item симметричность
\item асимметричность
\item антисимметричность
\item транзитивность
\end{enumerate}

Обоснуйте свой ответ по каждому из приведенных ниже вопросов:
\begin{enumerate} [a)]\setcounter{enumi}{0}
    \item Является ли это отношение отношением эквивалентности?
    \item Является ли это отношение функциональным?
    \item Каким из отношений соответствия (одно-многозначным, много-многозначный и т.д.) оно является?
    \item К каким из отношений порядка (полного, частичного и т.д.) можно отнести данное отношение?
\end{enumerate}


\question
Установите, является ли каждое из перечисленных ниже отношений на А ($R \subseteq A \times A$) отношением эквивалентности (обоснование ответа обязательно). Для каждого отношения эквивалентности постройте классы 
эквивалентности и постройте граф отношения:
\begin{enumerate} [a)]\setcounter{enumi}{0}
\item $A = \{-10, -9, … , 9, 10\}$ и отношение $R = \{(a,b)|a^{2} = b^{2}\}$
\item $A = \{a, b, c, d, p, t\}$ задано отношение $R = \{(a, a), (b, b), (b, c), (b, d), (c, b), (c, c), (c, d), (d, b), (d, c), (d, d), (p,p), (t,t)\}$
\item Пусть A – множество имен. $A = \{ $Алексей, Иван, Петр, Александр, Павел, Андрей$ \}$. Тогда отношение $R$ верно на парах имен, начинающихся с одной и той же буквы, и только на них.
\end{enumerate}\question Составьте полную таблицу истинности, определите, какие переменные являются фиктивными и проверьте, является ли формула тавтологией:
$(( P \land \neg Q) \rightarrow (R \land \neg R)) \rightarrow (P \rightarrow Q)$

\end{questions}
\newpage
%%% begin test
\begin{flushright}
\begin{tabular}{p{2.8in} r l}
%\textbf{\class} & \textbf{ФИО:} & \makebox[2.5in]{\hrulefill}\\
\textbf{\class} & \textbf{ФИО:} &Сырма Тимур Эрсин
\\

\textbf{\examdate} &&\\
%\textbf{Time Limit: \timelimit} & Teaching Assistant & \makebox[2in]{\hrulefill}
\end{tabular}\\
\end{flushright}
\rule[1ex]{\textwidth}{.1pt}


\begin{questions}
\question
Найдите и упростите P:
\begin{equation*}
\overline{P} = \overline{A} \cap B \cup \overline{A} \cap C \cup A \cap \overline{B} \cup \overline{B} \cap C
\end{equation*}
Затем найдите элементы множества P, выраженного через множества:
\begin{equation*}
A = \{0, 3, 4, 9\}; 
B = \{1, 3, 4, 7\};
C = \{0, 1, 2, 4, 7, 8, 9\};
I = \{0, 1, 2, 3, 4, 5, 6, 7, 8, 9\}.
\end{equation*}\question
Упростите следующее выражение с учетом того, что $A\subset B \subset C \subset D \subset U; A \neq \O$
\begin{equation*}
A \cap C  \cap D \cup B \cap \overline{C} \cap D \cup B \cap C \cap D
\end{equation*}

Примечание: U — универсум\question
Дано отношение на множестве $\{1, 2, 3, 4, 5\}$ 
\begin{equation*}
aRb \iff |a-b| = 1
\end{equation*}
Напишите обоснованный ответ какими свойствами обладает или не обладает отношение и почему:   
\begin{enumerate} [a)]\setcounter{enumi}{0}
\item рефлексивность
\item антирефлексивность
\item симметричность
\item асимметричность
\item антисимметричность
\item транзитивность
\end{enumerate}

Обоснуйте свой ответ по каждому из приведенных ниже вопросов:
\begin{enumerate} [a)]\setcounter{enumi}{0}
    \item Является ли это отношение отношением эквивалентности?
    \item Является ли это отношение функциональным?
    \item Каким из отношений соответствия (одно-многозначным, много-многозначный и т.д.) оно является?
    \item К каким из отношений порядка (полного, частичного и т.д.) можно отнести данное отношение?
\end{enumerate}

\question
Установите, является ли каждое из перечисленных ниже отношений на А ($R \subseteq A \times A$) отношением эквивалентности (обоснование ответа обязательно). Для каждого отношения эквивалентности постройте классы 
эквивалентности и постройте граф отношения:
\begin{enumerate} [a)]\setcounter{enumi}{0}
\item А - множество целых чисел и отношение $R = \{(a,b)|a + b = 5\}$
\item Пусть A – множество имен. $A = \{ $Алексей, Иван, Петр, Александр, Павел, Андрей$ \}$. Тогда отношение $R $ верно на парах имен, начинающихся с одной и той же буквы, и только на них.
\item На множестве $A = \{1; 2; 3; 4; 5\}$ задано отношение $R = \{(1; 2); (1; 3); (1; 5); (2; 3); (2; 4); (2; 5); (3; 4); (3; 5); (4; 5)\}$
\end{enumerate}\question Составьте полную таблицу истинности, определите, какие переменные являются фиктивными и проверьте, является ли формула тавтологией:
$((P \rightarrow Q) \land (R \rightarrow S) \land \neg (Q \lor S)) \rightarrow \neg (P \lor R)$

\end{questions}
\newpage
%%% begin test
\begin{flushright}
\begin{tabular}{p{2.8in} r l}
%\textbf{\class} & \textbf{ФИО:} & \makebox[2.5in]{\hrulefill}\\
\textbf{\class} & \textbf{ФИО:} &Сыромятников Данил Максимович
\\

\textbf{\examdate} &&\\
%\textbf{Time Limit: \timelimit} & Teaching Assistant & \makebox[2in]{\hrulefill}
\end{tabular}\\
\end{flushright}
\rule[1ex]{\textwidth}{.1pt}


\begin{questions}
\question
Найдите и упростите P:
\begin{equation*}
\overline{P} = A \cap B \cup \overline{A} \cap \overline{B} \cup A \cap C \cup \overline{B} \cap C
\end{equation*}
Затем найдите элементы множества P, выраженного через множества:
\begin{equation*}
A = \{0, 3, 4, 9\}; 
B = \{1, 3, 4, 7\};
C = \{0, 1, 2, 4, 7, 8, 9\};
I = \{0, 1, 2, 3, 4, 5, 6, 7, 8, 9\}.
\end{equation*}\question
Упростите следующее выражение с учетом того, что $A\subset B \subset C \subset D \subset U; A \neq \O$
\begin{equation*}
A \cap B \cup \overline{A} \cap \overline{C} \cup A \cap C \cup \overline{B} \cap \overline{C}
\end{equation*}

Примечание: U — универсум\question
Дано отношение на множестве $\{1, 2, 3, 4, 5\}$ 
\begin{equation*}
aRb \iff |a-b| = 1
\end{equation*}
Напишите обоснованный ответ какими свойствами обладает или не обладает отношение и почему:   
\begin{enumerate} [a)]\setcounter{enumi}{0}
\item рефлексивность
\item антирефлексивность
\item симметричность
\item асимметричность
\item антисимметричность
\item транзитивность
\end{enumerate}

Обоснуйте свой ответ по каждому из приведенных ниже вопросов:
\begin{enumerate} [a)]\setcounter{enumi}{0}
    \item Является ли это отношение отношением эквивалентности?
    \item Является ли это отношение функциональным?
    \item Каким из отношений соответствия (одно-многозначным, много-многозначный и т.д.) оно является?
    \item К каким из отношений порядка (полного, частичного и т.д.) можно отнести данное отношение?
\end{enumerate}

\question
Установите, является ли каждое из перечисленных ниже отношений на А ($R \subseteq A \times A$) отношением эквивалентности (обоснование ответа обязательно). Для каждого отношения эквивалентности постройте классы 
эквивалентности и постройте граф отношения:
\begin{enumerate} [a)]\setcounter{enumi}{0}
\item $A = \{-10, -9, … , 9, 10\}$ и отношение $R = \{(a,b)|a^{2} = b^{2}\}$
\item $A = \{a, b, c, d, p, t\}$ задано отношение $R = \{(a, a), (b, b), (b, c), (b, d), (c, b), (c, c), (c, d), (d, b), (d, c), (d, d), (p,p), (t,t)\}$
\item Пусть A – множество имен. $A = \{ $Алексей, Иван, Петр, Александр, Павел, Андрей$ \}$. Тогда отношение $R$ верно на парах имен, начинающихся с одной и той же буквы, и только на них.
\end{enumerate}\question Составьте полную таблицу истинности, определите, какие переменные являются фиктивными и проверьте, является ли формула тавтологией:
$((P \rightarrow Q) \lor R) \leftrightarrow (P \rightarrow (Q \lor R))$

\end{questions}
\newpage
%%% begin test
\begin{flushright}
\begin{tabular}{p{2.8in} r l}
%\textbf{\class} & \textbf{ФИО:} & \makebox[2.5in]{\hrulefill}\\
\textbf{\class} & \textbf{ФИО:} &Ушакова Алёна Игоревна
\\

\textbf{\examdate} &&\\
%\textbf{Time Limit: \timelimit} & Teaching Assistant & \makebox[2in]{\hrulefill}
\end{tabular}\\
\end{flushright}
\rule[1ex]{\textwidth}{.1pt}


\begin{questions}
\question
Найдите и упростите P:
\begin{equation*}
\overline{P} = A \cap \overline{B} \cup A \cap C \cup B \cap C \cup \overline{A} \cap C
\end{equation*}
Затем найдите элементы множества P, выраженного через множества:
\begin{equation*}
A = \{0, 3, 4, 9\}; 
B = \{1, 3, 4, 7\};
C = \{0, 1, 2, 4, 7, 8, 9\};
I = \{0, 1, 2, 3, 4, 5, 6, 7, 8, 9\}.
\end{equation*}\question
Упростите следующее выражение с учетом того, что $A\subset B \subset C \subset D \subset U; A \neq \O$
\begin{equation*}
A \cap B  \cap \overline{C} \cup \overline{C} \cap D \cup B \cap C \cap D
\end{equation*}

Примечание: U — универсум\question
Дано отношение на множестве $\{1, 2, 3, 4, 5\}$ 
\begin{equation*}
aRb \iff a \leq b
\end{equation*}
Напишите обоснованный ответ какими свойствами обладает или не обладает отношение и почему:   
\begin{enumerate} [a)]\setcounter{enumi}{0}
\item рефлексивность
\item антирефлексивность
\item симметричность
\item асимметричность
\item антисимметричность
\item транзитивность
\end{enumerate}

Обоснуйте свой ответ по каждому из приведенных ниже вопросов:
\begin{enumerate} [a)]\setcounter{enumi}{0}
    \item Является ли это отношение отношением эквивалентности?
    \item Является ли это отношение функциональным?
    \item Каким из отношений соответствия (одно-многозначным, много-многозначный и т.д.) оно является?
    \item К каким из отношений порядка (полного, частичного и т.д.) можно отнести данное отношение?
\end{enumerate}


\question
Установите, является ли каждое из перечисленных ниже отношений на А ($R \subseteq A \times A$) отношением эквивалентности (обоснование ответа обязательно). Для каждого отношения эквивалентности постройте классы эквивалентности и постройте граф отношения:
\begin{enumerate} [a)]\setcounter{enumi}{0}
\item $F(x)=x^{2}+1$, где $x \in A = [-2, 4]$ и отношение $R = \{(a,b)|F(a) = F(b)\}$
\item А - множество целых чисел и отношение $R = \{(a,b)|a + b = 5\}$
\item На множестве $A = \{1; 2; 3\}$ задано отношение $R = \{(1; 1); (2; 2); (3; 3); (3; 2); (1; 2); (2; 1)\}$

\end{enumerate}\question Составьте полную таблицу истинности, определите, какие переменные являются фиктивными и проверьте, является ли формула тавтологией:

$(P \rightarrow (Q \land R)) \leftrightarrow ((P \rightarrow Q) \land (P \rightarrow R))$

\end{questions}
\newpage
%%% begin test
\begin{flushright}
\begin{tabular}{p{2.8in} r l}
%\textbf{\class} & \textbf{ФИО:} & \makebox[2.5in]{\hrulefill}\\
\textbf{\class} & \textbf{ФИО:} &Хан Андрей Андреевич
\\

\textbf{\examdate} &&\\
%\textbf{Time Limit: \timelimit} & Teaching Assistant & \makebox[2in]{\hrulefill}
\end{tabular}\\
\end{flushright}
\rule[1ex]{\textwidth}{.1pt}


\begin{questions}
\question
Найдите и упростите P:
\begin{equation*}
\overline{P} = A \cap \overline{B} \cup \overline{B} \cap C \cup \overline{A} \cap \overline{B} \cup \overline{A} \cap C
\end{equation*}
Затем найдите элементы множества P, выраженного через множества:
\begin{equation*}
A = \{0, 3, 4, 9\}; 
B = \{1, 3, 4, 7\};
C = \{0, 1, 2, 4, 7, 8, 9\};
I = \{0, 1, 2, 3, 4, 5, 6, 7, 8, 9\}.
\end{equation*}\question
Упростите следующее выражение с учетом того, что $A\subset B \subset C \subset D \subset U; A \neq \O$
\begin{equation*}
A \cap C  \cap D \cup B \cap \overline{C} \cap D \cup B \cap C \cap D
\end{equation*}

Примечание: U — универсум\question
Дано отношение на множестве $\{1, 2, 3, 4, 5\}$ 
\begin{equation*}
aRb \iff a \geq b^2
\end{equation*}
Напишите обоснованный ответ какими свойствами обладает или не обладает отношение и почему:   
\begin{enumerate} [a)]\setcounter{enumi}{0}
\item рефлексивность
\item антирефлексивность
\item симметричность
\item асимметричность
\item антисимметричность
\item транзитивность
\end{enumerate}

Обоснуйте свой ответ по каждому из приведенных ниже вопросов:
\begin{enumerate} [a)]\setcounter{enumi}{0}
    \item Является ли это отношение отношением эквивалентности?
    \item Является ли это отношение функциональным?
    \item Каким из отношений соответствия (одно-многозначным, много-многозначный и т.д.) оно является?
    \item К каким из отношений порядка (полного, частичного и т.д.) можно отнести данное отношение?
\end{enumerate}


\question
Установите, является ли каждое из перечисленных ниже отношений на А ($R \subseteq A \times A$) отношением эквивалентности (обоснование ответа обязательно). Для каждого отношения эквивалентности постройте классы 
эквивалентности и постройте граф отношения:
\begin{enumerate} [a)]\setcounter{enumi}{0}
\item Пусть A – множество имен. $A = \{ $Алексей, Иван, Петр, Александр, Павел, Андрей$ \}$. Тогда отношение $R$ верно на парах имен, начинающихся с одной и той же буквы, и только на них.
\item $A = \{-10, -9, … , 9, 10\}$ и отношение $ R = \{(a,b)|a^{2} = b^{2}\}$
\item На множестве $A = \{1; 2; 3\}$ задано отношение $R = \{(1; 1); (2; 2); (3; 3); (3; 2); (1; 2); (2; 1)\}$
\end{enumerate}\question Составьте полную таблицу истинности, определите, какие переменные являются фиктивными и проверьте, является ли формула тавтологией:
$(( P \land \neg Q) \rightarrow (R \land \neg R)) \rightarrow (P \rightarrow Q)$

\end{questions}
\newpage
%%% begin test
\begin{flushright}
\begin{tabular}{p{2.8in} r l}
%\textbf{\class} & \textbf{ФИО:} & \makebox[2.5in]{\hrulefill}\\
\textbf{\class} & \textbf{ФИО:} &Хафизов Рузаль Ильгамович
\\

\textbf{\examdate} &&\\
%\textbf{Time Limit: \timelimit} & Teaching Assistant & \makebox[2in]{\hrulefill}
\end{tabular}\\
\end{flushright}
\rule[1ex]{\textwidth}{.1pt}


\begin{questions}
\question
Найдите и упростите P:
\begin{equation*}
\overline{P} = A \cap \overline{B} \cup A \cap C \cup B \cap C \cup \overline{A} \cap C
\end{equation*}
Затем найдите элементы множества P, выраженного через множества:
\begin{equation*}
A = \{0, 3, 4, 9\}; 
B = \{1, 3, 4, 7\};
C = \{0, 1, 2, 4, 7, 8, 9\};
I = \{0, 1, 2, 3, 4, 5, 6, 7, 8, 9\}.
\end{equation*}\question
Упростите следующее выражение с учетом того, что $A\subset B \subset C \subset D \subset U; A \neq \O$
\begin{equation*}
\overline{A} \cap \overline{B} \cup B \cap \overline{C} \cup \overline{C} \cap D
\end{equation*}

Примечание: U — универсум\question
Дано отношение на множестве $\{1, 2, 3, 4, 5\}$ 
\begin{equation*}
aRb \iff a \geq b^2
\end{equation*}
Напишите обоснованный ответ какими свойствами обладает или не обладает отношение и почему:   
\begin{enumerate} [a)]\setcounter{enumi}{0}
\item рефлексивность
\item антирефлексивность
\item симметричность
\item асимметричность
\item антисимметричность
\item транзитивность
\end{enumerate}

Обоснуйте свой ответ по каждому из приведенных ниже вопросов:
\begin{enumerate} [a)]\setcounter{enumi}{0}
    \item Является ли это отношение отношением эквивалентности?
    \item Является ли это отношение функциональным?
    \item Каким из отношений соответствия (одно-многозначным, много-многозначный и т.д.) оно является?
    \item К каким из отношений порядка (полного, частичного и т.д.) можно отнести данное отношение?
\end{enumerate}


\question
Установите, является ли каждое из перечисленных ниже отношений на А ($R \subseteq A \times A$) отношением эквивалентности (обоснование ответа обязательно). Для каждого отношения эквивалентности постройте классы 
эквивалентности и постройте граф отношения:
\begin{enumerate} [a)]\setcounter{enumi}{0}
\item А - множество целых чисел и отношение $R = \{(a,b)|a + b = 5\}$
\item Пусть A – множество имен. $A = \{ $Алексей, Иван, Петр, Александр, Павел, Андрей$ \}$. Тогда отношение $R $ верно на парах имен, начинающихся с одной и той же буквы, и только на них.
\item На множестве $A = \{1; 2; 3; 4; 5\}$ задано отношение $R = \{(1; 2); (1; 3); (1; 5); (2; 3); (2; 4); (2; 5); (3; 4); (3; 5); (4; 5)\}$
\end{enumerate}\question Составьте полную таблицу истинности, определите, какие переменные являются фиктивными и проверьте, является ли формула тавтологией:
$((P \rightarrow Q) \lor R) \leftrightarrow (P \rightarrow (Q \lor R))$

\end{questions}
\newpage
%%% begin test
\begin{flushright}
\begin{tabular}{p{2.8in} r l}
%\textbf{\class} & \textbf{ФИО:} & \makebox[2.5in]{\hrulefill}\\
\textbf{\class} & \textbf{ФИО:} &Шаров Дмитрий Олегович
\\

\textbf{\examdate} &&\\
%\textbf{Time Limit: \timelimit} & Teaching Assistant & \makebox[2in]{\hrulefill}
\end{tabular}\\
\end{flushright}
\rule[1ex]{\textwidth}{.1pt}


\begin{questions}
\question
Найдите и упростите P:
\begin{equation*}
\overline{P} = A \cap C \cup \overline{A} \cap \overline{C} \cup \overline{B} \cap C \cup \overline{A} \cap \overline{B}
\end{equation*}
Затем найдите элементы множества P, выраженного через множества:
\begin{equation*}
A = \{0, 3, 4, 9\}; 
B = \{1, 3, 4, 7\};
C = \{0, 1, 2, 4, 7, 8, 9\};
I = \{0, 1, 2, 3, 4, 5, 6, 7, 8, 9\}.
\end{equation*}\question
Упростите следующее выражение с учетом того, что $A\subset B \subset C \subset D \subset U; A \neq \O$
\begin{equation*}
A \cap B \cup \overline{A} \cap \overline{C} \cup A \cap C \cup \overline{B} \cap \overline{C}
\end{equation*}

Примечание: U — универсум\question
Дано отношение на множестве $\{1, 2, 3, 4, 5\}$ 
\begin{equation*}
aRb \iff a \leq b
\end{equation*}
Напишите обоснованный ответ какими свойствами обладает или не обладает отношение и почему:   
\begin{enumerate} [a)]\setcounter{enumi}{0}
\item рефлексивность
\item антирефлексивность
\item симметричность
\item асимметричность
\item антисимметричность
\item транзитивность
\end{enumerate}

Обоснуйте свой ответ по каждому из приведенных ниже вопросов:
\begin{enumerate} [a)]\setcounter{enumi}{0}
    \item Является ли это отношение отношением эквивалентности?
    \item Является ли это отношение функциональным?
    \item Каким из отношений соответствия (одно-многозначным, много-многозначный и т.д.) оно является?
    \item К каким из отношений порядка (полного, частичного и т.д.) можно отнести данное отношение?
\end{enumerate}


\question
Установите, является ли каждое из перечисленных ниже отношений на А ($R \subseteq A \times A$) отношением эквивалентности (обоснование ответа обязательно). Для каждого отношения эквивалентности постройте классы 
эквивалентности и постройте граф отношения:
\begin{enumerate} [a)]\setcounter{enumi}{0}
\item На множестве $A = \{1; 2; 3\}$ задано отношение $R = \{(1; 1); (2; 2); (3; 3); (2; 1); (1; 2); (2; 3); (3; 2); (3; 1); (1; 3)\}$
\item На множестве $A = \{1; 2; 3; 4; 5\}$ задано отношение $R = \{(1; 2); (1; 3); (1; 5); (2; 3); (2; 4); (2; 5); (3; 4); (3; 5); (4; 5)\}$
\item А - множество целых чисел и отношение $R = \{(a,b)|a + b = 0\}$
\end{enumerate}\question Составьте полную таблицу истинности, определите, какие переменные являются фиктивными и проверьте, является ли формула тавтологией:
$ P \rightarrow (Q \rightarrow ((P \lor Q) \rightarrow (P \land Q)))$

\end{questions}
\newpage
%%% begin test
\begin{flushright}
\begin{tabular}{p{2.8in} r l}
%\textbf{\class} & \textbf{ФИО:} & \makebox[2.5in]{\hrulefill}\\
\textbf{\class} & \textbf{ФИО:} &Шафиков Александр Наильевич
\\

\textbf{\examdate} &&\\
%\textbf{Time Limit: \timelimit} & Teaching Assistant & \makebox[2in]{\hrulefill}
\end{tabular}\\
\end{flushright}
\rule[1ex]{\textwidth}{.1pt}


\begin{questions}
\question
Найдите и упростите P:
\begin{equation*}
\overline{P} = A \cap \overline{B} \cup A \cap C \cup B \cap C \cup \overline{A} \cap C
\end{equation*}
Затем найдите элементы множества P, выраженного через множества:
\begin{equation*}
A = \{0, 3, 4, 9\}; 
B = \{1, 3, 4, 7\};
C = \{0, 1, 2, 4, 7, 8, 9\};
I = \{0, 1, 2, 3, 4, 5, 6, 7, 8, 9\}.
\end{equation*}\question
Упростите следующее выражение с учетом того, что $A\subset B \subset C \subset D \subset U; A \neq \O$
\begin{equation*}
A \cap C  \cap D \cup B \cap \overline{C} \cap D \cup B \cap C \cap D
\end{equation*}

Примечание: U — универсум\question
Дано отношение на множестве $\{1, 2, 3, 4, 5\}$ 
\begin{equation*}
aRb \iff |a-b| = 1
\end{equation*}
Напишите обоснованный ответ какими свойствами обладает или не обладает отношение и почему:   
\begin{enumerate} [a)]\setcounter{enumi}{0}
\item рефлексивность
\item антирефлексивность
\item симметричность
\item асимметричность
\item антисимметричность
\item транзитивность
\end{enumerate}

Обоснуйте свой ответ по каждому из приведенных ниже вопросов:
\begin{enumerate} [a)]\setcounter{enumi}{0}
    \item Является ли это отношение отношением эквивалентности?
    \item Является ли это отношение функциональным?
    \item Каким из отношений соответствия (одно-многозначным, много-многозначный и т.д.) оно является?
    \item К каким из отношений порядка (полного, частичного и т.д.) можно отнести данное отношение?
\end{enumerate}

\question
Установите, является ли каждое из перечисленных ниже отношений на А ($R \subseteq A \times A$) отношением эквивалентности (обоснование ответа обязательно). Для каждого отношения эквивалентности постройте классы 
эквивалентности и постройте граф отношения:
\begin{enumerate} [a)]\setcounter{enumi}{0}
\item Пусть A – множество имен. $A = \{ $Алексей, Иван, Петр, Александр, Павел, Андрей$ \}$. Тогда отношение $R$ верно на парах имен, начинающихся с одной и той же буквы, и только на них.
\item $A = \{-10, -9, … , 9, 10\}$ и отношение $ R = \{(a,b)|a^{2} = b^{2}\}$
\item На множестве $A = \{1; 2; 3\}$ задано отношение $R = \{(1; 1); (2; 2); (3; 3); (3; 2); (1; 2); (2; 1)\}$
\end{enumerate}\question Составьте полную таблицу истинности, определите, какие переменные являются фиктивными и проверьте, является ли формула тавтологией:
$ P \rightarrow (Q \rightarrow ((P \lor Q) \rightarrow (P \land Q)))$

\end{questions}
\newpage
%%% begin test
\begin{flushright}
\begin{tabular}{p{2.8in} r l}
%\textbf{\class} & \textbf{ФИО:} & \makebox[2.5in]{\hrulefill}\\
\textbf{\class} & \textbf{ФИО:} &Шеврина Мария Дмитриевна
\\

\textbf{\examdate} &&\\
%\textbf{Time Limit: \timelimit} & Teaching Assistant & \makebox[2in]{\hrulefill}
\end{tabular}\\
\end{flushright}
\rule[1ex]{\textwidth}{.1pt}


\begin{questions}
\question
Найдите и упростите P:
\begin{equation*}
\overline{P} = A \cap \overline{B} \cup \overline{B} \cap C \cup \overline{A} \cap \overline{B} \cup \overline{A} \cap C
\end{equation*}
Затем найдите элементы множества P, выраженного через множества:
\begin{equation*}
A = \{0, 3, 4, 9\}; 
B = \{1, 3, 4, 7\};
C = \{0, 1, 2, 4, 7, 8, 9\};
I = \{0, 1, 2, 3, 4, 5, 6, 7, 8, 9\}.
\end{equation*}\question
Упростите следующее выражение с учетом того, что $A\subset B \subset C \subset D \subset U; A \neq \O$
\begin{equation*}
\overline{A} \cap \overline{B} \cup B \cap \overline{C} \cup \overline{C} \cap D
\end{equation*}

Примечание: U — универсум\question
Дано отношение на множестве $\{1, 2, 3, 4, 5\}$ 
\begin{equation*}
aRb \iff b > a
\end{equation*}
Напишите обоснованный ответ какими свойствами обладает или не обладает отношение и почему:   
\begin{enumerate} [a)]\setcounter{enumi}{0}
\item рефлексивность
\item антирефлексивность
\item симметричность
\item асимметричность
\item антисимметричность
\item транзитивность
\end{enumerate}

Обоснуйте свой ответ по каждому из приведенных ниже вопросов:
\begin{enumerate} [a)]\setcounter{enumi}{0}
    \item Является ли это отношение отношением эквивалентности?
    \item Является ли это отношение функциональным?
    \item Каким из отношений соответствия (одно-многозначным, много-многозначный и т.д.) оно является?
    \item К каким из отношений порядка (полного, частичного и т.д.) можно отнести данное отношение?
\end{enumerate}

\question
Установите, является ли каждое из перечисленных ниже отношений на А ($R \subseteq A \times A$) отношением эквивалентности (обоснование ответа обязательно). Для каждого отношения эквивалентности 
постройте классы эквивалентности и постройте граф отношения:
\begin{enumerate}[a)]\setcounter{enumi}{0}
\item А - множество целых чисел и отношение $R = \{(a,b)|a + b = 0\}$
\item $A = \{-10, -9, …, 9, 10\}$ и отношение $R = \{(a,b)|a^{3} = b^{3}\}$
\item На множестве $A = \{1; 2; 3\}$ задано отношение $R = \{(1; 1); (2; 2); (3; 3); (2; 1); (1; 2); (2; 3); (3; 2); (3; 1); (1; 3)\}$

\end{enumerate}\question Составьте полную таблицу истинности, определите, какие переменные являются фиктивными и проверьте, является ли формула тавтологией:
$ P \rightarrow (Q \rightarrow ((P \lor Q) \rightarrow (P \land Q)))$

\end{questions}
\newpage
%%% begin test
\begin{flushright}
\begin{tabular}{p{2.8in} r l}
%\textbf{\class} & \textbf{ФИО:} & \makebox[2.5in]{\hrulefill}\\
\textbf{\class} & \textbf{ФИО:} &Янбарисов Илья Александрович
\\

\textbf{\examdate} &&\\
%\textbf{Time Limit: \timelimit} & Teaching Assistant & \makebox[2in]{\hrulefill}
\end{tabular}\\
\end{flushright}
\rule[1ex]{\textwidth}{.1pt}


\begin{questions}
\question
Найдите и упростите P:
\begin{equation*}
\overline{P} = A \cap \overline{B} \cup \overline{B} \cap C \cup \overline{A} \cap \overline{B} \cup \overline{A} \cap C
\end{equation*}
Затем найдите элементы множества P, выраженного через множества:
\begin{equation*}
A = \{0, 3, 4, 9\}; 
B = \{1, 3, 4, 7\};
C = \{0, 1, 2, 4, 7, 8, 9\};
I = \{0, 1, 2, 3, 4, 5, 6, 7, 8, 9\}.
\end{equation*}\question
Упростите следующее выражение с учетом того, что $A\subset B \subset C \subset D \subset U; A \neq \O$
\begin{equation*}
A \cap B  \cap \overline{C} \cup \overline{C} \cap D \cup B \cap C \cap D
\end{equation*}

Примечание: U — универсум\question
Дано отношение на множестве $\{1, 2, 3, 4, 5\}$ 
\begin{equation*}
aRb \iff  \text{НОД}(a,b) =1
\end{equation*}
Напишите обоснованный ответ какими свойствами обладает или не обладает отношение и почему:   
\begin{enumerate} [a)]\setcounter{enumi}{0}
\item рефлексивность
\item антирефлексивность
\item симметричность
\item асимметричность
\item антисимметричность
\item транзитивность
\end{enumerate}

Обоснуйте свой ответ по каждому из приведенных ниже вопросов:
\begin{enumerate} [a)]\setcounter{enumi}{0}
    \item Является ли это отношение отношением эквивалентности?
    \item Является ли это отношение функциональным?
    \item Каким из отношений соответствия (одно-многозначным, много-многозначный и т.д.) оно является?
    \item К каким из отношений порядка (полного, частичного и т.д.) можно отнести данное отношение?
\end{enumerate}


\question
Установите, является ли каждое из перечисленных ниже отношений на А ($R \subseteq A \times A$) отношением эквивалентности (обоснование ответа обязательно). Для каждого отношения эквивалентности постройте классы эквивалентности и постройте граф отношения:
\begin{enumerate} [a)]\setcounter{enumi}{0}
\item $F(x)=x^{2}+1$, где $x \in A = [-2, 4]$ и отношение $R = \{(a,b)|F(a) = F(b)\}$
\item А - множество целых чисел и отношение $R = \{(a,b)|a + b = 5\}$
\item На множестве $A = \{1; 2; 3\}$ задано отношение $R = \{(1; 1); (2; 2); (3; 3); (3; 2); (1; 2); (2; 1)\}$

\end{enumerate}\question Составьте полную таблицу истинности, определите, какие переменные являются фиктивными и проверьте, является ли формула тавтологией:
$(( P \rightarrow Q) \land (Q \rightarrow P)) \rightarrow (P \rightarrow R)$

\end{questions}
\newpage
%%% begin test
\begin{flushright}
\begin{tabular}{p{2.8in} r l}
%\textbf{\class} & \textbf{ФИО:} & \makebox[2.5in]{\hrulefill}\\
\textbf{\class} & \textbf{ФИО:} &М3109
\\

\textbf{\examdate} &&\\
%\textbf{Time Limit: \timelimit} & Teaching Assistant & \makebox[2in]{\hrulefill}
\end{tabular}\\
\end{flushright}
\rule[1ex]{\textwidth}{.1pt}


\begin{questions}
\question
Найдите и упростите P:
\begin{equation*}
\overline{P} = B \cap \overline{C} \cup A \cap B \cup \overline{A} \cap C \cup \overline{A} \cap B
\end{equation*}
Затем найдите элементы множества P, выраженного через множества:
\begin{equation*}
A = \{0, 3, 4, 9\}; 
B = \{1, 3, 4, 7\};
C = \{0, 1, 2, 4, 7, 8, 9\};
I = \{0, 1, 2, 3, 4, 5, 6, 7, 8, 9\}.
\end{equation*}\question
Упростите следующее выражение с учетом того, что $A\subset B \subset C \subset D \subset U; A \neq \O$
\begin{equation*}
A \cap B  \cap \overline{C} \cup \overline{C} \cap D \cup B \cap C \cap D
\end{equation*}

Примечание: U — универсум\question
Дано отношение на множестве $\{1, 2, 3, 4, 5\}$ 
\begin{equation*}
aRb \iff b > a
\end{equation*}
Напишите обоснованный ответ какими свойствами обладает или не обладает отношение и почему:   
\begin{enumerate} [a)]\setcounter{enumi}{0}
\item рефлексивность
\item антирефлексивность
\item симметричность
\item асимметричность
\item антисимметричность
\item транзитивность
\end{enumerate}

Обоснуйте свой ответ по каждому из приведенных ниже вопросов:
\begin{enumerate} [a)]\setcounter{enumi}{0}
    \item Является ли это отношение отношением эквивалентности?
    \item Является ли это отношение функциональным?
    \item Каким из отношений соответствия (одно-многозначным, много-многозначный и т.д.) оно является?
    \item К каким из отношений порядка (полного, частичного и т.д.) можно отнести данное отношение?
\end{enumerate}

\question
Установите, является ли каждое из перечисленных ниже отношений на А ($R \subseteq A \times A$) отношением эквивалентности (обоснование ответа обязательно). Для каждого отношения эквивалентности постройте классы эквивалентности и постройте граф отношения:
\begin{enumerate} [a)]\setcounter{enumi}{0}
\item $F(x)=x^{2}+1$, где $x \in A = [-2, 4]$ и отношение $R = \{(a,b)|F(a) = F(b)\}$
\item А - множество целых чисел и отношение $R = \{(a,b)|a + b = 5\}$
\item На множестве $A = \{1; 2; 3\}$ задано отношение $R = \{(1; 1); (2; 2); (3; 3); (3; 2); (1; 2); (2; 1)\}$

\end{enumerate}\question Составьте полную таблицу истинности, определите, какие переменные являются фиктивными и проверьте, является ли формула тавтологией:
$((P \rightarrow Q) \lor R) \leftrightarrow (P \rightarrow (Q \lor R))$

\end{questions}
\newpage
%%% begin test
\begin{flushright}
\begin{tabular}{p{2.8in} r l}
%\textbf{\class} & \textbf{ФИО:} & \makebox[2.5in]{\hrulefill}\\
\textbf{\class} & \textbf{ФИО:} &Бабич Артём Антонович
\\

\textbf{\examdate} &&\\
%\textbf{Time Limit: \timelimit} & Teaching Assistant & \makebox[2in]{\hrulefill}
\end{tabular}\\
\end{flushright}
\rule[1ex]{\textwidth}{.1pt}


\begin{questions}
\question
Найдите и упростите P:
\begin{equation*}
\overline{P} = A \cap B \cup \overline{A} \cap \overline{B} \cup A \cap C \cup \overline{B} \cap C
\end{equation*}
Затем найдите элементы множества P, выраженного через множества:
\begin{equation*}
A = \{0, 3, 4, 9\}; 
B = \{1, 3, 4, 7\};
C = \{0, 1, 2, 4, 7, 8, 9\};
I = \{0, 1, 2, 3, 4, 5, 6, 7, 8, 9\}.
\end{equation*}\question
Упростите следующее выражение с учетом того, что $A\subset B \subset C \subset D \subset U; A \neq \O$
\begin{equation*}
\overline{B} \cap \overline{C} \cap D \cup \overline{A} \cap \overline{C} \cap D \cup \overline{A} \cap B
\end{equation*}

Примечание: U — универсум\question
Дано отношение на множестве $\{1, 2, 3, 4, 5\}$ 
\begin{equation*}
aRb \iff |a-b| = 1
\end{equation*}
Напишите обоснованный ответ какими свойствами обладает или не обладает отношение и почему:   
\begin{enumerate} [a)]\setcounter{enumi}{0}
\item рефлексивность
\item антирефлексивность
\item симметричность
\item асимметричность
\item антисимметричность
\item транзитивность
\end{enumerate}

Обоснуйте свой ответ по каждому из приведенных ниже вопросов:
\begin{enumerate} [a)]\setcounter{enumi}{0}
    \item Является ли это отношение отношением эквивалентности?
    \item Является ли это отношение функциональным?
    \item Каким из отношений соответствия (одно-многозначным, много-многозначный и т.д.) оно является?
    \item К каким из отношений порядка (полного, частичного и т.д.) можно отнести данное отношение?
\end{enumerate}

\question
Установите, является ли каждое из перечисленных ниже отношений на А ($R \subseteq A \times A$) отношением эквивалентности (обоснование ответа обязательно). Для каждого отношения эквивалентности постройте классы эквивалентности и постройте граф отношения:
\begin{enumerate} [a)]\setcounter{enumi}{0}
\item $F(x)=x^{2}+1$, где $x \in A = [-2, 4]$ и отношение $R = \{(a,b)|F(a) = F(b)\}$
\item А - множество целых чисел и отношение $R = \{(a,b)|a + b = 5\}$
\item На множестве $A = \{1; 2; 3\}$ задано отношение $R = \{(1; 1); (2; 2); (3; 3); (3; 2); (1; 2); (2; 1)\}$

\end{enumerate}\question Составьте полную таблицу истинности, определите, какие переменные являются фиктивными и проверьте, является ли формула тавтологией:
$((P \rightarrow Q) \lor R) \leftrightarrow (P \rightarrow (Q \lor R))$

\end{questions}
\newpage
%%% begin test
\begin{flushright}
\begin{tabular}{p{2.8in} r l}
%\textbf{\class} & \textbf{ФИО:} & \makebox[2.5in]{\hrulefill}\\
\textbf{\class} & \textbf{ФИО:} &Беликов Владимир Владимирович
\\

\textbf{\examdate} &&\\
%\textbf{Time Limit: \timelimit} & Teaching Assistant & \makebox[2in]{\hrulefill}
\end{tabular}\\
\end{flushright}
\rule[1ex]{\textwidth}{.1pt}


\begin{questions}
\question
Найдите и упростите P:
\begin{equation*}
\overline{P} = A \cap \overline{B} \cup \overline{B} \cap C \cup \overline{A} \cap \overline{B} \cup \overline{A} \cap C
\end{equation*}
Затем найдите элементы множества P, выраженного через множества:
\begin{equation*}
A = \{0, 3, 4, 9\}; 
B = \{1, 3, 4, 7\};
C = \{0, 1, 2, 4, 7, 8, 9\};
I = \{0, 1, 2, 3, 4, 5, 6, 7, 8, 9\}.
\end{equation*}\question
Упростите следующее выражение с учетом того, что $A\subset B \subset C \subset D \subset U; A \neq \O$
\begin{equation*}
A \cap  \overline{C} \cup B \cap \overline{D} \cup  \overline{A} \cap C \cap  \overline{D}
\end{equation*}

Примечание: U — универсум\question
Дано отношение на множестве $\{1, 2, 3, 4, 5\}$ 
\begin{equation*}
aRb \iff b > a
\end{equation*}
Напишите обоснованный ответ какими свойствами обладает или не обладает отношение и почему:   
\begin{enumerate} [a)]\setcounter{enumi}{0}
\item рефлексивность
\item антирефлексивность
\item симметричность
\item асимметричность
\item антисимметричность
\item транзитивность
\end{enumerate}

Обоснуйте свой ответ по каждому из приведенных ниже вопросов:
\begin{enumerate} [a)]\setcounter{enumi}{0}
    \item Является ли это отношение отношением эквивалентности?
    \item Является ли это отношение функциональным?
    \item Каким из отношений соответствия (одно-многозначным, много-многозначный и т.д.) оно является?
    \item К каким из отношений порядка (полного, частичного и т.д.) можно отнести данное отношение?
\end{enumerate}

\question
Установите, является ли каждое из перечисленных ниже отношений на А ($R \subseteq A \times A$) отношением эквивалентности (обоснование ответа обязательно). Для каждого отношения эквивалентности постройте классы эквивалентности и постройте граф отношения:
\begin{enumerate} [a)]\setcounter{enumi}{0}
\item $F(x)=x^{2}+1$, где $x \in A = [-2, 4]$ и отношение $R = \{(a,b)|F(a) = F(b)\}$
\item А - множество целых чисел и отношение $R = \{(a,b)|a + b = 5\}$
\item На множестве $A = \{1; 2; 3\}$ задано отношение $R = \{(1; 1); (2; 2); (3; 3); (3; 2); (1; 2); (2; 1)\}$

\end{enumerate}\question Составьте полную таблицу истинности, определите, какие переменные являются фиктивными и проверьте, является ли формула тавтологией:
$(( P \land \neg Q) \rightarrow (R \land \neg R)) \rightarrow (P \rightarrow Q)$

\end{questions}
\newpage
%%% begin test
\begin{flushright}
\begin{tabular}{p{2.8in} r l}
%\textbf{\class} & \textbf{ФИО:} & \makebox[2.5in]{\hrulefill}\\
\textbf{\class} & \textbf{ФИО:} &Ганжа Дарья
\\

\textbf{\examdate} &&\\
%\textbf{Time Limit: \timelimit} & Teaching Assistant & \makebox[2in]{\hrulefill}
\end{tabular}\\
\end{flushright}
\rule[1ex]{\textwidth}{.1pt}


\begin{questions}
\question
Найдите и упростите P:
\begin{equation*}
\overline{P} = A \cap \overline{B} \cup \overline{B} \cap C \cup \overline{A} \cap \overline{B} \cup \overline{A} \cap C
\end{equation*}
Затем найдите элементы множества P, выраженного через множества:
\begin{equation*}
A = \{0, 3, 4, 9\}; 
B = \{1, 3, 4, 7\};
C = \{0, 1, 2, 4, 7, 8, 9\};
I = \{0, 1, 2, 3, 4, 5, 6, 7, 8, 9\}.
\end{equation*}\question
Упростите следующее выражение с учетом того, что $A\subset B \subset C \subset D \subset U; A \neq \O$
\begin{equation*}
\overline{A} \cap \overline{C} \cap D \cup \overline{B} \cap \overline{C} \cap D \cup A \cap B
\end{equation*}

Примечание: U — универсум\question
Для следующего отношения на множестве $\{1, 2, 3, 4, 5\}$ 
\begin{equation*}
aRb \iff 0 < a-b<2
\end{equation*}
Напишите обоснованный ответ какими свойствами обладает или не обладает отношение и почему:   
\begin{enumerate} [a)]\setcounter{enumi}{0}
\item рефлексивность
\item антирефлексивность
\item симметричность
\item асимметричность
\item антисимметричность
\item транзитивность
\end{enumerate}

Обоснуйте свой ответ по каждому из приведенных ниже вопросов:
\begin{enumerate} [a)]\setcounter{enumi}{0}
    \item Является ли это отношение отношением эквивалентности?
    \item Является ли это отношение функциональным?
    \item Каким из отношений соответствия (одно-многозначным, много-многозначный и т.д.) оно является?
    \item К каким из отношений порядка (полного, частичного и т.д.) можно отнести данное отношение?
\end{enumerate}
\question
Установите, является ли каждое из перечисленных ниже отношений на А ($R \subseteq A \times A$) отношением эквивалентности (обоснование ответа обязательно). Для каждого отношения эквивалентности постройте классы 
эквивалентности и постройте граф отношения:
\begin{enumerate} [a)]\setcounter{enumi}{0}
\item $A = \{-10, -9, … , 9, 10\}$ и отношение $R = \{(a,b)|a^{2} = b^{2}\}$
\item $A = \{a, b, c, d, p, t\}$ задано отношение $R = \{(a, a), (b, b), (b, c), (b, d), (c, b), (c, c), (c, d), (d, b), (d, c), (d, d), (p,p), (t,t)\}$
\item Пусть A – множество имен. $A = \{ $Алексей, Иван, Петр, Александр, Павел, Андрей$ \}$. Тогда отношение $R$ верно на парах имен, начинающихся с одной и той же буквы, и только на них.
\end{enumerate}\question Составьте полную таблицу истинности, определите, какие переменные являются фиктивными и проверьте, является ли формула тавтологией:
$((P \rightarrow Q) \lor R) \leftrightarrow (P \rightarrow (Q \lor R))$

\end{questions}
\newpage
%%% begin test
\begin{flushright}
\begin{tabular}{p{2.8in} r l}
%\textbf{\class} & \textbf{ФИО:} & \makebox[2.5in]{\hrulefill}\\
\textbf{\class} & \textbf{ФИО:} &Грыу Андрей Анатольевич
\\

\textbf{\examdate} &&\\
%\textbf{Time Limit: \timelimit} & Teaching Assistant & \makebox[2in]{\hrulefill}
\end{tabular}\\
\end{flushright}
\rule[1ex]{\textwidth}{.1pt}


\begin{questions}
\question
Найдите и упростите P:
\begin{equation*}
\overline{P} = A \cap \overline{C} \cup A \cap \overline{B} \cup B \cap \overline{C} \cup A \cap C
\end{equation*}
Затем найдите элементы множества P, выраженного через множества:
\begin{equation*}
A = \{0, 3, 4, 9\}; 
B = \{1, 3, 4, 7\};
C = \{0, 1, 2, 4, 7, 8, 9\};
I = \{0, 1, 2, 3, 4, 5, 6, 7, 8, 9\}.
\end{equation*}\question
Упростите следующее выражение с учетом того, что $A\subset B \subset C \subset D \subset U; A \neq \O$
\begin{equation*}
A \cap B \cup \overline{A} \cap \overline{C} \cup A \cap C \cup \overline{B} \cap \overline{C}
\end{equation*}

Примечание: U — универсум\question
Дано отношение на множестве $\{1, 2, 3, 4, 5\}$ 
\begin{equation*}
aRb \iff (a+b) \bmod 2 =0
\end{equation*}
Напишите обоснованный ответ какими свойствами обладает или не обладает отношение и почему:   
\begin{enumerate} [a)]\setcounter{enumi}{0}
\item рефлексивность
\item антирефлексивность
\item симметричность
\item асимметричность
\item антисимметричность
\item транзитивность
\end{enumerate}

Обоснуйте свой ответ по каждому из приведенных ниже вопросов:
\begin{enumerate} [a)]\setcounter{enumi}{0}
    \item Является ли это отношение отношением эквивалентности?
    \item Является ли это отношение функциональным?
    \item Каким из отношений соответствия (одно-многозначным, много-многозначный и т.д.) оно является?
    \item К каким из отношений порядка (полного, частичного и т.д.) можно отнести данное отношение?
\end{enumerate}



\question
Установите, является ли каждое из перечисленных ниже отношений на А ($R \subseteq A \times A$) отношением эквивалентности (обоснование ответа обязательно). Для каждого отношения эквивалентности постройте классы 
эквивалентности и постройте граф отношения:
\begin{enumerate} [a)]\setcounter{enumi}{0}
\item А - множество целых чисел и отношение $R = \{(a,b)|a + b = 5\}$
\item Пусть A – множество имен. $A = \{ $Алексей, Иван, Петр, Александр, Павел, Андрей$ \}$. Тогда отношение $R $ верно на парах имен, начинающихся с одной и той же буквы, и только на них.
\item На множестве $A = \{1; 2; 3; 4; 5\}$ задано отношение $R = \{(1; 2); (1; 3); (1; 5); (2; 3); (2; 4); (2; 5); (3; 4); (3; 5); (4; 5)\}$
\end{enumerate}\question Составьте полную таблицу истинности, определите, какие переменные являются фиктивными и проверьте, является ли формула тавтологией:
$(( P \land \neg Q) \rightarrow (R \land \neg R)) \rightarrow (P \rightarrow Q)$

\end{questions}
\newpage
%%% begin test
\begin{flushright}
\begin{tabular}{p{2.8in} r l}
%\textbf{\class} & \textbf{ФИО:} & \makebox[2.5in]{\hrulefill}\\
\textbf{\class} & \textbf{ФИО:} &Жаркова Екатерина Сергеевна
\\

\textbf{\examdate} &&\\
%\textbf{Time Limit: \timelimit} & Teaching Assistant & \makebox[2in]{\hrulefill}
\end{tabular}\\
\end{flushright}
\rule[1ex]{\textwidth}{.1pt}


\begin{questions}
\question
Найдите и упростите P:
\begin{equation*}
\overline{P} = A \cap \overline{B} \cup \overline{B} \cap C \cup \overline{A} \cap \overline{B} \cup \overline{A} \cap C
\end{equation*}
Затем найдите элементы множества P, выраженного через множества:
\begin{equation*}
A = \{0, 3, 4, 9\}; 
B = \{1, 3, 4, 7\};
C = \{0, 1, 2, 4, 7, 8, 9\};
I = \{0, 1, 2, 3, 4, 5, 6, 7, 8, 9\}.
\end{equation*}\question
Упростите следующее выражение с учетом того, что $A\subset B \subset C \subset D \subset U; A \neq \O$
\begin{equation*}
A \cap B  \cap \overline{C} \cup \overline{C} \cap D \cup B \cap C \cap D
\end{equation*}

Примечание: U — универсум\question
Дано отношение на множестве $\{1, 2, 3, 4, 5\}$ 
\begin{equation*}
aRb \iff a \geq b^2
\end{equation*}
Напишите обоснованный ответ какими свойствами обладает или не обладает отношение и почему:   
\begin{enumerate} [a)]\setcounter{enumi}{0}
\item рефлексивность
\item антирефлексивность
\item симметричность
\item асимметричность
\item антисимметричность
\item транзитивность
\end{enumerate}

Обоснуйте свой ответ по каждому из приведенных ниже вопросов:
\begin{enumerate} [a)]\setcounter{enumi}{0}
    \item Является ли это отношение отношением эквивалентности?
    \item Является ли это отношение функциональным?
    \item Каким из отношений соответствия (одно-многозначным, много-многозначный и т.д.) оно является?
    \item К каким из отношений порядка (полного, частичного и т.д.) можно отнести данное отношение?
\end{enumerate}


\question
Установите, является ли каждое из перечисленных ниже отношений на А ($R \subseteq A \times A$) отношением эквивалентности (обоснование ответа обязательно). Для каждого отношения эквивалентности постройте классы 
эквивалентности и постройте граф отношения:
\begin{enumerate} [a)]\setcounter{enumi}{0}
\item $A = \{a, b, c, d, p, t\}$ задано отношение $R = \{(a, a), (b, b), (b, c), (b, d), (c, b), (c, c), (c, d), (d, b), (d, c), (d, d), (p,p), (t,t)\}$
\item $A = \{-10, -9, … , 9, 10\}$ и отношение $R = \{(a,b)|a^{3} = b^{3}\}$

\item $F(x)=x^{2}+1$, где $x \in A = [-2, 4]$ и отношение $R = \{(a,b)|F(a) = F(b)\}$
\end{enumerate}\question Составьте полную таблицу истинности, определите, какие переменные являются фиктивными и проверьте, является ли формула тавтологией:
$(( P \land \neg Q) \rightarrow (R \land \neg R)) \rightarrow (P \rightarrow Q)$

\end{questions}
\newpage
%%% begin test
\begin{flushright}
\begin{tabular}{p{2.8in} r l}
%\textbf{\class} & \textbf{ФИО:} & \makebox[2.5in]{\hrulefill}\\
\textbf{\class} & \textbf{ФИО:} &Жиделев Илья Дмитриевич
\\

\textbf{\examdate} &&\\
%\textbf{Time Limit: \timelimit} & Teaching Assistant & \makebox[2in]{\hrulefill}
\end{tabular}\\
\end{flushright}
\rule[1ex]{\textwidth}{.1pt}


\begin{questions}
\question
Найдите и упростите P:
\begin{equation*}
\overline{P} = A \cap \overline{C} \cup A \cap \overline{B} \cup B \cap \overline{C} \cup A \cap C
\end{equation*}
Затем найдите элементы множества P, выраженного через множества:
\begin{equation*}
A = \{0, 3, 4, 9\}; 
B = \{1, 3, 4, 7\};
C = \{0, 1, 2, 4, 7, 8, 9\};
I = \{0, 1, 2, 3, 4, 5, 6, 7, 8, 9\}.
\end{equation*}\question
Упростите следующее выражение с учетом того, что $A\subset B \subset C \subset D \subset U; A \neq \O$
\begin{equation*}
\overline{A} \cap \overline{C} \cap D \cup \overline{B} \cap \overline{C} \cap D \cup A \cap B
\end{equation*}

Примечание: U — универсум\question
Дано отношение на множестве $\{1, 2, 3, 4, 5\}$ 
\begin{equation*}
aRb \iff a \leq b
\end{equation*}
Напишите обоснованный ответ какими свойствами обладает или не обладает отношение и почему:   
\begin{enumerate} [a)]\setcounter{enumi}{0}
\item рефлексивность
\item антирефлексивность
\item симметричность
\item асимметричность
\item антисимметричность
\item транзитивность
\end{enumerate}

Обоснуйте свой ответ по каждому из приведенных ниже вопросов:
\begin{enumerate} [a)]\setcounter{enumi}{0}
    \item Является ли это отношение отношением эквивалентности?
    \item Является ли это отношение функциональным?
    \item Каким из отношений соответствия (одно-многозначным, много-многозначный и т.д.) оно является?
    \item К каким из отношений порядка (полного, частичного и т.д.) можно отнести данное отношение?
\end{enumerate}


\question
Установите, является ли каждое из перечисленных ниже отношений на А ($R \subseteq A \times A$) отношением эквивалентности (обоснование ответа обязательно). Для каждого отношения эквивалентности постройте классы 
эквивалентности и постройте граф отношения:
\begin{enumerate} [a)]\setcounter{enumi}{0}
\item $A = \{-10, -9, … , 9, 10\}$ и отношение $R = \{(a,b)|a^{2} = b^{2}\}$
\item $A = \{a, b, c, d, p, t\}$ задано отношение $R = \{(a, a), (b, b), (b, c), (b, d), (c, b), (c, c), (c, d), (d, b), (d, c), (d, d), (p,p), (t,t)\}$
\item Пусть A – множество имен. $A = \{ $Алексей, Иван, Петр, Александр, Павел, Андрей$ \}$. Тогда отношение $R$ верно на парах имен, начинающихся с одной и той же буквы, и только на них.
\end{enumerate}\question Составьте полную таблицу истинности, определите, какие переменные являются фиктивными и проверьте, является ли формула тавтологией:
$(( P \land \neg Q) \rightarrow (R \land \neg R)) \rightarrow (P \rightarrow Q)$

\end{questions}
\newpage
%%% begin test
\begin{flushright}
\begin{tabular}{p{2.8in} r l}
%\textbf{\class} & \textbf{ФИО:} & \makebox[2.5in]{\hrulefill}\\
\textbf{\class} & \textbf{ФИО:} &Захаров Илья Константинович
\\

\textbf{\examdate} &&\\
%\textbf{Time Limit: \timelimit} & Teaching Assistant & \makebox[2in]{\hrulefill}
\end{tabular}\\
\end{flushright}
\rule[1ex]{\textwidth}{.1pt}


\begin{questions}
\question
Найдите и упростите P:
\begin{equation*}
\overline{P} = A \cap \overline{B} \cup \overline{B} \cap C \cup \overline{A} \cap \overline{B} \cup \overline{A} \cap C
\end{equation*}
Затем найдите элементы множества P, выраженного через множества:
\begin{equation*}
A = \{0, 3, 4, 9\}; 
B = \{1, 3, 4, 7\};
C = \{0, 1, 2, 4, 7, 8, 9\};
I = \{0, 1, 2, 3, 4, 5, 6, 7, 8, 9\}.
\end{equation*}\question
Упростите следующее выражение с учетом того, что $A\subset B \subset C \subset D \subset U; A \neq \O$
\begin{equation*}
A \cap B \cup \overline{A} \cap \overline{C} \cup A \cap C \cup \overline{B} \cap \overline{C}
\end{equation*}

Примечание: U — универсум\question
Дано отношение на множестве $\{1, 2, 3, 4, 5\}$ 
\begin{equation*}
aRb \iff a \geq b^2
\end{equation*}
Напишите обоснованный ответ какими свойствами обладает или не обладает отношение и почему:   
\begin{enumerate} [a)]\setcounter{enumi}{0}
\item рефлексивность
\item антирефлексивность
\item симметричность
\item асимметричность
\item антисимметричность
\item транзитивность
\end{enumerate}

Обоснуйте свой ответ по каждому из приведенных ниже вопросов:
\begin{enumerate} [a)]\setcounter{enumi}{0}
    \item Является ли это отношение отношением эквивалентности?
    \item Является ли это отношение функциональным?
    \item Каким из отношений соответствия (одно-многозначным, много-многозначный и т.д.) оно является?
    \item К каким из отношений порядка (полного, частичного и т.д.) можно отнести данное отношение?
\end{enumerate}


\question
Установите, является ли каждое из перечисленных ниже отношений на А ($R \subseteq A \times A$) отношением эквивалентности (обоснование ответа обязательно). Для каждого отношения эквивалентности постройте классы 
эквивалентности и постройте граф отношения:
\begin{enumerate} [a)]\setcounter{enumi}{0}
\item Пусть A – множество имен. $A = \{ $Алексей, Иван, Петр, Александр, Павел, Андрей$ \}$. Тогда отношение $R$ верно на парах имен, начинающихся с одной и той же буквы, и только на них.
\item $A = \{-10, -9, … , 9, 10\}$ и отношение $ R = \{(a,b)|a^{2} = b^{2}\}$
\item На множестве $A = \{1; 2; 3\}$ задано отношение $R = \{(1; 1); (2; 2); (3; 3); (3; 2); (1; 2); (2; 1)\}$
\end{enumerate}\question Составьте полную таблицу истинности, определите, какие переменные являются фиктивными и проверьте, является ли формула тавтологией:
$(( P \rightarrow Q) \land (Q \rightarrow P)) \rightarrow (P \rightarrow R)$

\end{questions}
\newpage
%%% begin test
\begin{flushright}
\begin{tabular}{p{2.8in} r l}
%\textbf{\class} & \textbf{ФИО:} & \makebox[2.5in]{\hrulefill}\\
\textbf{\class} & \textbf{ФИО:} &Иванушкин Севастьян Сергеевич
\\

\textbf{\examdate} &&\\
%\textbf{Time Limit: \timelimit} & Teaching Assistant & \makebox[2in]{\hrulefill}
\end{tabular}\\
\end{flushright}
\rule[1ex]{\textwidth}{.1pt}


\begin{questions}
\question
Найдите и упростите P:
\begin{equation*}
\overline{P} = B \cap \overline{C} \cup A \cap B \cup \overline{A} \cap C \cup \overline{A} \cap B
\end{equation*}
Затем найдите элементы множества P, выраженного через множества:
\begin{equation*}
A = \{0, 3, 4, 9\}; 
B = \{1, 3, 4, 7\};
C = \{0, 1, 2, 4, 7, 8, 9\};
I = \{0, 1, 2, 3, 4, 5, 6, 7, 8, 9\}.
\end{equation*}\question
Упростите следующее выражение с учетом того, что $A\subset B \subset C \subset D \subset U; A \neq \O$
\begin{equation*}
A \cap B  \cap \overline{C} \cup \overline{C} \cap D \cup B \cap C \cap D
\end{equation*}

Примечание: U — универсум\question
Для следующего отношения на множестве $\{1, 2, 3, 4, 5\}$ 
\begin{equation*}
aRb \iff 0 < a-b<2
\end{equation*}
Напишите обоснованный ответ какими свойствами обладает или не обладает отношение и почему:   
\begin{enumerate} [a)]\setcounter{enumi}{0}
\item рефлексивность
\item антирефлексивность
\item симметричность
\item асимметричность
\item антисимметричность
\item транзитивность
\end{enumerate}

Обоснуйте свой ответ по каждому из приведенных ниже вопросов:
\begin{enumerate} [a)]\setcounter{enumi}{0}
    \item Является ли это отношение отношением эквивалентности?
    \item Является ли это отношение функциональным?
    \item Каким из отношений соответствия (одно-многозначным, много-многозначный и т.д.) оно является?
    \item К каким из отношений порядка (полного, частичного и т.д.) можно отнести данное отношение?
\end{enumerate}
\question
Установите, является ли каждое из перечисленных ниже отношений на А ($R \subseteq A \times A$) отношением эквивалентности (обоснование ответа обязательно). Для каждого отношения эквивалентности постройте классы эквивалентности и постройте граф отношения:
\begin{enumerate} [a)]\setcounter{enumi}{0}
\item $F(x)=x^{2}+1$, где $x \in A = [-2, 4]$ и отношение $R = \{(a,b)|F(a) = F(b)\}$
\item А - множество целых чисел и отношение $R = \{(a,b)|a + b = 5\}$
\item На множестве $A = \{1; 2; 3\}$ задано отношение $R = \{(1; 1); (2; 2); (3; 3); (3; 2); (1; 2); (2; 1)\}$

\end{enumerate}\question Составьте полную таблицу истинности, определите, какие переменные являются фиктивными и проверьте, является ли формула тавтологией:

$(P \rightarrow (Q \land R)) \leftrightarrow ((P \rightarrow Q) \land (P \rightarrow R))$

\end{questions}
\newpage
%%% begin test
\begin{flushright}
\begin{tabular}{p{2.8in} r l}
%\textbf{\class} & \textbf{ФИО:} & \makebox[2.5in]{\hrulefill}\\
\textbf{\class} & \textbf{ФИО:} &Кожемякин Иван Сергеевич
\\

\textbf{\examdate} &&\\
%\textbf{Time Limit: \timelimit} & Teaching Assistant & \makebox[2in]{\hrulefill}
\end{tabular}\\
\end{flushright}
\rule[1ex]{\textwidth}{.1pt}


\begin{questions}
\question
Найдите и упростите P:
\begin{equation*}
\overline{P} = B \cap \overline{C} \cup A \cap B \cup \overline{A} \cap C \cup \overline{A} \cap B
\end{equation*}
Затем найдите элементы множества P, выраженного через множества:
\begin{equation*}
A = \{0, 3, 4, 9\}; 
B = \{1, 3, 4, 7\};
C = \{0, 1, 2, 4, 7, 8, 9\};
I = \{0, 1, 2, 3, 4, 5, 6, 7, 8, 9\}.
\end{equation*}\question
Упростите следующее выражение с учетом того, что $A\subset B \subset C \subset D \subset U; A \neq \O$
\begin{equation*}
A \cap B  \cap \overline{C} \cup \overline{C} \cap D \cup B \cap C \cap D
\end{equation*}

Примечание: U — универсум\question
Дано отношение на множестве $\{1, 2, 3, 4, 5\}$ 
\begin{equation*}
aRb \iff a \geq b^2
\end{equation*}
Напишите обоснованный ответ какими свойствами обладает или не обладает отношение и почему:   
\begin{enumerate} [a)]\setcounter{enumi}{0}
\item рефлексивность
\item антирефлексивность
\item симметричность
\item асимметричность
\item антисимметричность
\item транзитивность
\end{enumerate}

Обоснуйте свой ответ по каждому из приведенных ниже вопросов:
\begin{enumerate} [a)]\setcounter{enumi}{0}
    \item Является ли это отношение отношением эквивалентности?
    \item Является ли это отношение функциональным?
    \item Каким из отношений соответствия (одно-многозначным, много-многозначный и т.д.) оно является?
    \item К каким из отношений порядка (полного, частичного и т.д.) можно отнести данное отношение?
\end{enumerate}


\question
Установите, является ли каждое из перечисленных ниже отношений на А ($R \subseteq A \times A$) отношением эквивалентности (обоснование ответа обязательно). Для каждого отношения эквивалентности постройте классы 
эквивалентности и постройте граф отношения:
\begin{enumerate} [a)]\setcounter{enumi}{0}
\item $A = \{a, b, c, d, p, t\}$ задано отношение $R = \{(a, a), (b, b), (b, c), (b, d), (c, b), (c, c), (c, d), (d, b), (d, c), (d, d), (p,p), (t,t)\}$
\item $A = \{-10, -9, … , 9, 10\}$ и отношение $R = \{(a,b)|a^{3} = b^{3}\}$

\item $F(x)=x^{2}+1$, где $x \in A = [-2, 4]$ и отношение $R = \{(a,b)|F(a) = F(b)\}$
\end{enumerate}\question Составьте полную таблицу истинности, определите, какие переменные являются фиктивными и проверьте, является ли формула тавтологией:
$ P \rightarrow (Q \rightarrow ((P \lor Q) \rightarrow (P \land Q)))$

\end{questions}
\newpage
%%% begin test
\begin{flushright}
\begin{tabular}{p{2.8in} r l}
%\textbf{\class} & \textbf{ФИО:} & \makebox[2.5in]{\hrulefill}\\
\textbf{\class} & \textbf{ФИО:} &Кравченко Сергей Вячеславович
\\

\textbf{\examdate} &&\\
%\textbf{Time Limit: \timelimit} & Teaching Assistant & \makebox[2in]{\hrulefill}
\end{tabular}\\
\end{flushright}
\rule[1ex]{\textwidth}{.1pt}


\begin{questions}
\question
Найдите и упростите P:
\begin{equation*}
\overline{P} = B \cap \overline{C} \cup A \cap B \cup \overline{A} \cap C \cup \overline{A} \cap B
\end{equation*}
Затем найдите элементы множества P, выраженного через множества:
\begin{equation*}
A = \{0, 3, 4, 9\}; 
B = \{1, 3, 4, 7\};
C = \{0, 1, 2, 4, 7, 8, 9\};
I = \{0, 1, 2, 3, 4, 5, 6, 7, 8, 9\}.
\end{equation*}\question
Упростите следующее выражение с учетом того, что $A\subset B \subset C \subset D \subset U; A \neq \O$
\begin{equation*}
\overline{B} \cap \overline{C} \cap D \cup \overline{A} \cap \overline{C} \cap D \cup \overline{A} \cap B
\end{equation*}

Примечание: U — универсум\question
Дано отношение на множестве $\{1, 2, 3, 4, 5\}$ 
\begin{equation*}
aRb \iff |a-b| = 1
\end{equation*}
Напишите обоснованный ответ какими свойствами обладает или не обладает отношение и почему:   
\begin{enumerate} [a)]\setcounter{enumi}{0}
\item рефлексивность
\item антирефлексивность
\item симметричность
\item асимметричность
\item антисимметричность
\item транзитивность
\end{enumerate}

Обоснуйте свой ответ по каждому из приведенных ниже вопросов:
\begin{enumerate} [a)]\setcounter{enumi}{0}
    \item Является ли это отношение отношением эквивалентности?
    \item Является ли это отношение функциональным?
    \item Каким из отношений соответствия (одно-многозначным, много-многозначный и т.д.) оно является?
    \item К каким из отношений порядка (полного, частичного и т.д.) можно отнести данное отношение?
\end{enumerate}

\question
Установите, является ли каждое из перечисленных ниже отношений на А ($R \subseteq A \times A$) отношением эквивалентности (обоснование ответа обязательно). Для каждого отношения эквивалентности 
постройте классы эквивалентности и постройте граф отношения:
\begin{enumerate}[a)]\setcounter{enumi}{0}
\item А - множество целых чисел и отношение $R = \{(a,b)|a + b = 0\}$
\item $A = \{-10, -9, …, 9, 10\}$ и отношение $R = \{(a,b)|a^{3} = b^{3}\}$
\item На множестве $A = \{1; 2; 3\}$ задано отношение $R = \{(1; 1); (2; 2); (3; 3); (2; 1); (1; 2); (2; 3); (3; 2); (3; 1); (1; 3)\}$

\end{enumerate}\question Составьте полную таблицу истинности, определите, какие переменные являются фиктивными и проверьте, является ли формула тавтологией:
$(( P \land \neg Q) \rightarrow (R \land \neg R)) \rightarrow (P \rightarrow Q)$

\end{questions}
\newpage
%%% begin test
\begin{flushright}
\begin{tabular}{p{2.8in} r l}
%\textbf{\class} & \textbf{ФИО:} & \makebox[2.5in]{\hrulefill}\\
\textbf{\class} & \textbf{ФИО:} &Лядов Дмитрий Евгеньевич
\\

\textbf{\examdate} &&\\
%\textbf{Time Limit: \timelimit} & Teaching Assistant & \makebox[2in]{\hrulefill}
\end{tabular}\\
\end{flushright}
\rule[1ex]{\textwidth}{.1pt}


\begin{questions}
\question
Найдите и упростите P:
\begin{equation*}
\overline{P} = \overline{A} \cap B \cup \overline{A} \cap C \cup A \cap \overline{B} \cup \overline{B} \cap C
\end{equation*}
Затем найдите элементы множества P, выраженного через множества:
\begin{equation*}
A = \{0, 3, 4, 9\}; 
B = \{1, 3, 4, 7\};
C = \{0, 1, 2, 4, 7, 8, 9\};
I = \{0, 1, 2, 3, 4, 5, 6, 7, 8, 9\}.
\end{equation*}\question
Упростите следующее выражение с учетом того, что $A\subset B \subset C \subset D \subset U; A \neq \O$
\begin{equation*}
A \cap B \cup \overline{A} \cap \overline{C} \cup A \cap C \cup \overline{B} \cap \overline{C}
\end{equation*}

Примечание: U — универсум\question
Дано отношение на множестве $\{1, 2, 3, 4, 5\}$ 
\begin{equation*}
aRb \iff a \leq b
\end{equation*}
Напишите обоснованный ответ какими свойствами обладает или не обладает отношение и почему:   
\begin{enumerate} [a)]\setcounter{enumi}{0}
\item рефлексивность
\item антирефлексивность
\item симметричность
\item асимметричность
\item антисимметричность
\item транзитивность
\end{enumerate}

Обоснуйте свой ответ по каждому из приведенных ниже вопросов:
\begin{enumerate} [a)]\setcounter{enumi}{0}
    \item Является ли это отношение отношением эквивалентности?
    \item Является ли это отношение функциональным?
    \item Каким из отношений соответствия (одно-многозначным, много-многозначный и т.д.) оно является?
    \item К каким из отношений порядка (полного, частичного и т.д.) можно отнести данное отношение?
\end{enumerate}


\question
Установите, является ли каждое из перечисленных ниже отношений на А ($R \subseteq A \times A$) отношением эквивалентности (обоснование ответа обязательно). Для каждого отношения эквивалентности постройте классы 
эквивалентности и постройте граф отношения:
\begin{enumerate} [a)]\setcounter{enumi}{0}
\item $A = \{a, b, c, d, p, t\}$ задано отношение $R = \{(a, a), (b, b), (b, c), (b, d), (c, b), (c, c), (c, d), (d, b), (d, c), (d, d), (p,p), (t,t)\}$
\item $A = \{-10, -9, … , 9, 10\}$ и отношение $R = \{(a,b)|a^{3} = b^{3}\}$

\item $F(x)=x^{2}+1$, где $x \in A = [-2, 4]$ и отношение $R = \{(a,b)|F(a) = F(b)\}$
\end{enumerate}\question Составьте полную таблицу истинности, определите, какие переменные являются фиктивными и проверьте, является ли формула тавтологией:
$ P \rightarrow (Q \rightarrow ((P \lor Q) \rightarrow (P \land Q)))$

\end{questions}
\newpage
%%% begin test
\begin{flushright}
\begin{tabular}{p{2.8in} r l}
%\textbf{\class} & \textbf{ФИО:} & \makebox[2.5in]{\hrulefill}\\
\textbf{\class} & \textbf{ФИО:} &Максимов Лев Сергеевич
\\

\textbf{\examdate} &&\\
%\textbf{Time Limit: \timelimit} & Teaching Assistant & \makebox[2in]{\hrulefill}
\end{tabular}\\
\end{flushright}
\rule[1ex]{\textwidth}{.1pt}


\begin{questions}
\question
Найдите и упростите P:
\begin{equation*}
\overline{P} = A \cap \overline{B} \cup A \cap C \cup B \cap C \cup \overline{A} \cap C
\end{equation*}
Затем найдите элементы множества P, выраженного через множества:
\begin{equation*}
A = \{0, 3, 4, 9\}; 
B = \{1, 3, 4, 7\};
C = \{0, 1, 2, 4, 7, 8, 9\};
I = \{0, 1, 2, 3, 4, 5, 6, 7, 8, 9\}.
\end{equation*}\question
Упростите следующее выражение с учетом того, что $A\subset B \subset C \subset D \subset U; A \neq \O$
\begin{equation*}
\overline{B} \cap \overline{C} \cap D \cup \overline{A} \cap \overline{C} \cap D \cup \overline{A} \cap B
\end{equation*}

Примечание: U — универсум\question
Дано отношение на множестве $\{1, 2, 3, 4, 5\}$ 
\begin{equation*}
aRb \iff |a-b| = 1
\end{equation*}
Напишите обоснованный ответ какими свойствами обладает или не обладает отношение и почему:   
\begin{enumerate} [a)]\setcounter{enumi}{0}
\item рефлексивность
\item антирефлексивность
\item симметричность
\item асимметричность
\item антисимметричность
\item транзитивность
\end{enumerate}

Обоснуйте свой ответ по каждому из приведенных ниже вопросов:
\begin{enumerate} [a)]\setcounter{enumi}{0}
    \item Является ли это отношение отношением эквивалентности?
    \item Является ли это отношение функциональным?
    \item Каким из отношений соответствия (одно-многозначным, много-многозначный и т.д.) оно является?
    \item К каким из отношений порядка (полного, частичного и т.д.) можно отнести данное отношение?
\end{enumerate}

\question
Установите, является ли каждое из перечисленных ниже отношений на А ($R \subseteq A \times A$) отношением эквивалентности (обоснование ответа обязательно). Для каждого отношения эквивалентности постройте классы 
эквивалентности и постройте граф отношения:
\begin{enumerate} [a)]\setcounter{enumi}{0}
\item $A = \{a, b, c, d, p, t\}$ задано отношение $R = \{(a, a), (b, b), (b, c), (b, d), (c, b), (c, c), (c, d), (d, b), (d, c), (d, d), (p,p), (t,t)\}$
\item $A = \{-10, -9, … , 9, 10\}$ и отношение $R = \{(a,b)|a^{3} = b^{3}\}$

\item $F(x)=x^{2}+1$, где $x \in A = [-2, 4]$ и отношение $R = \{(a,b)|F(a) = F(b)\}$
\end{enumerate}\question Составьте полную таблицу истинности, определите, какие переменные являются фиктивными и проверьте, является ли формула тавтологией:
$ P \rightarrow (Q \rightarrow ((P \lor Q) \rightarrow (P \land Q)))$

\end{questions}
\newpage
%%% begin test
\begin{flushright}
\begin{tabular}{p{2.8in} r l}
%\textbf{\class} & \textbf{ФИО:} & \makebox[2.5in]{\hrulefill}\\
\textbf{\class} & \textbf{ФИО:} &Мамин Илья Игоревич
\\

\textbf{\examdate} &&\\
%\textbf{Time Limit: \timelimit} & Teaching Assistant & \makebox[2in]{\hrulefill}
\end{tabular}\\
\end{flushright}
\rule[1ex]{\textwidth}{.1pt}


\begin{questions}
\question
Найдите и упростите P:
\begin{equation*}
\overline{P} = A \cap \overline{C} \cup A \cap \overline{B} \cup B \cap \overline{C} \cup A \cap C
\end{equation*}
Затем найдите элементы множества P, выраженного через множества:
\begin{equation*}
A = \{0, 3, 4, 9\}; 
B = \{1, 3, 4, 7\};
C = \{0, 1, 2, 4, 7, 8, 9\};
I = \{0, 1, 2, 3, 4, 5, 6, 7, 8, 9\}.
\end{equation*}\question
Упростите следующее выражение с учетом того, что $A\subset B \subset C \subset D \subset U; A \neq \O$
\begin{equation*}
\overline{B} \cap \overline{C} \cap D \cup \overline{A} \cap \overline{C} \cap D \cup \overline{A} \cap B
\end{equation*}

Примечание: U — универсум\question
Дано отношение на множестве $\{1, 2, 3, 4, 5\}$ 
\begin{equation*}
aRb \iff a \geq b^2
\end{equation*}
Напишите обоснованный ответ какими свойствами обладает или не обладает отношение и почему:   
\begin{enumerate} [a)]\setcounter{enumi}{0}
\item рефлексивность
\item антирефлексивность
\item симметричность
\item асимметричность
\item антисимметричность
\item транзитивность
\end{enumerate}

Обоснуйте свой ответ по каждому из приведенных ниже вопросов:
\begin{enumerate} [a)]\setcounter{enumi}{0}
    \item Является ли это отношение отношением эквивалентности?
    \item Является ли это отношение функциональным?
    \item Каким из отношений соответствия (одно-многозначным, много-многозначный и т.д.) оно является?
    \item К каким из отношений порядка (полного, частичного и т.д.) можно отнести данное отношение?
\end{enumerate}


\question
Установите, является ли каждое из перечисленных ниже отношений на А ($R \subseteq A \times A$) отношением эквивалентности (обоснование ответа обязательно). Для каждого отношения эквивалентности 
постройте классы эквивалентности и постройте граф отношения:
\begin{enumerate}[a)]\setcounter{enumi}{0}
\item А - множество целых чисел и отношение $R = \{(a,b)|a + b = 0\}$
\item $A = \{-10, -9, …, 9, 10\}$ и отношение $R = \{(a,b)|a^{3} = b^{3}\}$
\item На множестве $A = \{1; 2; 3\}$ задано отношение $R = \{(1; 1); (2; 2); (3; 3); (2; 1); (1; 2); (2; 3); (3; 2); (3; 1); (1; 3)\}$

\end{enumerate}\question Составьте полную таблицу истинности, определите, какие переменные являются фиктивными и проверьте, является ли формула тавтологией:
$(( P \land \neg Q) \rightarrow (R \land \neg R)) \rightarrow (P \rightarrow Q)$

\end{questions}
\newpage
%%% begin test
\begin{flushright}
\begin{tabular}{p{2.8in} r l}
%\textbf{\class} & \textbf{ФИО:} & \makebox[2.5in]{\hrulefill}\\
\textbf{\class} & \textbf{ФИО:} &Мирзабеков Ренат Эльмарович
\\

\textbf{\examdate} &&\\
%\textbf{Time Limit: \timelimit} & Teaching Assistant & \makebox[2in]{\hrulefill}
\end{tabular}\\
\end{flushright}
\rule[1ex]{\textwidth}{.1pt}


\begin{questions}
\question
Найдите и упростите P:
\begin{equation*}
\overline{P} = A \cap \overline{B} \cup A \cap C \cup B \cap C \cup \overline{A} \cap C
\end{equation*}
Затем найдите элементы множества P, выраженного через множества:
\begin{equation*}
A = \{0, 3, 4, 9\}; 
B = \{1, 3, 4, 7\};
C = \{0, 1, 2, 4, 7, 8, 9\};
I = \{0, 1, 2, 3, 4, 5, 6, 7, 8, 9\}.
\end{equation*}\question
Упростите следующее выражение с учетом того, что $A\subset B \subset C \subset D \subset U; A \neq \O$
\begin{equation*}
A \cap C  \cap D \cup B \cap \overline{C} \cap D \cup B \cap C \cap D
\end{equation*}

Примечание: U — универсум\question
Дано отношение на множестве $\{1, 2, 3, 4, 5\}$ 
\begin{equation*}
aRb \iff a \leq b
\end{equation*}
Напишите обоснованный ответ какими свойствами обладает или не обладает отношение и почему:   
\begin{enumerate} [a)]\setcounter{enumi}{0}
\item рефлексивность
\item антирефлексивность
\item симметричность
\item асимметричность
\item антисимметричность
\item транзитивность
\end{enumerate}

Обоснуйте свой ответ по каждому из приведенных ниже вопросов:
\begin{enumerate} [a)]\setcounter{enumi}{0}
    \item Является ли это отношение отношением эквивалентности?
    \item Является ли это отношение функциональным?
    \item Каким из отношений соответствия (одно-многозначным, много-многозначный и т.д.) оно является?
    \item К каким из отношений порядка (полного, частичного и т.д.) можно отнести данное отношение?
\end{enumerate}


\question
Установите, является ли каждое из перечисленных ниже отношений на А ($R \subseteq A \times A$) отношением эквивалентности (обоснование ответа обязательно). Для каждого отношения эквивалентности постройте классы 
эквивалентности и постройте граф отношения:
\begin{enumerate} [a)]\setcounter{enumi}{0}
\item На множестве $A = \{1; 2; 3\}$ задано отношение $R = \{(1; 1); (2; 2); (3; 3); (2; 1); (1; 2); (2; 3); (3; 2); (3; 1); (1; 3)\}$
\item На множестве $A = \{1; 2; 3; 4; 5\}$ задано отношение $R = \{(1; 2); (1; 3); (1; 5); (2; 3); (2; 4); (2; 5); (3; 4); (3; 5); (4; 5)\}$
\item А - множество целых чисел и отношение $R = \{(a,b)|a + b = 0\}$
\end{enumerate}\question Составьте полную таблицу истинности, определите, какие переменные являются фиктивными и проверьте, является ли формула тавтологией:
$((P \rightarrow Q) \lor R) \leftrightarrow (P \rightarrow (Q \lor R))$

\end{questions}
\newpage
%%% begin test
\begin{flushright}
\begin{tabular}{p{2.8in} r l}
%\textbf{\class} & \textbf{ФИО:} & \makebox[2.5in]{\hrulefill}\\
\textbf{\class} & \textbf{ФИО:} &Нгуен Туан Киет
\\

\textbf{\examdate} &&\\
%\textbf{Time Limit: \timelimit} & Teaching Assistant & \makebox[2in]{\hrulefill}
\end{tabular}\\
\end{flushright}
\rule[1ex]{\textwidth}{.1pt}


\begin{questions}
\question
Найдите и упростите P:
\begin{equation*}
\overline{P} = A \cap \overline{B} \cup \overline{B} \cap C \cup \overline{A} \cap \overline{B} \cup \overline{A} \cap C
\end{equation*}
Затем найдите элементы множества P, выраженного через множества:
\begin{equation*}
A = \{0, 3, 4, 9\}; 
B = \{1, 3, 4, 7\};
C = \{0, 1, 2, 4, 7, 8, 9\};
I = \{0, 1, 2, 3, 4, 5, 6, 7, 8, 9\}.
\end{equation*}\question
Упростите следующее выражение с учетом того, что $A\subset B \subset C \subset D \subset U; A \neq \O$
\begin{equation*}
A \cap C  \cap D \cup B \cap \overline{C} \cap D \cup B \cap C \cap D
\end{equation*}

Примечание: U — универсум\question
Дано отношение на множестве $\{1, 2, 3, 4, 5\}$ 
\begin{equation*}
aRb \iff a \leq b
\end{equation*}
Напишите обоснованный ответ какими свойствами обладает или не обладает отношение и почему:   
\begin{enumerate} [a)]\setcounter{enumi}{0}
\item рефлексивность
\item антирефлексивность
\item симметричность
\item асимметричность
\item антисимметричность
\item транзитивность
\end{enumerate}

Обоснуйте свой ответ по каждому из приведенных ниже вопросов:
\begin{enumerate} [a)]\setcounter{enumi}{0}
    \item Является ли это отношение отношением эквивалентности?
    \item Является ли это отношение функциональным?
    \item Каким из отношений соответствия (одно-многозначным, много-многозначный и т.д.) оно является?
    \item К каким из отношений порядка (полного, частичного и т.д.) можно отнести данное отношение?
\end{enumerate}


\question
Установите, является ли каждое из перечисленных ниже отношений на А ($R \subseteq A \times A$) отношением эквивалентности (обоснование ответа обязательно). Для каждого отношения эквивалентности постройте классы 
эквивалентности и постройте граф отношения:
\begin{enumerate} [a)]\setcounter{enumi}{0}
\item А - множество целых чисел и отношение $R = \{(a,b)|a + b = 5\}$
\item Пусть A – множество имен. $A = \{ $Алексей, Иван, Петр, Александр, Павел, Андрей$ \}$. Тогда отношение $R $ верно на парах имен, начинающихся с одной и той же буквы, и только на них.
\item На множестве $A = \{1; 2; 3; 4; 5\}$ задано отношение $R = \{(1; 2); (1; 3); (1; 5); (2; 3); (2; 4); (2; 5); (3; 4); (3; 5); (4; 5)\}$
\end{enumerate}\question Составьте полную таблицу истинности, определите, какие переменные являются фиктивными и проверьте, является ли формула тавтологией:
$(( P \rightarrow Q) \land (Q \rightarrow P)) \rightarrow (P \rightarrow R)$

\end{questions}
\newpage
%%% begin test
\begin{flushright}
\begin{tabular}{p{2.8in} r l}
%\textbf{\class} & \textbf{ФИО:} & \makebox[2.5in]{\hrulefill}\\
\textbf{\class} & \textbf{ФИО:} &Пелевин Евгений Николаевич
\\

\textbf{\examdate} &&\\
%\textbf{Time Limit: \timelimit} & Teaching Assistant & \makebox[2in]{\hrulefill}
\end{tabular}\\
\end{flushright}
\rule[1ex]{\textwidth}{.1pt}


\begin{questions}
\question
Найдите и упростите P:
\begin{equation*}
\overline{P} = B \cap \overline{C} \cup A \cap B \cup \overline{A} \cap C \cup \overline{A} \cap B
\end{equation*}
Затем найдите элементы множества P, выраженного через множества:
\begin{equation*}
A = \{0, 3, 4, 9\}; 
B = \{1, 3, 4, 7\};
C = \{0, 1, 2, 4, 7, 8, 9\};
I = \{0, 1, 2, 3, 4, 5, 6, 7, 8, 9\}.
\end{equation*}\question
Упростите следующее выражение с учетом того, что $A\subset B \subset C \subset D \subset U; A \neq \O$
\begin{equation*}
\overline{A} \cap \overline{B} \cup B \cap \overline{C} \cup \overline{C} \cap D
\end{equation*}

Примечание: U — универсум\question
Дано отношение на множестве $\{1, 2, 3, 4, 5\}$ 
\begin{equation*}
aRb \iff b > a
\end{equation*}
Напишите обоснованный ответ какими свойствами обладает или не обладает отношение и почему:   
\begin{enumerate} [a)]\setcounter{enumi}{0}
\item рефлексивность
\item антирефлексивность
\item симметричность
\item асимметричность
\item антисимметричность
\item транзитивность
\end{enumerate}

Обоснуйте свой ответ по каждому из приведенных ниже вопросов:
\begin{enumerate} [a)]\setcounter{enumi}{0}
    \item Является ли это отношение отношением эквивалентности?
    \item Является ли это отношение функциональным?
    \item Каким из отношений соответствия (одно-многозначным, много-многозначный и т.д.) оно является?
    \item К каким из отношений порядка (полного, частичного и т.д.) можно отнести данное отношение?
\end{enumerate}

\question
Установите, является ли каждое из перечисленных ниже отношений на А ($R \subseteq A \times A$) отношением эквивалентности (обоснование ответа обязательно). Для каждого отношения эквивалентности постройте классы 
эквивалентности и постройте граф отношения:
\begin{enumerate} [a)]\setcounter{enumi}{0}
\item $A = \{a, b, c, d, p, t\}$ задано отношение $R = \{(a, a), (b, b), (b, c), (b, d), (c, b), (c, c), (c, d), (d, b), (d, c), (d, d), (p,p), (t,t)\}$
\item $A = \{-10, -9, … , 9, 10\}$ и отношение $R = \{(a,b)|a^{3} = b^{3}\}$

\item $F(x)=x^{2}+1$, где $x \in A = [-2, 4]$ и отношение $R = \{(a,b)|F(a) = F(b)\}$
\end{enumerate}\question Составьте полную таблицу истинности, определите, какие переменные являются фиктивными и проверьте, является ли формула тавтологией:
$((P \rightarrow Q) \lor R) \leftrightarrow (P \rightarrow (Q \lor R))$

\end{questions}
\newpage
%%% begin test
\begin{flushright}
\begin{tabular}{p{2.8in} r l}
%\textbf{\class} & \textbf{ФИО:} & \makebox[2.5in]{\hrulefill}\\
\textbf{\class} & \textbf{ФИО:} &Сердитов Максим Анатольевич
\\

\textbf{\examdate} &&\\
%\textbf{Time Limit: \timelimit} & Teaching Assistant & \makebox[2in]{\hrulefill}
\end{tabular}\\
\end{flushright}
\rule[1ex]{\textwidth}{.1pt}


\begin{questions}
\question
Найдите и упростите P:
\begin{equation*}
\overline{P} = A \cap \overline{B} \cup A \cap C \cup B \cap C \cup \overline{A} \cap C
\end{equation*}
Затем найдите элементы множества P, выраженного через множества:
\begin{equation*}
A = \{0, 3, 4, 9\}; 
B = \{1, 3, 4, 7\};
C = \{0, 1, 2, 4, 7, 8, 9\};
I = \{0, 1, 2, 3, 4, 5, 6, 7, 8, 9\}.
\end{equation*}\question
Упростите следующее выражение с учетом того, что $A\subset B \subset C \subset D \subset U; A \neq \O$
\begin{equation*}
\overline{B} \cap \overline{C} \cap D \cup \overline{A} \cap \overline{C} \cap D \cup \overline{A} \cap B
\end{equation*}

Примечание: U — универсум\question
Дано отношение на множестве $\{1, 2, 3, 4, 5\}$ 
\begin{equation*}
aRb \iff a \leq b
\end{equation*}
Напишите обоснованный ответ какими свойствами обладает или не обладает отношение и почему:   
\begin{enumerate} [a)]\setcounter{enumi}{0}
\item рефлексивность
\item антирефлексивность
\item симметричность
\item асимметричность
\item антисимметричность
\item транзитивность
\end{enumerate}

Обоснуйте свой ответ по каждому из приведенных ниже вопросов:
\begin{enumerate} [a)]\setcounter{enumi}{0}
    \item Является ли это отношение отношением эквивалентности?
    \item Является ли это отношение функциональным?
    \item Каким из отношений соответствия (одно-многозначным, много-многозначный и т.д.) оно является?
    \item К каким из отношений порядка (полного, частичного и т.д.) можно отнести данное отношение?
\end{enumerate}


\question
Установите, является ли каждое из перечисленных ниже отношений на А ($R \subseteq A \times A$) отношением эквивалентности (обоснование ответа обязательно). Для каждого отношения эквивалентности постройте классы 
эквивалентности и постройте граф отношения:
\begin{enumerate} [a)]\setcounter{enumi}{0}
\item На множестве $A = \{1; 2; 3\}$ задано отношение $R = \{(1; 1); (2; 2); (3; 3); (2; 1); (1; 2); (2; 3); (3; 2); (3; 1); (1; 3)\}$
\item На множестве $A = \{1; 2; 3; 4; 5\}$ задано отношение $R = \{(1; 2); (1; 3); (1; 5); (2; 3); (2; 4); (2; 5); (3; 4); (3; 5); (4; 5)\}$
\item А - множество целых чисел и отношение $R = \{(a,b)|a + b = 0\}$
\end{enumerate}\question Составьте полную таблицу истинности, определите, какие переменные являются фиктивными и проверьте, является ли формула тавтологией:
$(( P \rightarrow Q) \land (Q \rightarrow P)) \rightarrow (P \rightarrow R)$

\end{questions}
\newpage
%%% begin test
\begin{flushright}
\begin{tabular}{p{2.8in} r l}
%\textbf{\class} & \textbf{ФИО:} & \makebox[2.5in]{\hrulefill}\\
\textbf{\class} & \textbf{ФИО:} &Сорокин Андрей Анатольевич
\\

\textbf{\examdate} &&\\
%\textbf{Time Limit: \timelimit} & Teaching Assistant & \makebox[2in]{\hrulefill}
\end{tabular}\\
\end{flushright}
\rule[1ex]{\textwidth}{.1pt}


\begin{questions}
\question
Найдите и упростите P:
\begin{equation*}
\overline{P} = \overline{A} \cap B \cup \overline{A} \cap C \cup A \cap \overline{B} \cup \overline{B} \cap C
\end{equation*}
Затем найдите элементы множества P, выраженного через множества:
\begin{equation*}
A = \{0, 3, 4, 9\}; 
B = \{1, 3, 4, 7\};
C = \{0, 1, 2, 4, 7, 8, 9\};
I = \{0, 1, 2, 3, 4, 5, 6, 7, 8, 9\}.
\end{equation*}\question
Упростите следующее выражение с учетом того, что $A\subset B \subset C \subset D \subset U; A \neq \O$
\begin{equation*}
A \cap  \overline{C} \cup B \cap \overline{D} \cup  \overline{A} \cap C \cap  \overline{D}
\end{equation*}

Примечание: U — универсум\question
Дано отношение на множестве $\{1, 2, 3, 4, 5\}$ 
\begin{equation*}
aRb \iff a \leq b
\end{equation*}
Напишите обоснованный ответ какими свойствами обладает или не обладает отношение и почему:   
\begin{enumerate} [a)]\setcounter{enumi}{0}
\item рефлексивность
\item антирефлексивность
\item симметричность
\item асимметричность
\item антисимметричность
\item транзитивность
\end{enumerate}

Обоснуйте свой ответ по каждому из приведенных ниже вопросов:
\begin{enumerate} [a)]\setcounter{enumi}{0}
    \item Является ли это отношение отношением эквивалентности?
    \item Является ли это отношение функциональным?
    \item Каким из отношений соответствия (одно-многозначным, много-многозначный и т.д.) оно является?
    \item К каким из отношений порядка (полного, частичного и т.д.) можно отнести данное отношение?
\end{enumerate}


\question
Установите, является ли каждое из перечисленных ниже отношений на А ($R \subseteq A \times A$) отношением эквивалентности (обоснование ответа обязательно). Для каждого отношения эквивалентности постройте классы 
эквивалентности и постройте граф отношения:
\begin{enumerate} [a)]\setcounter{enumi}{0}
\item А - множество целых чисел и отношение $R = \{(a,b)|a + b = 5\}$
\item Пусть A – множество имен. $A = \{ $Алексей, Иван, Петр, Александр, Павел, Андрей$ \}$. Тогда отношение $R $ верно на парах имен, начинающихся с одной и той же буквы, и только на них.
\item На множестве $A = \{1; 2; 3; 4; 5\}$ задано отношение $R = \{(1; 2); (1; 3); (1; 5); (2; 3); (2; 4); (2; 5); (3; 4); (3; 5); (4; 5)\}$
\end{enumerate}\question Составьте полную таблицу истинности, определите, какие переменные являются фиктивными и проверьте, является ли формула тавтологией:

$(P \rightarrow (Q \land R)) \leftrightarrow ((P \rightarrow Q) \land (P \rightarrow R))$

\end{questions}
\newpage
%%% begin test
\begin{flushright}
\begin{tabular}{p{2.8in} r l}
%\textbf{\class} & \textbf{ФИО:} & \makebox[2.5in]{\hrulefill}\\
\textbf{\class} & \textbf{ФИО:} &Стрельцов Алексей Павлович
\\

\textbf{\examdate} &&\\
%\textbf{Time Limit: \timelimit} & Teaching Assistant & \makebox[2in]{\hrulefill}
\end{tabular}\\
\end{flushright}
\rule[1ex]{\textwidth}{.1pt}


\begin{questions}
\question
Найдите и упростите P:
\begin{equation*}
\overline{P} = A \cap \overline{B} \cup \overline{B} \cap C \cup \overline{A} \cap \overline{B} \cup \overline{A} \cap C
\end{equation*}
Затем найдите элементы множества P, выраженного через множества:
\begin{equation*}
A = \{0, 3, 4, 9\}; 
B = \{1, 3, 4, 7\};
C = \{0, 1, 2, 4, 7, 8, 9\};
I = \{0, 1, 2, 3, 4, 5, 6, 7, 8, 9\}.
\end{equation*}\question
Упростите следующее выражение с учетом того, что $A\subset B \subset C \subset D \subset U; A \neq \O$
\begin{equation*}
\overline{A} \cap \overline{B} \cup B \cap \overline{C} \cup \overline{C} \cap D
\end{equation*}

Примечание: U — универсум\question
Дано отношение на множестве $\{1, 2, 3, 4, 5\}$ 
\begin{equation*}
aRb \iff a \geq b^2
\end{equation*}
Напишите обоснованный ответ какими свойствами обладает или не обладает отношение и почему:   
\begin{enumerate} [a)]\setcounter{enumi}{0}
\item рефлексивность
\item антирефлексивность
\item симметричность
\item асимметричность
\item антисимметричность
\item транзитивность
\end{enumerate}

Обоснуйте свой ответ по каждому из приведенных ниже вопросов:
\begin{enumerate} [a)]\setcounter{enumi}{0}
    \item Является ли это отношение отношением эквивалентности?
    \item Является ли это отношение функциональным?
    \item Каким из отношений соответствия (одно-многозначным, много-многозначный и т.д.) оно является?
    \item К каким из отношений порядка (полного, частичного и т.д.) можно отнести данное отношение?
\end{enumerate}


\question
Установите, является ли каждое из перечисленных ниже отношений на А ($R \subseteq A \times A$) отношением эквивалентности (обоснование ответа обязательно). Для каждого отношения эквивалентности постройте классы 
эквивалентности и постройте граф отношения:
\begin{enumerate} [a)]\setcounter{enumi}{0}
\item $A = \{-10, -9, … , 9, 10\}$ и отношение $R = \{(a,b)|a^{2} = b^{2}\}$
\item $A = \{a, b, c, d, p, t\}$ задано отношение $R = \{(a, a), (b, b), (b, c), (b, d), (c, b), (c, c), (c, d), (d, b), (d, c), (d, d), (p,p), (t,t)\}$
\item Пусть A – множество имен. $A = \{ $Алексей, Иван, Петр, Александр, Павел, Андрей$ \}$. Тогда отношение $R$ верно на парах имен, начинающихся с одной и той же буквы, и только на них.
\end{enumerate}\question Составьте полную таблицу истинности, определите, какие переменные являются фиктивными и проверьте, является ли формула тавтологией:
$((P \rightarrow Q) \land (R \rightarrow S) \land \neg (Q \lor S)) \rightarrow \neg (P \lor R)$

\end{questions}
\newpage
%%% begin test
\begin{flushright}
\begin{tabular}{p{2.8in} r l}
%\textbf{\class} & \textbf{ФИО:} & \makebox[2.5in]{\hrulefill}\\
\textbf{\class} & \textbf{ФИО:} &Тарасов Михаил Евгеньевич
\\

\textbf{\examdate} &&\\
%\textbf{Time Limit: \timelimit} & Teaching Assistant & \makebox[2in]{\hrulefill}
\end{tabular}\\
\end{flushright}
\rule[1ex]{\textwidth}{.1pt}


\begin{questions}
\question
Найдите и упростите P:
\begin{equation*}
\overline{P} = \overline{A} \cap B \cup \overline{A} \cap C \cup A \cap \overline{B} \cup \overline{B} \cap C
\end{equation*}
Затем найдите элементы множества P, выраженного через множества:
\begin{equation*}
A = \{0, 3, 4, 9\}; 
B = \{1, 3, 4, 7\};
C = \{0, 1, 2, 4, 7, 8, 9\};
I = \{0, 1, 2, 3, 4, 5, 6, 7, 8, 9\}.
\end{equation*}\question
Упростите следующее выражение с учетом того, что $A\subset B \subset C \subset D \subset U; A \neq \O$
\begin{equation*}
\overline{A} \cap \overline{B} \cup B \cap \overline{C} \cup \overline{C} \cap D
\end{equation*}

Примечание: U — универсум\question
Дано отношение на множестве $\{1, 2, 3, 4, 5\}$ 
\begin{equation*}
aRb \iff |a-b| = 1
\end{equation*}
Напишите обоснованный ответ какими свойствами обладает или не обладает отношение и почему:   
\begin{enumerate} [a)]\setcounter{enumi}{0}
\item рефлексивность
\item антирефлексивность
\item симметричность
\item асимметричность
\item антисимметричность
\item транзитивность
\end{enumerate}

Обоснуйте свой ответ по каждому из приведенных ниже вопросов:
\begin{enumerate} [a)]\setcounter{enumi}{0}
    \item Является ли это отношение отношением эквивалентности?
    \item Является ли это отношение функциональным?
    \item Каким из отношений соответствия (одно-многозначным, много-многозначный и т.д.) оно является?
    \item К каким из отношений порядка (полного, частичного и т.д.) можно отнести данное отношение?
\end{enumerate}

\question
Установите, является ли каждое из перечисленных ниже отношений на А ($R \subseteq A \times A$) отношением эквивалентности (обоснование ответа обязательно). Для каждого отношения эквивалентности постройте классы 
эквивалентности и постройте граф отношения:
\begin{enumerate} [a)]\setcounter{enumi}{0}
\item А - множество целых чисел и отношение $R = \{(a,b)|a + b = 5\}$
\item Пусть A – множество имен. $A = \{ $Алексей, Иван, Петр, Александр, Павел, Андрей$ \}$. Тогда отношение $R $ верно на парах имен, начинающихся с одной и той же буквы, и только на них.
\item На множестве $A = \{1; 2; 3; 4; 5\}$ задано отношение $R = \{(1; 2); (1; 3); (1; 5); (2; 3); (2; 4); (2; 5); (3; 4); (3; 5); (4; 5)\}$
\end{enumerate}\question Составьте полную таблицу истинности, определите, какие переменные являются фиктивными и проверьте, является ли формула тавтологией:
$((P \rightarrow Q) \lor R) \leftrightarrow (P \rightarrow (Q \lor R))$

\end{questions}
\newpage
%%% begin test
\begin{flushright}
\begin{tabular}{p{2.8in} r l}
%\textbf{\class} & \textbf{ФИО:} & \makebox[2.5in]{\hrulefill}\\
\textbf{\class} & \textbf{ФИО:} &Трущев Сергей Андреевич
\\

\textbf{\examdate} &&\\
%\textbf{Time Limit: \timelimit} & Teaching Assistant & \makebox[2in]{\hrulefill}
\end{tabular}\\
\end{flushright}
\rule[1ex]{\textwidth}{.1pt}


\begin{questions}
\question
Найдите и упростите P:
\begin{equation*}
\overline{P} = A \cap C \cup \overline{A} \cap \overline{C} \cup \overline{B} \cap C \cup \overline{A} \cap \overline{B}
\end{equation*}
Затем найдите элементы множества P, выраженного через множества:
\begin{equation*}
A = \{0, 3, 4, 9\}; 
B = \{1, 3, 4, 7\};
C = \{0, 1, 2, 4, 7, 8, 9\};
I = \{0, 1, 2, 3, 4, 5, 6, 7, 8, 9\}.
\end{equation*}\question
Упростите следующее выражение с учетом того, что $A\subset B \subset C \subset D \subset U; A \neq \O$
\begin{equation*}
A \cap B \cup \overline{A} \cap \overline{C} \cup A \cap C \cup \overline{B} \cap \overline{C}
\end{equation*}

Примечание: U — универсум\question
Дано отношение на множестве $\{1, 2, 3, 4, 5\}$ 
\begin{equation*}
aRb \iff  \text{НОД}(a,b) =1
\end{equation*}
Напишите обоснованный ответ какими свойствами обладает или не обладает отношение и почему:   
\begin{enumerate} [a)]\setcounter{enumi}{0}
\item рефлексивность
\item антирефлексивность
\item симметричность
\item асимметричность
\item антисимметричность
\item транзитивность
\end{enumerate}

Обоснуйте свой ответ по каждому из приведенных ниже вопросов:
\begin{enumerate} [a)]\setcounter{enumi}{0}
    \item Является ли это отношение отношением эквивалентности?
    \item Является ли это отношение функциональным?
    \item Каким из отношений соответствия (одно-многозначным, много-многозначный и т.д.) оно является?
    \item К каким из отношений порядка (полного, частичного и т.д.) можно отнести данное отношение?
\end{enumerate}


\question
Установите, является ли каждое из перечисленных ниже отношений на А ($R \subseteq A \times A$) отношением эквивалентности (обоснование ответа обязательно). Для каждого отношения эквивалентности постройте классы 
эквивалентности и постройте граф отношения:
\begin{enumerate} [a)]\setcounter{enumi}{0}
\item А - множество целых чисел и отношение $R = \{(a,b)|a + b = 5\}$
\item Пусть A – множество имен. $A = \{ $Алексей, Иван, Петр, Александр, Павел, Андрей$ \}$. Тогда отношение $R $ верно на парах имен, начинающихся с одной и той же буквы, и только на них.
\item На множестве $A = \{1; 2; 3; 4; 5\}$ задано отношение $R = \{(1; 2); (1; 3); (1; 5); (2; 3); (2; 4); (2; 5); (3; 4); (3; 5); (4; 5)\}$
\end{enumerate}\question Составьте полную таблицу истинности, определите, какие переменные являются фиктивными и проверьте, является ли формула тавтологией:
$(( P \rightarrow Q) \land (Q \rightarrow P)) \rightarrow (P \rightarrow R)$

\end{questions}
\newpage
%%% begin test
\begin{flushright}
\begin{tabular}{p{2.8in} r l}
%\textbf{\class} & \textbf{ФИО:} & \makebox[2.5in]{\hrulefill}\\
\textbf{\class} & \textbf{ФИО:} &Филипцев Дмитрий Витальевич
\\

\textbf{\examdate} &&\\
%\textbf{Time Limit: \timelimit} & Teaching Assistant & \makebox[2in]{\hrulefill}
\end{tabular}\\
\end{flushright}
\rule[1ex]{\textwidth}{.1pt}


\begin{questions}
\question
Найдите и упростите P:
\begin{equation*}
\overline{P} = A \cap C \cup \overline{A} \cap \overline{C} \cup \overline{B} \cap C \cup \overline{A} \cap \overline{B}
\end{equation*}
Затем найдите элементы множества P, выраженного через множества:
\begin{equation*}
A = \{0, 3, 4, 9\}; 
B = \{1, 3, 4, 7\};
C = \{0, 1, 2, 4, 7, 8, 9\};
I = \{0, 1, 2, 3, 4, 5, 6, 7, 8, 9\}.
\end{equation*}\question
Упростите следующее выражение с учетом того, что $A\subset B \subset C \subset D \subset U; A \neq \O$
\begin{equation*}
A \cap  \overline{C} \cup B \cap \overline{D} \cup  \overline{A} \cap C \cap  \overline{D}
\end{equation*}

Примечание: U — универсум\question
Дано отношение на множестве $\{1, 2, 3, 4, 5\}$ 
\begin{equation*}
aRb \iff |a-b| = 1
\end{equation*}
Напишите обоснованный ответ какими свойствами обладает или не обладает отношение и почему:   
\begin{enumerate} [a)]\setcounter{enumi}{0}
\item рефлексивность
\item антирефлексивность
\item симметричность
\item асимметричность
\item антисимметричность
\item транзитивность
\end{enumerate}

Обоснуйте свой ответ по каждому из приведенных ниже вопросов:
\begin{enumerate} [a)]\setcounter{enumi}{0}
    \item Является ли это отношение отношением эквивалентности?
    \item Является ли это отношение функциональным?
    \item Каким из отношений соответствия (одно-многозначным, много-многозначный и т.д.) оно является?
    \item К каким из отношений порядка (полного, частичного и т.д.) можно отнести данное отношение?
\end{enumerate}

\question
Установите, является ли каждое из перечисленных ниже отношений на А ($R \subseteq A \times A$) отношением эквивалентности (обоснование ответа обязательно). Для каждого отношения эквивалентности постройте классы 
эквивалентности и постройте граф отношения:
\begin{enumerate} [a)]\setcounter{enumi}{0}
\item $A = \{-10, -9, … , 9, 10\}$ и отношение $R = \{(a,b)|a^{2} = b^{2}\}$
\item $A = \{a, b, c, d, p, t\}$ задано отношение $R = \{(a, a), (b, b), (b, c), (b, d), (c, b), (c, c), (c, d), (d, b), (d, c), (d, d), (p,p), (t,t)\}$
\item Пусть A – множество имен. $A = \{ $Алексей, Иван, Петр, Александр, Павел, Андрей$ \}$. Тогда отношение $R$ верно на парах имен, начинающихся с одной и той же буквы, и только на них.
\end{enumerate}\question Составьте полную таблицу истинности, определите, какие переменные являются фиктивными и проверьте, является ли формула тавтологией:
$(P \rightarrow (Q \rightarrow R)) \rightarrow ((P \rightarrow Q) \rightarrow (P \rightarrow R))$

\end{questions}
\newpage
%%% begin test
\begin{flushright}
\begin{tabular}{p{2.8in} r l}
%\textbf{\class} & \textbf{ФИО:} & \makebox[2.5in]{\hrulefill}\\
\textbf{\class} & \textbf{ФИО:} &Фомкин Никита Сергеевич
\\

\textbf{\examdate} &&\\
%\textbf{Time Limit: \timelimit} & Teaching Assistant & \makebox[2in]{\hrulefill}
\end{tabular}\\
\end{flushright}
\rule[1ex]{\textwidth}{.1pt}


\begin{questions}
\question
Найдите и упростите P:
\begin{equation*}
\overline{P} = A \cap B \cup \overline{A} \cap \overline{B} \cup A \cap C \cup \overline{B} \cap C
\end{equation*}
Затем найдите элементы множества P, выраженного через множества:
\begin{equation*}
A = \{0, 3, 4, 9\}; 
B = \{1, 3, 4, 7\};
C = \{0, 1, 2, 4, 7, 8, 9\};
I = \{0, 1, 2, 3, 4, 5, 6, 7, 8, 9\}.
\end{equation*}\question
Упростите следующее выражение с учетом того, что $A\subset B \subset C \subset D \subset U; A \neq \O$
\begin{equation*}
A \cap B  \cap \overline{C} \cup \overline{C} \cap D \cup B \cap C \cap D
\end{equation*}

Примечание: U — универсум\question
Дано отношение на множестве $\{1, 2, 3, 4, 5\}$ 
\begin{equation*}
aRb \iff a \geq b^2
\end{equation*}
Напишите обоснованный ответ какими свойствами обладает или не обладает отношение и почему:   
\begin{enumerate} [a)]\setcounter{enumi}{0}
\item рефлексивность
\item антирефлексивность
\item симметричность
\item асимметричность
\item антисимметричность
\item транзитивность
\end{enumerate}

Обоснуйте свой ответ по каждому из приведенных ниже вопросов:
\begin{enumerate} [a)]\setcounter{enumi}{0}
    \item Является ли это отношение отношением эквивалентности?
    \item Является ли это отношение функциональным?
    \item Каким из отношений соответствия (одно-многозначным, много-многозначный и т.д.) оно является?
    \item К каким из отношений порядка (полного, частичного и т.д.) можно отнести данное отношение?
\end{enumerate}


\question
Установите, является ли каждое из перечисленных ниже отношений на А ($R \subseteq A \times A$) отношением эквивалентности (обоснование ответа обязательно). Для каждого отношения эквивалентности постройте классы 
эквивалентности и постройте граф отношения:
\begin{enumerate} [a)]\setcounter{enumi}{0}
\item На множестве $A = \{1; 2; 3\}$ задано отношение $R = \{(1; 1); (2; 2); (3; 3); (2; 1); (1; 2); (2; 3); (3; 2); (3; 1); (1; 3)\}$
\item На множестве $A = \{1; 2; 3; 4; 5\}$ задано отношение $R = \{(1; 2); (1; 3); (1; 5); (2; 3); (2; 4); (2; 5); (3; 4); (3; 5); (4; 5)\}$
\item А - множество целых чисел и отношение $R = \{(a,b)|a + b = 0\}$
\end{enumerate}\question Составьте полную таблицу истинности, определите, какие переменные являются фиктивными и проверьте, является ли формула тавтологией:
$(( P \land \neg Q) \rightarrow (R \land \neg R)) \rightarrow (P \rightarrow Q)$

\end{questions}
\newpage
%%% begin test
\begin{flushright}
\begin{tabular}{p{2.8in} r l}
%\textbf{\class} & \textbf{ФИО:} & \makebox[2.5in]{\hrulefill}\\
\textbf{\class} & \textbf{ФИО:} &Хаваева Алина Владимировна
\\

\textbf{\examdate} &&\\
%\textbf{Time Limit: \timelimit} & Teaching Assistant & \makebox[2in]{\hrulefill}
\end{tabular}\\
\end{flushright}
\rule[1ex]{\textwidth}{.1pt}


\begin{questions}
\question
Найдите и упростите P:
\begin{equation*}
\overline{P} = \overline{A} \cap B \cup \overline{A} \cap C \cup A \cap \overline{B} \cup \overline{B} \cap C
\end{equation*}
Затем найдите элементы множества P, выраженного через множества:
\begin{equation*}
A = \{0, 3, 4, 9\}; 
B = \{1, 3, 4, 7\};
C = \{0, 1, 2, 4, 7, 8, 9\};
I = \{0, 1, 2, 3, 4, 5, 6, 7, 8, 9\}.
\end{equation*}\question
Упростите следующее выражение с учетом того, что $A\subset B \subset C \subset D \subset U; A \neq \O$
\begin{equation*}
A \cap B \cup \overline{A} \cap \overline{C} \cup A \cap C \cup \overline{B} \cap \overline{C}
\end{equation*}

Примечание: U — универсум\question
Дано отношение на множестве $\{1, 2, 3, 4, 5\}$ 
\begin{equation*}
aRb \iff  \text{НОД}(a,b) =1
\end{equation*}
Напишите обоснованный ответ какими свойствами обладает или не обладает отношение и почему:   
\begin{enumerate} [a)]\setcounter{enumi}{0}
\item рефлексивность
\item антирефлексивность
\item симметричность
\item асимметричность
\item антисимметричность
\item транзитивность
\end{enumerate}

Обоснуйте свой ответ по каждому из приведенных ниже вопросов:
\begin{enumerate} [a)]\setcounter{enumi}{0}
    \item Является ли это отношение отношением эквивалентности?
    \item Является ли это отношение функциональным?
    \item Каким из отношений соответствия (одно-многозначным, много-многозначный и т.д.) оно является?
    \item К каким из отношений порядка (полного, частичного и т.д.) можно отнести данное отношение?
\end{enumerate}


\question
Установите, является ли каждое из перечисленных ниже отношений на А ($R \subseteq A \times A$) отношением эквивалентности (обоснование ответа обязательно). Для каждого отношения эквивалентности постройте классы 
эквивалентности и постройте граф отношения:
\begin{enumerate} [a)]\setcounter{enumi}{0}
\item Пусть A – множество имен. $A = \{ $Алексей, Иван, Петр, Александр, Павел, Андрей$ \}$. Тогда отношение $R$ верно на парах имен, начинающихся с одной и той же буквы, и только на них.
\item $A = \{-10, -9, … , 9, 10\}$ и отношение $ R = \{(a,b)|a^{2} = b^{2}\}$
\item На множестве $A = \{1; 2; 3\}$ задано отношение $R = \{(1; 1); (2; 2); (3; 3); (3; 2); (1; 2); (2; 1)\}$
\end{enumerate}\question Составьте полную таблицу истинности, определите, какие переменные являются фиктивными и проверьте, является ли формула тавтологией:
$((P \rightarrow Q) \lor R) \leftrightarrow (P \rightarrow (Q \lor R))$

\end{questions}
\newpage
%%% begin test
\begin{flushright}
\begin{tabular}{p{2.8in} r l}
%\textbf{\class} & \textbf{ФИО:} & \makebox[2.5in]{\hrulefill}\\
\textbf{\class} & \textbf{ФИО:} &Цыденов Алексей Гомбоевич
\\

\textbf{\examdate} &&\\
%\textbf{Time Limit: \timelimit} & Teaching Assistant & \makebox[2in]{\hrulefill}
\end{tabular}\\
\end{flushright}
\rule[1ex]{\textwidth}{.1pt}


\begin{questions}
\question
Найдите и упростите P:
\begin{equation*}
\overline{P} = A \cap \overline{C} \cup A \cap \overline{B} \cup B \cap \overline{C} \cup A \cap C
\end{equation*}
Затем найдите элементы множества P, выраженного через множества:
\begin{equation*}
A = \{0, 3, 4, 9\}; 
B = \{1, 3, 4, 7\};
C = \{0, 1, 2, 4, 7, 8, 9\};
I = \{0, 1, 2, 3, 4, 5, 6, 7, 8, 9\}.
\end{equation*}\question
Упростите следующее выражение с учетом того, что $A\subset B \subset C \subset D \subset U; A \neq \O$
\begin{equation*}
\overline{A} \cap \overline{B} \cup B \cap \overline{C} \cup \overline{C} \cap D
\end{equation*}

Примечание: U — универсум\question
Для следующего отношения на множестве $\{1, 2, 3, 4, 5\}$ 
\begin{equation*}
aRb \iff 0 < a-b<2
\end{equation*}
Напишите обоснованный ответ какими свойствами обладает или не обладает отношение и почему:   
\begin{enumerate} [a)]\setcounter{enumi}{0}
\item рефлексивность
\item антирефлексивность
\item симметричность
\item асимметричность
\item антисимметричность
\item транзитивность
\end{enumerate}

Обоснуйте свой ответ по каждому из приведенных ниже вопросов:
\begin{enumerate} [a)]\setcounter{enumi}{0}
    \item Является ли это отношение отношением эквивалентности?
    \item Является ли это отношение функциональным?
    \item Каким из отношений соответствия (одно-многозначным, много-многозначный и т.д.) оно является?
    \item К каким из отношений порядка (полного, частичного и т.д.) можно отнести данное отношение?
\end{enumerate}
\question
Установите, является ли каждое из перечисленных ниже отношений на А ($R \subseteq A \times A$) отношением эквивалентности (обоснование ответа обязательно). Для каждого отношения эквивалентности постройте классы 
эквивалентности и постройте граф отношения:
\begin{enumerate} [a)]\setcounter{enumi}{0}
\item $A = \{a, b, c, d, p, t\}$ задано отношение $R = \{(a, a), (b, b), (b, c), (b, d), (c, b), (c, c), (c, d), (d, b), (d, c), (d, d), (p,p), (t,t)\}$
\item $A = \{-10, -9, … , 9, 10\}$ и отношение $R = \{(a,b)|a^{3} = b^{3}\}$

\item $F(x)=x^{2}+1$, где $x \in A = [-2, 4]$ и отношение $R = \{(a,b)|F(a) = F(b)\}$
\end{enumerate}\question Составьте полную таблицу истинности, определите, какие переменные являются фиктивными и проверьте, является ли формула тавтологией:
$(( P \rightarrow Q) \land (Q \rightarrow P)) \rightarrow (P \rightarrow R)$

\end{questions}
\newpage
%%% begin test
\begin{flushright}
\begin{tabular}{p{2.8in} r l}
%\textbf{\class} & \textbf{ФИО:} & \makebox[2.5in]{\hrulefill}\\
\textbf{\class} & \textbf{ФИО:} &Шлегель Александр Ярославович
\\

\textbf{\examdate} &&\\
%\textbf{Time Limit: \timelimit} & Teaching Assistant & \makebox[2in]{\hrulefill}
\end{tabular}\\
\end{flushright}
\rule[1ex]{\textwidth}{.1pt}


\begin{questions}
\question
Найдите и упростите P:
\begin{equation*}
\overline{P} = A \cap \overline{B} \cup \overline{B} \cap C \cup \overline{A} \cap \overline{B} \cup \overline{A} \cap C
\end{equation*}
Затем найдите элементы множества P, выраженного через множества:
\begin{equation*}
A = \{0, 3, 4, 9\}; 
B = \{1, 3, 4, 7\};
C = \{0, 1, 2, 4, 7, 8, 9\};
I = \{0, 1, 2, 3, 4, 5, 6, 7, 8, 9\}.
\end{equation*}\question
Упростите следующее выражение с учетом того, что $A\subset B \subset C \subset D \subset U; A \neq \O$
\begin{equation*}
A \cap B  \cap \overline{C} \cup \overline{C} \cap D \cup B \cap C \cap D
\end{equation*}

Примечание: U — универсум\question
Дано отношение на множестве $\{1, 2, 3, 4, 5\}$ 
\begin{equation*}
aRb \iff (a+b) \bmod 2 =0
\end{equation*}
Напишите обоснованный ответ какими свойствами обладает или не обладает отношение и почему:   
\begin{enumerate} [a)]\setcounter{enumi}{0}
\item рефлексивность
\item антирефлексивность
\item симметричность
\item асимметричность
\item антисимметричность
\item транзитивность
\end{enumerate}

Обоснуйте свой ответ по каждому из приведенных ниже вопросов:
\begin{enumerate} [a)]\setcounter{enumi}{0}
    \item Является ли это отношение отношением эквивалентности?
    \item Является ли это отношение функциональным?
    \item Каким из отношений соответствия (одно-многозначным, много-многозначный и т.д.) оно является?
    \item К каким из отношений порядка (полного, частичного и т.д.) можно отнести данное отношение?
\end{enumerate}



\question
Установите, является ли каждое из перечисленных ниже отношений на А ($R \subseteq A \times A$) отношением эквивалентности (обоснование ответа обязательно). Для каждого отношения эквивалентности постройте классы 
эквивалентности и постройте граф отношения:
\begin{enumerate} [a)]\setcounter{enumi}{0}
\item А - множество целых чисел и отношение $R = \{(a,b)|a + b = 5\}$
\item Пусть A – множество имен. $A = \{ $Алексей, Иван, Петр, Александр, Павел, Андрей$ \}$. Тогда отношение $R $ верно на парах имен, начинающихся с одной и той же буквы, и только на них.
\item На множестве $A = \{1; 2; 3; 4; 5\}$ задано отношение $R = \{(1; 2); (1; 3); (1; 5); (2; 3); (2; 4); (2; 5); (3; 4); (3; 5); (4; 5)\}$
\end{enumerate}\question Составьте полную таблицу истинности, определите, какие переменные являются фиктивными и проверьте, является ли формула тавтологией:
$ P \rightarrow (Q \rightarrow ((P \lor Q) \rightarrow (P \land Q)))$

\end{questions}
\newpage
%%% begin test
\begin{flushright}
\begin{tabular}{p{2.8in} r l}
%\textbf{\class} & \textbf{ФИО:} & \makebox[2.5in]{\hrulefill}\\
\textbf{\class} & \textbf{ФИО:} &Штыб Александр Сергеевич
\\

\textbf{\examdate} &&\\
%\textbf{Time Limit: \timelimit} & Teaching Assistant & \makebox[2in]{\hrulefill}
\end{tabular}\\
\end{flushright}
\rule[1ex]{\textwidth}{.1pt}


\begin{questions}
\question
Найдите и упростите P:
\begin{equation*}
\overline{P} = A \cap \overline{B} \cup A \cap C \cup B \cap C \cup \overline{A} \cap C
\end{equation*}
Затем найдите элементы множества P, выраженного через множества:
\begin{equation*}
A = \{0, 3, 4, 9\}; 
B = \{1, 3, 4, 7\};
C = \{0, 1, 2, 4, 7, 8, 9\};
I = \{0, 1, 2, 3, 4, 5, 6, 7, 8, 9\}.
\end{equation*}\question
Упростите следующее выражение с учетом того, что $A\subset B \subset C \subset D \subset U; A \neq \O$
\begin{equation*}
\overline{B} \cap \overline{C} \cap D \cup \overline{A} \cap \overline{C} \cap D \cup \overline{A} \cap B
\end{equation*}

Примечание: U — универсум\question
Дано отношение на множестве $\{1, 2, 3, 4, 5\}$ 
\begin{equation*}
aRb \iff  \text{НОД}(a,b) =1
\end{equation*}
Напишите обоснованный ответ какими свойствами обладает или не обладает отношение и почему:   
\begin{enumerate} [a)]\setcounter{enumi}{0}
\item рефлексивность
\item антирефлексивность
\item симметричность
\item асимметричность
\item антисимметричность
\item транзитивность
\end{enumerate}

Обоснуйте свой ответ по каждому из приведенных ниже вопросов:
\begin{enumerate} [a)]\setcounter{enumi}{0}
    \item Является ли это отношение отношением эквивалентности?
    \item Является ли это отношение функциональным?
    \item Каким из отношений соответствия (одно-многозначным, много-многозначный и т.д.) оно является?
    \item К каким из отношений порядка (полного, частичного и т.д.) можно отнести данное отношение?
\end{enumerate}


\question
Установите, является ли каждое из перечисленных ниже отношений на А ($R \subseteq A \times A$) отношением эквивалентности (обоснование ответа обязательно). Для каждого отношения эквивалентности постройте классы 
эквивалентности и постройте граф отношения:
\begin{enumerate} [a)]\setcounter{enumi}{0}
\item $A = \{a, b, c, d, p, t\}$ задано отношение $R = \{(a, a), (b, b), (b, c), (b, d), (c, b), (c, c), (c, d), (d, b), (d, c), (d, d), (p,p), (t,t)\}$
\item $A = \{-10, -9, … , 9, 10\}$ и отношение $R = \{(a,b)|a^{3} = b^{3}\}$

\item $F(x)=x^{2}+1$, где $x \in A = [-2, 4]$ и отношение $R = \{(a,b)|F(a) = F(b)\}$
\end{enumerate}\question Составьте полную таблицу истинности, определите, какие переменные являются фиктивными и проверьте, является ли формула тавтологией:
$(( P \land \neg Q) \rightarrow (R \land \neg R)) \rightarrow (P \rightarrow Q)$

\end{questions}
\newpage
%%% begin test
\begin{flushright}
\begin{tabular}{p{2.8in} r l}
%\textbf{\class} & \textbf{ФИО:} & \makebox[2.5in]{\hrulefill}\\
\textbf{\class} & \textbf{ФИО:} &М3110
\\

\textbf{\examdate} &&\\
%\textbf{Time Limit: \timelimit} & Teaching Assistant & \makebox[2in]{\hrulefill}
\end{tabular}\\
\end{flushright}
\rule[1ex]{\textwidth}{.1pt}


\begin{questions}
\question
Найдите и упростите P:
\begin{equation*}
\overline{P} = \overline{A} \cap B \cup \overline{A} \cap C \cup A \cap \overline{B} \cup \overline{B} \cap C
\end{equation*}
Затем найдите элементы множества P, выраженного через множества:
\begin{equation*}
A = \{0, 3, 4, 9\}; 
B = \{1, 3, 4, 7\};
C = \{0, 1, 2, 4, 7, 8, 9\};
I = \{0, 1, 2, 3, 4, 5, 6, 7, 8, 9\}.
\end{equation*}\question
Упростите следующее выражение с учетом того, что $A\subset B \subset C \subset D \subset U; A \neq \O$
\begin{equation*}
\overline{A} \cap \overline{B} \cup B \cap \overline{C} \cup \overline{C} \cap D
\end{equation*}

Примечание: U — универсум\question
Дано отношение на множестве $\{1, 2, 3, 4, 5\}$ 
\begin{equation*}
aRb \iff b > a
\end{equation*}
Напишите обоснованный ответ какими свойствами обладает или не обладает отношение и почему:   
\begin{enumerate} [a)]\setcounter{enumi}{0}
\item рефлексивность
\item антирефлексивность
\item симметричность
\item асимметричность
\item антисимметричность
\item транзитивность
\end{enumerate}

Обоснуйте свой ответ по каждому из приведенных ниже вопросов:
\begin{enumerate} [a)]\setcounter{enumi}{0}
    \item Является ли это отношение отношением эквивалентности?
    \item Является ли это отношение функциональным?
    \item Каким из отношений соответствия (одно-многозначным, много-многозначный и т.д.) оно является?
    \item К каким из отношений порядка (полного, частичного и т.д.) можно отнести данное отношение?
\end{enumerate}

\question
Установите, является ли каждое из перечисленных ниже отношений на А ($R \subseteq A \times A$) отношением эквивалентности (обоснование ответа обязательно). Для каждого отношения эквивалентности постройте классы эквивалентности и постройте граф отношения:
\begin{enumerate} [a)]\setcounter{enumi}{0}
\item $F(x)=x^{2}+1$, где $x \in A = [-2, 4]$ и отношение $R = \{(a,b)|F(a) = F(b)\}$
\item А - множество целых чисел и отношение $R = \{(a,b)|a + b = 5\}$
\item На множестве $A = \{1; 2; 3\}$ задано отношение $R = \{(1; 1); (2; 2); (3; 3); (3; 2); (1; 2); (2; 1)\}$

\end{enumerate}\question Составьте полную таблицу истинности, определите, какие переменные являются фиктивными и проверьте, является ли формула тавтологией:
$ P \rightarrow (Q \rightarrow ((P \lor Q) \rightarrow (P \land Q)))$

\end{questions}
\newpage
%%% begin test
\begin{flushright}
\begin{tabular}{p{2.8in} r l}
%\textbf{\class} & \textbf{ФИО:} & \makebox[2.5in]{\hrulefill}\\
\textbf{\class} & \textbf{ФИО:} &Адрианов Константин Сергеевич
\\

\textbf{\examdate} &&\\
%\textbf{Time Limit: \timelimit} & Teaching Assistant & \makebox[2in]{\hrulefill}
\end{tabular}\\
\end{flushright}
\rule[1ex]{\textwidth}{.1pt}


\begin{questions}
\question
Найдите и упростите P:
\begin{equation*}
\overline{P} = A \cap C \cup \overline{A} \cap \overline{C} \cup \overline{B} \cap C \cup \overline{A} \cap \overline{B}
\end{equation*}
Затем найдите элементы множества P, выраженного через множества:
\begin{equation*}
A = \{0, 3, 4, 9\}; 
B = \{1, 3, 4, 7\};
C = \{0, 1, 2, 4, 7, 8, 9\};
I = \{0, 1, 2, 3, 4, 5, 6, 7, 8, 9\}.
\end{equation*}\question
Упростите следующее выражение с учетом того, что $A\subset B \subset C \subset D \subset U; A \neq \O$
\begin{equation*}
A \cap B  \cap \overline{C} \cup \overline{C} \cap D \cup B \cap C \cap D
\end{equation*}

Примечание: U — универсум\question
Дано отношение на множестве $\{1, 2, 3, 4, 5\}$ 
\begin{equation*}
aRb \iff  \text{НОД}(a,b) =1
\end{equation*}
Напишите обоснованный ответ какими свойствами обладает или не обладает отношение и почему:   
\begin{enumerate} [a)]\setcounter{enumi}{0}
\item рефлексивность
\item антирефлексивность
\item симметричность
\item асимметричность
\item антисимметричность
\item транзитивность
\end{enumerate}

Обоснуйте свой ответ по каждому из приведенных ниже вопросов:
\begin{enumerate} [a)]\setcounter{enumi}{0}
    \item Является ли это отношение отношением эквивалентности?
    \item Является ли это отношение функциональным?
    \item Каким из отношений соответствия (одно-многозначным, много-многозначный и т.д.) оно является?
    \item К каким из отношений порядка (полного, частичного и т.д.) можно отнести данное отношение?
\end{enumerate}


\question
Установите, является ли каждое из перечисленных ниже отношений на А ($R \subseteq A \times A$) отношением эквивалентности (обоснование ответа обязательно). Для каждого отношения эквивалентности постройте классы 
эквивалентности и постройте граф отношения:
\begin{enumerate} [a)]\setcounter{enumi}{0}
\item А - множество целых чисел и отношение $R = \{(a,b)|a + b = 5\}$
\item Пусть A – множество имен. $A = \{ $Алексей, Иван, Петр, Александр, Павел, Андрей$ \}$. Тогда отношение $R $ верно на парах имен, начинающихся с одной и той же буквы, и только на них.
\item На множестве $A = \{1; 2; 3; 4; 5\}$ задано отношение $R = \{(1; 2); (1; 3); (1; 5); (2; 3); (2; 4); (2; 5); (3; 4); (3; 5); (4; 5)\}$
\end{enumerate}\question Составьте полную таблицу истинности, определите, какие переменные являются фиктивными и проверьте, является ли формула тавтологией:
$(( P \rightarrow Q) \land (Q \rightarrow P)) \rightarrow (P \rightarrow R)$

\end{questions}
\newpage
%%% begin test
\begin{flushright}
\begin{tabular}{p{2.8in} r l}
%\textbf{\class} & \textbf{ФИО:} & \makebox[2.5in]{\hrulefill}\\
\textbf{\class} & \textbf{ФИО:} &Бахтиева Айсылу Робертовна
\\

\textbf{\examdate} &&\\
%\textbf{Time Limit: \timelimit} & Teaching Assistant & \makebox[2in]{\hrulefill}
\end{tabular}\\
\end{flushright}
\rule[1ex]{\textwidth}{.1pt}


\begin{questions}
\question
Найдите и упростите P:
\begin{equation*}
\overline{P} = A \cap C \cup \overline{A} \cap \overline{C} \cup \overline{B} \cap C \cup \overline{A} \cap \overline{B}
\end{equation*}
Затем найдите элементы множества P, выраженного через множества:
\begin{equation*}
A = \{0, 3, 4, 9\}; 
B = \{1, 3, 4, 7\};
C = \{0, 1, 2, 4, 7, 8, 9\};
I = \{0, 1, 2, 3, 4, 5, 6, 7, 8, 9\}.
\end{equation*}\question
Упростите следующее выражение с учетом того, что $A\subset B \subset C \subset D \subset U; A \neq \O$
\begin{equation*}
A \cap B \cup \overline{A} \cap \overline{C} \cup A \cap C \cup \overline{B} \cap \overline{C}
\end{equation*}

Примечание: U — универсум\question
Дано отношение на множестве $\{1, 2, 3, 4, 5\}$ 
\begin{equation*}
aRb \iff (a+b) \bmod 2 =0
\end{equation*}
Напишите обоснованный ответ какими свойствами обладает или не обладает отношение и почему:   
\begin{enumerate} [a)]\setcounter{enumi}{0}
\item рефлексивность
\item антирефлексивность
\item симметричность
\item асимметричность
\item антисимметричность
\item транзитивность
\end{enumerate}

Обоснуйте свой ответ по каждому из приведенных ниже вопросов:
\begin{enumerate} [a)]\setcounter{enumi}{0}
    \item Является ли это отношение отношением эквивалентности?
    \item Является ли это отношение функциональным?
    \item Каким из отношений соответствия (одно-многозначным, много-многозначный и т.д.) оно является?
    \item К каким из отношений порядка (полного, частичного и т.д.) можно отнести данное отношение?
\end{enumerate}



\question
Установите, является ли каждое из перечисленных ниже отношений на А ($R \subseteq A \times A$) отношением эквивалентности (обоснование ответа обязательно). Для каждого отношения эквивалентности 
постройте классы эквивалентности и постройте граф отношения:
\begin{enumerate}[a)]\setcounter{enumi}{0}
\item А - множество целых чисел и отношение $R = \{(a,b)|a + b = 0\}$
\item $A = \{-10, -9, …, 9, 10\}$ и отношение $R = \{(a,b)|a^{3} = b^{3}\}$
\item На множестве $A = \{1; 2; 3\}$ задано отношение $R = \{(1; 1); (2; 2); (3; 3); (2; 1); (1; 2); (2; 3); (3; 2); (3; 1); (1; 3)\}$

\end{enumerate}\question Составьте полную таблицу истинности, определите, какие переменные являются фиктивными и проверьте, является ли формула тавтологией:
$((P \rightarrow Q) \lor R) \leftrightarrow (P \rightarrow (Q \lor R))$

\end{questions}
\newpage
%%% begin test
\begin{flushright}
\begin{tabular}{p{2.8in} r l}
%\textbf{\class} & \textbf{ФИО:} & \makebox[2.5in]{\hrulefill}\\
\textbf{\class} & \textbf{ФИО:} &Блик Антон Сергеевич
\\

\textbf{\examdate} &&\\
%\textbf{Time Limit: \timelimit} & Teaching Assistant & \makebox[2in]{\hrulefill}
\end{tabular}\\
\end{flushright}
\rule[1ex]{\textwidth}{.1pt}


\begin{questions}
\question
Найдите и упростите P:
\begin{equation*}
\overline{P} = A \cap \overline{B} \cup \overline{B} \cap C \cup \overline{A} \cap \overline{B} \cup \overline{A} \cap C
\end{equation*}
Затем найдите элементы множества P, выраженного через множества:
\begin{equation*}
A = \{0, 3, 4, 9\}; 
B = \{1, 3, 4, 7\};
C = \{0, 1, 2, 4, 7, 8, 9\};
I = \{0, 1, 2, 3, 4, 5, 6, 7, 8, 9\}.
\end{equation*}\question
Упростите следующее выражение с учетом того, что $A\subset B \subset C \subset D \subset U; A \neq \O$
\begin{equation*}
A \cap B \cup \overline{A} \cap \overline{C} \cup A \cap C \cup \overline{B} \cap \overline{C}
\end{equation*}

Примечание: U — универсум\question
Дано отношение на множестве $\{1, 2, 3, 4, 5\}$ 
\begin{equation*}
aRb \iff b > a
\end{equation*}
Напишите обоснованный ответ какими свойствами обладает или не обладает отношение и почему:   
\begin{enumerate} [a)]\setcounter{enumi}{0}
\item рефлексивность
\item антирефлексивность
\item симметричность
\item асимметричность
\item антисимметричность
\item транзитивность
\end{enumerate}

Обоснуйте свой ответ по каждому из приведенных ниже вопросов:
\begin{enumerate} [a)]\setcounter{enumi}{0}
    \item Является ли это отношение отношением эквивалентности?
    \item Является ли это отношение функциональным?
    \item Каким из отношений соответствия (одно-многозначным, много-многозначный и т.д.) оно является?
    \item К каким из отношений порядка (полного, частичного и т.д.) можно отнести данное отношение?
\end{enumerate}

\question
Установите, является ли каждое из перечисленных ниже отношений на А ($R \subseteq A \times A$) отношением эквивалентности (обоснование ответа обязательно). Для каждого отношения эквивалентности постройте классы 
эквивалентности и постройте граф отношения:
\begin{enumerate} [a)]\setcounter{enumi}{0}
\item Пусть A – множество имен. $A = \{ $Алексей, Иван, Петр, Александр, Павел, Андрей$ \}$. Тогда отношение $R$ верно на парах имен, начинающихся с одной и той же буквы, и только на них.
\item $A = \{-10, -9, … , 9, 10\}$ и отношение $ R = \{(a,b)|a^{2} = b^{2}\}$
\item На множестве $A = \{1; 2; 3\}$ задано отношение $R = \{(1; 1); (2; 2); (3; 3); (3; 2); (1; 2); (2; 1)\}$
\end{enumerate}\question Составьте полную таблицу истинности, определите, какие переменные являются фиктивными и проверьте, является ли формула тавтологией:
$(P \rightarrow (Q \rightarrow R)) \rightarrow ((P \rightarrow Q) \rightarrow (P \rightarrow R))$

\end{questions}
\newpage
%%% begin test
\begin{flushright}
\begin{tabular}{p{2.8in} r l}
%\textbf{\class} & \textbf{ФИО:} & \makebox[2.5in]{\hrulefill}\\
\textbf{\class} & \textbf{ФИО:} &Босов Александр Владимирович
\\

\textbf{\examdate} &&\\
%\textbf{Time Limit: \timelimit} & Teaching Assistant & \makebox[2in]{\hrulefill}
\end{tabular}\\
\end{flushright}
\rule[1ex]{\textwidth}{.1pt}


\begin{questions}
\question
Найдите и упростите P:
\begin{equation*}
\overline{P} = A \cap C \cup \overline{A} \cap \overline{C} \cup \overline{B} \cap C \cup \overline{A} \cap \overline{B}
\end{equation*}
Затем найдите элементы множества P, выраженного через множества:
\begin{equation*}
A = \{0, 3, 4, 9\}; 
B = \{1, 3, 4, 7\};
C = \{0, 1, 2, 4, 7, 8, 9\};
I = \{0, 1, 2, 3, 4, 5, 6, 7, 8, 9\}.
\end{equation*}\question
Упростите следующее выражение с учетом того, что $A\subset B \subset C \subset D \subset U; A \neq \O$
\begin{equation*}
\overline{A} \cap \overline{B} \cup B \cap \overline{C} \cup \overline{C} \cap D
\end{equation*}

Примечание: U — универсум\question
Дано отношение на множестве $\{1, 2, 3, 4, 5\}$ 
\begin{equation*}
aRb \iff (a+b) \bmod 2 =0
\end{equation*}
Напишите обоснованный ответ какими свойствами обладает или не обладает отношение и почему:   
\begin{enumerate} [a)]\setcounter{enumi}{0}
\item рефлексивность
\item антирефлексивность
\item симметричность
\item асимметричность
\item антисимметричность
\item транзитивность
\end{enumerate}

Обоснуйте свой ответ по каждому из приведенных ниже вопросов:
\begin{enumerate} [a)]\setcounter{enumi}{0}
    \item Является ли это отношение отношением эквивалентности?
    \item Является ли это отношение функциональным?
    \item Каким из отношений соответствия (одно-многозначным, много-многозначный и т.д.) оно является?
    \item К каким из отношений порядка (полного, частичного и т.д.) можно отнести данное отношение?
\end{enumerate}



\question
Установите, является ли каждое из перечисленных ниже отношений на А ($R \subseteq A \times A$) отношением эквивалентности (обоснование ответа обязательно). Для каждого отношения эквивалентности постройте классы 
эквивалентности и постройте граф отношения:
\begin{enumerate} [a)]\setcounter{enumi}{0}
\item $A = \{-10, -9, … , 9, 10\}$ и отношение $R = \{(a,b)|a^{2} = b^{2}\}$
\item $A = \{a, b, c, d, p, t\}$ задано отношение $R = \{(a, a), (b, b), (b, c), (b, d), (c, b), (c, c), (c, d), (d, b), (d, c), (d, d), (p,p), (t,t)\}$
\item Пусть A – множество имен. $A = \{ $Алексей, Иван, Петр, Александр, Павел, Андрей$ \}$. Тогда отношение $R$ верно на парах имен, начинающихся с одной и той же буквы, и только на них.
\end{enumerate}\question Составьте полную таблицу истинности, определите, какие переменные являются фиктивными и проверьте, является ли формула тавтологией:
$(( P \rightarrow Q) \land (Q \rightarrow P)) \rightarrow (P \rightarrow R)$

\end{questions}
\newpage
%%% begin test
\begin{flushright}
\begin{tabular}{p{2.8in} r l}
%\textbf{\class} & \textbf{ФИО:} & \makebox[2.5in]{\hrulefill}\\
\textbf{\class} & \textbf{ФИО:} &Гуськов Егор Дмитриевич
\\

\textbf{\examdate} &&\\
%\textbf{Time Limit: \timelimit} & Teaching Assistant & \makebox[2in]{\hrulefill}
\end{tabular}\\
\end{flushright}
\rule[1ex]{\textwidth}{.1pt}


\begin{questions}
\question
Найдите и упростите P:
\begin{equation*}
\overline{P} = A \cap C \cup \overline{A} \cap \overline{C} \cup \overline{B} \cap C \cup \overline{A} \cap \overline{B}
\end{equation*}
Затем найдите элементы множества P, выраженного через множества:
\begin{equation*}
A = \{0, 3, 4, 9\}; 
B = \{1, 3, 4, 7\};
C = \{0, 1, 2, 4, 7, 8, 9\};
I = \{0, 1, 2, 3, 4, 5, 6, 7, 8, 9\}.
\end{equation*}\question
Упростите следующее выражение с учетом того, что $A\subset B \subset C \subset D \subset U; A \neq \O$
\begin{equation*}
A \cap C  \cap D \cup B \cap \overline{C} \cap D \cup B \cap C \cap D
\end{equation*}

Примечание: U — универсум\question
Дано отношение на множестве $\{1, 2, 3, 4, 5\}$ 
\begin{equation*}
aRb \iff a \leq b
\end{equation*}
Напишите обоснованный ответ какими свойствами обладает или не обладает отношение и почему:   
\begin{enumerate} [a)]\setcounter{enumi}{0}
\item рефлексивность
\item антирефлексивность
\item симметричность
\item асимметричность
\item антисимметричность
\item транзитивность
\end{enumerate}

Обоснуйте свой ответ по каждому из приведенных ниже вопросов:
\begin{enumerate} [a)]\setcounter{enumi}{0}
    \item Является ли это отношение отношением эквивалентности?
    \item Является ли это отношение функциональным?
    \item Каким из отношений соответствия (одно-многозначным, много-многозначный и т.д.) оно является?
    \item К каким из отношений порядка (полного, частичного и т.д.) можно отнести данное отношение?
\end{enumerate}


\question
Установите, является ли каждое из перечисленных ниже отношений на А ($R \subseteq A \times A$) отношением эквивалентности (обоснование ответа обязательно). Для каждого отношения эквивалентности постройте классы 
эквивалентности и постройте граф отношения:
\begin{enumerate} [a)]\setcounter{enumi}{0}
\item А - множество целых чисел и отношение $R = \{(a,b)|a + b = 5\}$
\item Пусть A – множество имен. $A = \{ $Алексей, Иван, Петр, Александр, Павел, Андрей$ \}$. Тогда отношение $R $ верно на парах имен, начинающихся с одной и той же буквы, и только на них.
\item На множестве $A = \{1; 2; 3; 4; 5\}$ задано отношение $R = \{(1; 2); (1; 3); (1; 5); (2; 3); (2; 4); (2; 5); (3; 4); (3; 5); (4; 5)\}$
\end{enumerate}\question Составьте полную таблицу истинности, определите, какие переменные являются фиктивными и проверьте, является ли формула тавтологией:
$(( P \rightarrow Q) \land (Q \rightarrow P)) \rightarrow (P \rightarrow R)$

\end{questions}
\newpage
%%% begin test
\begin{flushright}
\begin{tabular}{p{2.8in} r l}
%\textbf{\class} & \textbf{ФИО:} & \makebox[2.5in]{\hrulefill}\\
\textbf{\class} & \textbf{ФИО:} &Дуксин Патрик Евгеньевич
\\

\textbf{\examdate} &&\\
%\textbf{Time Limit: \timelimit} & Teaching Assistant & \makebox[2in]{\hrulefill}
\end{tabular}\\
\end{flushright}
\rule[1ex]{\textwidth}{.1pt}


\begin{questions}
\question
Найдите и упростите P:
\begin{equation*}
\overline{P} = A \cap C \cup \overline{A} \cap \overline{C} \cup \overline{B} \cap C \cup \overline{A} \cap \overline{B}
\end{equation*}
Затем найдите элементы множества P, выраженного через множества:
\begin{equation*}
A = \{0, 3, 4, 9\}; 
B = \{1, 3, 4, 7\};
C = \{0, 1, 2, 4, 7, 8, 9\};
I = \{0, 1, 2, 3, 4, 5, 6, 7, 8, 9\}.
\end{equation*}\question
Упростите следующее выражение с учетом того, что $A\subset B \subset C \subset D \subset U; A \neq \O$
\begin{equation*}
\overline{A} \cap \overline{C} \cap D \cup \overline{B} \cap \overline{C} \cap D \cup A \cap B
\end{equation*}

Примечание: U — универсум\question
Дано отношение на множестве $\{1, 2, 3, 4, 5\}$ 
\begin{equation*}
aRb \iff  \text{НОД}(a,b) =1
\end{equation*}
Напишите обоснованный ответ какими свойствами обладает или не обладает отношение и почему:   
\begin{enumerate} [a)]\setcounter{enumi}{0}
\item рефлексивность
\item антирефлексивность
\item симметричность
\item асимметричность
\item антисимметричность
\item транзитивность
\end{enumerate}

Обоснуйте свой ответ по каждому из приведенных ниже вопросов:
\begin{enumerate} [a)]\setcounter{enumi}{0}
    \item Является ли это отношение отношением эквивалентности?
    \item Является ли это отношение функциональным?
    \item Каким из отношений соответствия (одно-многозначным, много-многозначный и т.д.) оно является?
    \item К каким из отношений порядка (полного, частичного и т.д.) можно отнести данное отношение?
\end{enumerate}


\question
Установите, является ли каждое из перечисленных ниже отношений на А ($R \subseteq A \times A$) отношением эквивалентности (обоснование ответа обязательно). Для каждого отношения эквивалентности постройте классы эквивалентности и постройте граф отношения:
\begin{enumerate} [a)]\setcounter{enumi}{0}
\item $F(x)=x^{2}+1$, где $x \in A = [-2, 4]$ и отношение $R = \{(a,b)|F(a) = F(b)\}$
\item А - множество целых чисел и отношение $R = \{(a,b)|a + b = 5\}$
\item На множестве $A = \{1; 2; 3\}$ задано отношение $R = \{(1; 1); (2; 2); (3; 3); (3; 2); (1; 2); (2; 1)\}$

\end{enumerate}\question Составьте полную таблицу истинности, определите, какие переменные являются фиктивными и проверьте, является ли формула тавтологией:
$(P \rightarrow (Q \rightarrow R)) \rightarrow ((P \rightarrow Q) \rightarrow (P \rightarrow R))$

\end{questions}
\newpage
%%% begin test
\begin{flushright}
\begin{tabular}{p{2.8in} r l}
%\textbf{\class} & \textbf{ФИО:} & \makebox[2.5in]{\hrulefill}\\
\textbf{\class} & \textbf{ФИО:} &Евгеев Олег Дмитриевич
\\

\textbf{\examdate} &&\\
%\textbf{Time Limit: \timelimit} & Teaching Assistant & \makebox[2in]{\hrulefill}
\end{tabular}\\
\end{flushright}
\rule[1ex]{\textwidth}{.1pt}


\begin{questions}
\question
Найдите и упростите P:
\begin{equation*}
\overline{P} = \overline{A} \cap B \cup \overline{A} \cap C \cup A \cap \overline{B} \cup \overline{B} \cap C
\end{equation*}
Затем найдите элементы множества P, выраженного через множества:
\begin{equation*}
A = \{0, 3, 4, 9\}; 
B = \{1, 3, 4, 7\};
C = \{0, 1, 2, 4, 7, 8, 9\};
I = \{0, 1, 2, 3, 4, 5, 6, 7, 8, 9\}.
\end{equation*}\question
Упростите следующее выражение с учетом того, что $A\subset B \subset C \subset D \subset U; A \neq \O$
\begin{equation*}
A \cap B  \cap \overline{C} \cup \overline{C} \cap D \cup B \cap C \cap D
\end{equation*}

Примечание: U — универсум\question
Для следующего отношения на множестве $\{1, 2, 3, 4, 5\}$ 
\begin{equation*}
aRb \iff 0 < a-b<2
\end{equation*}
Напишите обоснованный ответ какими свойствами обладает или не обладает отношение и почему:   
\begin{enumerate} [a)]\setcounter{enumi}{0}
\item рефлексивность
\item антирефлексивность
\item симметричность
\item асимметричность
\item антисимметричность
\item транзитивность
\end{enumerate}

Обоснуйте свой ответ по каждому из приведенных ниже вопросов:
\begin{enumerate} [a)]\setcounter{enumi}{0}
    \item Является ли это отношение отношением эквивалентности?
    \item Является ли это отношение функциональным?
    \item Каким из отношений соответствия (одно-многозначным, много-многозначный и т.д.) оно является?
    \item К каким из отношений порядка (полного, частичного и т.д.) можно отнести данное отношение?
\end{enumerate}
\question
Установите, является ли каждое из перечисленных ниже отношений на А ($R \subseteq A \times A$) отношением эквивалентности (обоснование ответа обязательно). Для каждого отношения эквивалентности постройте классы 
эквивалентности и постройте граф отношения:
\begin{enumerate} [a)]\setcounter{enumi}{0}
\item А - множество целых чисел и отношение $R = \{(a,b)|a + b = 5\}$
\item Пусть A – множество имен. $A = \{ $Алексей, Иван, Петр, Александр, Павел, Андрей$ \}$. Тогда отношение $R $ верно на парах имен, начинающихся с одной и той же буквы, и только на них.
\item На множестве $A = \{1; 2; 3; 4; 5\}$ задано отношение $R = \{(1; 2); (1; 3); (1; 5); (2; 3); (2; 4); (2; 5); (3; 4); (3; 5); (4; 5)\}$
\end{enumerate}\question Составьте полную таблицу истинности, определите, какие переменные являются фиктивными и проверьте, является ли формула тавтологией:
$(( P \rightarrow Q) \land (Q \rightarrow P)) \rightarrow (P \rightarrow R)$

\end{questions}
\newpage
%%% begin test
\begin{flushright}
\begin{tabular}{p{2.8in} r l}
%\textbf{\class} & \textbf{ФИО:} & \makebox[2.5in]{\hrulefill}\\
\textbf{\class} & \textbf{ФИО:} &Зайцев Михаил Михайлович
\\

\textbf{\examdate} &&\\
%\textbf{Time Limit: \timelimit} & Teaching Assistant & \makebox[2in]{\hrulefill}
\end{tabular}\\
\end{flushright}
\rule[1ex]{\textwidth}{.1pt}


\begin{questions}
\question
Найдите и упростите P:
\begin{equation*}
\overline{P} = A \cap \overline{B} \cup \overline{B} \cap C \cup \overline{A} \cap \overline{B} \cup \overline{A} \cap C
\end{equation*}
Затем найдите элементы множества P, выраженного через множества:
\begin{equation*}
A = \{0, 3, 4, 9\}; 
B = \{1, 3, 4, 7\};
C = \{0, 1, 2, 4, 7, 8, 9\};
I = \{0, 1, 2, 3, 4, 5, 6, 7, 8, 9\}.
\end{equation*}\question
Упростите следующее выражение с учетом того, что $A\subset B \subset C \subset D \subset U; A \neq \O$
\begin{equation*}
A \cap C  \cap D \cup B \cap \overline{C} \cap D \cup B \cap C \cap D
\end{equation*}

Примечание: U — универсум\question
Дано отношение на множестве $\{1, 2, 3, 4, 5\}$ 
\begin{equation*}
aRb \iff (a+b) \bmod 2 =0
\end{equation*}
Напишите обоснованный ответ какими свойствами обладает или не обладает отношение и почему:   
\begin{enumerate} [a)]\setcounter{enumi}{0}
\item рефлексивность
\item антирефлексивность
\item симметричность
\item асимметричность
\item антисимметричность
\item транзитивность
\end{enumerate}

Обоснуйте свой ответ по каждому из приведенных ниже вопросов:
\begin{enumerate} [a)]\setcounter{enumi}{0}
    \item Является ли это отношение отношением эквивалентности?
    \item Является ли это отношение функциональным?
    \item Каким из отношений соответствия (одно-многозначным, много-многозначный и т.д.) оно является?
    \item К каким из отношений порядка (полного, частичного и т.д.) можно отнести данное отношение?
\end{enumerate}



\question
Установите, является ли каждое из перечисленных ниже отношений на А ($R \subseteq A \times A$) отношением эквивалентности (обоснование ответа обязательно). Для каждого отношения эквивалентности постройте классы 
эквивалентности и постройте граф отношения:
\begin{enumerate} [a)]\setcounter{enumi}{0}
\item На множестве $A = \{1; 2; 3\}$ задано отношение $R = \{(1; 1); (2; 2); (3; 3); (2; 1); (1; 2); (2; 3); (3; 2); (3; 1); (1; 3)\}$
\item На множестве $A = \{1; 2; 3; 4; 5\}$ задано отношение $R = \{(1; 2); (1; 3); (1; 5); (2; 3); (2; 4); (2; 5); (3; 4); (3; 5); (4; 5)\}$
\item А - множество целых чисел и отношение $R = \{(a,b)|a + b = 0\}$
\end{enumerate}\question Составьте полную таблицу истинности, определите, какие переменные являются фиктивными и проверьте, является ли формула тавтологией:
$(P \rightarrow (Q \rightarrow R)) \rightarrow ((P \rightarrow Q) \rightarrow (P \rightarrow R))$

\end{questions}
\newpage
%%% begin test
\begin{flushright}
\begin{tabular}{p{2.8in} r l}
%\textbf{\class} & \textbf{ФИО:} & \makebox[2.5in]{\hrulefill}\\
\textbf{\class} & \textbf{ФИО:} &Каталков Георгий Александрович
\\

\textbf{\examdate} &&\\
%\textbf{Time Limit: \timelimit} & Teaching Assistant & \makebox[2in]{\hrulefill}
\end{tabular}\\
\end{flushright}
\rule[1ex]{\textwidth}{.1pt}


\begin{questions}
\question
Найдите и упростите P:
\begin{equation*}
\overline{P} = A \cap C \cup \overline{A} \cap \overline{C} \cup \overline{B} \cap C \cup \overline{A} \cap \overline{B}
\end{equation*}
Затем найдите элементы множества P, выраженного через множества:
\begin{equation*}
A = \{0, 3, 4, 9\}; 
B = \{1, 3, 4, 7\};
C = \{0, 1, 2, 4, 7, 8, 9\};
I = \{0, 1, 2, 3, 4, 5, 6, 7, 8, 9\}.
\end{equation*}\question
Упростите следующее выражение с учетом того, что $A\subset B \subset C \subset D \subset U; A \neq \O$
\begin{equation*}
\overline{A} \cap \overline{C} \cap D \cup \overline{B} \cap \overline{C} \cap D \cup A \cap B
\end{equation*}

Примечание: U — универсум\question
Для следующего отношения на множестве $\{1, 2, 3, 4, 5\}$ 
\begin{equation*}
aRb \iff 0 < a-b<2
\end{equation*}
Напишите обоснованный ответ какими свойствами обладает или не обладает отношение и почему:   
\begin{enumerate} [a)]\setcounter{enumi}{0}
\item рефлексивность
\item антирефлексивность
\item симметричность
\item асимметричность
\item антисимметричность
\item транзитивность
\end{enumerate}

Обоснуйте свой ответ по каждому из приведенных ниже вопросов:
\begin{enumerate} [a)]\setcounter{enumi}{0}
    \item Является ли это отношение отношением эквивалентности?
    \item Является ли это отношение функциональным?
    \item Каким из отношений соответствия (одно-многозначным, много-многозначный и т.д.) оно является?
    \item К каким из отношений порядка (полного, частичного и т.д.) можно отнести данное отношение?
\end{enumerate}
\question
Установите, является ли каждое из перечисленных ниже отношений на А ($R \subseteq A \times A$) отношением эквивалентности (обоснование ответа обязательно). Для каждого отношения эквивалентности постройте классы 
эквивалентности и постройте граф отношения:
\begin{enumerate} [a)]\setcounter{enumi}{0}
\item Пусть A – множество имен. $A = \{ $Алексей, Иван, Петр, Александр, Павел, Андрей$ \}$. Тогда отношение $R$ верно на парах имен, начинающихся с одной и той же буквы, и только на них.
\item $A = \{-10, -9, … , 9, 10\}$ и отношение $ R = \{(a,b)|a^{2} = b^{2}\}$
\item На множестве $A = \{1; 2; 3\}$ задано отношение $R = \{(1; 1); (2; 2); (3; 3); (3; 2); (1; 2); (2; 1)\}$
\end{enumerate}\question Составьте полную таблицу истинности, определите, какие переменные являются фиктивными и проверьте, является ли формула тавтологией:
$(( P \rightarrow Q) \land (Q \rightarrow P)) \rightarrow (P \rightarrow R)$

\end{questions}
\newpage
%%% begin test
\begin{flushright}
\begin{tabular}{p{2.8in} r l}
%\textbf{\class} & \textbf{ФИО:} & \makebox[2.5in]{\hrulefill}\\
\textbf{\class} & \textbf{ФИО:} &Кветный Михаил Аркадьевич
\\

\textbf{\examdate} &&\\
%\textbf{Time Limit: \timelimit} & Teaching Assistant & \makebox[2in]{\hrulefill}
\end{tabular}\\
\end{flushright}
\rule[1ex]{\textwidth}{.1pt}


\begin{questions}
\question
Найдите и упростите P:
\begin{equation*}
\overline{P} = A \cap \overline{B} \cup \overline{B} \cap C \cup \overline{A} \cap \overline{B} \cup \overline{A} \cap C
\end{equation*}
Затем найдите элементы множества P, выраженного через множества:
\begin{equation*}
A = \{0, 3, 4, 9\}; 
B = \{1, 3, 4, 7\};
C = \{0, 1, 2, 4, 7, 8, 9\};
I = \{0, 1, 2, 3, 4, 5, 6, 7, 8, 9\}.
\end{equation*}\question
Упростите следующее выражение с учетом того, что $A\subset B \subset C \subset D \subset U; A \neq \O$
\begin{equation*}
\overline{A} \cap \overline{B} \cup B \cap \overline{C} \cup \overline{C} \cap D
\end{equation*}

Примечание: U — универсум\question
Дано отношение на множестве $\{1, 2, 3, 4, 5\}$ 
\begin{equation*}
aRb \iff |a-b| = 1
\end{equation*}
Напишите обоснованный ответ какими свойствами обладает или не обладает отношение и почему:   
\begin{enumerate} [a)]\setcounter{enumi}{0}
\item рефлексивность
\item антирефлексивность
\item симметричность
\item асимметричность
\item антисимметричность
\item транзитивность
\end{enumerate}

Обоснуйте свой ответ по каждому из приведенных ниже вопросов:
\begin{enumerate} [a)]\setcounter{enumi}{0}
    \item Является ли это отношение отношением эквивалентности?
    \item Является ли это отношение функциональным?
    \item Каким из отношений соответствия (одно-многозначным, много-многозначный и т.д.) оно является?
    \item К каким из отношений порядка (полного, частичного и т.д.) можно отнести данное отношение?
\end{enumerate}

\question
Установите, является ли каждое из перечисленных ниже отношений на А ($R \subseteq A \times A$) отношением эквивалентности (обоснование ответа обязательно). Для каждого отношения эквивалентности постройте классы 
эквивалентности и постройте граф отношения:
\begin{enumerate} [a)]\setcounter{enumi}{0}
\item Пусть A – множество имен. $A = \{ $Алексей, Иван, Петр, Александр, Павел, Андрей$ \}$. Тогда отношение $R$ верно на парах имен, начинающихся с одной и той же буквы, и только на них.
\item $A = \{-10, -9, … , 9, 10\}$ и отношение $ R = \{(a,b)|a^{2} = b^{2}\}$
\item На множестве $A = \{1; 2; 3\}$ задано отношение $R = \{(1; 1); (2; 2); (3; 3); (3; 2); (1; 2); (2; 1)\}$
\end{enumerate}\question Составьте полную таблицу истинности, определите, какие переменные являются фиктивными и проверьте, является ли формула тавтологией:
$(( P \rightarrow Q) \land (Q \rightarrow P)) \rightarrow (P \rightarrow R)$

\end{questions}
\newpage
%%% begin test
\begin{flushright}
\begin{tabular}{p{2.8in} r l}
%\textbf{\class} & \textbf{ФИО:} & \makebox[2.5in]{\hrulefill}\\
\textbf{\class} & \textbf{ФИО:} &Колпикова Ксения Денисовна
\\

\textbf{\examdate} &&\\
%\textbf{Time Limit: \timelimit} & Teaching Assistant & \makebox[2in]{\hrulefill}
\end{tabular}\\
\end{flushright}
\rule[1ex]{\textwidth}{.1pt}


\begin{questions}
\question
Найдите и упростите P:
\begin{equation*}
\overline{P} = A \cap \overline{B} \cup A \cap C \cup B \cap C \cup \overline{A} \cap C
\end{equation*}
Затем найдите элементы множества P, выраженного через множества:
\begin{equation*}
A = \{0, 3, 4, 9\}; 
B = \{1, 3, 4, 7\};
C = \{0, 1, 2, 4, 7, 8, 9\};
I = \{0, 1, 2, 3, 4, 5, 6, 7, 8, 9\}.
\end{equation*}\question
Упростите следующее выражение с учетом того, что $A\subset B \subset C \subset D \subset U; A \neq \O$
\begin{equation*}
\overline{A} \cap \overline{B} \cup B \cap \overline{C} \cup \overline{C} \cap D
\end{equation*}

Примечание: U — универсум\question
Дано отношение на множестве $\{1, 2, 3, 4, 5\}$ 
\begin{equation*}
aRb \iff a \geq b^2
\end{equation*}
Напишите обоснованный ответ какими свойствами обладает или не обладает отношение и почему:   
\begin{enumerate} [a)]\setcounter{enumi}{0}
\item рефлексивность
\item антирефлексивность
\item симметричность
\item асимметричность
\item антисимметричность
\item транзитивность
\end{enumerate}

Обоснуйте свой ответ по каждому из приведенных ниже вопросов:
\begin{enumerate} [a)]\setcounter{enumi}{0}
    \item Является ли это отношение отношением эквивалентности?
    \item Является ли это отношение функциональным?
    \item Каким из отношений соответствия (одно-многозначным, много-многозначный и т.д.) оно является?
    \item К каким из отношений порядка (полного, частичного и т.д.) можно отнести данное отношение?
\end{enumerate}


\question
Установите, является ли каждое из перечисленных ниже отношений на А ($R \subseteq A \times A$) отношением эквивалентности (обоснование ответа обязательно). Для каждого отношения эквивалентности постройте классы 
эквивалентности и постройте граф отношения:
\begin{enumerate} [a)]\setcounter{enumi}{0}
\item А - множество целых чисел и отношение $R = \{(a,b)|a + b = 5\}$
\item Пусть A – множество имен. $A = \{ $Алексей, Иван, Петр, Александр, Павел, Андрей$ \}$. Тогда отношение $R $ верно на парах имен, начинающихся с одной и той же буквы, и только на них.
\item На множестве $A = \{1; 2; 3; 4; 5\}$ задано отношение $R = \{(1; 2); (1; 3); (1; 5); (2; 3); (2; 4); (2; 5); (3; 4); (3; 5); (4; 5)\}$
\end{enumerate}\question Составьте полную таблицу истинности, определите, какие переменные являются фиктивными и проверьте, является ли формула тавтологией:
$ P \rightarrow (Q \rightarrow ((P \lor Q) \rightarrow (P \land Q)))$

\end{questions}
\newpage
%%% begin test
\begin{flushright}
\begin{tabular}{p{2.8in} r l}
%\textbf{\class} & \textbf{ФИО:} & \makebox[2.5in]{\hrulefill}\\
\textbf{\class} & \textbf{ФИО:} &Косарский Александр Андреевич
\\

\textbf{\examdate} &&\\
%\textbf{Time Limit: \timelimit} & Teaching Assistant & \makebox[2in]{\hrulefill}
\end{tabular}\\
\end{flushright}
\rule[1ex]{\textwidth}{.1pt}


\begin{questions}
\question
Найдите и упростите P:
\begin{equation*}
\overline{P} = A \cap \overline{B} \cup A \cap C \cup B \cap C \cup \overline{A} \cap C
\end{equation*}
Затем найдите элементы множества P, выраженного через множества:
\begin{equation*}
A = \{0, 3, 4, 9\}; 
B = \{1, 3, 4, 7\};
C = \{0, 1, 2, 4, 7, 8, 9\};
I = \{0, 1, 2, 3, 4, 5, 6, 7, 8, 9\}.
\end{equation*}\question
Упростите следующее выражение с учетом того, что $A\subset B \subset C \subset D \subset U; A \neq \O$
\begin{equation*}
\overline{B} \cap \overline{C} \cap D \cup \overline{A} \cap \overline{C} \cap D \cup \overline{A} \cap B
\end{equation*}

Примечание: U — универсум\question
Дано отношение на множестве $\{1, 2, 3, 4, 5\}$ 
\begin{equation*}
aRb \iff a \leq b
\end{equation*}
Напишите обоснованный ответ какими свойствами обладает или не обладает отношение и почему:   
\begin{enumerate} [a)]\setcounter{enumi}{0}
\item рефлексивность
\item антирефлексивность
\item симметричность
\item асимметричность
\item антисимметричность
\item транзитивность
\end{enumerate}

Обоснуйте свой ответ по каждому из приведенных ниже вопросов:
\begin{enumerate} [a)]\setcounter{enumi}{0}
    \item Является ли это отношение отношением эквивалентности?
    \item Является ли это отношение функциональным?
    \item Каким из отношений соответствия (одно-многозначным, много-многозначный и т.д.) оно является?
    \item К каким из отношений порядка (полного, частичного и т.д.) можно отнести данное отношение?
\end{enumerate}


\question
Установите, является ли каждое из перечисленных ниже отношений на А ($R \subseteq A \times A$) отношением эквивалентности (обоснование ответа обязательно). Для каждого отношения эквивалентности постройте классы 
эквивалентности и постройте граф отношения:
\begin{enumerate} [a)]\setcounter{enumi}{0}
\item $A = \{a, b, c, d, p, t\}$ задано отношение $R = \{(a, a), (b, b), (b, c), (b, d), (c, b), (c, c), (c, d), (d, b), (d, c), (d, d), (p,p), (t,t)\}$
\item $A = \{-10, -9, … , 9, 10\}$ и отношение $R = \{(a,b)|a^{3} = b^{3}\}$

\item $F(x)=x^{2}+1$, где $x \in A = [-2, 4]$ и отношение $R = \{(a,b)|F(a) = F(b)\}$
\end{enumerate}\question Составьте полную таблицу истинности, определите, какие переменные являются фиктивными и проверьте, является ли формула тавтологией:
$(( P \rightarrow Q) \land (Q \rightarrow P)) \rightarrow (P \rightarrow R)$

\end{questions}
\newpage
%%% begin test
\begin{flushright}
\begin{tabular}{p{2.8in} r l}
%\textbf{\class} & \textbf{ФИО:} & \makebox[2.5in]{\hrulefill}\\
\textbf{\class} & \textbf{ФИО:} &Куляев Игорь Александрович
\\

\textbf{\examdate} &&\\
%\textbf{Time Limit: \timelimit} & Teaching Assistant & \makebox[2in]{\hrulefill}
\end{tabular}\\
\end{flushright}
\rule[1ex]{\textwidth}{.1pt}


\begin{questions}
\question
Найдите и упростите P:
\begin{equation*}
\overline{P} = A \cap \overline{C} \cup A \cap \overline{B} \cup B \cap \overline{C} \cup A \cap C
\end{equation*}
Затем найдите элементы множества P, выраженного через множества:
\begin{equation*}
A = \{0, 3, 4, 9\}; 
B = \{1, 3, 4, 7\};
C = \{0, 1, 2, 4, 7, 8, 9\};
I = \{0, 1, 2, 3, 4, 5, 6, 7, 8, 9\}.
\end{equation*}\question
Упростите следующее выражение с учетом того, что $A\subset B \subset C \subset D \subset U; A \neq \O$
\begin{equation*}
A \cap  \overline{C} \cup B \cap \overline{D} \cup  \overline{A} \cap C \cap  \overline{D}
\end{equation*}

Примечание: U — универсум\question
Дано отношение на множестве $\{1, 2, 3, 4, 5\}$ 
\begin{equation*}
aRb \iff a \geq b^2
\end{equation*}
Напишите обоснованный ответ какими свойствами обладает или не обладает отношение и почему:   
\begin{enumerate} [a)]\setcounter{enumi}{0}
\item рефлексивность
\item антирефлексивность
\item симметричность
\item асимметричность
\item антисимметричность
\item транзитивность
\end{enumerate}

Обоснуйте свой ответ по каждому из приведенных ниже вопросов:
\begin{enumerate} [a)]\setcounter{enumi}{0}
    \item Является ли это отношение отношением эквивалентности?
    \item Является ли это отношение функциональным?
    \item Каким из отношений соответствия (одно-многозначным, много-многозначный и т.д.) оно является?
    \item К каким из отношений порядка (полного, частичного и т.д.) можно отнести данное отношение?
\end{enumerate}


\question
Установите, является ли каждое из перечисленных ниже отношений на А ($R \subseteq A \times A$) отношением эквивалентности (обоснование ответа обязательно). Для каждого отношения эквивалентности постройте классы 
эквивалентности и постройте граф отношения:
\begin{enumerate} [a)]\setcounter{enumi}{0}
\item Пусть A – множество имен. $A = \{ $Алексей, Иван, Петр, Александр, Павел, Андрей$ \}$. Тогда отношение $R$ верно на парах имен, начинающихся с одной и той же буквы, и только на них.
\item $A = \{-10, -9, … , 9, 10\}$ и отношение $ R = \{(a,b)|a^{2} = b^{2}\}$
\item На множестве $A = \{1; 2; 3\}$ задано отношение $R = \{(1; 1); (2; 2); (3; 3); (3; 2); (1; 2); (2; 1)\}$
\end{enumerate}\question Составьте полную таблицу истинности, определите, какие переменные являются фиктивными и проверьте, является ли формула тавтологией:

$(P \rightarrow (Q \land R)) \leftrightarrow ((P \rightarrow Q) \land (P \rightarrow R))$

\end{questions}
\newpage
%%% begin test
\begin{flushright}
\begin{tabular}{p{2.8in} r l}
%\textbf{\class} & \textbf{ФИО:} & \makebox[2.5in]{\hrulefill}\\
\textbf{\class} & \textbf{ФИО:} &Марков Фёдор Дмитриевич
\\

\textbf{\examdate} &&\\
%\textbf{Time Limit: \timelimit} & Teaching Assistant & \makebox[2in]{\hrulefill}
\end{tabular}\\
\end{flushright}
\rule[1ex]{\textwidth}{.1pt}


\begin{questions}
\question
Найдите и упростите P:
\begin{equation*}
\overline{P} = \overline{A} \cap B \cup \overline{A} \cap C \cup A \cap \overline{B} \cup \overline{B} \cap C
\end{equation*}
Затем найдите элементы множества P, выраженного через множества:
\begin{equation*}
A = \{0, 3, 4, 9\}; 
B = \{1, 3, 4, 7\};
C = \{0, 1, 2, 4, 7, 8, 9\};
I = \{0, 1, 2, 3, 4, 5, 6, 7, 8, 9\}.
\end{equation*}\question
Упростите следующее выражение с учетом того, что $A\subset B \subset C \subset D \subset U; A \neq \O$
\begin{equation*}
\overline{A} \cap \overline{B} \cup B \cap \overline{C} \cup \overline{C} \cap D
\end{equation*}

Примечание: U — универсум\question
Дано отношение на множестве $\{1, 2, 3, 4, 5\}$ 
\begin{equation*}
aRb \iff |a-b| = 1
\end{equation*}
Напишите обоснованный ответ какими свойствами обладает или не обладает отношение и почему:   
\begin{enumerate} [a)]\setcounter{enumi}{0}
\item рефлексивность
\item антирефлексивность
\item симметричность
\item асимметричность
\item антисимметричность
\item транзитивность
\end{enumerate}

Обоснуйте свой ответ по каждому из приведенных ниже вопросов:
\begin{enumerate} [a)]\setcounter{enumi}{0}
    \item Является ли это отношение отношением эквивалентности?
    \item Является ли это отношение функциональным?
    \item Каким из отношений соответствия (одно-многозначным, много-многозначный и т.д.) оно является?
    \item К каким из отношений порядка (полного, частичного и т.д.) можно отнести данное отношение?
\end{enumerate}

\question
Установите, является ли каждое из перечисленных ниже отношений на А ($R \subseteq A \times A$) отношением эквивалентности (обоснование ответа обязательно). Для каждого отношения эквивалентности постройте классы 
эквивалентности и постройте граф отношения:
\begin{enumerate} [a)]\setcounter{enumi}{0}
\item А - множество целых чисел и отношение $R = \{(a,b)|a + b = 5\}$
\item Пусть A – множество имен. $A = \{ $Алексей, Иван, Петр, Александр, Павел, Андрей$ \}$. Тогда отношение $R $ верно на парах имен, начинающихся с одной и той же буквы, и только на них.
\item На множестве $A = \{1; 2; 3; 4; 5\}$ задано отношение $R = \{(1; 2); (1; 3); (1; 5); (2; 3); (2; 4); (2; 5); (3; 4); (3; 5); (4; 5)\}$
\end{enumerate}\question Составьте полную таблицу истинности, определите, какие переменные являются фиктивными и проверьте, является ли формула тавтологией:
$(( P \land \neg Q) \rightarrow (R \land \neg R)) \rightarrow (P \rightarrow Q)$

\end{questions}
\newpage
%%% begin test
\begin{flushright}
\begin{tabular}{p{2.8in} r l}
%\textbf{\class} & \textbf{ФИО:} & \makebox[2.5in]{\hrulefill}\\
\textbf{\class} & \textbf{ФИО:} &Медведева Злата Андреевна
\\

\textbf{\examdate} &&\\
%\textbf{Time Limit: \timelimit} & Teaching Assistant & \makebox[2in]{\hrulefill}
\end{tabular}\\
\end{flushright}
\rule[1ex]{\textwidth}{.1pt}


\begin{questions}
\question
Найдите и упростите P:
\begin{equation*}
\overline{P} = A \cap \overline{B} \cup \overline{B} \cap C \cup \overline{A} \cap \overline{B} \cup \overline{A} \cap C
\end{equation*}
Затем найдите элементы множества P, выраженного через множества:
\begin{equation*}
A = \{0, 3, 4, 9\}; 
B = \{1, 3, 4, 7\};
C = \{0, 1, 2, 4, 7, 8, 9\};
I = \{0, 1, 2, 3, 4, 5, 6, 7, 8, 9\}.
\end{equation*}\question
Упростите следующее выражение с учетом того, что $A\subset B \subset C \subset D \subset U; A \neq \O$
\begin{equation*}
A \cap  \overline{C} \cup B \cap \overline{D} \cup  \overline{A} \cap C \cap  \overline{D}
\end{equation*}

Примечание: U — универсум\question
Дано отношение на множестве $\{1, 2, 3, 4, 5\}$ 
\begin{equation*}
aRb \iff a \leq b
\end{equation*}
Напишите обоснованный ответ какими свойствами обладает или не обладает отношение и почему:   
\begin{enumerate} [a)]\setcounter{enumi}{0}
\item рефлексивность
\item антирефлексивность
\item симметричность
\item асимметричность
\item антисимметричность
\item транзитивность
\end{enumerate}

Обоснуйте свой ответ по каждому из приведенных ниже вопросов:
\begin{enumerate} [a)]\setcounter{enumi}{0}
    \item Является ли это отношение отношением эквивалентности?
    \item Является ли это отношение функциональным?
    \item Каким из отношений соответствия (одно-многозначным, много-многозначный и т.д.) оно является?
    \item К каким из отношений порядка (полного, частичного и т.д.) можно отнести данное отношение?
\end{enumerate}


\question
Установите, является ли каждое из перечисленных ниже отношений на А ($R \subseteq A \times A$) отношением эквивалентности (обоснование ответа обязательно). Для каждого отношения эквивалентности 
постройте классы эквивалентности и постройте граф отношения:
\begin{enumerate}[a)]\setcounter{enumi}{0}
\item А - множество целых чисел и отношение $R = \{(a,b)|a + b = 0\}$
\item $A = \{-10, -9, …, 9, 10\}$ и отношение $R = \{(a,b)|a^{3} = b^{3}\}$
\item На множестве $A = \{1; 2; 3\}$ задано отношение $R = \{(1; 1); (2; 2); (3; 3); (2; 1); (1; 2); (2; 3); (3; 2); (3; 1); (1; 3)\}$

\end{enumerate}\question Составьте полную таблицу истинности, определите, какие переменные являются фиктивными и проверьте, является ли формула тавтологией:
$(( P \rightarrow Q) \land (Q \rightarrow P)) \rightarrow (P \rightarrow R)$

\end{questions}
\newpage
%%% begin test
\begin{flushright}
\begin{tabular}{p{2.8in} r l}
%\textbf{\class} & \textbf{ФИО:} & \makebox[2.5in]{\hrulefill}\\
\textbf{\class} & \textbf{ФИО:} &Мизиев Эльдар Ибрагимович
\\

\textbf{\examdate} &&\\
%\textbf{Time Limit: \timelimit} & Teaching Assistant & \makebox[2in]{\hrulefill}
\end{tabular}\\
\end{flushright}
\rule[1ex]{\textwidth}{.1pt}


\begin{questions}
\question
Найдите и упростите P:
\begin{equation*}
\overline{P} = A \cap \overline{B} \cup \overline{B} \cap C \cup \overline{A} \cap \overline{B} \cup \overline{A} \cap C
\end{equation*}
Затем найдите элементы множества P, выраженного через множества:
\begin{equation*}
A = \{0, 3, 4, 9\}; 
B = \{1, 3, 4, 7\};
C = \{0, 1, 2, 4, 7, 8, 9\};
I = \{0, 1, 2, 3, 4, 5, 6, 7, 8, 9\}.
\end{equation*}\question
Упростите следующее выражение с учетом того, что $A\subset B \subset C \subset D \subset U; A \neq \O$
\begin{equation*}
\overline{A} \cap \overline{C} \cap D \cup \overline{B} \cap \overline{C} \cap D \cup A \cap B
\end{equation*}

Примечание: U — универсум\question
Дано отношение на множестве $\{1, 2, 3, 4, 5\}$ 
\begin{equation*}
aRb \iff (a+b) \bmod 2 =0
\end{equation*}
Напишите обоснованный ответ какими свойствами обладает или не обладает отношение и почему:   
\begin{enumerate} [a)]\setcounter{enumi}{0}
\item рефлексивность
\item антирефлексивность
\item симметричность
\item асимметричность
\item антисимметричность
\item транзитивность
\end{enumerate}

Обоснуйте свой ответ по каждому из приведенных ниже вопросов:
\begin{enumerate} [a)]\setcounter{enumi}{0}
    \item Является ли это отношение отношением эквивалентности?
    \item Является ли это отношение функциональным?
    \item Каким из отношений соответствия (одно-многозначным, много-многозначный и т.д.) оно является?
    \item К каким из отношений порядка (полного, частичного и т.д.) можно отнести данное отношение?
\end{enumerate}



\question
Установите, является ли каждое из перечисленных ниже отношений на А ($R \subseteq A \times A$) отношением эквивалентности (обоснование ответа обязательно). Для каждого отношения эквивалентности постройте классы 
эквивалентности и постройте граф отношения:
\begin{enumerate} [a)]\setcounter{enumi}{0}
\item $A = \{-10, -9, … , 9, 10\}$ и отношение $R = \{(a,b)|a^{2} = b^{2}\}$
\item $A = \{a, b, c, d, p, t\}$ задано отношение $R = \{(a, a), (b, b), (b, c), (b, d), (c, b), (c, c), (c, d), (d, b), (d, c), (d, d), (p,p), (t,t)\}$
\item Пусть A – множество имен. $A = \{ $Алексей, Иван, Петр, Александр, Павел, Андрей$ \}$. Тогда отношение $R$ верно на парах имен, начинающихся с одной и той же буквы, и только на них.
\end{enumerate}\question Составьте полную таблицу истинности, определите, какие переменные являются фиктивными и проверьте, является ли формула тавтологией:
$((P \rightarrow Q) \lor R) \leftrightarrow (P \rightarrow (Q \lor R))$

\end{questions}
\newpage
%%% begin test
\begin{flushright}
\begin{tabular}{p{2.8in} r l}
%\textbf{\class} & \textbf{ФИО:} & \makebox[2.5in]{\hrulefill}\\
\textbf{\class} & \textbf{ФИО:} &Мурзин Тимофей Романович
\\

\textbf{\examdate} &&\\
%\textbf{Time Limit: \timelimit} & Teaching Assistant & \makebox[2in]{\hrulefill}
\end{tabular}\\
\end{flushright}
\rule[1ex]{\textwidth}{.1pt}


\begin{questions}
\question
Найдите и упростите P:
\begin{equation*}
\overline{P} = B \cap \overline{C} \cup A \cap B \cup \overline{A} \cap C \cup \overline{A} \cap B
\end{equation*}
Затем найдите элементы множества P, выраженного через множества:
\begin{equation*}
A = \{0, 3, 4, 9\}; 
B = \{1, 3, 4, 7\};
C = \{0, 1, 2, 4, 7, 8, 9\};
I = \{0, 1, 2, 3, 4, 5, 6, 7, 8, 9\}.
\end{equation*}\question
Упростите следующее выражение с учетом того, что $A\subset B \subset C \subset D \subset U; A \neq \O$
\begin{equation*}
A \cap B \cup \overline{A} \cap \overline{C} \cup A \cap C \cup \overline{B} \cap \overline{C}
\end{equation*}

Примечание: U — универсум\question
Дано отношение на множестве $\{1, 2, 3, 4, 5\}$ 
\begin{equation*}
aRb \iff  \text{НОД}(a,b) =1
\end{equation*}
Напишите обоснованный ответ какими свойствами обладает или не обладает отношение и почему:   
\begin{enumerate} [a)]\setcounter{enumi}{0}
\item рефлексивность
\item антирефлексивность
\item симметричность
\item асимметричность
\item антисимметричность
\item транзитивность
\end{enumerate}

Обоснуйте свой ответ по каждому из приведенных ниже вопросов:
\begin{enumerate} [a)]\setcounter{enumi}{0}
    \item Является ли это отношение отношением эквивалентности?
    \item Является ли это отношение функциональным?
    \item Каким из отношений соответствия (одно-многозначным, много-многозначный и т.д.) оно является?
    \item К каким из отношений порядка (полного, частичного и т.д.) можно отнести данное отношение?
\end{enumerate}


\question
Установите, является ли каждое из перечисленных ниже отношений на А ($R \subseteq A \times A$) отношением эквивалентности (обоснование ответа обязательно). Для каждого отношения эквивалентности постройте классы 
эквивалентности и постройте граф отношения:
\begin{enumerate} [a)]\setcounter{enumi}{0}
\item Пусть A – множество имен. $A = \{ $Алексей, Иван, Петр, Александр, Павел, Андрей$ \}$. Тогда отношение $R$ верно на парах имен, начинающихся с одной и той же буквы, и только на них.
\item $A = \{-10, -9, … , 9, 10\}$ и отношение $ R = \{(a,b)|a^{2} = b^{2}\}$
\item На множестве $A = \{1; 2; 3\}$ задано отношение $R = \{(1; 1); (2; 2); (3; 3); (3; 2); (1; 2); (2; 1)\}$
\end{enumerate}\question Составьте полную таблицу истинности, определите, какие переменные являются фиктивными и проверьте, является ли формула тавтологией:
$ P \rightarrow (Q \rightarrow ((P \lor Q) \rightarrow (P \land Q)))$

\end{questions}
\newpage
%%% begin test
\begin{flushright}
\begin{tabular}{p{2.8in} r l}
%\textbf{\class} & \textbf{ФИО:} & \makebox[2.5in]{\hrulefill}\\
\textbf{\class} & \textbf{ФИО:} &Осипов Лев Александрович
\\

\textbf{\examdate} &&\\
%\textbf{Time Limit: \timelimit} & Teaching Assistant & \makebox[2in]{\hrulefill}
\end{tabular}\\
\end{flushright}
\rule[1ex]{\textwidth}{.1pt}


\begin{questions}
\question
Найдите и упростите P:
\begin{equation*}
\overline{P} = \overline{A} \cap B \cup \overline{A} \cap C \cup A \cap \overline{B} \cup \overline{B} \cap C
\end{equation*}
Затем найдите элементы множества P, выраженного через множества:
\begin{equation*}
A = \{0, 3, 4, 9\}; 
B = \{1, 3, 4, 7\};
C = \{0, 1, 2, 4, 7, 8, 9\};
I = \{0, 1, 2, 3, 4, 5, 6, 7, 8, 9\}.
\end{equation*}\question
Упростите следующее выражение с учетом того, что $A\subset B \subset C \subset D \subset U; A \neq \O$
\begin{equation*}
A \cap C  \cap D \cup B \cap \overline{C} \cap D \cup B \cap C \cap D
\end{equation*}

Примечание: U — универсум\question
Дано отношение на множестве $\{1, 2, 3, 4, 5\}$ 
\begin{equation*}
aRb \iff (a+b) \bmod 2 =0
\end{equation*}
Напишите обоснованный ответ какими свойствами обладает или не обладает отношение и почему:   
\begin{enumerate} [a)]\setcounter{enumi}{0}
\item рефлексивность
\item антирефлексивность
\item симметричность
\item асимметричность
\item антисимметричность
\item транзитивность
\end{enumerate}

Обоснуйте свой ответ по каждому из приведенных ниже вопросов:
\begin{enumerate} [a)]\setcounter{enumi}{0}
    \item Является ли это отношение отношением эквивалентности?
    \item Является ли это отношение функциональным?
    \item Каким из отношений соответствия (одно-многозначным, много-многозначный и т.д.) оно является?
    \item К каким из отношений порядка (полного, частичного и т.д.) можно отнести данное отношение?
\end{enumerate}



\question
Установите, является ли каждое из перечисленных ниже отношений на А ($R \subseteq A \times A$) отношением эквивалентности (обоснование ответа обязательно). Для каждого отношения эквивалентности постройте классы эквивалентности и постройте граф отношения:
\begin{enumerate} [a)]\setcounter{enumi}{0}
\item $F(x)=x^{2}+1$, где $x \in A = [-2, 4]$ и отношение $R = \{(a,b)|F(a) = F(b)\}$
\item А - множество целых чисел и отношение $R = \{(a,b)|a + b = 5\}$
\item На множестве $A = \{1; 2; 3\}$ задано отношение $R = \{(1; 1); (2; 2); (3; 3); (3; 2); (1; 2); (2; 1)\}$

\end{enumerate}\question Составьте полную таблицу истинности, определите, какие переменные являются фиктивными и проверьте, является ли формула тавтологией:

$(P \rightarrow (Q \land R)) \leftrightarrow ((P \rightarrow Q) \land (P \rightarrow R))$

\end{questions}
\newpage
%%% begin test
\begin{flushright}
\begin{tabular}{p{2.8in} r l}
%\textbf{\class} & \textbf{ФИО:} & \makebox[2.5in]{\hrulefill}\\
\textbf{\class} & \textbf{ФИО:} &Парфененков Алексей Дмитриевич
\\

\textbf{\examdate} &&\\
%\textbf{Time Limit: \timelimit} & Teaching Assistant & \makebox[2in]{\hrulefill}
\end{tabular}\\
\end{flushright}
\rule[1ex]{\textwidth}{.1pt}


\begin{questions}
\question
Найдите и упростите P:
\begin{equation*}
\overline{P} = B \cap \overline{C} \cup A \cap B \cup \overline{A} \cap C \cup \overline{A} \cap B
\end{equation*}
Затем найдите элементы множества P, выраженного через множества:
\begin{equation*}
A = \{0, 3, 4, 9\}; 
B = \{1, 3, 4, 7\};
C = \{0, 1, 2, 4, 7, 8, 9\};
I = \{0, 1, 2, 3, 4, 5, 6, 7, 8, 9\}.
\end{equation*}\question
Упростите следующее выражение с учетом того, что $A\subset B \subset C \subset D \subset U; A \neq \O$
\begin{equation*}
A \cap B  \cap \overline{C} \cup \overline{C} \cap D \cup B \cap C \cap D
\end{equation*}

Примечание: U — универсум\question
Дано отношение на множестве $\{1, 2, 3, 4, 5\}$ 
\begin{equation*}
aRb \iff  \text{НОД}(a,b) =1
\end{equation*}
Напишите обоснованный ответ какими свойствами обладает или не обладает отношение и почему:   
\begin{enumerate} [a)]\setcounter{enumi}{0}
\item рефлексивность
\item антирефлексивность
\item симметричность
\item асимметричность
\item антисимметричность
\item транзитивность
\end{enumerate}

Обоснуйте свой ответ по каждому из приведенных ниже вопросов:
\begin{enumerate} [a)]\setcounter{enumi}{0}
    \item Является ли это отношение отношением эквивалентности?
    \item Является ли это отношение функциональным?
    \item Каким из отношений соответствия (одно-многозначным, много-многозначный и т.д.) оно является?
    \item К каким из отношений порядка (полного, частичного и т.д.) можно отнести данное отношение?
\end{enumerate}


\question
Установите, является ли каждое из перечисленных ниже отношений на А ($R \subseteq A \times A$) отношением эквивалентности (обоснование ответа обязательно). Для каждого отношения эквивалентности постройте классы 
эквивалентности и постройте граф отношения:
\begin{enumerate} [a)]\setcounter{enumi}{0}
\item $A = \{-10, -9, … , 9, 10\}$ и отношение $R = \{(a,b)|a^{2} = b^{2}\}$
\item $A = \{a, b, c, d, p, t\}$ задано отношение $R = \{(a, a), (b, b), (b, c), (b, d), (c, b), (c, c), (c, d), (d, b), (d, c), (d, d), (p,p), (t,t)\}$
\item Пусть A – множество имен. $A = \{ $Алексей, Иван, Петр, Александр, Павел, Андрей$ \}$. Тогда отношение $R$ верно на парах имен, начинающихся с одной и той же буквы, и только на них.
\end{enumerate}\question Составьте полную таблицу истинности, определите, какие переменные являются фиктивными и проверьте, является ли формула тавтологией:
$ P \rightarrow (Q \rightarrow ((P \lor Q) \rightarrow (P \land Q)))$

\end{questions}
\newpage
%%% begin test
\begin{flushright}
\begin{tabular}{p{2.8in} r l}
%\textbf{\class} & \textbf{ФИО:} & \makebox[2.5in]{\hrulefill}\\
\textbf{\class} & \textbf{ФИО:} &Подшивалов Олег Игоревич
\\

\textbf{\examdate} &&\\
%\textbf{Time Limit: \timelimit} & Teaching Assistant & \makebox[2in]{\hrulefill}
\end{tabular}\\
\end{flushright}
\rule[1ex]{\textwidth}{.1pt}


\begin{questions}
\question
Найдите и упростите P:
\begin{equation*}
\overline{P} = A \cap C \cup \overline{A} \cap \overline{C} \cup \overline{B} \cap C \cup \overline{A} \cap \overline{B}
\end{equation*}
Затем найдите элементы множества P, выраженного через множества:
\begin{equation*}
A = \{0, 3, 4, 9\}; 
B = \{1, 3, 4, 7\};
C = \{0, 1, 2, 4, 7, 8, 9\};
I = \{0, 1, 2, 3, 4, 5, 6, 7, 8, 9\}.
\end{equation*}\question
Упростите следующее выражение с учетом того, что $A\subset B \subset C \subset D \subset U; A \neq \O$
\begin{equation*}
A \cap B  \cap \overline{C} \cup \overline{C} \cap D \cup B \cap C \cap D
\end{equation*}

Примечание: U — универсум\question
Дано отношение на множестве $\{1, 2, 3, 4, 5\}$ 
\begin{equation*}
aRb \iff a \geq b^2
\end{equation*}
Напишите обоснованный ответ какими свойствами обладает или не обладает отношение и почему:   
\begin{enumerate} [a)]\setcounter{enumi}{0}
\item рефлексивность
\item антирефлексивность
\item симметричность
\item асимметричность
\item антисимметричность
\item транзитивность
\end{enumerate}

Обоснуйте свой ответ по каждому из приведенных ниже вопросов:
\begin{enumerate} [a)]\setcounter{enumi}{0}
    \item Является ли это отношение отношением эквивалентности?
    \item Является ли это отношение функциональным?
    \item Каким из отношений соответствия (одно-многозначным, много-многозначный и т.д.) оно является?
    \item К каким из отношений порядка (полного, частичного и т.д.) можно отнести данное отношение?
\end{enumerate}


\question
Установите, является ли каждое из перечисленных ниже отношений на А ($R \subseteq A \times A$) отношением эквивалентности (обоснование ответа обязательно). Для каждого отношения эквивалентности постройте классы 
эквивалентности и постройте граф отношения:
\begin{enumerate} [a)]\setcounter{enumi}{0}
\item А - множество целых чисел и отношение $R = \{(a,b)|a + b = 5\}$
\item Пусть A – множество имен. $A = \{ $Алексей, Иван, Петр, Александр, Павел, Андрей$ \}$. Тогда отношение $R $ верно на парах имен, начинающихся с одной и той же буквы, и только на них.
\item На множестве $A = \{1; 2; 3; 4; 5\}$ задано отношение $R = \{(1; 2); (1; 3); (1; 5); (2; 3); (2; 4); (2; 5); (3; 4); (3; 5); (4; 5)\}$
\end{enumerate}\question Составьте полную таблицу истинности, определите, какие переменные являются фиктивными и проверьте, является ли формула тавтологией:
$((P \rightarrow Q) \land (R \rightarrow S) \land \neg (Q \lor S)) \rightarrow \neg (P \lor R)$

\end{questions}
\newpage
%%% begin test
\begin{flushright}
\begin{tabular}{p{2.8in} r l}
%\textbf{\class} & \textbf{ФИО:} & \makebox[2.5in]{\hrulefill}\\
\textbf{\class} & \textbf{ФИО:} &Реброва Татьяна Ивановна
\\

\textbf{\examdate} &&\\
%\textbf{Time Limit: \timelimit} & Teaching Assistant & \makebox[2in]{\hrulefill}
\end{tabular}\\
\end{flushright}
\rule[1ex]{\textwidth}{.1pt}


\begin{questions}
\question
Найдите и упростите P:
\begin{equation*}
\overline{P} = A \cap C \cup \overline{A} \cap \overline{C} \cup \overline{B} \cap C \cup \overline{A} \cap \overline{B}
\end{equation*}
Затем найдите элементы множества P, выраженного через множества:
\begin{equation*}
A = \{0, 3, 4, 9\}; 
B = \{1, 3, 4, 7\};
C = \{0, 1, 2, 4, 7, 8, 9\};
I = \{0, 1, 2, 3, 4, 5, 6, 7, 8, 9\}.
\end{equation*}\question
Упростите следующее выражение с учетом того, что $A\subset B \subset C \subset D \subset U; A \neq \O$
\begin{equation*}
\overline{B} \cap \overline{C} \cap D \cup \overline{A} \cap \overline{C} \cap D \cup \overline{A} \cap B
\end{equation*}

Примечание: U — универсум\question
Дано отношение на множестве $\{1, 2, 3, 4, 5\}$ 
\begin{equation*}
aRb \iff  \text{НОД}(a,b) =1
\end{equation*}
Напишите обоснованный ответ какими свойствами обладает или не обладает отношение и почему:   
\begin{enumerate} [a)]\setcounter{enumi}{0}
\item рефлексивность
\item антирефлексивность
\item симметричность
\item асимметричность
\item антисимметричность
\item транзитивность
\end{enumerate}

Обоснуйте свой ответ по каждому из приведенных ниже вопросов:
\begin{enumerate} [a)]\setcounter{enumi}{0}
    \item Является ли это отношение отношением эквивалентности?
    \item Является ли это отношение функциональным?
    \item Каким из отношений соответствия (одно-многозначным, много-многозначный и т.д.) оно является?
    \item К каким из отношений порядка (полного, частичного и т.д.) можно отнести данное отношение?
\end{enumerate}


\question
Установите, является ли каждое из перечисленных ниже отношений на А ($R \subseteq A \times A$) отношением эквивалентности (обоснование ответа обязательно). Для каждого отношения эквивалентности 
постройте классы эквивалентности и постройте граф отношения:
\begin{enumerate}[a)]\setcounter{enumi}{0}
\item А - множество целых чисел и отношение $R = \{(a,b)|a + b = 0\}$
\item $A = \{-10, -9, …, 9, 10\}$ и отношение $R = \{(a,b)|a^{3} = b^{3}\}$
\item На множестве $A = \{1; 2; 3\}$ задано отношение $R = \{(1; 1); (2; 2); (3; 3); (2; 1); (1; 2); (2; 3); (3; 2); (3; 1); (1; 3)\}$

\end{enumerate}\question Составьте полную таблицу истинности, определите, какие переменные являются фиктивными и проверьте, является ли формула тавтологией:
$((P \rightarrow Q) \lor R) \leftrightarrow (P \rightarrow (Q \lor R))$

\end{questions}
\newpage
%%% begin test
\begin{flushright}
\begin{tabular}{p{2.8in} r l}
%\textbf{\class} & \textbf{ФИО:} & \makebox[2.5in]{\hrulefill}\\
\textbf{\class} & \textbf{ФИО:} &Ремизов Ростислав Олегович
\\

\textbf{\examdate} &&\\
%\textbf{Time Limit: \timelimit} & Teaching Assistant & \makebox[2in]{\hrulefill}
\end{tabular}\\
\end{flushright}
\rule[1ex]{\textwidth}{.1pt}


\begin{questions}
\question
Найдите и упростите P:
\begin{equation*}
\overline{P} = A \cap \overline{B} \cup \overline{B} \cap C \cup \overline{A} \cap \overline{B} \cup \overline{A} \cap C
\end{equation*}
Затем найдите элементы множества P, выраженного через множества:
\begin{equation*}
A = \{0, 3, 4, 9\}; 
B = \{1, 3, 4, 7\};
C = \{0, 1, 2, 4, 7, 8, 9\};
I = \{0, 1, 2, 3, 4, 5, 6, 7, 8, 9\}.
\end{equation*}\question
Упростите следующее выражение с учетом того, что $A\subset B \subset C \subset D \subset U; A \neq \O$
\begin{equation*}
A \cap B \cup \overline{A} \cap \overline{C} \cup A \cap C \cup \overline{B} \cap \overline{C}
\end{equation*}

Примечание: U — универсум\question
Дано отношение на множестве $\{1, 2, 3, 4, 5\}$ 
\begin{equation*}
aRb \iff a \geq b^2
\end{equation*}
Напишите обоснованный ответ какими свойствами обладает или не обладает отношение и почему:   
\begin{enumerate} [a)]\setcounter{enumi}{0}
\item рефлексивность
\item антирефлексивность
\item симметричность
\item асимметричность
\item антисимметричность
\item транзитивность
\end{enumerate}

Обоснуйте свой ответ по каждому из приведенных ниже вопросов:
\begin{enumerate} [a)]\setcounter{enumi}{0}
    \item Является ли это отношение отношением эквивалентности?
    \item Является ли это отношение функциональным?
    \item Каким из отношений соответствия (одно-многозначным, много-многозначный и т.д.) оно является?
    \item К каким из отношений порядка (полного, частичного и т.д.) можно отнести данное отношение?
\end{enumerate}


\question
Установите, является ли каждое из перечисленных ниже отношений на А ($R \subseteq A \times A$) отношением эквивалентности (обоснование ответа обязательно). Для каждого отношения эквивалентности постройте классы 
эквивалентности и постройте граф отношения:
\begin{enumerate} [a)]\setcounter{enumi}{0}
\item Пусть A – множество имен. $A = \{ $Алексей, Иван, Петр, Александр, Павел, Андрей$ \}$. Тогда отношение $R$ верно на парах имен, начинающихся с одной и той же буквы, и только на них.
\item $A = \{-10, -9, … , 9, 10\}$ и отношение $ R = \{(a,b)|a^{2} = b^{2}\}$
\item На множестве $A = \{1; 2; 3\}$ задано отношение $R = \{(1; 1); (2; 2); (3; 3); (3; 2); (1; 2); (2; 1)\}$
\end{enumerate}\question Составьте полную таблицу истинности, определите, какие переменные являются фиктивными и проверьте, является ли формула тавтологией:
$(P \rightarrow (Q \rightarrow R)) \rightarrow ((P \rightarrow Q) \rightarrow (P \rightarrow R))$

\end{questions}
\newpage
%%% begin test
\begin{flushright}
\begin{tabular}{p{2.8in} r l}
%\textbf{\class} & \textbf{ФИО:} & \makebox[2.5in]{\hrulefill}\\
\textbf{\class} & \textbf{ФИО:} &Свириденков Владимир Анатольевич
\\

\textbf{\examdate} &&\\
%\textbf{Time Limit: \timelimit} & Teaching Assistant & \makebox[2in]{\hrulefill}
\end{tabular}\\
\end{flushright}
\rule[1ex]{\textwidth}{.1pt}


\begin{questions}
\question
Найдите и упростите P:
\begin{equation*}
\overline{P} = B \cap \overline{C} \cup A \cap B \cup \overline{A} \cap C \cup \overline{A} \cap B
\end{equation*}
Затем найдите элементы множества P, выраженного через множества:
\begin{equation*}
A = \{0, 3, 4, 9\}; 
B = \{1, 3, 4, 7\};
C = \{0, 1, 2, 4, 7, 8, 9\};
I = \{0, 1, 2, 3, 4, 5, 6, 7, 8, 9\}.
\end{equation*}\question
Упростите следующее выражение с учетом того, что $A\subset B \subset C \subset D \subset U; A \neq \O$
\begin{equation*}
A \cap B  \cap \overline{C} \cup \overline{C} \cap D \cup B \cap C \cap D
\end{equation*}

Примечание: U — универсум\question
Для следующего отношения на множестве $\{1, 2, 3, 4, 5\}$ 
\begin{equation*}
aRb \iff 0 < a-b<2
\end{equation*}
Напишите обоснованный ответ какими свойствами обладает или не обладает отношение и почему:   
\begin{enumerate} [a)]\setcounter{enumi}{0}
\item рефлексивность
\item антирефлексивность
\item симметричность
\item асимметричность
\item антисимметричность
\item транзитивность
\end{enumerate}

Обоснуйте свой ответ по каждому из приведенных ниже вопросов:
\begin{enumerate} [a)]\setcounter{enumi}{0}
    \item Является ли это отношение отношением эквивалентности?
    \item Является ли это отношение функциональным?
    \item Каким из отношений соответствия (одно-многозначным, много-многозначный и т.д.) оно является?
    \item К каким из отношений порядка (полного, частичного и т.д.) можно отнести данное отношение?
\end{enumerate}
\question
Установите, является ли каждое из перечисленных ниже отношений на А ($R \subseteq A \times A$) отношением эквивалентности (обоснование ответа обязательно). Для каждого отношения эквивалентности постройте классы 
эквивалентности и постройте граф отношения:
\begin{enumerate} [a)]\setcounter{enumi}{0}
\item $A = \{a, b, c, d, p, t\}$ задано отношение $R = \{(a, a), (b, b), (b, c), (b, d), (c, b), (c, c), (c, d), (d, b), (d, c), (d, d), (p,p), (t,t)\}$
\item $A = \{-10, -9, … , 9, 10\}$ и отношение $R = \{(a,b)|a^{3} = b^{3}\}$

\item $F(x)=x^{2}+1$, где $x \in A = [-2, 4]$ и отношение $R = \{(a,b)|F(a) = F(b)\}$
\end{enumerate}\question Составьте полную таблицу истинности, определите, какие переменные являются фиктивными и проверьте, является ли формула тавтологией:
$(( P \rightarrow Q) \land (Q \rightarrow P)) \rightarrow (P \rightarrow R)$

\end{questions}
\newpage
%%% begin test
\begin{flushright}
\begin{tabular}{p{2.8in} r l}
%\textbf{\class} & \textbf{ФИО:} & \makebox[2.5in]{\hrulefill}\\
\textbf{\class} & \textbf{ФИО:} &Семенов Кирилл Александрович
\\

\textbf{\examdate} &&\\
%\textbf{Time Limit: \timelimit} & Teaching Assistant & \makebox[2in]{\hrulefill}
\end{tabular}\\
\end{flushright}
\rule[1ex]{\textwidth}{.1pt}


\begin{questions}
\question
Найдите и упростите P:
\begin{equation*}
\overline{P} = A \cap \overline{B} \cup A \cap C \cup B \cap C \cup \overline{A} \cap C
\end{equation*}
Затем найдите элементы множества P, выраженного через множества:
\begin{equation*}
A = \{0, 3, 4, 9\}; 
B = \{1, 3, 4, 7\};
C = \{0, 1, 2, 4, 7, 8, 9\};
I = \{0, 1, 2, 3, 4, 5, 6, 7, 8, 9\}.
\end{equation*}\question
Упростите следующее выражение с учетом того, что $A\subset B \subset C \subset D \subset U; A \neq \O$
\begin{equation*}
A \cap B \cup \overline{A} \cap \overline{C} \cup A \cap C \cup \overline{B} \cap \overline{C}
\end{equation*}

Примечание: U — универсум\question
Дано отношение на множестве $\{1, 2, 3, 4, 5\}$ 
\begin{equation*}
aRb \iff (a+b) \bmod 2 =0
\end{equation*}
Напишите обоснованный ответ какими свойствами обладает или не обладает отношение и почему:   
\begin{enumerate} [a)]\setcounter{enumi}{0}
\item рефлексивность
\item антирефлексивность
\item симметричность
\item асимметричность
\item антисимметричность
\item транзитивность
\end{enumerate}

Обоснуйте свой ответ по каждому из приведенных ниже вопросов:
\begin{enumerate} [a)]\setcounter{enumi}{0}
    \item Является ли это отношение отношением эквивалентности?
    \item Является ли это отношение функциональным?
    \item Каким из отношений соответствия (одно-многозначным, много-многозначный и т.д.) оно является?
    \item К каким из отношений порядка (полного, частичного и т.д.) можно отнести данное отношение?
\end{enumerate}



\question
Установите, является ли каждое из перечисленных ниже отношений на А ($R \subseteq A \times A$) отношением эквивалентности (обоснование ответа обязательно). Для каждого отношения эквивалентности постройте классы 
эквивалентности и постройте граф отношения:
\begin{enumerate} [a)]\setcounter{enumi}{0}
\item А - множество целых чисел и отношение $R = \{(a,b)|a + b = 5\}$
\item Пусть A – множество имен. $A = \{ $Алексей, Иван, Петр, Александр, Павел, Андрей$ \}$. Тогда отношение $R $ верно на парах имен, начинающихся с одной и той же буквы, и только на них.
\item На множестве $A = \{1; 2; 3; 4; 5\}$ задано отношение $R = \{(1; 2); (1; 3); (1; 5); (2; 3); (2; 4); (2; 5); (3; 4); (3; 5); (4; 5)\}$
\end{enumerate}\question Составьте полную таблицу истинности, определите, какие переменные являются фиктивными и проверьте, является ли формула тавтологией:
$((P \rightarrow Q) \lor R) \leftrightarrow (P \rightarrow (Q \lor R))$

\end{questions}
\newpage
%%% begin test
\begin{flushright}
\begin{tabular}{p{2.8in} r l}
%\textbf{\class} & \textbf{ФИО:} & \makebox[2.5in]{\hrulefill}\\
\textbf{\class} & \textbf{ФИО:} &Соболева Елена Васильевна
\\

\textbf{\examdate} &&\\
%\textbf{Time Limit: \timelimit} & Teaching Assistant & \makebox[2in]{\hrulefill}
\end{tabular}\\
\end{flushright}
\rule[1ex]{\textwidth}{.1pt}


\begin{questions}
\question
Найдите и упростите P:
\begin{equation*}
\overline{P} = A \cap \overline{B} \cup \overline{B} \cap C \cup \overline{A} \cap \overline{B} \cup \overline{A} \cap C
\end{equation*}
Затем найдите элементы множества P, выраженного через множества:
\begin{equation*}
A = \{0, 3, 4, 9\}; 
B = \{1, 3, 4, 7\};
C = \{0, 1, 2, 4, 7, 8, 9\};
I = \{0, 1, 2, 3, 4, 5, 6, 7, 8, 9\}.
\end{equation*}\question
Упростите следующее выражение с учетом того, что $A\subset B \subset C \subset D \subset U; A \neq \O$
\begin{equation*}
A \cap C  \cap D \cup B \cap \overline{C} \cap D \cup B \cap C \cap D
\end{equation*}

Примечание: U — универсум\question
Для следующего отношения на множестве $\{1, 2, 3, 4, 5\}$ 
\begin{equation*}
aRb \iff 0 < a-b<2
\end{equation*}
Напишите обоснованный ответ какими свойствами обладает или не обладает отношение и почему:   
\begin{enumerate} [a)]\setcounter{enumi}{0}
\item рефлексивность
\item антирефлексивность
\item симметричность
\item асимметричность
\item антисимметричность
\item транзитивность
\end{enumerate}

Обоснуйте свой ответ по каждому из приведенных ниже вопросов:
\begin{enumerate} [a)]\setcounter{enumi}{0}
    \item Является ли это отношение отношением эквивалентности?
    \item Является ли это отношение функциональным?
    \item Каким из отношений соответствия (одно-многозначным, много-многозначный и т.д.) оно является?
    \item К каким из отношений порядка (полного, частичного и т.д.) можно отнести данное отношение?
\end{enumerate}
\question
Установите, является ли каждое из перечисленных ниже отношений на А ($R \subseteq A \times A$) отношением эквивалентности (обоснование ответа обязательно). Для каждого отношения эквивалентности постройте классы 
эквивалентности и постройте граф отношения:
\begin{enumerate} [a)]\setcounter{enumi}{0}
\item На множестве $A = \{1; 2; 3\}$ задано отношение $R = \{(1; 1); (2; 2); (3; 3); (2; 1); (1; 2); (2; 3); (3; 2); (3; 1); (1; 3)\}$
\item На множестве $A = \{1; 2; 3; 4; 5\}$ задано отношение $R = \{(1; 2); (1; 3); (1; 5); (2; 3); (2; 4); (2; 5); (3; 4); (3; 5); (4; 5)\}$
\item А - множество целых чисел и отношение $R = \{(a,b)|a + b = 0\}$
\end{enumerate}\question Составьте полную таблицу истинности, определите, какие переменные являются фиктивными и проверьте, является ли формула тавтологией:
$((P \rightarrow Q) \lor R) \leftrightarrow (P \rightarrow (Q \lor R))$

\end{questions}
\newpage
%%% begin test
\begin{flushright}
\begin{tabular}{p{2.8in} r l}
%\textbf{\class} & \textbf{ФИО:} & \makebox[2.5in]{\hrulefill}\\
\textbf{\class} & \textbf{ФИО:} &Федорова Алёна Алексеевна
\\

\textbf{\examdate} &&\\
%\textbf{Time Limit: \timelimit} & Teaching Assistant & \makebox[2in]{\hrulefill}
\end{tabular}\\
\end{flushright}
\rule[1ex]{\textwidth}{.1pt}


\begin{questions}
\question
Найдите и упростите P:
\begin{equation*}
\overline{P} = \overline{A} \cap B \cup \overline{A} \cap C \cup A \cap \overline{B} \cup \overline{B} \cap C
\end{equation*}
Затем найдите элементы множества P, выраженного через множества:
\begin{equation*}
A = \{0, 3, 4, 9\}; 
B = \{1, 3, 4, 7\};
C = \{0, 1, 2, 4, 7, 8, 9\};
I = \{0, 1, 2, 3, 4, 5, 6, 7, 8, 9\}.
\end{equation*}\question
Упростите следующее выражение с учетом того, что $A\subset B \subset C \subset D \subset U; A \neq \O$
\begin{equation*}
\overline{B} \cap \overline{C} \cap D \cup \overline{A} \cap \overline{C} \cap D \cup \overline{A} \cap B
\end{equation*}

Примечание: U — универсум\question
Дано отношение на множестве $\{1, 2, 3, 4, 5\}$ 
\begin{equation*}
aRb \iff b > a
\end{equation*}
Напишите обоснованный ответ какими свойствами обладает или не обладает отношение и почему:   
\begin{enumerate} [a)]\setcounter{enumi}{0}
\item рефлексивность
\item антирефлексивность
\item симметричность
\item асимметричность
\item антисимметричность
\item транзитивность
\end{enumerate}

Обоснуйте свой ответ по каждому из приведенных ниже вопросов:
\begin{enumerate} [a)]\setcounter{enumi}{0}
    \item Является ли это отношение отношением эквивалентности?
    \item Является ли это отношение функциональным?
    \item Каким из отношений соответствия (одно-многозначным, много-многозначный и т.д.) оно является?
    \item К каким из отношений порядка (полного, частичного и т.д.) можно отнести данное отношение?
\end{enumerate}

\question
Установите, является ли каждое из перечисленных ниже отношений на А ($R \subseteq A \times A$) отношением эквивалентности (обоснование ответа обязательно). Для каждого отношения эквивалентности 
постройте классы эквивалентности и постройте граф отношения:
\begin{enumerate}[a)]\setcounter{enumi}{0}
\item А - множество целых чисел и отношение $R = \{(a,b)|a + b = 0\}$
\item $A = \{-10, -9, …, 9, 10\}$ и отношение $R = \{(a,b)|a^{3} = b^{3}\}$
\item На множестве $A = \{1; 2; 3\}$ задано отношение $R = \{(1; 1); (2; 2); (3; 3); (2; 1); (1; 2); (2; 3); (3; 2); (3; 1); (1; 3)\}$

\end{enumerate}\question Составьте полную таблицу истинности, определите, какие переменные являются фиктивными и проверьте, является ли формула тавтологией:
$(( P \land \neg Q) \rightarrow (R \land \neg R)) \rightarrow (P \rightarrow Q)$

\end{questions}
\newpage
%%% begin test
\begin{flushright}
\begin{tabular}{p{2.8in} r l}
%\textbf{\class} & \textbf{ФИО:} & \makebox[2.5in]{\hrulefill}\\
\textbf{\class} & \textbf{ФИО:} &Шамсутдинов Ислам Альбертович
\\

\textbf{\examdate} &&\\
%\textbf{Time Limit: \timelimit} & Teaching Assistant & \makebox[2in]{\hrulefill}
\end{tabular}\\
\end{flushright}
\rule[1ex]{\textwidth}{.1pt}


\begin{questions}
\question
Найдите и упростите P:
\begin{equation*}
\overline{P} = A \cap \overline{B} \cup \overline{B} \cap C \cup \overline{A} \cap \overline{B} \cup \overline{A} \cap C
\end{equation*}
Затем найдите элементы множества P, выраженного через множества:
\begin{equation*}
A = \{0, 3, 4, 9\}; 
B = \{1, 3, 4, 7\};
C = \{0, 1, 2, 4, 7, 8, 9\};
I = \{0, 1, 2, 3, 4, 5, 6, 7, 8, 9\}.
\end{equation*}\question
Упростите следующее выражение с учетом того, что $A\subset B \subset C \subset D \subset U; A \neq \O$
\begin{equation*}
A \cap B \cup \overline{A} \cap \overline{C} \cup A \cap C \cup \overline{B} \cap \overline{C}
\end{equation*}

Примечание: U — универсум\question
Дано отношение на множестве $\{1, 2, 3, 4, 5\}$ 
\begin{equation*}
aRb \iff b > a
\end{equation*}
Напишите обоснованный ответ какими свойствами обладает или не обладает отношение и почему:   
\begin{enumerate} [a)]\setcounter{enumi}{0}
\item рефлексивность
\item антирефлексивность
\item симметричность
\item асимметричность
\item антисимметричность
\item транзитивность
\end{enumerate}

Обоснуйте свой ответ по каждому из приведенных ниже вопросов:
\begin{enumerate} [a)]\setcounter{enumi}{0}
    \item Является ли это отношение отношением эквивалентности?
    \item Является ли это отношение функциональным?
    \item Каким из отношений соответствия (одно-многозначным, много-многозначный и т.д.) оно является?
    \item К каким из отношений порядка (полного, частичного и т.д.) можно отнести данное отношение?
\end{enumerate}

\question
Установите, является ли каждое из перечисленных ниже отношений на А ($R \subseteq A \times A$) отношением эквивалентности (обоснование ответа обязательно). Для каждого отношения эквивалентности 
постройте классы эквивалентности и постройте граф отношения:
\begin{enumerate}[a)]\setcounter{enumi}{0}
\item А - множество целых чисел и отношение $R = \{(a,b)|a + b = 0\}$
\item $A = \{-10, -9, …, 9, 10\}$ и отношение $R = \{(a,b)|a^{3} = b^{3}\}$
\item На множестве $A = \{1; 2; 3\}$ задано отношение $R = \{(1; 1); (2; 2); (3; 3); (2; 1); (1; 2); (2; 3); (3; 2); (3; 1); (1; 3)\}$

\end{enumerate}\question Составьте полную таблицу истинности, определите, какие переменные являются фиктивными и проверьте, является ли формула тавтологией:
$(( P \land \neg Q) \rightarrow (R \land \neg R)) \rightarrow (P \rightarrow Q)$

\end{questions}
\newpage
%%% begin test
\begin{flushright}
\begin{tabular}{p{2.8in} r l}
%\textbf{\class} & \textbf{ФИО:} & \makebox[2.5in]{\hrulefill}\\
\textbf{\class} & \textbf{ФИО:} &М3111
\\

\textbf{\examdate} &&\\
%\textbf{Time Limit: \timelimit} & Teaching Assistant & \makebox[2in]{\hrulefill}
\end{tabular}\\
\end{flushright}
\rule[1ex]{\textwidth}{.1pt}


\begin{questions}
\question
Найдите и упростите P:
\begin{equation*}
\overline{P} = A \cap \overline{C} \cup A \cap \overline{B} \cup B \cap \overline{C} \cup A \cap C
\end{equation*}
Затем найдите элементы множества P, выраженного через множества:
\begin{equation*}
A = \{0, 3, 4, 9\}; 
B = \{1, 3, 4, 7\};
C = \{0, 1, 2, 4, 7, 8, 9\};
I = \{0, 1, 2, 3, 4, 5, 6, 7, 8, 9\}.
\end{equation*}\question
Упростите следующее выражение с учетом того, что $A\subset B \subset C \subset D \subset U; A \neq \O$
\begin{equation*}
A \cap B  \cap \overline{C} \cup \overline{C} \cap D \cup B \cap C \cap D
\end{equation*}

Примечание: U — универсум\question
Дано отношение на множестве $\{1, 2, 3, 4, 5\}$ 
\begin{equation*}
aRb \iff (a+b) \bmod 2 =0
\end{equation*}
Напишите обоснованный ответ какими свойствами обладает или не обладает отношение и почему:   
\begin{enumerate} [a)]\setcounter{enumi}{0}
\item рефлексивность
\item антирефлексивность
\item симметричность
\item асимметричность
\item антисимметричность
\item транзитивность
\end{enumerate}

Обоснуйте свой ответ по каждому из приведенных ниже вопросов:
\begin{enumerate} [a)]\setcounter{enumi}{0}
    \item Является ли это отношение отношением эквивалентности?
    \item Является ли это отношение функциональным?
    \item Каким из отношений соответствия (одно-многозначным, много-многозначный и т.д.) оно является?
    \item К каким из отношений порядка (полного, частичного и т.д.) можно отнести данное отношение?
\end{enumerate}



\question
Установите, является ли каждое из перечисленных ниже отношений на А ($R \subseteq A \times A$) отношением эквивалентности (обоснование ответа обязательно). Для каждого отношения эквивалентности постройте классы 
эквивалентности и постройте граф отношения:
\begin{enumerate} [a)]\setcounter{enumi}{0}
\item $A = \{a, b, c, d, p, t\}$ задано отношение $R = \{(a, a), (b, b), (b, c), (b, d), (c, b), (c, c), (c, d), (d, b), (d, c), (d, d), (p,p), (t,t)\}$
\item $A = \{-10, -9, … , 9, 10\}$ и отношение $R = \{(a,b)|a^{3} = b^{3}\}$

\item $F(x)=x^{2}+1$, где $x \in A = [-2, 4]$ и отношение $R = \{(a,b)|F(a) = F(b)\}$
\end{enumerate}\question Составьте полную таблицу истинности, определите, какие переменные являются фиктивными и проверьте, является ли формула тавтологией:
$((P \rightarrow Q) \lor R) \leftrightarrow (P \rightarrow (Q \lor R))$

\end{questions}
\newpage
%%% begin test
\begin{flushright}
\begin{tabular}{p{2.8in} r l}
%\textbf{\class} & \textbf{ФИО:} & \makebox[2.5in]{\hrulefill}\\
\textbf{\class} & \textbf{ФИО:} &Бурнашева Марина Мариковна
\\

\textbf{\examdate} &&\\
%\textbf{Time Limit: \timelimit} & Teaching Assistant & \makebox[2in]{\hrulefill}
\end{tabular}\\
\end{flushright}
\rule[1ex]{\textwidth}{.1pt}


\begin{questions}
\question
Найдите и упростите P:
\begin{equation*}
\overline{P} = A \cap \overline{B} \cup \overline{B} \cap C \cup \overline{A} \cap \overline{B} \cup \overline{A} \cap C
\end{equation*}
Затем найдите элементы множества P, выраженного через множества:
\begin{equation*}
A = \{0, 3, 4, 9\}; 
B = \{1, 3, 4, 7\};
C = \{0, 1, 2, 4, 7, 8, 9\};
I = \{0, 1, 2, 3, 4, 5, 6, 7, 8, 9\}.
\end{equation*}\question
Упростите следующее выражение с учетом того, что $A\subset B \subset C \subset D \subset U; A \neq \O$
\begin{equation*}
\overline{B} \cap \overline{C} \cap D \cup \overline{A} \cap \overline{C} \cap D \cup \overline{A} \cap B
\end{equation*}

Примечание: U — универсум\question
Дано отношение на множестве $\{1, 2, 3, 4, 5\}$ 
\begin{equation*}
aRb \iff b > a
\end{equation*}
Напишите обоснованный ответ какими свойствами обладает или не обладает отношение и почему:   
\begin{enumerate} [a)]\setcounter{enumi}{0}
\item рефлексивность
\item антирефлексивность
\item симметричность
\item асимметричность
\item антисимметричность
\item транзитивность
\end{enumerate}

Обоснуйте свой ответ по каждому из приведенных ниже вопросов:
\begin{enumerate} [a)]\setcounter{enumi}{0}
    \item Является ли это отношение отношением эквивалентности?
    \item Является ли это отношение функциональным?
    \item Каким из отношений соответствия (одно-многозначным, много-многозначный и т.д.) оно является?
    \item К каким из отношений порядка (полного, частичного и т.д.) можно отнести данное отношение?
\end{enumerate}

\question
Установите, является ли каждое из перечисленных ниже отношений на А ($R \subseteq A \times A$) отношением эквивалентности (обоснование ответа обязательно). Для каждого отношения эквивалентности постройте классы 
эквивалентности и постройте граф отношения:
\begin{enumerate} [a)]\setcounter{enumi}{0}
\item Пусть A – множество имен. $A = \{ $Алексей, Иван, Петр, Александр, Павел, Андрей$ \}$. Тогда отношение $R$ верно на парах имен, начинающихся с одной и той же буквы, и только на них.
\item $A = \{-10, -9, … , 9, 10\}$ и отношение $ R = \{(a,b)|a^{2} = b^{2}\}$
\item На множестве $A = \{1; 2; 3\}$ задано отношение $R = \{(1; 1); (2; 2); (3; 3); (3; 2); (1; 2); (2; 1)\}$
\end{enumerate}\question Составьте полную таблицу истинности, определите, какие переменные являются фиктивными и проверьте, является ли формула тавтологией:
$(P \rightarrow (Q \rightarrow R)) \rightarrow ((P \rightarrow Q) \rightarrow (P \rightarrow R))$

\end{questions}
\newpage
%%% begin test
\begin{flushright}
\begin{tabular}{p{2.8in} r l}
%\textbf{\class} & \textbf{ФИО:} & \makebox[2.5in]{\hrulefill}\\
\textbf{\class} & \textbf{ФИО:} &Васютинская Ксения Сергеевна
\\

\textbf{\examdate} &&\\
%\textbf{Time Limit: \timelimit} & Teaching Assistant & \makebox[2in]{\hrulefill}
\end{tabular}\\
\end{flushright}
\rule[1ex]{\textwidth}{.1pt}


\begin{questions}
\question
Найдите и упростите P:
\begin{equation*}
\overline{P} = \overline{A} \cap B \cup \overline{A} \cap C \cup A \cap \overline{B} \cup \overline{B} \cap C
\end{equation*}
Затем найдите элементы множества P, выраженного через множества:
\begin{equation*}
A = \{0, 3, 4, 9\}; 
B = \{1, 3, 4, 7\};
C = \{0, 1, 2, 4, 7, 8, 9\};
I = \{0, 1, 2, 3, 4, 5, 6, 7, 8, 9\}.
\end{equation*}\question
Упростите следующее выражение с учетом того, что $A\subset B \subset C \subset D \subset U; A \neq \O$
\begin{equation*}
A \cap B \cup \overline{A} \cap \overline{C} \cup A \cap C \cup \overline{B} \cap \overline{C}
\end{equation*}

Примечание: U — универсум\question
Дано отношение на множестве $\{1, 2, 3, 4, 5\}$ 
\begin{equation*}
aRb \iff a \leq b
\end{equation*}
Напишите обоснованный ответ какими свойствами обладает или не обладает отношение и почему:   
\begin{enumerate} [a)]\setcounter{enumi}{0}
\item рефлексивность
\item антирефлексивность
\item симметричность
\item асимметричность
\item антисимметричность
\item транзитивность
\end{enumerate}

Обоснуйте свой ответ по каждому из приведенных ниже вопросов:
\begin{enumerate} [a)]\setcounter{enumi}{0}
    \item Является ли это отношение отношением эквивалентности?
    \item Является ли это отношение функциональным?
    \item Каким из отношений соответствия (одно-многозначным, много-многозначный и т.д.) оно является?
    \item К каким из отношений порядка (полного, частичного и т.д.) можно отнести данное отношение?
\end{enumerate}


\question
Установите, является ли каждое из перечисленных ниже отношений на А ($R \subseteq A \times A$) отношением эквивалентности (обоснование ответа обязательно). Для каждого отношения эквивалентности постройте классы 
эквивалентности и постройте граф отношения:
\begin{enumerate} [a)]\setcounter{enumi}{0}
\item $A = \{-10, -9, … , 9, 10\}$ и отношение $R = \{(a,b)|a^{2} = b^{2}\}$
\item $A = \{a, b, c, d, p, t\}$ задано отношение $R = \{(a, a), (b, b), (b, c), (b, d), (c, b), (c, c), (c, d), (d, b), (d, c), (d, d), (p,p), (t,t)\}$
\item Пусть A – множество имен. $A = \{ $Алексей, Иван, Петр, Александр, Павел, Андрей$ \}$. Тогда отношение $R$ верно на парах имен, начинающихся с одной и той же буквы, и только на них.
\end{enumerate}\question Составьте полную таблицу истинности, определите, какие переменные являются фиктивными и проверьте, является ли формула тавтологией:
$(P \rightarrow (Q \rightarrow R)) \rightarrow ((P \rightarrow Q) \rightarrow (P \rightarrow R))$

\end{questions}
\newpage
%%% begin test
\begin{flushright}
\begin{tabular}{p{2.8in} r l}
%\textbf{\class} & \textbf{ФИО:} & \makebox[2.5in]{\hrulefill}\\
\textbf{\class} & \textbf{ФИО:} &Галиев Искандер Фаргатович
\\

\textbf{\examdate} &&\\
%\textbf{Time Limit: \timelimit} & Teaching Assistant & \makebox[2in]{\hrulefill}
\end{tabular}\\
\end{flushright}
\rule[1ex]{\textwidth}{.1pt}


\begin{questions}
\question
Найдите и упростите P:
\begin{equation*}
\overline{P} = B \cap \overline{C} \cup A \cap B \cup \overline{A} \cap C \cup \overline{A} \cap B
\end{equation*}
Затем найдите элементы множества P, выраженного через множества:
\begin{equation*}
A = \{0, 3, 4, 9\}; 
B = \{1, 3, 4, 7\};
C = \{0, 1, 2, 4, 7, 8, 9\};
I = \{0, 1, 2, 3, 4, 5, 6, 7, 8, 9\}.
\end{equation*}\question
Упростите следующее выражение с учетом того, что $A\subset B \subset C \subset D \subset U; A \neq \O$
\begin{equation*}
A \cap B \cup \overline{A} \cap \overline{C} \cup A \cap C \cup \overline{B} \cap \overline{C}
\end{equation*}

Примечание: U — универсум\question
Дано отношение на множестве $\{1, 2, 3, 4, 5\}$ 
\begin{equation*}
aRb \iff  \text{НОД}(a,b) =1
\end{equation*}
Напишите обоснованный ответ какими свойствами обладает или не обладает отношение и почему:   
\begin{enumerate} [a)]\setcounter{enumi}{0}
\item рефлексивность
\item антирефлексивность
\item симметричность
\item асимметричность
\item антисимметричность
\item транзитивность
\end{enumerate}

Обоснуйте свой ответ по каждому из приведенных ниже вопросов:
\begin{enumerate} [a)]\setcounter{enumi}{0}
    \item Является ли это отношение отношением эквивалентности?
    \item Является ли это отношение функциональным?
    \item Каким из отношений соответствия (одно-многозначным, много-многозначный и т.д.) оно является?
    \item К каким из отношений порядка (полного, частичного и т.д.) можно отнести данное отношение?
\end{enumerate}


\question
Установите, является ли каждое из перечисленных ниже отношений на А ($R \subseteq A \times A$) отношением эквивалентности (обоснование ответа обязательно). Для каждого отношения эквивалентности постройте классы 
эквивалентности и постройте граф отношения:
\begin{enumerate} [a)]\setcounter{enumi}{0}
\item А - множество целых чисел и отношение $R = \{(a,b)|a + b = 5\}$
\item Пусть A – множество имен. $A = \{ $Алексей, Иван, Петр, Александр, Павел, Андрей$ \}$. Тогда отношение $R $ верно на парах имен, начинающихся с одной и той же буквы, и только на них.
\item На множестве $A = \{1; 2; 3; 4; 5\}$ задано отношение $R = \{(1; 2); (1; 3); (1; 5); (2; 3); (2; 4); (2; 5); (3; 4); (3; 5); (4; 5)\}$
\end{enumerate}\question Составьте полную таблицу истинности, определите, какие переменные являются фиктивными и проверьте, является ли формула тавтологией:
$((P \rightarrow Q) \land (R \rightarrow S) \land \neg (Q \lor S)) \rightarrow \neg (P \lor R)$

\end{questions}
\newpage
%%% begin test
\begin{flushright}
\begin{tabular}{p{2.8in} r l}
%\textbf{\class} & \textbf{ФИО:} & \makebox[2.5in]{\hrulefill}\\
\textbf{\class} & \textbf{ФИО:} &Горячева Екатерина Николаевна
\\

\textbf{\examdate} &&\\
%\textbf{Time Limit: \timelimit} & Teaching Assistant & \makebox[2in]{\hrulefill}
\end{tabular}\\
\end{flushright}
\rule[1ex]{\textwidth}{.1pt}


\begin{questions}
\question
Найдите и упростите P:
\begin{equation*}
\overline{P} = \overline{A} \cap B \cup \overline{A} \cap C \cup A \cap \overline{B} \cup \overline{B} \cap C
\end{equation*}
Затем найдите элементы множества P, выраженного через множества:
\begin{equation*}
A = \{0, 3, 4, 9\}; 
B = \{1, 3, 4, 7\};
C = \{0, 1, 2, 4, 7, 8, 9\};
I = \{0, 1, 2, 3, 4, 5, 6, 7, 8, 9\}.
\end{equation*}\question
Упростите следующее выражение с учетом того, что $A\subset B \subset C \subset D \subset U; A \neq \O$
\begin{equation*}
\overline{A} \cap \overline{B} \cup B \cap \overline{C} \cup \overline{C} \cap D
\end{equation*}

Примечание: U — универсум\question
Дано отношение на множестве $\{1, 2, 3, 4, 5\}$ 
\begin{equation*}
aRb \iff b > a
\end{equation*}
Напишите обоснованный ответ какими свойствами обладает или не обладает отношение и почему:   
\begin{enumerate} [a)]\setcounter{enumi}{0}
\item рефлексивность
\item антирефлексивность
\item симметричность
\item асимметричность
\item антисимметричность
\item транзитивность
\end{enumerate}

Обоснуйте свой ответ по каждому из приведенных ниже вопросов:
\begin{enumerate} [a)]\setcounter{enumi}{0}
    \item Является ли это отношение отношением эквивалентности?
    \item Является ли это отношение функциональным?
    \item Каким из отношений соответствия (одно-многозначным, много-многозначный и т.д.) оно является?
    \item К каким из отношений порядка (полного, частичного и т.д.) можно отнести данное отношение?
\end{enumerate}

\question
Установите, является ли каждое из перечисленных ниже отношений на А ($R \subseteq A \times A$) отношением эквивалентности (обоснование ответа обязательно). Для каждого отношения эквивалентности постройте классы 
эквивалентности и постройте граф отношения:
\begin{enumerate} [a)]\setcounter{enumi}{0}
\item $A = \{-10, -9, … , 9, 10\}$ и отношение $R = \{(a,b)|a^{2} = b^{2}\}$
\item $A = \{a, b, c, d, p, t\}$ задано отношение $R = \{(a, a), (b, b), (b, c), (b, d), (c, b), (c, c), (c, d), (d, b), (d, c), (d, d), (p,p), (t,t)\}$
\item Пусть A – множество имен. $A = \{ $Алексей, Иван, Петр, Александр, Павел, Андрей$ \}$. Тогда отношение $R$ верно на парах имен, начинающихся с одной и той же буквы, и только на них.
\end{enumerate}\question Составьте полную таблицу истинности, определите, какие переменные являются фиктивными и проверьте, является ли формула тавтологией:
$(P \rightarrow (Q \rightarrow R)) \rightarrow ((P \rightarrow Q) \rightarrow (P \rightarrow R))$

\end{questions}
\newpage
%%% begin test
\begin{flushright}
\begin{tabular}{p{2.8in} r l}
%\textbf{\class} & \textbf{ФИО:} & \makebox[2.5in]{\hrulefill}\\
\textbf{\class} & \textbf{ФИО:} &Гукоян Эрик Арменович
\\

\textbf{\examdate} &&\\
%\textbf{Time Limit: \timelimit} & Teaching Assistant & \makebox[2in]{\hrulefill}
\end{tabular}\\
\end{flushright}
\rule[1ex]{\textwidth}{.1pt}


\begin{questions}
\question
Найдите и упростите P:
\begin{equation*}
\overline{P} = B \cap \overline{C} \cup A \cap B \cup \overline{A} \cap C \cup \overline{A} \cap B
\end{equation*}
Затем найдите элементы множества P, выраженного через множества:
\begin{equation*}
A = \{0, 3, 4, 9\}; 
B = \{1, 3, 4, 7\};
C = \{0, 1, 2, 4, 7, 8, 9\};
I = \{0, 1, 2, 3, 4, 5, 6, 7, 8, 9\}.
\end{equation*}\question
Упростите следующее выражение с учетом того, что $A\subset B \subset C \subset D \subset U; A \neq \O$
\begin{equation*}
A \cap B \cup \overline{A} \cap \overline{C} \cup A \cap C \cup \overline{B} \cap \overline{C}
\end{equation*}

Примечание: U — универсум\question
Дано отношение на множестве $\{1, 2, 3, 4, 5\}$ 
\begin{equation*}
aRb \iff (a+b) \bmod 2 =0
\end{equation*}
Напишите обоснованный ответ какими свойствами обладает или не обладает отношение и почему:   
\begin{enumerate} [a)]\setcounter{enumi}{0}
\item рефлексивность
\item антирефлексивность
\item симметричность
\item асимметричность
\item антисимметричность
\item транзитивность
\end{enumerate}

Обоснуйте свой ответ по каждому из приведенных ниже вопросов:
\begin{enumerate} [a)]\setcounter{enumi}{0}
    \item Является ли это отношение отношением эквивалентности?
    \item Является ли это отношение функциональным?
    \item Каким из отношений соответствия (одно-многозначным, много-многозначный и т.д.) оно является?
    \item К каким из отношений порядка (полного, частичного и т.д.) можно отнести данное отношение?
\end{enumerate}



\question
Установите, является ли каждое из перечисленных ниже отношений на А ($R \subseteq A \times A$) отношением эквивалентности (обоснование ответа обязательно). Для каждого отношения эквивалентности постройте классы 
эквивалентности и постройте граф отношения:
\begin{enumerate} [a)]\setcounter{enumi}{0}
\item $A = \{a, b, c, d, p, t\}$ задано отношение $R = \{(a, a), (b, b), (b, c), (b, d), (c, b), (c, c), (c, d), (d, b), (d, c), (d, d), (p,p), (t,t)\}$
\item $A = \{-10, -9, … , 9, 10\}$ и отношение $R = \{(a,b)|a^{3} = b^{3}\}$

\item $F(x)=x^{2}+1$, где $x \in A = [-2, 4]$ и отношение $R = \{(a,b)|F(a) = F(b)\}$
\end{enumerate}\question Составьте полную таблицу истинности, определите, какие переменные являются фиктивными и проверьте, является ли формула тавтологией:
$(P \rightarrow (Q \rightarrow R)) \rightarrow ((P \rightarrow Q) \rightarrow (P \rightarrow R))$

\end{questions}
\newpage
%%% begin test
\begin{flushright}
\begin{tabular}{p{2.8in} r l}
%\textbf{\class} & \textbf{ФИО:} & \makebox[2.5in]{\hrulefill}\\
\textbf{\class} & \textbf{ФИО:} &Доронин Дмитрий Сергеевич
\\

\textbf{\examdate} &&\\
%\textbf{Time Limit: \timelimit} & Teaching Assistant & \makebox[2in]{\hrulefill}
\end{tabular}\\
\end{flushright}
\rule[1ex]{\textwidth}{.1pt}


\begin{questions}
\question
Найдите и упростите P:
\begin{equation*}
\overline{P} = \overline{A} \cap B \cup \overline{A} \cap C \cup A \cap \overline{B} \cup \overline{B} \cap C
\end{equation*}
Затем найдите элементы множества P, выраженного через множества:
\begin{equation*}
A = \{0, 3, 4, 9\}; 
B = \{1, 3, 4, 7\};
C = \{0, 1, 2, 4, 7, 8, 9\};
I = \{0, 1, 2, 3, 4, 5, 6, 7, 8, 9\}.
\end{equation*}\question
Упростите следующее выражение с учетом того, что $A\subset B \subset C \subset D \subset U; A \neq \O$
\begin{equation*}
A \cap B  \cap \overline{C} \cup \overline{C} \cap D \cup B \cap C \cap D
\end{equation*}

Примечание: U — универсум\question
Дано отношение на множестве $\{1, 2, 3, 4, 5\}$ 
\begin{equation*}
aRb \iff b > a
\end{equation*}
Напишите обоснованный ответ какими свойствами обладает или не обладает отношение и почему:   
\begin{enumerate} [a)]\setcounter{enumi}{0}
\item рефлексивность
\item антирефлексивность
\item симметричность
\item асимметричность
\item антисимметричность
\item транзитивность
\end{enumerate}

Обоснуйте свой ответ по каждому из приведенных ниже вопросов:
\begin{enumerate} [a)]\setcounter{enumi}{0}
    \item Является ли это отношение отношением эквивалентности?
    \item Является ли это отношение функциональным?
    \item Каким из отношений соответствия (одно-многозначным, много-многозначный и т.д.) оно является?
    \item К каким из отношений порядка (полного, частичного и т.д.) можно отнести данное отношение?
\end{enumerate}

\question
Установите, является ли каждое из перечисленных ниже отношений на А ($R \subseteq A \times A$) отношением эквивалентности (обоснование ответа обязательно). Для каждого отношения эквивалентности постройте классы 
эквивалентности и постройте граф отношения:
\begin{enumerate} [a)]\setcounter{enumi}{0}
\item $A = \{a, b, c, d, p, t\}$ задано отношение $R = \{(a, a), (b, b), (b, c), (b, d), (c, b), (c, c), (c, d), (d, b), (d, c), (d, d), (p,p), (t,t)\}$
\item $A = \{-10, -9, … , 9, 10\}$ и отношение $R = \{(a,b)|a^{3} = b^{3}\}$

\item $F(x)=x^{2}+1$, где $x \in A = [-2, 4]$ и отношение $R = \{(a,b)|F(a) = F(b)\}$
\end{enumerate}\question Составьте полную таблицу истинности, определите, какие переменные являются фиктивными и проверьте, является ли формула тавтологией:
$(P \rightarrow (Q \rightarrow R)) \rightarrow ((P \rightarrow Q) \rightarrow (P \rightarrow R))$

\end{questions}
\newpage
%%% begin test
\begin{flushright}
\begin{tabular}{p{2.8in} r l}
%\textbf{\class} & \textbf{ФИО:} & \makebox[2.5in]{\hrulefill}\\
\textbf{\class} & \textbf{ФИО:} &Дорошенко Семен Михайлович
\\

\textbf{\examdate} &&\\
%\textbf{Time Limit: \timelimit} & Teaching Assistant & \makebox[2in]{\hrulefill}
\end{tabular}\\
\end{flushright}
\rule[1ex]{\textwidth}{.1pt}


\begin{questions}
\question
Найдите и упростите P:
\begin{equation*}
\overline{P} = A \cap C \cup \overline{A} \cap \overline{C} \cup \overline{B} \cap C \cup \overline{A} \cap \overline{B}
\end{equation*}
Затем найдите элементы множества P, выраженного через множества:
\begin{equation*}
A = \{0, 3, 4, 9\}; 
B = \{1, 3, 4, 7\};
C = \{0, 1, 2, 4, 7, 8, 9\};
I = \{0, 1, 2, 3, 4, 5, 6, 7, 8, 9\}.
\end{equation*}\question
Упростите следующее выражение с учетом того, что $A\subset B \subset C \subset D \subset U; A \neq \O$
\begin{equation*}
\overline{A} \cap \overline{B} \cup B \cap \overline{C} \cup \overline{C} \cap D
\end{equation*}

Примечание: U — универсум\question
Дано отношение на множестве $\{1, 2, 3, 4, 5\}$ 
\begin{equation*}
aRb \iff  \text{НОД}(a,b) =1
\end{equation*}
Напишите обоснованный ответ какими свойствами обладает или не обладает отношение и почему:   
\begin{enumerate} [a)]\setcounter{enumi}{0}
\item рефлексивность
\item антирефлексивность
\item симметричность
\item асимметричность
\item антисимметричность
\item транзитивность
\end{enumerate}

Обоснуйте свой ответ по каждому из приведенных ниже вопросов:
\begin{enumerate} [a)]\setcounter{enumi}{0}
    \item Является ли это отношение отношением эквивалентности?
    \item Является ли это отношение функциональным?
    \item Каким из отношений соответствия (одно-многозначным, много-многозначный и т.д.) оно является?
    \item К каким из отношений порядка (полного, частичного и т.д.) можно отнести данное отношение?
\end{enumerate}


\question
Установите, является ли каждое из перечисленных ниже отношений на А ($R \subseteq A \times A$) отношением эквивалентности (обоснование ответа обязательно). Для каждого отношения эквивалентности постройте классы 
эквивалентности и постройте граф отношения:
\begin{enumerate} [a)]\setcounter{enumi}{0}
\item На множестве $A = \{1; 2; 3\}$ задано отношение $R = \{(1; 1); (2; 2); (3; 3); (2; 1); (1; 2); (2; 3); (3; 2); (3; 1); (1; 3)\}$
\item На множестве $A = \{1; 2; 3; 4; 5\}$ задано отношение $R = \{(1; 2); (1; 3); (1; 5); (2; 3); (2; 4); (2; 5); (3; 4); (3; 5); (4; 5)\}$
\item А - множество целых чисел и отношение $R = \{(a,b)|a + b = 0\}$
\end{enumerate}\question Составьте полную таблицу истинности, определите, какие переменные являются фиктивными и проверьте, является ли формула тавтологией:
$(( P \rightarrow Q) \land (Q \rightarrow P)) \rightarrow (P \rightarrow R)$

\end{questions}
\newpage
%%% begin test
\begin{flushright}
\begin{tabular}{p{2.8in} r l}
%\textbf{\class} & \textbf{ФИО:} & \makebox[2.5in]{\hrulefill}\\
\textbf{\class} & \textbf{ФИО:} &Дудко Александр Романович
\\

\textbf{\examdate} &&\\
%\textbf{Time Limit: \timelimit} & Teaching Assistant & \makebox[2in]{\hrulefill}
\end{tabular}\\
\end{flushright}
\rule[1ex]{\textwidth}{.1pt}


\begin{questions}
\question
Найдите и упростите P:
\begin{equation*}
\overline{P} = A \cap \overline{C} \cup A \cap \overline{B} \cup B \cap \overline{C} \cup A \cap C
\end{equation*}
Затем найдите элементы множества P, выраженного через множества:
\begin{equation*}
A = \{0, 3, 4, 9\}; 
B = \{1, 3, 4, 7\};
C = \{0, 1, 2, 4, 7, 8, 9\};
I = \{0, 1, 2, 3, 4, 5, 6, 7, 8, 9\}.
\end{equation*}\question
Упростите следующее выражение с учетом того, что $A\subset B \subset C \subset D \subset U; A \neq \O$
\begin{equation*}
A \cap C  \cap D \cup B \cap \overline{C} \cap D \cup B \cap C \cap D
\end{equation*}

Примечание: U — универсум\question
Дано отношение на множестве $\{1, 2, 3, 4, 5\}$ 
\begin{equation*}
aRb \iff (a+b) \bmod 2 =0
\end{equation*}
Напишите обоснованный ответ какими свойствами обладает или не обладает отношение и почему:   
\begin{enumerate} [a)]\setcounter{enumi}{0}
\item рефлексивность
\item антирефлексивность
\item симметричность
\item асимметричность
\item антисимметричность
\item транзитивность
\end{enumerate}

Обоснуйте свой ответ по каждому из приведенных ниже вопросов:
\begin{enumerate} [a)]\setcounter{enumi}{0}
    \item Является ли это отношение отношением эквивалентности?
    \item Является ли это отношение функциональным?
    \item Каким из отношений соответствия (одно-многозначным, много-многозначный и т.д.) оно является?
    \item К каким из отношений порядка (полного, частичного и т.д.) можно отнести данное отношение?
\end{enumerate}



\question
Установите, является ли каждое из перечисленных ниже отношений на А ($R \subseteq A \times A$) отношением эквивалентности (обоснование ответа обязательно). Для каждого отношения эквивалентности постройте классы 
эквивалентности и постройте граф отношения:
\begin{enumerate} [a)]\setcounter{enumi}{0}
\item Пусть A – множество имен. $A = \{ $Алексей, Иван, Петр, Александр, Павел, Андрей$ \}$. Тогда отношение $R$ верно на парах имен, начинающихся с одной и той же буквы, и только на них.
\item $A = \{-10, -9, … , 9, 10\}$ и отношение $ R = \{(a,b)|a^{2} = b^{2}\}$
\item На множестве $A = \{1; 2; 3\}$ задано отношение $R = \{(1; 1); (2; 2); (3; 3); (3; 2); (1; 2); (2; 1)\}$
\end{enumerate}\question Составьте полную таблицу истинности, определите, какие переменные являются фиктивными и проверьте, является ли формула тавтологией:
$(P \rightarrow (Q \rightarrow R)) \rightarrow ((P \rightarrow Q) \rightarrow (P \rightarrow R))$

\end{questions}
\newpage
%%% begin test
\begin{flushright}
\begin{tabular}{p{2.8in} r l}
%\textbf{\class} & \textbf{ФИО:} & \makebox[2.5in]{\hrulefill}\\
\textbf{\class} & \textbf{ФИО:} &Жевлаков Андрей Олегович
\\

\textbf{\examdate} &&\\
%\textbf{Time Limit: \timelimit} & Teaching Assistant & \makebox[2in]{\hrulefill}
\end{tabular}\\
\end{flushright}
\rule[1ex]{\textwidth}{.1pt}


\begin{questions}
\question
Найдите и упростите P:
\begin{equation*}
\overline{P} = A \cap \overline{C} \cup A \cap \overline{B} \cup B \cap \overline{C} \cup A \cap C
\end{equation*}
Затем найдите элементы множества P, выраженного через множества:
\begin{equation*}
A = \{0, 3, 4, 9\}; 
B = \{1, 3, 4, 7\};
C = \{0, 1, 2, 4, 7, 8, 9\};
I = \{0, 1, 2, 3, 4, 5, 6, 7, 8, 9\}.
\end{equation*}\question
Упростите следующее выражение с учетом того, что $A\subset B \subset C \subset D \subset U; A \neq \O$
\begin{equation*}
A \cap B \cup \overline{A} \cap \overline{C} \cup A \cap C \cup \overline{B} \cap \overline{C}
\end{equation*}

Примечание: U — универсум\question
Для следующего отношения на множестве $\{1, 2, 3, 4, 5\}$ 
\begin{equation*}
aRb \iff 0 < a-b<2
\end{equation*}
Напишите обоснованный ответ какими свойствами обладает или не обладает отношение и почему:   
\begin{enumerate} [a)]\setcounter{enumi}{0}
\item рефлексивность
\item антирефлексивность
\item симметричность
\item асимметричность
\item антисимметричность
\item транзитивность
\end{enumerate}

Обоснуйте свой ответ по каждому из приведенных ниже вопросов:
\begin{enumerate} [a)]\setcounter{enumi}{0}
    \item Является ли это отношение отношением эквивалентности?
    \item Является ли это отношение функциональным?
    \item Каким из отношений соответствия (одно-многозначным, много-многозначный и т.д.) оно является?
    \item К каким из отношений порядка (полного, частичного и т.д.) можно отнести данное отношение?
\end{enumerate}
\question
Установите, является ли каждое из перечисленных ниже отношений на А ($R \subseteq A \times A$) отношением эквивалентности (обоснование ответа обязательно). Для каждого отношения эквивалентности постройте классы 
эквивалентности и постройте граф отношения:
\begin{enumerate} [a)]\setcounter{enumi}{0}
\item На множестве $A = \{1; 2; 3\}$ задано отношение $R = \{(1; 1); (2; 2); (3; 3); (2; 1); (1; 2); (2; 3); (3; 2); (3; 1); (1; 3)\}$
\item На множестве $A = \{1; 2; 3; 4; 5\}$ задано отношение $R = \{(1; 2); (1; 3); (1; 5); (2; 3); (2; 4); (2; 5); (3; 4); (3; 5); (4; 5)\}$
\item А - множество целых чисел и отношение $R = \{(a,b)|a + b = 0\}$
\end{enumerate}\question Составьте полную таблицу истинности, определите, какие переменные являются фиктивными и проверьте, является ли формула тавтологией:

$(P \rightarrow (Q \land R)) \leftrightarrow ((P \rightarrow Q) \land (P \rightarrow R))$

\end{questions}
\newpage
%%% begin test
\begin{flushright}
\begin{tabular}{p{2.8in} r l}
%\textbf{\class} & \textbf{ФИО:} & \makebox[2.5in]{\hrulefill}\\
\textbf{\class} & \textbf{ФИО:} &Завальнюк Павел Борисович
\\

\textbf{\examdate} &&\\
%\textbf{Time Limit: \timelimit} & Teaching Assistant & \makebox[2in]{\hrulefill}
\end{tabular}\\
\end{flushright}
\rule[1ex]{\textwidth}{.1pt}


\begin{questions}
\question
Найдите и упростите P:
\begin{equation*}
\overline{P} = A \cap \overline{B} \cup A \cap C \cup B \cap C \cup \overline{A} \cap C
\end{equation*}
Затем найдите элементы множества P, выраженного через множества:
\begin{equation*}
A = \{0, 3, 4, 9\}; 
B = \{1, 3, 4, 7\};
C = \{0, 1, 2, 4, 7, 8, 9\};
I = \{0, 1, 2, 3, 4, 5, 6, 7, 8, 9\}.
\end{equation*}\question
Упростите следующее выражение с учетом того, что $A\subset B \subset C \subset D \subset U; A \neq \O$
\begin{equation*}
A \cap  \overline{C} \cup B \cap \overline{D} \cup  \overline{A} \cap C \cap  \overline{D}
\end{equation*}

Примечание: U — универсум\question
Дано отношение на множестве $\{1, 2, 3, 4, 5\}$ 
\begin{equation*}
aRb \iff a \leq b
\end{equation*}
Напишите обоснованный ответ какими свойствами обладает или не обладает отношение и почему:   
\begin{enumerate} [a)]\setcounter{enumi}{0}
\item рефлексивность
\item антирефлексивность
\item симметричность
\item асимметричность
\item антисимметричность
\item транзитивность
\end{enumerate}

Обоснуйте свой ответ по каждому из приведенных ниже вопросов:
\begin{enumerate} [a)]\setcounter{enumi}{0}
    \item Является ли это отношение отношением эквивалентности?
    \item Является ли это отношение функциональным?
    \item Каким из отношений соответствия (одно-многозначным, много-многозначный и т.д.) оно является?
    \item К каким из отношений порядка (полного, частичного и т.д.) можно отнести данное отношение?
\end{enumerate}


\question
Установите, является ли каждое из перечисленных ниже отношений на А ($R \subseteq A \times A$) отношением эквивалентности (обоснование ответа обязательно). Для каждого отношения эквивалентности постройте классы 
эквивалентности и постройте граф отношения:
\begin{enumerate} [a)]\setcounter{enumi}{0}
\item Пусть A – множество имен. $A = \{ $Алексей, Иван, Петр, Александр, Павел, Андрей$ \}$. Тогда отношение $R$ верно на парах имен, начинающихся с одной и той же буквы, и только на них.
\item $A = \{-10, -9, … , 9, 10\}$ и отношение $ R = \{(a,b)|a^{2} = b^{2}\}$
\item На множестве $A = \{1; 2; 3\}$ задано отношение $R = \{(1; 1); (2; 2); (3; 3); (3; 2); (1; 2); (2; 1)\}$
\end{enumerate}\question Составьте полную таблицу истинности, определите, какие переменные являются фиктивными и проверьте, является ли формула тавтологией:
$(( P \rightarrow Q) \land (Q \rightarrow P)) \rightarrow (P \rightarrow R)$

\end{questions}
\newpage
%%% begin test
\begin{flushright}
\begin{tabular}{p{2.8in} r l}
%\textbf{\class} & \textbf{ФИО:} & \makebox[2.5in]{\hrulefill}\\
\textbf{\class} & \textbf{ФИО:} &Кондратьев Николай Евгеньевич
\\

\textbf{\examdate} &&\\
%\textbf{Time Limit: \timelimit} & Teaching Assistant & \makebox[2in]{\hrulefill}
\end{tabular}\\
\end{flushright}
\rule[1ex]{\textwidth}{.1pt}


\begin{questions}
\question
Найдите и упростите P:
\begin{equation*}
\overline{P} = A \cap \overline{C} \cup A \cap \overline{B} \cup B \cap \overline{C} \cup A \cap C
\end{equation*}
Затем найдите элементы множества P, выраженного через множества:
\begin{equation*}
A = \{0, 3, 4, 9\}; 
B = \{1, 3, 4, 7\};
C = \{0, 1, 2, 4, 7, 8, 9\};
I = \{0, 1, 2, 3, 4, 5, 6, 7, 8, 9\}.
\end{equation*}\question
Упростите следующее выражение с учетом того, что $A\subset B \subset C \subset D \subset U; A \neq \O$
\begin{equation*}
A \cap B \cup \overline{A} \cap \overline{C} \cup A \cap C \cup \overline{B} \cap \overline{C}
\end{equation*}

Примечание: U — универсум\question
Дано отношение на множестве $\{1, 2, 3, 4, 5\}$ 
\begin{equation*}
aRb \iff a \geq b^2
\end{equation*}
Напишите обоснованный ответ какими свойствами обладает или не обладает отношение и почему:   
\begin{enumerate} [a)]\setcounter{enumi}{0}
\item рефлексивность
\item антирефлексивность
\item симметричность
\item асимметричность
\item антисимметричность
\item транзитивность
\end{enumerate}

Обоснуйте свой ответ по каждому из приведенных ниже вопросов:
\begin{enumerate} [a)]\setcounter{enumi}{0}
    \item Является ли это отношение отношением эквивалентности?
    \item Является ли это отношение функциональным?
    \item Каким из отношений соответствия (одно-многозначным, много-многозначный и т.д.) оно является?
    \item К каким из отношений порядка (полного, частичного и т.д.) можно отнести данное отношение?
\end{enumerate}


\question
Установите, является ли каждое из перечисленных ниже отношений на А ($R \subseteq A \times A$) отношением эквивалентности (обоснование ответа обязательно). Для каждого отношения эквивалентности постройте классы 
эквивалентности и постройте граф отношения:
\begin{enumerate} [a)]\setcounter{enumi}{0}
\item Пусть A – множество имен. $A = \{ $Алексей, Иван, Петр, Александр, Павел, Андрей$ \}$. Тогда отношение $R$ верно на парах имен, начинающихся с одной и той же буквы, и только на них.
\item $A = \{-10, -9, … , 9, 10\}$ и отношение $ R = \{(a,b)|a^{2} = b^{2}\}$
\item На множестве $A = \{1; 2; 3\}$ задано отношение $R = \{(1; 1); (2; 2); (3; 3); (3; 2); (1; 2); (2; 1)\}$
\end{enumerate}\question Составьте полную таблицу истинности, определите, какие переменные являются фиктивными и проверьте, является ли формула тавтологией:
$((P \rightarrow Q) \land (R \rightarrow S) \land \neg (Q \lor S)) \rightarrow \neg (P \lor R)$

\end{questions}
\newpage
%%% begin test
\begin{flushright}
\begin{tabular}{p{2.8in} r l}
%\textbf{\class} & \textbf{ФИО:} & \makebox[2.5in]{\hrulefill}\\
\textbf{\class} & \textbf{ФИО:} &Копытин Алексей
\\

\textbf{\examdate} &&\\
%\textbf{Time Limit: \timelimit} & Teaching Assistant & \makebox[2in]{\hrulefill}
\end{tabular}\\
\end{flushright}
\rule[1ex]{\textwidth}{.1pt}


\begin{questions}
\question
Найдите и упростите P:
\begin{equation*}
\overline{P} = A \cap \overline{C} \cup A \cap \overline{B} \cup B \cap \overline{C} \cup A \cap C
\end{equation*}
Затем найдите элементы множества P, выраженного через множества:
\begin{equation*}
A = \{0, 3, 4, 9\}; 
B = \{1, 3, 4, 7\};
C = \{0, 1, 2, 4, 7, 8, 9\};
I = \{0, 1, 2, 3, 4, 5, 6, 7, 8, 9\}.
\end{equation*}\question
Упростите следующее выражение с учетом того, что $A\subset B \subset C \subset D \subset U; A \neq \O$
\begin{equation*}
\overline{B} \cap \overline{C} \cap D \cup \overline{A} \cap \overline{C} \cap D \cup \overline{A} \cap B
\end{equation*}

Примечание: U — универсум\question
Дано отношение на множестве $\{1, 2, 3, 4, 5\}$ 
\begin{equation*}
aRb \iff a \leq b
\end{equation*}
Напишите обоснованный ответ какими свойствами обладает или не обладает отношение и почему:   
\begin{enumerate} [a)]\setcounter{enumi}{0}
\item рефлексивность
\item антирефлексивность
\item симметричность
\item асимметричность
\item антисимметричность
\item транзитивность
\end{enumerate}

Обоснуйте свой ответ по каждому из приведенных ниже вопросов:
\begin{enumerate} [a)]\setcounter{enumi}{0}
    \item Является ли это отношение отношением эквивалентности?
    \item Является ли это отношение функциональным?
    \item Каким из отношений соответствия (одно-многозначным, много-многозначный и т.д.) оно является?
    \item К каким из отношений порядка (полного, частичного и т.д.) можно отнести данное отношение?
\end{enumerate}


\question
Установите, является ли каждое из перечисленных ниже отношений на А ($R \subseteq A \times A$) отношением эквивалентности (обоснование ответа обязательно). Для каждого отношения эквивалентности постройте классы эквивалентности и постройте граф отношения:
\begin{enumerate} [a)]\setcounter{enumi}{0}
\item $F(x)=x^{2}+1$, где $x \in A = [-2, 4]$ и отношение $R = \{(a,b)|F(a) = F(b)\}$
\item А - множество целых чисел и отношение $R = \{(a,b)|a + b = 5\}$
\item На множестве $A = \{1; 2; 3\}$ задано отношение $R = \{(1; 1); (2; 2); (3; 3); (3; 2); (1; 2); (2; 1)\}$

\end{enumerate}\question Составьте полную таблицу истинности, определите, какие переменные являются фиктивными и проверьте, является ли формула тавтологией:
$(P \rightarrow (Q \rightarrow R)) \rightarrow ((P \rightarrow Q) \rightarrow (P \rightarrow R))$

\end{questions}
\newpage
%%% begin test
\begin{flushright}
\begin{tabular}{p{2.8in} r l}
%\textbf{\class} & \textbf{ФИО:} & \makebox[2.5in]{\hrulefill}\\
\textbf{\class} & \textbf{ФИО:} &Магарьян Юрий Александрович
\\

\textbf{\examdate} &&\\
%\textbf{Time Limit: \timelimit} & Teaching Assistant & \makebox[2in]{\hrulefill}
\end{tabular}\\
\end{flushright}
\rule[1ex]{\textwidth}{.1pt}


\begin{questions}
\question
Найдите и упростите P:
\begin{equation*}
\overline{P} = A \cap C \cup \overline{A} \cap \overline{C} \cup \overline{B} \cap C \cup \overline{A} \cap \overline{B}
\end{equation*}
Затем найдите элементы множества P, выраженного через множества:
\begin{equation*}
A = \{0, 3, 4, 9\}; 
B = \{1, 3, 4, 7\};
C = \{0, 1, 2, 4, 7, 8, 9\};
I = \{0, 1, 2, 3, 4, 5, 6, 7, 8, 9\}.
\end{equation*}\question
Упростите следующее выражение с учетом того, что $A\subset B \subset C \subset D \subset U; A \neq \O$
\begin{equation*}
A \cap  \overline{C} \cup B \cap \overline{D} \cup  \overline{A} \cap C \cap  \overline{D}
\end{equation*}

Примечание: U — универсум\question
Дано отношение на множестве $\{1, 2, 3, 4, 5\}$ 
\begin{equation*}
aRb \iff  \text{НОД}(a,b) =1
\end{equation*}
Напишите обоснованный ответ какими свойствами обладает или не обладает отношение и почему:   
\begin{enumerate} [a)]\setcounter{enumi}{0}
\item рефлексивность
\item антирефлексивность
\item симметричность
\item асимметричность
\item антисимметричность
\item транзитивность
\end{enumerate}

Обоснуйте свой ответ по каждому из приведенных ниже вопросов:
\begin{enumerate} [a)]\setcounter{enumi}{0}
    \item Является ли это отношение отношением эквивалентности?
    \item Является ли это отношение функциональным?
    \item Каким из отношений соответствия (одно-многозначным, много-многозначный и т.д.) оно является?
    \item К каким из отношений порядка (полного, частичного и т.д.) можно отнести данное отношение?
\end{enumerate}


\question
Установите, является ли каждое из перечисленных ниже отношений на А ($R \subseteq A \times A$) отношением эквивалентности (обоснование ответа обязательно). Для каждого отношения эквивалентности постройте классы 
эквивалентности и постройте граф отношения:
\begin{enumerate} [a)]\setcounter{enumi}{0}
\item Пусть A – множество имен. $A = \{ $Алексей, Иван, Петр, Александр, Павел, Андрей$ \}$. Тогда отношение $R$ верно на парах имен, начинающихся с одной и той же буквы, и только на них.
\item $A = \{-10, -9, … , 9, 10\}$ и отношение $ R = \{(a,b)|a^{2} = b^{2}\}$
\item На множестве $A = \{1; 2; 3\}$ задано отношение $R = \{(1; 1); (2; 2); (3; 3); (3; 2); (1; 2); (2; 1)\}$
\end{enumerate}\question Составьте полную таблицу истинности, определите, какие переменные являются фиктивными и проверьте, является ли формула тавтологией:
$ P \rightarrow (Q \rightarrow ((P \lor Q) \rightarrow (P \land Q)))$

\end{questions}
\newpage
%%% begin test
\begin{flushright}
\begin{tabular}{p{2.8in} r l}
%\textbf{\class} & \textbf{ФИО:} & \makebox[2.5in]{\hrulefill}\\
\textbf{\class} & \textbf{ФИО:} &Малиева Дарья Игоревна
\\

\textbf{\examdate} &&\\
%\textbf{Time Limit: \timelimit} & Teaching Assistant & \makebox[2in]{\hrulefill}
\end{tabular}\\
\end{flushright}
\rule[1ex]{\textwidth}{.1pt}


\begin{questions}
\question
Найдите и упростите P:
\begin{equation*}
\overline{P} = A \cap \overline{C} \cup A \cap \overline{B} \cup B \cap \overline{C} \cup A \cap C
\end{equation*}
Затем найдите элементы множества P, выраженного через множества:
\begin{equation*}
A = \{0, 3, 4, 9\}; 
B = \{1, 3, 4, 7\};
C = \{0, 1, 2, 4, 7, 8, 9\};
I = \{0, 1, 2, 3, 4, 5, 6, 7, 8, 9\}.
\end{equation*}\question
Упростите следующее выражение с учетом того, что $A\subset B \subset C \subset D \subset U; A \neq \O$
\begin{equation*}
\overline{B} \cap \overline{C} \cap D \cup \overline{A} \cap \overline{C} \cap D \cup \overline{A} \cap B
\end{equation*}

Примечание: U — универсум\question
Для следующего отношения на множестве $\{1, 2, 3, 4, 5\}$ 
\begin{equation*}
aRb \iff 0 < a-b<2
\end{equation*}
Напишите обоснованный ответ какими свойствами обладает или не обладает отношение и почему:   
\begin{enumerate} [a)]\setcounter{enumi}{0}
\item рефлексивность
\item антирефлексивность
\item симметричность
\item асимметричность
\item антисимметричность
\item транзитивность
\end{enumerate}

Обоснуйте свой ответ по каждому из приведенных ниже вопросов:
\begin{enumerate} [a)]\setcounter{enumi}{0}
    \item Является ли это отношение отношением эквивалентности?
    \item Является ли это отношение функциональным?
    \item Каким из отношений соответствия (одно-многозначным, много-многозначный и т.д.) оно является?
    \item К каким из отношений порядка (полного, частичного и т.д.) можно отнести данное отношение?
\end{enumerate}
\question
Установите, является ли каждое из перечисленных ниже отношений на А ($R \subseteq A \times A$) отношением эквивалентности (обоснование ответа обязательно). Для каждого отношения эквивалентности постройте классы 
эквивалентности и постройте граф отношения:
\begin{enumerate} [a)]\setcounter{enumi}{0}
\item $A = \{a, b, c, d, p, t\}$ задано отношение $R = \{(a, a), (b, b), (b, c), (b, d), (c, b), (c, c), (c, d), (d, b), (d, c), (d, d), (p,p), (t,t)\}$
\item $A = \{-10, -9, … , 9, 10\}$ и отношение $R = \{(a,b)|a^{3} = b^{3}\}$

\item $F(x)=x^{2}+1$, где $x \in A = [-2, 4]$ и отношение $R = \{(a,b)|F(a) = F(b)\}$
\end{enumerate}\question Составьте полную таблицу истинности, определите, какие переменные являются фиктивными и проверьте, является ли формула тавтологией:
$(( P \land \neg Q) \rightarrow (R \land \neg R)) \rightarrow (P \rightarrow Q)$

\end{questions}
\newpage
%%% begin test
\begin{flushright}
\begin{tabular}{p{2.8in} r l}
%\textbf{\class} & \textbf{ФИО:} & \makebox[2.5in]{\hrulefill}\\
\textbf{\class} & \textbf{ФИО:} &Марсавин Егор Олегович
\\

\textbf{\examdate} &&\\
%\textbf{Time Limit: \timelimit} & Teaching Assistant & \makebox[2in]{\hrulefill}
\end{tabular}\\
\end{flushright}
\rule[1ex]{\textwidth}{.1pt}


\begin{questions}
\question
Найдите и упростите P:
\begin{equation*}
\overline{P} = A \cap \overline{C} \cup A \cap \overline{B} \cup B \cap \overline{C} \cup A \cap C
\end{equation*}
Затем найдите элементы множества P, выраженного через множества:
\begin{equation*}
A = \{0, 3, 4, 9\}; 
B = \{1, 3, 4, 7\};
C = \{0, 1, 2, 4, 7, 8, 9\};
I = \{0, 1, 2, 3, 4, 5, 6, 7, 8, 9\}.
\end{equation*}\question
Упростите следующее выражение с учетом того, что $A\subset B \subset C \subset D \subset U; A \neq \O$
\begin{equation*}
\overline{A} \cap \overline{C} \cap D \cup \overline{B} \cap \overline{C} \cap D \cup A \cap B
\end{equation*}

Примечание: U — универсум\question
Дано отношение на множестве $\{1, 2, 3, 4, 5\}$ 
\begin{equation*}
aRb \iff |a-b| = 1
\end{equation*}
Напишите обоснованный ответ какими свойствами обладает или не обладает отношение и почему:   
\begin{enumerate} [a)]\setcounter{enumi}{0}
\item рефлексивность
\item антирефлексивность
\item симметричность
\item асимметричность
\item антисимметричность
\item транзитивность
\end{enumerate}

Обоснуйте свой ответ по каждому из приведенных ниже вопросов:
\begin{enumerate} [a)]\setcounter{enumi}{0}
    \item Является ли это отношение отношением эквивалентности?
    \item Является ли это отношение функциональным?
    \item Каким из отношений соответствия (одно-многозначным, много-многозначный и т.д.) оно является?
    \item К каким из отношений порядка (полного, частичного и т.д.) можно отнести данное отношение?
\end{enumerate}

\question
Установите, является ли каждое из перечисленных ниже отношений на А ($R \subseteq A \times A$) отношением эквивалентности (обоснование ответа обязательно). Для каждого отношения эквивалентности 
постройте классы эквивалентности и постройте граф отношения:
\begin{enumerate}[a)]\setcounter{enumi}{0}
\item А - множество целых чисел и отношение $R = \{(a,b)|a + b = 0\}$
\item $A = \{-10, -9, …, 9, 10\}$ и отношение $R = \{(a,b)|a^{3} = b^{3}\}$
\item На множестве $A = \{1; 2; 3\}$ задано отношение $R = \{(1; 1); (2; 2); (3; 3); (2; 1); (1; 2); (2; 3); (3; 2); (3; 1); (1; 3)\}$

\end{enumerate}\question Составьте полную таблицу истинности, определите, какие переменные являются фиктивными и проверьте, является ли формула тавтологией:
$ P \rightarrow (Q \rightarrow ((P \lor Q) \rightarrow (P \land Q)))$

\end{questions}
\newpage
%%% begin test
\begin{flushright}
\begin{tabular}{p{2.8in} r l}
%\textbf{\class} & \textbf{ФИО:} & \makebox[2.5in]{\hrulefill}\\
\textbf{\class} & \textbf{ФИО:} &Насимов Манучехрхон Мансурхонович
\\

\textbf{\examdate} &&\\
%\textbf{Time Limit: \timelimit} & Teaching Assistant & \makebox[2in]{\hrulefill}
\end{tabular}\\
\end{flushright}
\rule[1ex]{\textwidth}{.1pt}


\begin{questions}
\question
Найдите и упростите P:
\begin{equation*}
\overline{P} = A \cap B \cup \overline{A} \cap \overline{B} \cup A \cap C \cup \overline{B} \cap C
\end{equation*}
Затем найдите элементы множества P, выраженного через множества:
\begin{equation*}
A = \{0, 3, 4, 9\}; 
B = \{1, 3, 4, 7\};
C = \{0, 1, 2, 4, 7, 8, 9\};
I = \{0, 1, 2, 3, 4, 5, 6, 7, 8, 9\}.
\end{equation*}\question
Упростите следующее выражение с учетом того, что $A\subset B \subset C \subset D \subset U; A \neq \O$
\begin{equation*}
\overline{A} \cap \overline{C} \cap D \cup \overline{B} \cap \overline{C} \cap D \cup A \cap B
\end{equation*}

Примечание: U — универсум\question
Дано отношение на множестве $\{1, 2, 3, 4, 5\}$ 
\begin{equation*}
aRb \iff a \geq b^2
\end{equation*}
Напишите обоснованный ответ какими свойствами обладает или не обладает отношение и почему:   
\begin{enumerate} [a)]\setcounter{enumi}{0}
\item рефлексивность
\item антирефлексивность
\item симметричность
\item асимметричность
\item антисимметричность
\item транзитивность
\end{enumerate}

Обоснуйте свой ответ по каждому из приведенных ниже вопросов:
\begin{enumerate} [a)]\setcounter{enumi}{0}
    \item Является ли это отношение отношением эквивалентности?
    \item Является ли это отношение функциональным?
    \item Каким из отношений соответствия (одно-многозначным, много-многозначный и т.д.) оно является?
    \item К каким из отношений порядка (полного, частичного и т.д.) можно отнести данное отношение?
\end{enumerate}


\question
Установите, является ли каждое из перечисленных ниже отношений на А ($R \subseteq A \times A$) отношением эквивалентности (обоснование ответа обязательно). Для каждого отношения эквивалентности постройте классы 
эквивалентности и постройте граф отношения:
\begin{enumerate} [a)]\setcounter{enumi}{0}
\item $A = \{-10, -9, … , 9, 10\}$ и отношение $R = \{(a,b)|a^{2} = b^{2}\}$
\item $A = \{a, b, c, d, p, t\}$ задано отношение $R = \{(a, a), (b, b), (b, c), (b, d), (c, b), (c, c), (c, d), (d, b), (d, c), (d, d), (p,p), (t,t)\}$
\item Пусть A – множество имен. $A = \{ $Алексей, Иван, Петр, Александр, Павел, Андрей$ \}$. Тогда отношение $R$ верно на парах имен, начинающихся с одной и той же буквы, и только на них.
\end{enumerate}\question Составьте полную таблицу истинности, определите, какие переменные являются фиктивными и проверьте, является ли формула тавтологией:
$((P \rightarrow Q) \land (R \rightarrow S) \land \neg (Q \lor S)) \rightarrow \neg (P \lor R)$

\end{questions}
\newpage
%%% begin test
\begin{flushright}
\begin{tabular}{p{2.8in} r l}
%\textbf{\class} & \textbf{ФИО:} & \makebox[2.5in]{\hrulefill}\\
\textbf{\class} & \textbf{ФИО:} &Никифоров Александр Алексеевич
\\

\textbf{\examdate} &&\\
%\textbf{Time Limit: \timelimit} & Teaching Assistant & \makebox[2in]{\hrulefill}
\end{tabular}\\
\end{flushright}
\rule[1ex]{\textwidth}{.1pt}


\begin{questions}
\question
Найдите и упростите P:
\begin{equation*}
\overline{P} = A \cap \overline{B} \cup \overline{B} \cap C \cup \overline{A} \cap \overline{B} \cup \overline{A} \cap C
\end{equation*}
Затем найдите элементы множества P, выраженного через множества:
\begin{equation*}
A = \{0, 3, 4, 9\}; 
B = \{1, 3, 4, 7\};
C = \{0, 1, 2, 4, 7, 8, 9\};
I = \{0, 1, 2, 3, 4, 5, 6, 7, 8, 9\}.
\end{equation*}\question
Упростите следующее выражение с учетом того, что $A\subset B \subset C \subset D \subset U; A \neq \O$
\begin{equation*}
\overline{A} \cap \overline{C} \cap D \cup \overline{B} \cap \overline{C} \cap D \cup A \cap B
\end{equation*}

Примечание: U — универсум\question
Дано отношение на множестве $\{1, 2, 3, 4, 5\}$ 
\begin{equation*}
aRb \iff |a-b| = 1
\end{equation*}
Напишите обоснованный ответ какими свойствами обладает или не обладает отношение и почему:   
\begin{enumerate} [a)]\setcounter{enumi}{0}
\item рефлексивность
\item антирефлексивность
\item симметричность
\item асимметричность
\item антисимметричность
\item транзитивность
\end{enumerate}

Обоснуйте свой ответ по каждому из приведенных ниже вопросов:
\begin{enumerate} [a)]\setcounter{enumi}{0}
    \item Является ли это отношение отношением эквивалентности?
    \item Является ли это отношение функциональным?
    \item Каким из отношений соответствия (одно-многозначным, много-многозначный и т.д.) оно является?
    \item К каким из отношений порядка (полного, частичного и т.д.) можно отнести данное отношение?
\end{enumerate}

\question
Установите, является ли каждое из перечисленных ниже отношений на А ($R \subseteq A \times A$) отношением эквивалентности (обоснование ответа обязательно). Для каждого отношения эквивалентности постройте классы эквивалентности и постройте граф отношения:
\begin{enumerate} [a)]\setcounter{enumi}{0}
\item $F(x)=x^{2}+1$, где $x \in A = [-2, 4]$ и отношение $R = \{(a,b)|F(a) = F(b)\}$
\item А - множество целых чисел и отношение $R = \{(a,b)|a + b = 5\}$
\item На множестве $A = \{1; 2; 3\}$ задано отношение $R = \{(1; 1); (2; 2); (3; 3); (3; 2); (1; 2); (2; 1)\}$

\end{enumerate}\question Составьте полную таблицу истинности, определите, какие переменные являются фиктивными и проверьте, является ли формула тавтологией:
$(( P \land \neg Q) \rightarrow (R \land \neg R)) \rightarrow (P \rightarrow Q)$

\end{questions}
\newpage
%%% begin test
\begin{flushright}
\begin{tabular}{p{2.8in} r l}
%\textbf{\class} & \textbf{ФИО:} & \makebox[2.5in]{\hrulefill}\\
\textbf{\class} & \textbf{ФИО:} &Нуреев Дамир
\\

\textbf{\examdate} &&\\
%\textbf{Time Limit: \timelimit} & Teaching Assistant & \makebox[2in]{\hrulefill}
\end{tabular}\\
\end{flushright}
\rule[1ex]{\textwidth}{.1pt}


\begin{questions}
\question
Найдите и упростите P:
\begin{equation*}
\overline{P} = A \cap B \cup \overline{A} \cap \overline{B} \cup A \cap C \cup \overline{B} \cap C
\end{equation*}
Затем найдите элементы множества P, выраженного через множества:
\begin{equation*}
A = \{0, 3, 4, 9\}; 
B = \{1, 3, 4, 7\};
C = \{0, 1, 2, 4, 7, 8, 9\};
I = \{0, 1, 2, 3, 4, 5, 6, 7, 8, 9\}.
\end{equation*}\question
Упростите следующее выражение с учетом того, что $A\subset B \subset C \subset D \subset U; A \neq \O$
\begin{equation*}
A \cap  \overline{C} \cup B \cap \overline{D} \cup  \overline{A} \cap C \cap  \overline{D}
\end{equation*}

Примечание: U — универсум\question
Дано отношение на множестве $\{1, 2, 3, 4, 5\}$ 
\begin{equation*}
aRb \iff (a+b) \bmod 2 =0
\end{equation*}
Напишите обоснованный ответ какими свойствами обладает или не обладает отношение и почему:   
\begin{enumerate} [a)]\setcounter{enumi}{0}
\item рефлексивность
\item антирефлексивность
\item симметричность
\item асимметричность
\item антисимметричность
\item транзитивность
\end{enumerate}

Обоснуйте свой ответ по каждому из приведенных ниже вопросов:
\begin{enumerate} [a)]\setcounter{enumi}{0}
    \item Является ли это отношение отношением эквивалентности?
    \item Является ли это отношение функциональным?
    \item Каким из отношений соответствия (одно-многозначным, много-многозначный и т.д.) оно является?
    \item К каким из отношений порядка (полного, частичного и т.д.) можно отнести данное отношение?
\end{enumerate}



\question
Установите, является ли каждое из перечисленных ниже отношений на А ($R \subseteq A \times A$) отношением эквивалентности (обоснование ответа обязательно). Для каждого отношения эквивалентности постройте классы 
эквивалентности и постройте граф отношения:
\begin{enumerate} [a)]\setcounter{enumi}{0}
\item $A = \{-10, -9, … , 9, 10\}$ и отношение $R = \{(a,b)|a^{2} = b^{2}\}$
\item $A = \{a, b, c, d, p, t\}$ задано отношение $R = \{(a, a), (b, b), (b, c), (b, d), (c, b), (c, c), (c, d), (d, b), (d, c), (d, d), (p,p), (t,t)\}$
\item Пусть A – множество имен. $A = \{ $Алексей, Иван, Петр, Александр, Павел, Андрей$ \}$. Тогда отношение $R$ верно на парах имен, начинающихся с одной и той же буквы, и только на них.
\end{enumerate}\question Составьте полную таблицу истинности, определите, какие переменные являются фиктивными и проверьте, является ли формула тавтологией:
$((P \rightarrow Q) \land (R \rightarrow S) \land \neg (Q \lor S)) \rightarrow \neg (P \lor R)$

\end{questions}
\newpage
%%% begin test
\begin{flushright}
\begin{tabular}{p{2.8in} r l}
%\textbf{\class} & \textbf{ФИО:} & \makebox[2.5in]{\hrulefill}\\
\textbf{\class} & \textbf{ФИО:} &Ондар Кежик Амирович
\\

\textbf{\examdate} &&\\
%\textbf{Time Limit: \timelimit} & Teaching Assistant & \makebox[2in]{\hrulefill}
\end{tabular}\\
\end{flushright}
\rule[1ex]{\textwidth}{.1pt}


\begin{questions}
\question
Найдите и упростите P:
\begin{equation*}
\overline{P} = A \cap \overline{B} \cup \overline{B} \cap C \cup \overline{A} \cap \overline{B} \cup \overline{A} \cap C
\end{equation*}
Затем найдите элементы множества P, выраженного через множества:
\begin{equation*}
A = \{0, 3, 4, 9\}; 
B = \{1, 3, 4, 7\};
C = \{0, 1, 2, 4, 7, 8, 9\};
I = \{0, 1, 2, 3, 4, 5, 6, 7, 8, 9\}.
\end{equation*}\question
Упростите следующее выражение с учетом того, что $A\subset B \subset C \subset D \subset U; A \neq \O$
\begin{equation*}
A \cap  \overline{C} \cup B \cap \overline{D} \cup  \overline{A} \cap C \cap  \overline{D}
\end{equation*}

Примечание: U — универсум\question
Дано отношение на множестве $\{1, 2, 3, 4, 5\}$ 
\begin{equation*}
aRb \iff  \text{НОД}(a,b) =1
\end{equation*}
Напишите обоснованный ответ какими свойствами обладает или не обладает отношение и почему:   
\begin{enumerate} [a)]\setcounter{enumi}{0}
\item рефлексивность
\item антирефлексивность
\item симметричность
\item асимметричность
\item антисимметричность
\item транзитивность
\end{enumerate}

Обоснуйте свой ответ по каждому из приведенных ниже вопросов:
\begin{enumerate} [a)]\setcounter{enumi}{0}
    \item Является ли это отношение отношением эквивалентности?
    \item Является ли это отношение функциональным?
    \item Каким из отношений соответствия (одно-многозначным, много-многозначный и т.д.) оно является?
    \item К каким из отношений порядка (полного, частичного и т.д.) можно отнести данное отношение?
\end{enumerate}


\question
Установите, является ли каждое из перечисленных ниже отношений на А ($R \subseteq A \times A$) отношением эквивалентности (обоснование ответа обязательно). Для каждого отношения эквивалентности постройте классы эквивалентности и постройте граф отношения:
\begin{enumerate} [a)]\setcounter{enumi}{0}
\item $F(x)=x^{2}+1$, где $x \in A = [-2, 4]$ и отношение $R = \{(a,b)|F(a) = F(b)\}$
\item А - множество целых чисел и отношение $R = \{(a,b)|a + b = 5\}$
\item На множестве $A = \{1; 2; 3\}$ задано отношение $R = \{(1; 1); (2; 2); (3; 3); (3; 2); (1; 2); (2; 1)\}$

\end{enumerate}\question Составьте полную таблицу истинности, определите, какие переменные являются фиктивными и проверьте, является ли формула тавтологией:
$((P \rightarrow Q) \land (R \rightarrow S) \land \neg (Q \lor S)) \rightarrow \neg (P \lor R)$

\end{questions}
\newpage
%%% begin test
\begin{flushright}
\begin{tabular}{p{2.8in} r l}
%\textbf{\class} & \textbf{ФИО:} & \makebox[2.5in]{\hrulefill}\\
\textbf{\class} & \textbf{ФИО:} &Орлова Софья Денисовна
\\

\textbf{\examdate} &&\\
%\textbf{Time Limit: \timelimit} & Teaching Assistant & \makebox[2in]{\hrulefill}
\end{tabular}\\
\end{flushright}
\rule[1ex]{\textwidth}{.1pt}


\begin{questions}
\question
Найдите и упростите P:
\begin{equation*}
\overline{P} = A \cap B \cup \overline{A} \cap \overline{B} \cup A \cap C \cup \overline{B} \cap C
\end{equation*}
Затем найдите элементы множества P, выраженного через множества:
\begin{equation*}
A = \{0, 3, 4, 9\}; 
B = \{1, 3, 4, 7\};
C = \{0, 1, 2, 4, 7, 8, 9\};
I = \{0, 1, 2, 3, 4, 5, 6, 7, 8, 9\}.
\end{equation*}\question
Упростите следующее выражение с учетом того, что $A\subset B \subset C \subset D \subset U; A \neq \O$
\begin{equation*}
A \cap B \cup \overline{A} \cap \overline{C} \cup A \cap C \cup \overline{B} \cap \overline{C}
\end{equation*}

Примечание: U — универсум\question
Дано отношение на множестве $\{1, 2, 3, 4, 5\}$ 
\begin{equation*}
aRb \iff a \geq b^2
\end{equation*}
Напишите обоснованный ответ какими свойствами обладает или не обладает отношение и почему:   
\begin{enumerate} [a)]\setcounter{enumi}{0}
\item рефлексивность
\item антирефлексивность
\item симметричность
\item асимметричность
\item антисимметричность
\item транзитивность
\end{enumerate}

Обоснуйте свой ответ по каждому из приведенных ниже вопросов:
\begin{enumerate} [a)]\setcounter{enumi}{0}
    \item Является ли это отношение отношением эквивалентности?
    \item Является ли это отношение функциональным?
    \item Каким из отношений соответствия (одно-многозначным, много-многозначный и т.д.) оно является?
    \item К каким из отношений порядка (полного, частичного и т.д.) можно отнести данное отношение?
\end{enumerate}


\question
Установите, является ли каждое из перечисленных ниже отношений на А ($R \subseteq A \times A$) отношением эквивалентности (обоснование ответа обязательно). Для каждого отношения эквивалентности постройте классы 
эквивалентности и постройте граф отношения:
\begin{enumerate} [a)]\setcounter{enumi}{0}
\item $A = \{a, b, c, d, p, t\}$ задано отношение $R = \{(a, a), (b, b), (b, c), (b, d), (c, b), (c, c), (c, d), (d, b), (d, c), (d, d), (p,p), (t,t)\}$
\item $A = \{-10, -9, … , 9, 10\}$ и отношение $R = \{(a,b)|a^{3} = b^{3}\}$

\item $F(x)=x^{2}+1$, где $x \in A = [-2, 4]$ и отношение $R = \{(a,b)|F(a) = F(b)\}$
\end{enumerate}\question Составьте полную таблицу истинности, определите, какие переменные являются фиктивными и проверьте, является ли формула тавтологией:
$(( P \land \neg Q) \rightarrow (R \land \neg R)) \rightarrow (P \rightarrow Q)$

\end{questions}
\newpage
%%% begin test
\begin{flushright}
\begin{tabular}{p{2.8in} r l}
%\textbf{\class} & \textbf{ФИО:} & \makebox[2.5in]{\hrulefill}\\
\textbf{\class} & \textbf{ФИО:} &Пизик Илья Александрович
\\

\textbf{\examdate} &&\\
%\textbf{Time Limit: \timelimit} & Teaching Assistant & \makebox[2in]{\hrulefill}
\end{tabular}\\
\end{flushright}
\rule[1ex]{\textwidth}{.1pt}


\begin{questions}
\question
Найдите и упростите P:
\begin{equation*}
\overline{P} = A \cap B \cup \overline{A} \cap \overline{B} \cup A \cap C \cup \overline{B} \cap C
\end{equation*}
Затем найдите элементы множества P, выраженного через множества:
\begin{equation*}
A = \{0, 3, 4, 9\}; 
B = \{1, 3, 4, 7\};
C = \{0, 1, 2, 4, 7, 8, 9\};
I = \{0, 1, 2, 3, 4, 5, 6, 7, 8, 9\}.
\end{equation*}\question
Упростите следующее выражение с учетом того, что $A\subset B \subset C \subset D \subset U; A \neq \O$
\begin{equation*}
A \cap C  \cap D \cup B \cap \overline{C} \cap D \cup B \cap C \cap D
\end{equation*}

Примечание: U — универсум\question
Для следующего отношения на множестве $\{1, 2, 3, 4, 5\}$ 
\begin{equation*}
aRb \iff 0 < a-b<2
\end{equation*}
Напишите обоснованный ответ какими свойствами обладает или не обладает отношение и почему:   
\begin{enumerate} [a)]\setcounter{enumi}{0}
\item рефлексивность
\item антирефлексивность
\item симметричность
\item асимметричность
\item антисимметричность
\item транзитивность
\end{enumerate}

Обоснуйте свой ответ по каждому из приведенных ниже вопросов:
\begin{enumerate} [a)]\setcounter{enumi}{0}
    \item Является ли это отношение отношением эквивалентности?
    \item Является ли это отношение функциональным?
    \item Каким из отношений соответствия (одно-многозначным, много-многозначный и т.д.) оно является?
    \item К каким из отношений порядка (полного, частичного и т.д.) можно отнести данное отношение?
\end{enumerate}
\question
Установите, является ли каждое из перечисленных ниже отношений на А ($R \subseteq A \times A$) отношением эквивалентности (обоснование ответа обязательно). Для каждого отношения эквивалентности постройте классы эквивалентности и постройте граф отношения:
\begin{enumerate} [a)]\setcounter{enumi}{0}
\item $F(x)=x^{2}+1$, где $x \in A = [-2, 4]$ и отношение $R = \{(a,b)|F(a) = F(b)\}$
\item А - множество целых чисел и отношение $R = \{(a,b)|a + b = 5\}$
\item На множестве $A = \{1; 2; 3\}$ задано отношение $R = \{(1; 1); (2; 2); (3; 3); (3; 2); (1; 2); (2; 1)\}$

\end{enumerate}\question Составьте полную таблицу истинности, определите, какие переменные являются фиктивными и проверьте, является ли формула тавтологией:
$((P \rightarrow Q) \lor R) \leftrightarrow (P \rightarrow (Q \lor R))$

\end{questions}
\newpage
%%% begin test
\begin{flushright}
\begin{tabular}{p{2.8in} r l}
%\textbf{\class} & \textbf{ФИО:} & \makebox[2.5in]{\hrulefill}\\
\textbf{\class} & \textbf{ФИО:} &Подвысоцкий Андрей Анатольевич
\\

\textbf{\examdate} &&\\
%\textbf{Time Limit: \timelimit} & Teaching Assistant & \makebox[2in]{\hrulefill}
\end{tabular}\\
\end{flushright}
\rule[1ex]{\textwidth}{.1pt}


\begin{questions}
\question
Найдите и упростите P:
\begin{equation*}
\overline{P} = B \cap \overline{C} \cup A \cap B \cup \overline{A} \cap C \cup \overline{A} \cap B
\end{equation*}
Затем найдите элементы множества P, выраженного через множества:
\begin{equation*}
A = \{0, 3, 4, 9\}; 
B = \{1, 3, 4, 7\};
C = \{0, 1, 2, 4, 7, 8, 9\};
I = \{0, 1, 2, 3, 4, 5, 6, 7, 8, 9\}.
\end{equation*}\question
Упростите следующее выражение с учетом того, что $A\subset B \subset C \subset D \subset U; A \neq \O$
\begin{equation*}
A \cap B  \cap \overline{C} \cup \overline{C} \cap D \cup B \cap C \cap D
\end{equation*}

Примечание: U — универсум\question
Дано отношение на множестве $\{1, 2, 3, 4, 5\}$ 
\begin{equation*}
aRb \iff a \geq b^2
\end{equation*}
Напишите обоснованный ответ какими свойствами обладает или не обладает отношение и почему:   
\begin{enumerate} [a)]\setcounter{enumi}{0}
\item рефлексивность
\item антирефлексивность
\item симметричность
\item асимметричность
\item антисимметричность
\item транзитивность
\end{enumerate}

Обоснуйте свой ответ по каждому из приведенных ниже вопросов:
\begin{enumerate} [a)]\setcounter{enumi}{0}
    \item Является ли это отношение отношением эквивалентности?
    \item Является ли это отношение функциональным?
    \item Каким из отношений соответствия (одно-многозначным, много-многозначный и т.д.) оно является?
    \item К каким из отношений порядка (полного, частичного и т.д.) можно отнести данное отношение?
\end{enumerate}


\question
Установите, является ли каждое из перечисленных ниже отношений на А ($R \subseteq A \times A$) отношением эквивалентности (обоснование ответа обязательно). Для каждого отношения эквивалентности постройте классы 
эквивалентности и постройте граф отношения:
\begin{enumerate} [a)]\setcounter{enumi}{0}
\item $A = \{-10, -9, … , 9, 10\}$ и отношение $R = \{(a,b)|a^{2} = b^{2}\}$
\item $A = \{a, b, c, d, p, t\}$ задано отношение $R = \{(a, a), (b, b), (b, c), (b, d), (c, b), (c, c), (c, d), (d, b), (d, c), (d, d), (p,p), (t,t)\}$
\item Пусть A – множество имен. $A = \{ $Алексей, Иван, Петр, Александр, Павел, Андрей$ \}$. Тогда отношение $R$ верно на парах имен, начинающихся с одной и той же буквы, и только на них.
\end{enumerate}\question Составьте полную таблицу истинности, определите, какие переменные являются фиктивными и проверьте, является ли формула тавтологией:
$ P \rightarrow (Q \rightarrow ((P \lor Q) \rightarrow (P \land Q)))$

\end{questions}
\newpage
%%% begin test
\begin{flushright}
\begin{tabular}{p{2.8in} r l}
%\textbf{\class} & \textbf{ФИО:} & \makebox[2.5in]{\hrulefill}\\
\textbf{\class} & \textbf{ФИО:} &Репенко Аннемария
\\

\textbf{\examdate} &&\\
%\textbf{Time Limit: \timelimit} & Teaching Assistant & \makebox[2in]{\hrulefill}
\end{tabular}\\
\end{flushright}
\rule[1ex]{\textwidth}{.1pt}


\begin{questions}
\question
Найдите и упростите P:
\begin{equation*}
\overline{P} = A \cap C \cup \overline{A} \cap \overline{C} \cup \overline{B} \cap C \cup \overline{A} \cap \overline{B}
\end{equation*}
Затем найдите элементы множества P, выраженного через множества:
\begin{equation*}
A = \{0, 3, 4, 9\}; 
B = \{1, 3, 4, 7\};
C = \{0, 1, 2, 4, 7, 8, 9\};
I = \{0, 1, 2, 3, 4, 5, 6, 7, 8, 9\}.
\end{equation*}\question
Упростите следующее выражение с учетом того, что $A\subset B \subset C \subset D \subset U; A \neq \O$
\begin{equation*}
\overline{A} \cap \overline{B} \cup B \cap \overline{C} \cup \overline{C} \cap D
\end{equation*}

Примечание: U — универсум\question
Дано отношение на множестве $\{1, 2, 3, 4, 5\}$ 
\begin{equation*}
aRb \iff |a-b| = 1
\end{equation*}
Напишите обоснованный ответ какими свойствами обладает или не обладает отношение и почему:   
\begin{enumerate} [a)]\setcounter{enumi}{0}
\item рефлексивность
\item антирефлексивность
\item симметричность
\item асимметричность
\item антисимметричность
\item транзитивность
\end{enumerate}

Обоснуйте свой ответ по каждому из приведенных ниже вопросов:
\begin{enumerate} [a)]\setcounter{enumi}{0}
    \item Является ли это отношение отношением эквивалентности?
    \item Является ли это отношение функциональным?
    \item Каким из отношений соответствия (одно-многозначным, много-многозначный и т.д.) оно является?
    \item К каким из отношений порядка (полного, частичного и т.д.) можно отнести данное отношение?
\end{enumerate}

\question
Установите, является ли каждое из перечисленных ниже отношений на А ($R \subseteq A \times A$) отношением эквивалентности (обоснование ответа обязательно). Для каждого отношения эквивалентности 
постройте классы эквивалентности и постройте граф отношения:
\begin{enumerate}[a)]\setcounter{enumi}{0}
\item А - множество целых чисел и отношение $R = \{(a,b)|a + b = 0\}$
\item $A = \{-10, -9, …, 9, 10\}$ и отношение $R = \{(a,b)|a^{3} = b^{3}\}$
\item На множестве $A = \{1; 2; 3\}$ задано отношение $R = \{(1; 1); (2; 2); (3; 3); (2; 1); (1; 2); (2; 3); (3; 2); (3; 1); (1; 3)\}$

\end{enumerate}\question Составьте полную таблицу истинности, определите, какие переменные являются фиктивными и проверьте, является ли формула тавтологией:
$(( P \rightarrow Q) \land (Q \rightarrow P)) \rightarrow (P \rightarrow R)$

\end{questions}
\newpage
%%% begin test
\begin{flushright}
\begin{tabular}{p{2.8in} r l}
%\textbf{\class} & \textbf{ФИО:} & \makebox[2.5in]{\hrulefill}\\
\textbf{\class} & \textbf{ФИО:} &Саидов Рустам Насруддинович
\\

\textbf{\examdate} &&\\
%\textbf{Time Limit: \timelimit} & Teaching Assistant & \makebox[2in]{\hrulefill}
\end{tabular}\\
\end{flushright}
\rule[1ex]{\textwidth}{.1pt}


\begin{questions}
\question
Найдите и упростите P:
\begin{equation*}
\overline{P} = A \cap \overline{B} \cup A \cap C \cup B \cap C \cup \overline{A} \cap C
\end{equation*}
Затем найдите элементы множества P, выраженного через множества:
\begin{equation*}
A = \{0, 3, 4, 9\}; 
B = \{1, 3, 4, 7\};
C = \{0, 1, 2, 4, 7, 8, 9\};
I = \{0, 1, 2, 3, 4, 5, 6, 7, 8, 9\}.
\end{equation*}\question
Упростите следующее выражение с учетом того, что $A\subset B \subset C \subset D \subset U; A \neq \O$
\begin{equation*}
\overline{A} \cap \overline{B} \cup B \cap \overline{C} \cup \overline{C} \cap D
\end{equation*}

Примечание: U — универсум\question
Дано отношение на множестве $\{1, 2, 3, 4, 5\}$ 
\begin{equation*}
aRb \iff a \geq b^2
\end{equation*}
Напишите обоснованный ответ какими свойствами обладает или не обладает отношение и почему:   
\begin{enumerate} [a)]\setcounter{enumi}{0}
\item рефлексивность
\item антирефлексивность
\item симметричность
\item асимметричность
\item антисимметричность
\item транзитивность
\end{enumerate}

Обоснуйте свой ответ по каждому из приведенных ниже вопросов:
\begin{enumerate} [a)]\setcounter{enumi}{0}
    \item Является ли это отношение отношением эквивалентности?
    \item Является ли это отношение функциональным?
    \item Каким из отношений соответствия (одно-многозначным, много-многозначный и т.д.) оно является?
    \item К каким из отношений порядка (полного, частичного и т.д.) можно отнести данное отношение?
\end{enumerate}


\question
Установите, является ли каждое из перечисленных ниже отношений на А ($R \subseteq A \times A$) отношением эквивалентности (обоснование ответа обязательно). Для каждого отношения эквивалентности 
постройте классы эквивалентности и постройте граф отношения:
\begin{enumerate}[a)]\setcounter{enumi}{0}
\item А - множество целых чисел и отношение $R = \{(a,b)|a + b = 0\}$
\item $A = \{-10, -9, …, 9, 10\}$ и отношение $R = \{(a,b)|a^{3} = b^{3}\}$
\item На множестве $A = \{1; 2; 3\}$ задано отношение $R = \{(1; 1); (2; 2); (3; 3); (2; 1); (1; 2); (2; 3); (3; 2); (3; 1); (1; 3)\}$

\end{enumerate}\question Составьте полную таблицу истинности, определите, какие переменные являются фиктивными и проверьте, является ли формула тавтологией:
$(P \rightarrow (Q \rightarrow R)) \rightarrow ((P \rightarrow Q) \rightarrow (P \rightarrow R))$

\end{questions}
\newpage
%%% begin test
\begin{flushright}
\begin{tabular}{p{2.8in} r l}
%\textbf{\class} & \textbf{ФИО:} & \makebox[2.5in]{\hrulefill}\\
\textbf{\class} & \textbf{ФИО:} &Терентьева Алена Витальевна
\\

\textbf{\examdate} &&\\
%\textbf{Time Limit: \timelimit} & Teaching Assistant & \makebox[2in]{\hrulefill}
\end{tabular}\\
\end{flushright}
\rule[1ex]{\textwidth}{.1pt}


\begin{questions}
\question
Найдите и упростите P:
\begin{equation*}
\overline{P} = A \cap B \cup \overline{A} \cap \overline{B} \cup A \cap C \cup \overline{B} \cap C
\end{equation*}
Затем найдите элементы множества P, выраженного через множества:
\begin{equation*}
A = \{0, 3, 4, 9\}; 
B = \{1, 3, 4, 7\};
C = \{0, 1, 2, 4, 7, 8, 9\};
I = \{0, 1, 2, 3, 4, 5, 6, 7, 8, 9\}.
\end{equation*}\question
Упростите следующее выражение с учетом того, что $A\subset B \subset C \subset D \subset U; A \neq \O$
\begin{equation*}
A \cap  \overline{C} \cup B \cap \overline{D} \cup  \overline{A} \cap C \cap  \overline{D}
\end{equation*}

Примечание: U — универсум\question
Дано отношение на множестве $\{1, 2, 3, 4, 5\}$ 
\begin{equation*}
aRb \iff b > a
\end{equation*}
Напишите обоснованный ответ какими свойствами обладает или не обладает отношение и почему:   
\begin{enumerate} [a)]\setcounter{enumi}{0}
\item рефлексивность
\item антирефлексивность
\item симметричность
\item асимметричность
\item антисимметричность
\item транзитивность
\end{enumerate}

Обоснуйте свой ответ по каждому из приведенных ниже вопросов:
\begin{enumerate} [a)]\setcounter{enumi}{0}
    \item Является ли это отношение отношением эквивалентности?
    \item Является ли это отношение функциональным?
    \item Каким из отношений соответствия (одно-многозначным, много-многозначный и т.д.) оно является?
    \item К каким из отношений порядка (полного, частичного и т.д.) можно отнести данное отношение?
\end{enumerate}

\question
Установите, является ли каждое из перечисленных ниже отношений на А ($R \subseteq A \times A$) отношением эквивалентности (обоснование ответа обязательно). Для каждого отношения эквивалентности постройте классы 
эквивалентности и постройте граф отношения:
\begin{enumerate} [a)]\setcounter{enumi}{0}
\item На множестве $A = \{1; 2; 3\}$ задано отношение $R = \{(1; 1); (2; 2); (3; 3); (2; 1); (1; 2); (2; 3); (3; 2); (3; 1); (1; 3)\}$
\item На множестве $A = \{1; 2; 3; 4; 5\}$ задано отношение $R = \{(1; 2); (1; 3); (1; 5); (2; 3); (2; 4); (2; 5); (3; 4); (3; 5); (4; 5)\}$
\item А - множество целых чисел и отношение $R = \{(a,b)|a + b = 0\}$
\end{enumerate}\question Составьте полную таблицу истинности, определите, какие переменные являются фиктивными и проверьте, является ли формула тавтологией:
$(( P \rightarrow Q) \land (Q \rightarrow P)) \rightarrow (P \rightarrow R)$

\end{questions}
\newpage
%%% begin test
\begin{flushright}
\begin{tabular}{p{2.8in} r l}
%\textbf{\class} & \textbf{ФИО:} & \makebox[2.5in]{\hrulefill}\\
\textbf{\class} & \textbf{ФИО:} &Уарова Ивалена Александровна
\\

\textbf{\examdate} &&\\
%\textbf{Time Limit: \timelimit} & Teaching Assistant & \makebox[2in]{\hrulefill}
\end{tabular}\\
\end{flushright}
\rule[1ex]{\textwidth}{.1pt}


\begin{questions}
\question
Найдите и упростите P:
\begin{equation*}
\overline{P} = \overline{A} \cap B \cup \overline{A} \cap C \cup A \cap \overline{B} \cup \overline{B} \cap C
\end{equation*}
Затем найдите элементы множества P, выраженного через множества:
\begin{equation*}
A = \{0, 3, 4, 9\}; 
B = \{1, 3, 4, 7\};
C = \{0, 1, 2, 4, 7, 8, 9\};
I = \{0, 1, 2, 3, 4, 5, 6, 7, 8, 9\}.
\end{equation*}\question
Упростите следующее выражение с учетом того, что $A\subset B \subset C \subset D \subset U; A \neq \O$
\begin{equation*}
\overline{B} \cap \overline{C} \cap D \cup \overline{A} \cap \overline{C} \cap D \cup \overline{A} \cap B
\end{equation*}

Примечание: U — универсум\question
Дано отношение на множестве $\{1, 2, 3, 4, 5\}$ 
\begin{equation*}
aRb \iff |a-b| = 1
\end{equation*}
Напишите обоснованный ответ какими свойствами обладает или не обладает отношение и почему:   
\begin{enumerate} [a)]\setcounter{enumi}{0}
\item рефлексивность
\item антирефлексивность
\item симметричность
\item асимметричность
\item антисимметричность
\item транзитивность
\end{enumerate}

Обоснуйте свой ответ по каждому из приведенных ниже вопросов:
\begin{enumerate} [a)]\setcounter{enumi}{0}
    \item Является ли это отношение отношением эквивалентности?
    \item Является ли это отношение функциональным?
    \item Каким из отношений соответствия (одно-многозначным, много-многозначный и т.д.) оно является?
    \item К каким из отношений порядка (полного, частичного и т.д.) можно отнести данное отношение?
\end{enumerate}

\question
Установите, является ли каждое из перечисленных ниже отношений на А ($R \subseteq A \times A$) отношением эквивалентности (обоснование ответа обязательно). Для каждого отношения эквивалентности 
постройте классы эквивалентности и постройте граф отношения:
\begin{enumerate}[a)]\setcounter{enumi}{0}
\item А - множество целых чисел и отношение $R = \{(a,b)|a + b = 0\}$
\item $A = \{-10, -9, …, 9, 10\}$ и отношение $R = \{(a,b)|a^{3} = b^{3}\}$
\item На множестве $A = \{1; 2; 3\}$ задано отношение $R = \{(1; 1); (2; 2); (3; 3); (2; 1); (1; 2); (2; 3); (3; 2); (3; 1); (1; 3)\}$

\end{enumerate}\question Составьте полную таблицу истинности, определите, какие переменные являются фиктивными и проверьте, является ли формула тавтологией:
$(( P \land \neg Q) \rightarrow (R \land \neg R)) \rightarrow (P \rightarrow Q)$

\end{questions}
\newpage
%%% begin test
\begin{flushright}
\begin{tabular}{p{2.8in} r l}
%\textbf{\class} & \textbf{ФИО:} & \makebox[2.5in]{\hrulefill}\\
\textbf{\class} & \textbf{ФИО:} &Шаламова Дарья Сергеевна
\\

\textbf{\examdate} &&\\
%\textbf{Time Limit: \timelimit} & Teaching Assistant & \makebox[2in]{\hrulefill}
\end{tabular}\\
\end{flushright}
\rule[1ex]{\textwidth}{.1pt}


\begin{questions}
\question
Найдите и упростите P:
\begin{equation*}
\overline{P} = A \cap \overline{B} \cup \overline{B} \cap C \cup \overline{A} \cap \overline{B} \cup \overline{A} \cap C
\end{equation*}
Затем найдите элементы множества P, выраженного через множества:
\begin{equation*}
A = \{0, 3, 4, 9\}; 
B = \{1, 3, 4, 7\};
C = \{0, 1, 2, 4, 7, 8, 9\};
I = \{0, 1, 2, 3, 4, 5, 6, 7, 8, 9\}.
\end{equation*}\question
Упростите следующее выражение с учетом того, что $A\subset B \subset C \subset D \subset U; A \neq \O$
\begin{equation*}
\overline{A} \cap \overline{C} \cap D \cup \overline{B} \cap \overline{C} \cap D \cup A \cap B
\end{equation*}

Примечание: U — универсум\question
Дано отношение на множестве $\{1, 2, 3, 4, 5\}$ 
\begin{equation*}
aRb \iff  \text{НОД}(a,b) =1
\end{equation*}
Напишите обоснованный ответ какими свойствами обладает или не обладает отношение и почему:   
\begin{enumerate} [a)]\setcounter{enumi}{0}
\item рефлексивность
\item антирефлексивность
\item симметричность
\item асимметричность
\item антисимметричность
\item транзитивность
\end{enumerate}

Обоснуйте свой ответ по каждому из приведенных ниже вопросов:
\begin{enumerate} [a)]\setcounter{enumi}{0}
    \item Является ли это отношение отношением эквивалентности?
    \item Является ли это отношение функциональным?
    \item Каким из отношений соответствия (одно-многозначным, много-многозначный и т.д.) оно является?
    \item К каким из отношений порядка (полного, частичного и т.д.) можно отнести данное отношение?
\end{enumerate}


\question
Установите, является ли каждое из перечисленных ниже отношений на А ($R \subseteq A \times A$) отношением эквивалентности (обоснование ответа обязательно). Для каждого отношения эквивалентности постройте классы 
эквивалентности и постройте граф отношения:
\begin{enumerate} [a)]\setcounter{enumi}{0}
\item А - множество целых чисел и отношение $R = \{(a,b)|a + b = 5\}$
\item Пусть A – множество имен. $A = \{ $Алексей, Иван, Петр, Александр, Павел, Андрей$ \}$. Тогда отношение $R $ верно на парах имен, начинающихся с одной и той же буквы, и только на них.
\item На множестве $A = \{1; 2; 3; 4; 5\}$ задано отношение $R = \{(1; 2); (1; 3); (1; 5); (2; 3); (2; 4); (2; 5); (3; 4); (3; 5); (4; 5)\}$
\end{enumerate}\question Составьте полную таблицу истинности, определите, какие переменные являются фиктивными и проверьте, является ли формула тавтологией:

$(P \rightarrow (Q \land R)) \leftrightarrow ((P \rightarrow Q) \land (P \rightarrow R))$

\end{questions}
\newpage
%%% begin test
\begin{flushright}
\begin{tabular}{p{2.8in} r l}
%\textbf{\class} & \textbf{ФИО:} & \makebox[2.5in]{\hrulefill}\\
\textbf{\class} & \textbf{ФИО:} &М3112
\\

\textbf{\examdate} &&\\
%\textbf{Time Limit: \timelimit} & Teaching Assistant & \makebox[2in]{\hrulefill}
\end{tabular}\\
\end{flushright}
\rule[1ex]{\textwidth}{.1pt}


\begin{questions}
\question
Найдите и упростите P:
\begin{equation*}
\overline{P} = \overline{A} \cap B \cup \overline{A} \cap C \cup A \cap \overline{B} \cup \overline{B} \cap C
\end{equation*}
Затем найдите элементы множества P, выраженного через множества:
\begin{equation*}
A = \{0, 3, 4, 9\}; 
B = \{1, 3, 4, 7\};
C = \{0, 1, 2, 4, 7, 8, 9\};
I = \{0, 1, 2, 3, 4, 5, 6, 7, 8, 9\}.
\end{equation*}\question
Упростите следующее выражение с учетом того, что $A\subset B \subset C \subset D \subset U; A \neq \O$
\begin{equation*}
A \cap B \cup \overline{A} \cap \overline{C} \cup A \cap C \cup \overline{B} \cap \overline{C}
\end{equation*}

Примечание: U — универсум\question
Дано отношение на множестве $\{1, 2, 3, 4, 5\}$ 
\begin{equation*}
aRb \iff (a+b) \bmod 2 =0
\end{equation*}
Напишите обоснованный ответ какими свойствами обладает или не обладает отношение и почему:   
\begin{enumerate} [a)]\setcounter{enumi}{0}
\item рефлексивность
\item антирефлексивность
\item симметричность
\item асимметричность
\item антисимметричность
\item транзитивность
\end{enumerate}

Обоснуйте свой ответ по каждому из приведенных ниже вопросов:
\begin{enumerate} [a)]\setcounter{enumi}{0}
    \item Является ли это отношение отношением эквивалентности?
    \item Является ли это отношение функциональным?
    \item Каким из отношений соответствия (одно-многозначным, много-многозначный и т.д.) оно является?
    \item К каким из отношений порядка (полного, частичного и т.д.) можно отнести данное отношение?
\end{enumerate}



\question
Установите, является ли каждое из перечисленных ниже отношений на А ($R \subseteq A \times A$) отношением эквивалентности (обоснование ответа обязательно). Для каждого отношения эквивалентности постройте классы 
эквивалентности и постройте граф отношения:
\begin{enumerate} [a)]\setcounter{enumi}{0}
\item А - множество целых чисел и отношение $R = \{(a,b)|a + b = 5\}$
\item Пусть A – множество имен. $A = \{ $Алексей, Иван, Петр, Александр, Павел, Андрей$ \}$. Тогда отношение $R $ верно на парах имен, начинающихся с одной и той же буквы, и только на них.
\item На множестве $A = \{1; 2; 3; 4; 5\}$ задано отношение $R = \{(1; 2); (1; 3); (1; 5); (2; 3); (2; 4); (2; 5); (3; 4); (3; 5); (4; 5)\}$
\end{enumerate}\question Составьте полную таблицу истинности, определите, какие переменные являются фиктивными и проверьте, является ли формула тавтологией:
$(( P \land \neg Q) \rightarrow (R \land \neg R)) \rightarrow (P \rightarrow Q)$

\end{questions}
\newpage
%%% begin test
\begin{flushright}
\begin{tabular}{p{2.8in} r l}
%\textbf{\class} & \textbf{ФИО:} & \makebox[2.5in]{\hrulefill}\\
\textbf{\class} & \textbf{ФИО:} &Абухатем Амру Мансур Ахмед Салех
\\

\textbf{\examdate} &&\\
%\textbf{Time Limit: \timelimit} & Teaching Assistant & \makebox[2in]{\hrulefill}
\end{tabular}\\
\end{flushright}
\rule[1ex]{\textwidth}{.1pt}


\begin{questions}
\question
Найдите и упростите P:
\begin{equation*}
\overline{P} = A \cap B \cup \overline{A} \cap \overline{B} \cup A \cap C \cup \overline{B} \cap C
\end{equation*}
Затем найдите элементы множества P, выраженного через множества:
\begin{equation*}
A = \{0, 3, 4, 9\}; 
B = \{1, 3, 4, 7\};
C = \{0, 1, 2, 4, 7, 8, 9\};
I = \{0, 1, 2, 3, 4, 5, 6, 7, 8, 9\}.
\end{equation*}\question
Упростите следующее выражение с учетом того, что $A\subset B \subset C \subset D \subset U; A \neq \O$
\begin{equation*}
A \cap C  \cap D \cup B \cap \overline{C} \cap D \cup B \cap C \cap D
\end{equation*}

Примечание: U — универсум\question
Дано отношение на множестве $\{1, 2, 3, 4, 5\}$ 
\begin{equation*}
aRb \iff |a-b| = 1
\end{equation*}
Напишите обоснованный ответ какими свойствами обладает или не обладает отношение и почему:   
\begin{enumerate} [a)]\setcounter{enumi}{0}
\item рефлексивность
\item антирефлексивность
\item симметричность
\item асимметричность
\item антисимметричность
\item транзитивность
\end{enumerate}

Обоснуйте свой ответ по каждому из приведенных ниже вопросов:
\begin{enumerate} [a)]\setcounter{enumi}{0}
    \item Является ли это отношение отношением эквивалентности?
    \item Является ли это отношение функциональным?
    \item Каким из отношений соответствия (одно-многозначным, много-многозначный и т.д.) оно является?
    \item К каким из отношений порядка (полного, частичного и т.д.) можно отнести данное отношение?
\end{enumerate}

\question
Установите, является ли каждое из перечисленных ниже отношений на А ($R \subseteq A \times A$) отношением эквивалентности (обоснование ответа обязательно). Для каждого отношения эквивалентности постройте классы 
эквивалентности и постройте граф отношения:
\begin{enumerate} [a)]\setcounter{enumi}{0}
\item На множестве $A = \{1; 2; 3\}$ задано отношение $R = \{(1; 1); (2; 2); (3; 3); (2; 1); (1; 2); (2; 3); (3; 2); (3; 1); (1; 3)\}$
\item На множестве $A = \{1; 2; 3; 4; 5\}$ задано отношение $R = \{(1; 2); (1; 3); (1; 5); (2; 3); (2; 4); (2; 5); (3; 4); (3; 5); (4; 5)\}$
\item А - множество целых чисел и отношение $R = \{(a,b)|a + b = 0\}$
\end{enumerate}\question Составьте полную таблицу истинности, определите, какие переменные являются фиктивными и проверьте, является ли формула тавтологией:
$ P \rightarrow (Q \rightarrow ((P \lor Q) \rightarrow (P \land Q)))$

\end{questions}
\newpage
%%% begin test
\begin{flushright}
\begin{tabular}{p{2.8in} r l}
%\textbf{\class} & \textbf{ФИО:} & \makebox[2.5in]{\hrulefill}\\
\textbf{\class} & \textbf{ФИО:} &Бушуева Александра Олеговна
\\

\textbf{\examdate} &&\\
%\textbf{Time Limit: \timelimit} & Teaching Assistant & \makebox[2in]{\hrulefill}
\end{tabular}\\
\end{flushright}
\rule[1ex]{\textwidth}{.1pt}


\begin{questions}
\question
Найдите и упростите P:
\begin{equation*}
\overline{P} = A \cap \overline{B} \cup A \cap C \cup B \cap C \cup \overline{A} \cap C
\end{equation*}
Затем найдите элементы множества P, выраженного через множества:
\begin{equation*}
A = \{0, 3, 4, 9\}; 
B = \{1, 3, 4, 7\};
C = \{0, 1, 2, 4, 7, 8, 9\};
I = \{0, 1, 2, 3, 4, 5, 6, 7, 8, 9\}.
\end{equation*}\question
Упростите следующее выражение с учетом того, что $A\subset B \subset C \subset D \subset U; A \neq \O$
\begin{equation*}
\overline{A} \cap \overline{C} \cap D \cup \overline{B} \cap \overline{C} \cap D \cup A \cap B
\end{equation*}

Примечание: U — универсум\question
Дано отношение на множестве $\{1, 2, 3, 4, 5\}$ 
\begin{equation*}
aRb \iff a \leq b
\end{equation*}
Напишите обоснованный ответ какими свойствами обладает или не обладает отношение и почему:   
\begin{enumerate} [a)]\setcounter{enumi}{0}
\item рефлексивность
\item антирефлексивность
\item симметричность
\item асимметричность
\item антисимметричность
\item транзитивность
\end{enumerate}

Обоснуйте свой ответ по каждому из приведенных ниже вопросов:
\begin{enumerate} [a)]\setcounter{enumi}{0}
    \item Является ли это отношение отношением эквивалентности?
    \item Является ли это отношение функциональным?
    \item Каким из отношений соответствия (одно-многозначным, много-многозначный и т.д.) оно является?
    \item К каким из отношений порядка (полного, частичного и т.д.) можно отнести данное отношение?
\end{enumerate}


\question
Установите, является ли каждое из перечисленных ниже отношений на А ($R \subseteq A \times A$) отношением эквивалентности (обоснование ответа обязательно). Для каждого отношения эквивалентности 
постройте классы эквивалентности и постройте граф отношения:
\begin{enumerate}[a)]\setcounter{enumi}{0}
\item А - множество целых чисел и отношение $R = \{(a,b)|a + b = 0\}$
\item $A = \{-10, -9, …, 9, 10\}$ и отношение $R = \{(a,b)|a^{3} = b^{3}\}$
\item На множестве $A = \{1; 2; 3\}$ задано отношение $R = \{(1; 1); (2; 2); (3; 3); (2; 1); (1; 2); (2; 3); (3; 2); (3; 1); (1; 3)\}$

\end{enumerate}\question Составьте полную таблицу истинности, определите, какие переменные являются фиктивными и проверьте, является ли формула тавтологией:
$(P \rightarrow (Q \rightarrow R)) \rightarrow ((P \rightarrow Q) \rightarrow (P \rightarrow R))$

\end{questions}
\newpage
%%% begin test
\begin{flushright}
\begin{tabular}{p{2.8in} r l}
%\textbf{\class} & \textbf{ФИО:} & \makebox[2.5in]{\hrulefill}\\
\textbf{\class} & \textbf{ФИО:} &Дудина Екатерина Михайловна
\\

\textbf{\examdate} &&\\
%\textbf{Time Limit: \timelimit} & Teaching Assistant & \makebox[2in]{\hrulefill}
\end{tabular}\\
\end{flushright}
\rule[1ex]{\textwidth}{.1pt}


\begin{questions}
\question
Найдите и упростите P:
\begin{equation*}
\overline{P} = A \cap \overline{C} \cup A \cap \overline{B} \cup B \cap \overline{C} \cup A \cap C
\end{equation*}
Затем найдите элементы множества P, выраженного через множества:
\begin{equation*}
A = \{0, 3, 4, 9\}; 
B = \{1, 3, 4, 7\};
C = \{0, 1, 2, 4, 7, 8, 9\};
I = \{0, 1, 2, 3, 4, 5, 6, 7, 8, 9\}.
\end{equation*}\question
Упростите следующее выражение с учетом того, что $A\subset B \subset C \subset D \subset U; A \neq \O$
\begin{equation*}
\overline{A} \cap \overline{B} \cup B \cap \overline{C} \cup \overline{C} \cap D
\end{equation*}

Примечание: U — универсум\question
Дано отношение на множестве $\{1, 2, 3, 4, 5\}$ 
\begin{equation*}
aRb \iff (a+b) \bmod 2 =0
\end{equation*}
Напишите обоснованный ответ какими свойствами обладает или не обладает отношение и почему:   
\begin{enumerate} [a)]\setcounter{enumi}{0}
\item рефлексивность
\item антирефлексивность
\item симметричность
\item асимметричность
\item антисимметричность
\item транзитивность
\end{enumerate}

Обоснуйте свой ответ по каждому из приведенных ниже вопросов:
\begin{enumerate} [a)]\setcounter{enumi}{0}
    \item Является ли это отношение отношением эквивалентности?
    \item Является ли это отношение функциональным?
    \item Каким из отношений соответствия (одно-многозначным, много-многозначный и т.д.) оно является?
    \item К каким из отношений порядка (полного, частичного и т.д.) можно отнести данное отношение?
\end{enumerate}



\question
Установите, является ли каждое из перечисленных ниже отношений на А ($R \subseteq A \times A$) отношением эквивалентности (обоснование ответа обязательно). Для каждого отношения эквивалентности постройте классы 
эквивалентности и постройте граф отношения:
\begin{enumerate} [a)]\setcounter{enumi}{0}
\item А - множество целых чисел и отношение $R = \{(a,b)|a + b = 5\}$
\item Пусть A – множество имен. $A = \{ $Алексей, Иван, Петр, Александр, Павел, Андрей$ \}$. Тогда отношение $R $ верно на парах имен, начинающихся с одной и той же буквы, и только на них.
\item На множестве $A = \{1; 2; 3; 4; 5\}$ задано отношение $R = \{(1; 2); (1; 3); (1; 5); (2; 3); (2; 4); (2; 5); (3; 4); (3; 5); (4; 5)\}$
\end{enumerate}\question Составьте полную таблицу истинности, определите, какие переменные являются фиктивными и проверьте, является ли формула тавтологией:

$(P \rightarrow (Q \land R)) \leftrightarrow ((P \rightarrow Q) \land (P \rightarrow R))$

\end{questions}
\newpage
%%% begin test
\begin{flushright}
\begin{tabular}{p{2.8in} r l}
%\textbf{\class} & \textbf{ФИО:} & \makebox[2.5in]{\hrulefill}\\
\textbf{\class} & \textbf{ФИО:} &Дун Цзеюй
\\

\textbf{\examdate} &&\\
%\textbf{Time Limit: \timelimit} & Teaching Assistant & \makebox[2in]{\hrulefill}
\end{tabular}\\
\end{flushright}
\rule[1ex]{\textwidth}{.1pt}


\begin{questions}
\question
Найдите и упростите P:
\begin{equation*}
\overline{P} = A \cap \overline{B} \cup A \cap C \cup B \cap C \cup \overline{A} \cap C
\end{equation*}
Затем найдите элементы множества P, выраженного через множества:
\begin{equation*}
A = \{0, 3, 4, 9\}; 
B = \{1, 3, 4, 7\};
C = \{0, 1, 2, 4, 7, 8, 9\};
I = \{0, 1, 2, 3, 4, 5, 6, 7, 8, 9\}.
\end{equation*}\question
Упростите следующее выражение с учетом того, что $A\subset B \subset C \subset D \subset U; A \neq \O$
\begin{equation*}
A \cap  \overline{C} \cup B \cap \overline{D} \cup  \overline{A} \cap C \cap  \overline{D}
\end{equation*}

Примечание: U — универсум\question
Дано отношение на множестве $\{1, 2, 3, 4, 5\}$ 
\begin{equation*}
aRb \iff |a-b| = 1
\end{equation*}
Напишите обоснованный ответ какими свойствами обладает или не обладает отношение и почему:   
\begin{enumerate} [a)]\setcounter{enumi}{0}
\item рефлексивность
\item антирефлексивность
\item симметричность
\item асимметричность
\item антисимметричность
\item транзитивность
\end{enumerate}

Обоснуйте свой ответ по каждому из приведенных ниже вопросов:
\begin{enumerate} [a)]\setcounter{enumi}{0}
    \item Является ли это отношение отношением эквивалентности?
    \item Является ли это отношение функциональным?
    \item Каким из отношений соответствия (одно-многозначным, много-многозначный и т.д.) оно является?
    \item К каким из отношений порядка (полного, частичного и т.д.) можно отнести данное отношение?
\end{enumerate}

\question
Установите, является ли каждое из перечисленных ниже отношений на А ($R \subseteq A \times A$) отношением эквивалентности (обоснование ответа обязательно). Для каждого отношения эквивалентности постройте классы 
эквивалентности и постройте граф отношения:
\begin{enumerate} [a)]\setcounter{enumi}{0}
\item А - множество целых чисел и отношение $R = \{(a,b)|a + b = 5\}$
\item Пусть A – множество имен. $A = \{ $Алексей, Иван, Петр, Александр, Павел, Андрей$ \}$. Тогда отношение $R $ верно на парах имен, начинающихся с одной и той же буквы, и только на них.
\item На множестве $A = \{1; 2; 3; 4; 5\}$ задано отношение $R = \{(1; 2); (1; 3); (1; 5); (2; 3); (2; 4); (2; 5); (3; 4); (3; 5); (4; 5)\}$
\end{enumerate}\question Составьте полную таблицу истинности, определите, какие переменные являются фиктивными и проверьте, является ли формула тавтологией:
$(P \rightarrow (Q \rightarrow R)) \rightarrow ((P \rightarrow Q) \rightarrow (P \rightarrow R))$

\end{questions}
\newpage
%%% begin test
\begin{flushright}
\begin{tabular}{p{2.8in} r l}
%\textbf{\class} & \textbf{ФИО:} & \makebox[2.5in]{\hrulefill}\\
\textbf{\class} & \textbf{ФИО:} &Ибрагимов Саид Исаевич
\\

\textbf{\examdate} &&\\
%\textbf{Time Limit: \timelimit} & Teaching Assistant & \makebox[2in]{\hrulefill}
\end{tabular}\\
\end{flushright}
\rule[1ex]{\textwidth}{.1pt}


\begin{questions}
\question
Найдите и упростите P:
\begin{equation*}
\overline{P} = A \cap \overline{B} \cup \overline{B} \cap C \cup \overline{A} \cap \overline{B} \cup \overline{A} \cap C
\end{equation*}
Затем найдите элементы множества P, выраженного через множества:
\begin{equation*}
A = \{0, 3, 4, 9\}; 
B = \{1, 3, 4, 7\};
C = \{0, 1, 2, 4, 7, 8, 9\};
I = \{0, 1, 2, 3, 4, 5, 6, 7, 8, 9\}.
\end{equation*}\question
Упростите следующее выражение с учетом того, что $A\subset B \subset C \subset D \subset U; A \neq \O$
\begin{equation*}
\overline{A} \cap \overline{C} \cap D \cup \overline{B} \cap \overline{C} \cap D \cup A \cap B
\end{equation*}

Примечание: U — универсум\question
Дано отношение на множестве $\{1, 2, 3, 4, 5\}$ 
\begin{equation*}
aRb \iff a \leq b
\end{equation*}
Напишите обоснованный ответ какими свойствами обладает или не обладает отношение и почему:   
\begin{enumerate} [a)]\setcounter{enumi}{0}
\item рефлексивность
\item антирефлексивность
\item симметричность
\item асимметричность
\item антисимметричность
\item транзитивность
\end{enumerate}

Обоснуйте свой ответ по каждому из приведенных ниже вопросов:
\begin{enumerate} [a)]\setcounter{enumi}{0}
    \item Является ли это отношение отношением эквивалентности?
    \item Является ли это отношение функциональным?
    \item Каким из отношений соответствия (одно-многозначным, много-многозначный и т.д.) оно является?
    \item К каким из отношений порядка (полного, частичного и т.д.) можно отнести данное отношение?
\end{enumerate}


\question
Установите, является ли каждое из перечисленных ниже отношений на А ($R \subseteq A \times A$) отношением эквивалентности (обоснование ответа обязательно). Для каждого отношения эквивалентности постройте классы 
эквивалентности и постройте граф отношения:
\begin{enumerate} [a)]\setcounter{enumi}{0}
\item А - множество целых чисел и отношение $R = \{(a,b)|a + b = 5\}$
\item Пусть A – множество имен. $A = \{ $Алексей, Иван, Петр, Александр, Павел, Андрей$ \}$. Тогда отношение $R $ верно на парах имен, начинающихся с одной и той же буквы, и только на них.
\item На множестве $A = \{1; 2; 3; 4; 5\}$ задано отношение $R = \{(1; 2); (1; 3); (1; 5); (2; 3); (2; 4); (2; 5); (3; 4); (3; 5); (4; 5)\}$
\end{enumerate}\question Составьте полную таблицу истинности, определите, какие переменные являются фиктивными и проверьте, является ли формула тавтологией:
$(P \rightarrow (Q \rightarrow R)) \rightarrow ((P \rightarrow Q) \rightarrow (P \rightarrow R))$

\end{questions}
\newpage
%%% begin test
\begin{flushright}
\begin{tabular}{p{2.8in} r l}
%\textbf{\class} & \textbf{ФИО:} & \makebox[2.5in]{\hrulefill}\\
\textbf{\class} & \textbf{ФИО:} &Клепиков Анатолий Викторович
\\

\textbf{\examdate} &&\\
%\textbf{Time Limit: \timelimit} & Teaching Assistant & \makebox[2in]{\hrulefill}
\end{tabular}\\
\end{flushright}
\rule[1ex]{\textwidth}{.1pt}


\begin{questions}
\question
Найдите и упростите P:
\begin{equation*}
\overline{P} = A \cap B \cup \overline{A} \cap \overline{B} \cup A \cap C \cup \overline{B} \cap C
\end{equation*}
Затем найдите элементы множества P, выраженного через множества:
\begin{equation*}
A = \{0, 3, 4, 9\}; 
B = \{1, 3, 4, 7\};
C = \{0, 1, 2, 4, 7, 8, 9\};
I = \{0, 1, 2, 3, 4, 5, 6, 7, 8, 9\}.
\end{equation*}\question
Упростите следующее выражение с учетом того, что $A\subset B \subset C \subset D \subset U; A \neq \O$
\begin{equation*}
A \cap B  \cap \overline{C} \cup \overline{C} \cap D \cup B \cap C \cap D
\end{equation*}

Примечание: U — универсум\question
Дано отношение на множестве $\{1, 2, 3, 4, 5\}$ 
\begin{equation*}
aRb \iff |a-b| = 1
\end{equation*}
Напишите обоснованный ответ какими свойствами обладает или не обладает отношение и почему:   
\begin{enumerate} [a)]\setcounter{enumi}{0}
\item рефлексивность
\item антирефлексивность
\item симметричность
\item асимметричность
\item антисимметричность
\item транзитивность
\end{enumerate}

Обоснуйте свой ответ по каждому из приведенных ниже вопросов:
\begin{enumerate} [a)]\setcounter{enumi}{0}
    \item Является ли это отношение отношением эквивалентности?
    \item Является ли это отношение функциональным?
    \item Каким из отношений соответствия (одно-многозначным, много-многозначный и т.д.) оно является?
    \item К каким из отношений порядка (полного, частичного и т.д.) можно отнести данное отношение?
\end{enumerate}

\question
Установите, является ли каждое из перечисленных ниже отношений на А ($R \subseteq A \times A$) отношением эквивалентности (обоснование ответа обязательно). Для каждого отношения эквивалентности 
постройте классы эквивалентности и постройте граф отношения:
\begin{enumerate}[a)]\setcounter{enumi}{0}
\item А - множество целых чисел и отношение $R = \{(a,b)|a + b = 0\}$
\item $A = \{-10, -9, …, 9, 10\}$ и отношение $R = \{(a,b)|a^{3} = b^{3}\}$
\item На множестве $A = \{1; 2; 3\}$ задано отношение $R = \{(1; 1); (2; 2); (3; 3); (2; 1); (1; 2); (2; 3); (3; 2); (3; 1); (1; 3)\}$

\end{enumerate}\question Составьте полную таблицу истинности, определите, какие переменные являются фиктивными и проверьте, является ли формула тавтологией:
$ P \rightarrow (Q \rightarrow ((P \lor Q) \rightarrow (P \land Q)))$

\end{questions}
\newpage
%%% begin test
\begin{flushright}
\begin{tabular}{p{2.8in} r l}
%\textbf{\class} & \textbf{ФИО:} & \makebox[2.5in]{\hrulefill}\\
\textbf{\class} & \textbf{ФИО:} &Климов Никита Валерьевич
\\

\textbf{\examdate} &&\\
%\textbf{Time Limit: \timelimit} & Teaching Assistant & \makebox[2in]{\hrulefill}
\end{tabular}\\
\end{flushright}
\rule[1ex]{\textwidth}{.1pt}


\begin{questions}
\question
Найдите и упростите P:
\begin{equation*}
\overline{P} = A \cap B \cup \overline{A} \cap \overline{B} \cup A \cap C \cup \overline{B} \cap C
\end{equation*}
Затем найдите элементы множества P, выраженного через множества:
\begin{equation*}
A = \{0, 3, 4, 9\}; 
B = \{1, 3, 4, 7\};
C = \{0, 1, 2, 4, 7, 8, 9\};
I = \{0, 1, 2, 3, 4, 5, 6, 7, 8, 9\}.
\end{equation*}\question
Упростите следующее выражение с учетом того, что $A\subset B \subset C \subset D \subset U; A \neq \O$
\begin{equation*}
A \cap  \overline{C} \cup B \cap \overline{D} \cup  \overline{A} \cap C \cap  \overline{D}
\end{equation*}

Примечание: U — универсум\question
Дано отношение на множестве $\{1, 2, 3, 4, 5\}$ 
\begin{equation*}
aRb \iff a \leq b
\end{equation*}
Напишите обоснованный ответ какими свойствами обладает или не обладает отношение и почему:   
\begin{enumerate} [a)]\setcounter{enumi}{0}
\item рефлексивность
\item антирефлексивность
\item симметричность
\item асимметричность
\item антисимметричность
\item транзитивность
\end{enumerate}

Обоснуйте свой ответ по каждому из приведенных ниже вопросов:
\begin{enumerate} [a)]\setcounter{enumi}{0}
    \item Является ли это отношение отношением эквивалентности?
    \item Является ли это отношение функциональным?
    \item Каким из отношений соответствия (одно-многозначным, много-многозначный и т.д.) оно является?
    \item К каким из отношений порядка (полного, частичного и т.д.) можно отнести данное отношение?
\end{enumerate}


\question
Установите, является ли каждое из перечисленных ниже отношений на А ($R \subseteq A \times A$) отношением эквивалентности (обоснование ответа обязательно). Для каждого отношения эквивалентности постройте классы 
эквивалентности и постройте граф отношения:
\begin{enumerate} [a)]\setcounter{enumi}{0}
\item Пусть A – множество имен. $A = \{ $Алексей, Иван, Петр, Александр, Павел, Андрей$ \}$. Тогда отношение $R$ верно на парах имен, начинающихся с одной и той же буквы, и только на них.
\item $A = \{-10, -9, … , 9, 10\}$ и отношение $ R = \{(a,b)|a^{2} = b^{2}\}$
\item На множестве $A = \{1; 2; 3\}$ задано отношение $R = \{(1; 1); (2; 2); (3; 3); (3; 2); (1; 2); (2; 1)\}$
\end{enumerate}\question Составьте полную таблицу истинности, определите, какие переменные являются фиктивными и проверьте, является ли формула тавтологией:
$ P \rightarrow (Q \rightarrow ((P \lor Q) \rightarrow (P \land Q)))$

\end{questions}
\newpage
%%% begin test
\begin{flushright}
\begin{tabular}{p{2.8in} r l}
%\textbf{\class} & \textbf{ФИО:} & \makebox[2.5in]{\hrulefill}\\
\textbf{\class} & \textbf{ФИО:} &Кононенко Филипп Алексеевич
\\

\textbf{\examdate} &&\\
%\textbf{Time Limit: \timelimit} & Teaching Assistant & \makebox[2in]{\hrulefill}
\end{tabular}\\
\end{flushright}
\rule[1ex]{\textwidth}{.1pt}


\begin{questions}
\question
Найдите и упростите P:
\begin{equation*}
\overline{P} = A \cap \overline{B} \cup \overline{B} \cap C \cup \overline{A} \cap \overline{B} \cup \overline{A} \cap C
\end{equation*}
Затем найдите элементы множества P, выраженного через множества:
\begin{equation*}
A = \{0, 3, 4, 9\}; 
B = \{1, 3, 4, 7\};
C = \{0, 1, 2, 4, 7, 8, 9\};
I = \{0, 1, 2, 3, 4, 5, 6, 7, 8, 9\}.
\end{equation*}\question
Упростите следующее выражение с учетом того, что $A\subset B \subset C \subset D \subset U; A \neq \O$
\begin{equation*}
A \cap B  \cap \overline{C} \cup \overline{C} \cap D \cup B \cap C \cap D
\end{equation*}

Примечание: U — универсум\question
Дано отношение на множестве $\{1, 2, 3, 4, 5\}$ 
\begin{equation*}
aRb \iff (a+b) \bmod 2 =0
\end{equation*}
Напишите обоснованный ответ какими свойствами обладает или не обладает отношение и почему:   
\begin{enumerate} [a)]\setcounter{enumi}{0}
\item рефлексивность
\item антирефлексивность
\item симметричность
\item асимметричность
\item антисимметричность
\item транзитивность
\end{enumerate}

Обоснуйте свой ответ по каждому из приведенных ниже вопросов:
\begin{enumerate} [a)]\setcounter{enumi}{0}
    \item Является ли это отношение отношением эквивалентности?
    \item Является ли это отношение функциональным?
    \item Каким из отношений соответствия (одно-многозначным, много-многозначный и т.д.) оно является?
    \item К каким из отношений порядка (полного, частичного и т.д.) можно отнести данное отношение?
\end{enumerate}



\question
Установите, является ли каждое из перечисленных ниже отношений на А ($R \subseteq A \times A$) отношением эквивалентности (обоснование ответа обязательно). Для каждого отношения эквивалентности постройте классы 
эквивалентности и постройте граф отношения:
\begin{enumerate} [a)]\setcounter{enumi}{0}
\item На множестве $A = \{1; 2; 3\}$ задано отношение $R = \{(1; 1); (2; 2); (3; 3); (2; 1); (1; 2); (2; 3); (3; 2); (3; 1); (1; 3)\}$
\item На множестве $A = \{1; 2; 3; 4; 5\}$ задано отношение $R = \{(1; 2); (1; 3); (1; 5); (2; 3); (2; 4); (2; 5); (3; 4); (3; 5); (4; 5)\}$
\item А - множество целых чисел и отношение $R = \{(a,b)|a + b = 0\}$
\end{enumerate}\question Составьте полную таблицу истинности, определите, какие переменные являются фиктивными и проверьте, является ли формула тавтологией:

$(P \rightarrow (Q \land R)) \leftrightarrow ((P \rightarrow Q) \land (P \rightarrow R))$

\end{questions}
\newpage
%%% begin test
\begin{flushright}
\begin{tabular}{p{2.8in} r l}
%\textbf{\class} & \textbf{ФИО:} & \makebox[2.5in]{\hrulefill}\\
\textbf{\class} & \textbf{ФИО:} &Кононова Юлия Александровна
\\

\textbf{\examdate} &&\\
%\textbf{Time Limit: \timelimit} & Teaching Assistant & \makebox[2in]{\hrulefill}
\end{tabular}\\
\end{flushright}
\rule[1ex]{\textwidth}{.1pt}


\begin{questions}
\question
Найдите и упростите P:
\begin{equation*}
\overline{P} = A \cap \overline{B} \cup \overline{B} \cap C \cup \overline{A} \cap \overline{B} \cup \overline{A} \cap C
\end{equation*}
Затем найдите элементы множества P, выраженного через множества:
\begin{equation*}
A = \{0, 3, 4, 9\}; 
B = \{1, 3, 4, 7\};
C = \{0, 1, 2, 4, 7, 8, 9\};
I = \{0, 1, 2, 3, 4, 5, 6, 7, 8, 9\}.
\end{equation*}\question
Упростите следующее выражение с учетом того, что $A\subset B \subset C \subset D \subset U; A \neq \O$
\begin{equation*}
\overline{A} \cap \overline{C} \cap D \cup \overline{B} \cap \overline{C} \cap D \cup A \cap B
\end{equation*}

Примечание: U — универсум\question
Дано отношение на множестве $\{1, 2, 3, 4, 5\}$ 
\begin{equation*}
aRb \iff a \leq b
\end{equation*}
Напишите обоснованный ответ какими свойствами обладает или не обладает отношение и почему:   
\begin{enumerate} [a)]\setcounter{enumi}{0}
\item рефлексивность
\item антирефлексивность
\item симметричность
\item асимметричность
\item антисимметричность
\item транзитивность
\end{enumerate}

Обоснуйте свой ответ по каждому из приведенных ниже вопросов:
\begin{enumerate} [a)]\setcounter{enumi}{0}
    \item Является ли это отношение отношением эквивалентности?
    \item Является ли это отношение функциональным?
    \item Каким из отношений соответствия (одно-многозначным, много-многозначный и т.д.) оно является?
    \item К каким из отношений порядка (полного, частичного и т.д.) можно отнести данное отношение?
\end{enumerate}


\question
Установите, является ли каждое из перечисленных ниже отношений на А ($R \subseteq A \times A$) отношением эквивалентности (обоснование ответа обязательно). Для каждого отношения эквивалентности постройте классы эквивалентности и постройте граф отношения:
\begin{enumerate} [a)]\setcounter{enumi}{0}
\item $F(x)=x^{2}+1$, где $x \in A = [-2, 4]$ и отношение $R = \{(a,b)|F(a) = F(b)\}$
\item А - множество целых чисел и отношение $R = \{(a,b)|a + b = 5\}$
\item На множестве $A = \{1; 2; 3\}$ задано отношение $R = \{(1; 1); (2; 2); (3; 3); (3; 2); (1; 2); (2; 1)\}$

\end{enumerate}\question Составьте полную таблицу истинности, определите, какие переменные являются фиктивными и проверьте, является ли формула тавтологией:
$(( P \land \neg Q) \rightarrow (R \land \neg R)) \rightarrow (P \rightarrow Q)$

\end{questions}
\newpage
%%% begin test
\begin{flushright}
\begin{tabular}{p{2.8in} r l}
%\textbf{\class} & \textbf{ФИО:} & \makebox[2.5in]{\hrulefill}\\
\textbf{\class} & \textbf{ФИО:} &Лаза Микаэль Феллис Захр
\\

\textbf{\examdate} &&\\
%\textbf{Time Limit: \timelimit} & Teaching Assistant & \makebox[2in]{\hrulefill}
\end{tabular}\\
\end{flushright}
\rule[1ex]{\textwidth}{.1pt}


\begin{questions}
\question
Найдите и упростите P:
\begin{equation*}
\overline{P} = A \cap \overline{B} \cup \overline{B} \cap C \cup \overline{A} \cap \overline{B} \cup \overline{A} \cap C
\end{equation*}
Затем найдите элементы множества P, выраженного через множества:
\begin{equation*}
A = \{0, 3, 4, 9\}; 
B = \{1, 3, 4, 7\};
C = \{0, 1, 2, 4, 7, 8, 9\};
I = \{0, 1, 2, 3, 4, 5, 6, 7, 8, 9\}.
\end{equation*}\question
Упростите следующее выражение с учетом того, что $A\subset B \subset C \subset D \subset U; A \neq \O$
\begin{equation*}
A \cap C  \cap D \cup B \cap \overline{C} \cap D \cup B \cap C \cap D
\end{equation*}

Примечание: U — универсум\question
Для следующего отношения на множестве $\{1, 2, 3, 4, 5\}$ 
\begin{equation*}
aRb \iff 0 < a-b<2
\end{equation*}
Напишите обоснованный ответ какими свойствами обладает или не обладает отношение и почему:   
\begin{enumerate} [a)]\setcounter{enumi}{0}
\item рефлексивность
\item антирефлексивность
\item симметричность
\item асимметричность
\item антисимметричность
\item транзитивность
\end{enumerate}

Обоснуйте свой ответ по каждому из приведенных ниже вопросов:
\begin{enumerate} [a)]\setcounter{enumi}{0}
    \item Является ли это отношение отношением эквивалентности?
    \item Является ли это отношение функциональным?
    \item Каким из отношений соответствия (одно-многозначным, много-многозначный и т.д.) оно является?
    \item К каким из отношений порядка (полного, частичного и т.д.) можно отнести данное отношение?
\end{enumerate}
\question
Установите, является ли каждое из перечисленных ниже отношений на А ($R \subseteq A \times A$) отношением эквивалентности (обоснование ответа обязательно). Для каждого отношения эквивалентности постройте классы 
эквивалентности и постройте граф отношения:
\begin{enumerate} [a)]\setcounter{enumi}{0}
\item $A = \{-10, -9, … , 9, 10\}$ и отношение $R = \{(a,b)|a^{2} = b^{2}\}$
\item $A = \{a, b, c, d, p, t\}$ задано отношение $R = \{(a, a), (b, b), (b, c), (b, d), (c, b), (c, c), (c, d), (d, b), (d, c), (d, d), (p,p), (t,t)\}$
\item Пусть A – множество имен. $A = \{ $Алексей, Иван, Петр, Александр, Павел, Андрей$ \}$. Тогда отношение $R$ верно на парах имен, начинающихся с одной и той же буквы, и только на них.
\end{enumerate}\question Составьте полную таблицу истинности, определите, какие переменные являются фиктивными и проверьте, является ли формула тавтологией:
$(( P \land \neg Q) \rightarrow (R \land \neg R)) \rightarrow (P \rightarrow Q)$

\end{questions}
\newpage
%%% begin test
\begin{flushright}
\begin{tabular}{p{2.8in} r l}
%\textbf{\class} & \textbf{ФИО:} & \makebox[2.5in]{\hrulefill}\\
\textbf{\class} & \textbf{ФИО:} &Ле Ба Киен
\\

\textbf{\examdate} &&\\
%\textbf{Time Limit: \timelimit} & Teaching Assistant & \makebox[2in]{\hrulefill}
\end{tabular}\\
\end{flushright}
\rule[1ex]{\textwidth}{.1pt}


\begin{questions}
\question
Найдите и упростите P:
\begin{equation*}
\overline{P} = \overline{A} \cap B \cup \overline{A} \cap C \cup A \cap \overline{B} \cup \overline{B} \cap C
\end{equation*}
Затем найдите элементы множества P, выраженного через множества:
\begin{equation*}
A = \{0, 3, 4, 9\}; 
B = \{1, 3, 4, 7\};
C = \{0, 1, 2, 4, 7, 8, 9\};
I = \{0, 1, 2, 3, 4, 5, 6, 7, 8, 9\}.
\end{equation*}\question
Упростите следующее выражение с учетом того, что $A\subset B \subset C \subset D \subset U; A \neq \O$
\begin{equation*}
A \cap  \overline{C} \cup B \cap \overline{D} \cup  \overline{A} \cap C \cap  \overline{D}
\end{equation*}

Примечание: U — универсум\question
Дано отношение на множестве $\{1, 2, 3, 4, 5\}$ 
\begin{equation*}
aRb \iff b > a
\end{equation*}
Напишите обоснованный ответ какими свойствами обладает или не обладает отношение и почему:   
\begin{enumerate} [a)]\setcounter{enumi}{0}
\item рефлексивность
\item антирефлексивность
\item симметричность
\item асимметричность
\item антисимметричность
\item транзитивность
\end{enumerate}

Обоснуйте свой ответ по каждому из приведенных ниже вопросов:
\begin{enumerate} [a)]\setcounter{enumi}{0}
    \item Является ли это отношение отношением эквивалентности?
    \item Является ли это отношение функциональным?
    \item Каким из отношений соответствия (одно-многозначным, много-многозначный и т.д.) оно является?
    \item К каким из отношений порядка (полного, частичного и т.д.) можно отнести данное отношение?
\end{enumerate}

\question
Установите, является ли каждое из перечисленных ниже отношений на А ($R \subseteq A \times A$) отношением эквивалентности (обоснование ответа обязательно). Для каждого отношения эквивалентности 
постройте классы эквивалентности и постройте граф отношения:
\begin{enumerate}[a)]\setcounter{enumi}{0}
\item А - множество целых чисел и отношение $R = \{(a,b)|a + b = 0\}$
\item $A = \{-10, -9, …, 9, 10\}$ и отношение $R = \{(a,b)|a^{3} = b^{3}\}$
\item На множестве $A = \{1; 2; 3\}$ задано отношение $R = \{(1; 1); (2; 2); (3; 3); (2; 1); (1; 2); (2; 3); (3; 2); (3; 1); (1; 3)\}$

\end{enumerate}\question Составьте полную таблицу истинности, определите, какие переменные являются фиктивными и проверьте, является ли формула тавтологией:
$(P \rightarrow (Q \rightarrow R)) \rightarrow ((P \rightarrow Q) \rightarrow (P \rightarrow R))$

\end{questions}
\newpage
%%% begin test
\begin{flushright}
\begin{tabular}{p{2.8in} r l}
%\textbf{\class} & \textbf{ФИО:} & \makebox[2.5in]{\hrulefill}\\
\textbf{\class} & \textbf{ФИО:} &Одиноченко Алексей Дмитриевич
\\

\textbf{\examdate} &&\\
%\textbf{Time Limit: \timelimit} & Teaching Assistant & \makebox[2in]{\hrulefill}
\end{tabular}\\
\end{flushright}
\rule[1ex]{\textwidth}{.1pt}


\begin{questions}
\question
Найдите и упростите P:
\begin{equation*}
\overline{P} = \overline{A} \cap B \cup \overline{A} \cap C \cup A \cap \overline{B} \cup \overline{B} \cap C
\end{equation*}
Затем найдите элементы множества P, выраженного через множества:
\begin{equation*}
A = \{0, 3, 4, 9\}; 
B = \{1, 3, 4, 7\};
C = \{0, 1, 2, 4, 7, 8, 9\};
I = \{0, 1, 2, 3, 4, 5, 6, 7, 8, 9\}.
\end{equation*}\question
Упростите следующее выражение с учетом того, что $A\subset B \subset C \subset D \subset U; A \neq \O$
\begin{equation*}
\overline{A} \cap \overline{B} \cup B \cap \overline{C} \cup \overline{C} \cap D
\end{equation*}

Примечание: U — универсум\question
Дано отношение на множестве $\{1, 2, 3, 4, 5\}$ 
\begin{equation*}
aRb \iff a \geq b^2
\end{equation*}
Напишите обоснованный ответ какими свойствами обладает или не обладает отношение и почему:   
\begin{enumerate} [a)]\setcounter{enumi}{0}
\item рефлексивность
\item антирефлексивность
\item симметричность
\item асимметричность
\item антисимметричность
\item транзитивность
\end{enumerate}

Обоснуйте свой ответ по каждому из приведенных ниже вопросов:
\begin{enumerate} [a)]\setcounter{enumi}{0}
    \item Является ли это отношение отношением эквивалентности?
    \item Является ли это отношение функциональным?
    \item Каким из отношений соответствия (одно-многозначным, много-многозначный и т.д.) оно является?
    \item К каким из отношений порядка (полного, частичного и т.д.) можно отнести данное отношение?
\end{enumerate}


\question
Установите, является ли каждое из перечисленных ниже отношений на А ($R \subseteq A \times A$) отношением эквивалентности (обоснование ответа обязательно). Для каждого отношения эквивалентности постройте классы 
эквивалентности и постройте граф отношения:
\begin{enumerate} [a)]\setcounter{enumi}{0}
\item А - множество целых чисел и отношение $R = \{(a,b)|a + b = 5\}$
\item Пусть A – множество имен. $A = \{ $Алексей, Иван, Петр, Александр, Павел, Андрей$ \}$. Тогда отношение $R $ верно на парах имен, начинающихся с одной и той же буквы, и только на них.
\item На множестве $A = \{1; 2; 3; 4; 5\}$ задано отношение $R = \{(1; 2); (1; 3); (1; 5); (2; 3); (2; 4); (2; 5); (3; 4); (3; 5); (4; 5)\}$
\end{enumerate}\question Составьте полную таблицу истинности, определите, какие переменные являются фиктивными и проверьте, является ли формула тавтологией:
$(( P \rightarrow Q) \land (Q \rightarrow P)) \rightarrow (P \rightarrow R)$

\end{questions}
\newpage
%%% begin test
\begin{flushright}
\begin{tabular}{p{2.8in} r l}
%\textbf{\class} & \textbf{ФИО:} & \makebox[2.5in]{\hrulefill}\\
\textbf{\class} & \textbf{ФИО:} &Пчелкин Алексей Юрьевич
\\

\textbf{\examdate} &&\\
%\textbf{Time Limit: \timelimit} & Teaching Assistant & \makebox[2in]{\hrulefill}
\end{tabular}\\
\end{flushright}
\rule[1ex]{\textwidth}{.1pt}


\begin{questions}
\question
Найдите и упростите P:
\begin{equation*}
\overline{P} = A \cap \overline{C} \cup A \cap \overline{B} \cup B \cap \overline{C} \cup A \cap C
\end{equation*}
Затем найдите элементы множества P, выраженного через множества:
\begin{equation*}
A = \{0, 3, 4, 9\}; 
B = \{1, 3, 4, 7\};
C = \{0, 1, 2, 4, 7, 8, 9\};
I = \{0, 1, 2, 3, 4, 5, 6, 7, 8, 9\}.
\end{equation*}\question
Упростите следующее выражение с учетом того, что $A\subset B \subset C \subset D \subset U; A \neq \O$
\begin{equation*}
A \cap  \overline{C} \cup B \cap \overline{D} \cup  \overline{A} \cap C \cap  \overline{D}
\end{equation*}

Примечание: U — универсум\question
Для следующего отношения на множестве $\{1, 2, 3, 4, 5\}$ 
\begin{equation*}
aRb \iff 0 < a-b<2
\end{equation*}
Напишите обоснованный ответ какими свойствами обладает или не обладает отношение и почему:   
\begin{enumerate} [a)]\setcounter{enumi}{0}
\item рефлексивность
\item антирефлексивность
\item симметричность
\item асимметричность
\item антисимметричность
\item транзитивность
\end{enumerate}

Обоснуйте свой ответ по каждому из приведенных ниже вопросов:
\begin{enumerate} [a)]\setcounter{enumi}{0}
    \item Является ли это отношение отношением эквивалентности?
    \item Является ли это отношение функциональным?
    \item Каким из отношений соответствия (одно-многозначным, много-многозначный и т.д.) оно является?
    \item К каким из отношений порядка (полного, частичного и т.д.) можно отнести данное отношение?
\end{enumerate}
\question
Установите, является ли каждое из перечисленных ниже отношений на А ($R \subseteq A \times A$) отношением эквивалентности (обоснование ответа обязательно). Для каждого отношения эквивалентности 
постройте классы эквивалентности и постройте граф отношения:
\begin{enumerate}[a)]\setcounter{enumi}{0}
\item А - множество целых чисел и отношение $R = \{(a,b)|a + b = 0\}$
\item $A = \{-10, -9, …, 9, 10\}$ и отношение $R = \{(a,b)|a^{3} = b^{3}\}$
\item На множестве $A = \{1; 2; 3\}$ задано отношение $R = \{(1; 1); (2; 2); (3; 3); (2; 1); (1; 2); (2; 3); (3; 2); (3; 1); (1; 3)\}$

\end{enumerate}\question Составьте полную таблицу истинности, определите, какие переменные являются фиктивными и проверьте, является ли формула тавтологией:

$(P \rightarrow (Q \land R)) \leftrightarrow ((P \rightarrow Q) \land (P \rightarrow R))$

\end{questions}
\newpage
%%% begin test
\begin{flushright}
\begin{tabular}{p{2.8in} r l}
%\textbf{\class} & \textbf{ФИО:} & \makebox[2.5in]{\hrulefill}\\
\textbf{\class} & \textbf{ФИО:} &Сандовал Торрес Пабло Андрес
\\

\textbf{\examdate} &&\\
%\textbf{Time Limit: \timelimit} & Teaching Assistant & \makebox[2in]{\hrulefill}
\end{tabular}\\
\end{flushright}
\rule[1ex]{\textwidth}{.1pt}


\begin{questions}
\question
Найдите и упростите P:
\begin{equation*}
\overline{P} = A \cap \overline{B} \cup A \cap C \cup B \cap C \cup \overline{A} \cap C
\end{equation*}
Затем найдите элементы множества P, выраженного через множества:
\begin{equation*}
A = \{0, 3, 4, 9\}; 
B = \{1, 3, 4, 7\};
C = \{0, 1, 2, 4, 7, 8, 9\};
I = \{0, 1, 2, 3, 4, 5, 6, 7, 8, 9\}.
\end{equation*}\question
Упростите следующее выражение с учетом того, что $A\subset B \subset C \subset D \subset U; A \neq \O$
\begin{equation*}
\overline{A} \cap \overline{B} \cup B \cap \overline{C} \cup \overline{C} \cap D
\end{equation*}

Примечание: U — универсум\question
Дано отношение на множестве $\{1, 2, 3, 4, 5\}$ 
\begin{equation*}
aRb \iff b > a
\end{equation*}
Напишите обоснованный ответ какими свойствами обладает или не обладает отношение и почему:   
\begin{enumerate} [a)]\setcounter{enumi}{0}
\item рефлексивность
\item антирефлексивность
\item симметричность
\item асимметричность
\item антисимметричность
\item транзитивность
\end{enumerate}

Обоснуйте свой ответ по каждому из приведенных ниже вопросов:
\begin{enumerate} [a)]\setcounter{enumi}{0}
    \item Является ли это отношение отношением эквивалентности?
    \item Является ли это отношение функциональным?
    \item Каким из отношений соответствия (одно-многозначным, много-многозначный и т.д.) оно является?
    \item К каким из отношений порядка (полного, частичного и т.д.) можно отнести данное отношение?
\end{enumerate}

\question
Установите, является ли каждое из перечисленных ниже отношений на А ($R \subseteq A \times A$) отношением эквивалентности (обоснование ответа обязательно). Для каждого отношения эквивалентности 
постройте классы эквивалентности и постройте граф отношения:
\begin{enumerate}[a)]\setcounter{enumi}{0}
\item А - множество целых чисел и отношение $R = \{(a,b)|a + b = 0\}$
\item $A = \{-10, -9, …, 9, 10\}$ и отношение $R = \{(a,b)|a^{3} = b^{3}\}$
\item На множестве $A = \{1; 2; 3\}$ задано отношение $R = \{(1; 1); (2; 2); (3; 3); (2; 1); (1; 2); (2; 3); (3; 2); (3; 1); (1; 3)\}$

\end{enumerate}\question Составьте полную таблицу истинности, определите, какие переменные являются фиктивными и проверьте, является ли формула тавтологией:
$(( P \rightarrow Q) \land (Q \rightarrow P)) \rightarrow (P \rightarrow R)$

\end{questions}
\newpage
%%% begin test
\begin{flushright}
\begin{tabular}{p{2.8in} r l}
%\textbf{\class} & \textbf{ФИО:} & \makebox[2.5in]{\hrulefill}\\
\textbf{\class} & \textbf{ФИО:} &Смолин Тимур Дмитриевич
\\

\textbf{\examdate} &&\\
%\textbf{Time Limit: \timelimit} & Teaching Assistant & \makebox[2in]{\hrulefill}
\end{tabular}\\
\end{flushright}
\rule[1ex]{\textwidth}{.1pt}


\begin{questions}
\question
Найдите и упростите P:
\begin{equation*}
\overline{P} = A \cap \overline{B} \cup \overline{B} \cap C \cup \overline{A} \cap \overline{B} \cup \overline{A} \cap C
\end{equation*}
Затем найдите элементы множества P, выраженного через множества:
\begin{equation*}
A = \{0, 3, 4, 9\}; 
B = \{1, 3, 4, 7\};
C = \{0, 1, 2, 4, 7, 8, 9\};
I = \{0, 1, 2, 3, 4, 5, 6, 7, 8, 9\}.
\end{equation*}\question
Упростите следующее выражение с учетом того, что $A\subset B \subset C \subset D \subset U; A \neq \O$
\begin{equation*}
\overline{A} \cap \overline{C} \cap D \cup \overline{B} \cap \overline{C} \cap D \cup A \cap B
\end{equation*}

Примечание: U — универсум\question
Дано отношение на множестве $\{1, 2, 3, 4, 5\}$ 
\begin{equation*}
aRb \iff b > a
\end{equation*}
Напишите обоснованный ответ какими свойствами обладает или не обладает отношение и почему:   
\begin{enumerate} [a)]\setcounter{enumi}{0}
\item рефлексивность
\item антирефлексивность
\item симметричность
\item асимметричность
\item антисимметричность
\item транзитивность
\end{enumerate}

Обоснуйте свой ответ по каждому из приведенных ниже вопросов:
\begin{enumerate} [a)]\setcounter{enumi}{0}
    \item Является ли это отношение отношением эквивалентности?
    \item Является ли это отношение функциональным?
    \item Каким из отношений соответствия (одно-многозначным, много-многозначный и т.д.) оно является?
    \item К каким из отношений порядка (полного, частичного и т.д.) можно отнести данное отношение?
\end{enumerate}

\question
Установите, является ли каждое из перечисленных ниже отношений на А ($R \subseteq A \times A$) отношением эквивалентности (обоснование ответа обязательно). Для каждого отношения эквивалентности постройте классы 
эквивалентности и постройте граф отношения:
\begin{enumerate} [a)]\setcounter{enumi}{0}
\item Пусть A – множество имен. $A = \{ $Алексей, Иван, Петр, Александр, Павел, Андрей$ \}$. Тогда отношение $R$ верно на парах имен, начинающихся с одной и той же буквы, и только на них.
\item $A = \{-10, -9, … , 9, 10\}$ и отношение $ R = \{(a,b)|a^{2} = b^{2}\}$
\item На множестве $A = \{1; 2; 3\}$ задано отношение $R = \{(1; 1); (2; 2); (3; 3); (3; 2); (1; 2); (2; 1)\}$
\end{enumerate}\question Составьте полную таблицу истинности, определите, какие переменные являются фиктивными и проверьте, является ли формула тавтологией:
$ P \rightarrow (Q \rightarrow ((P \lor Q) \rightarrow (P \land Q)))$

\end{questions}
\newpage
%%% begin test
\begin{flushright}
\begin{tabular}{p{2.8in} r l}
%\textbf{\class} & \textbf{ФИО:} & \makebox[2.5in]{\hrulefill}\\
\textbf{\class} & \textbf{ФИО:} &Терещенко Ярослав Вячеславович
\\

\textbf{\examdate} &&\\
%\textbf{Time Limit: \timelimit} & Teaching Assistant & \makebox[2in]{\hrulefill}
\end{tabular}\\
\end{flushright}
\rule[1ex]{\textwidth}{.1pt}


\begin{questions}
\question
Найдите и упростите P:
\begin{equation*}
\overline{P} = A \cap C \cup \overline{A} \cap \overline{C} \cup \overline{B} \cap C \cup \overline{A} \cap \overline{B}
\end{equation*}
Затем найдите элементы множества P, выраженного через множества:
\begin{equation*}
A = \{0, 3, 4, 9\}; 
B = \{1, 3, 4, 7\};
C = \{0, 1, 2, 4, 7, 8, 9\};
I = \{0, 1, 2, 3, 4, 5, 6, 7, 8, 9\}.
\end{equation*}\question
Упростите следующее выражение с учетом того, что $A\subset B \subset C \subset D \subset U; A \neq \O$
\begin{equation*}
A \cap B  \cap \overline{C} \cup \overline{C} \cap D \cup B \cap C \cap D
\end{equation*}

Примечание: U — универсум\question
Дано отношение на множестве $\{1, 2, 3, 4, 5\}$ 
\begin{equation*}
aRb \iff  \text{НОД}(a,b) =1
\end{equation*}
Напишите обоснованный ответ какими свойствами обладает или не обладает отношение и почему:   
\begin{enumerate} [a)]\setcounter{enumi}{0}
\item рефлексивность
\item антирефлексивность
\item симметричность
\item асимметричность
\item антисимметричность
\item транзитивность
\end{enumerate}

Обоснуйте свой ответ по каждому из приведенных ниже вопросов:
\begin{enumerate} [a)]\setcounter{enumi}{0}
    \item Является ли это отношение отношением эквивалентности?
    \item Является ли это отношение функциональным?
    \item Каким из отношений соответствия (одно-многозначным, много-многозначный и т.д.) оно является?
    \item К каким из отношений порядка (полного, частичного и т.д.) можно отнести данное отношение?
\end{enumerate}


\question
Установите, является ли каждое из перечисленных ниже отношений на А ($R \subseteq A \times A$) отношением эквивалентности (обоснование ответа обязательно). Для каждого отношения эквивалентности 
постройте классы эквивалентности и постройте граф отношения:
\begin{enumerate}[a)]\setcounter{enumi}{0}
\item А - множество целых чисел и отношение $R = \{(a,b)|a + b = 0\}$
\item $A = \{-10, -9, …, 9, 10\}$ и отношение $R = \{(a,b)|a^{3} = b^{3}\}$
\item На множестве $A = \{1; 2; 3\}$ задано отношение $R = \{(1; 1); (2; 2); (3; 3); (2; 1); (1; 2); (2; 3); (3; 2); (3; 1); (1; 3)\}$

\end{enumerate}\question Составьте полную таблицу истинности, определите, какие переменные являются фиктивными и проверьте, является ли формула тавтологией:
$((P \rightarrow Q) \lor R) \leftrightarrow (P \rightarrow (Q \lor R))$

\end{questions}
\newpage
%%% begin test
\begin{flushright}
\begin{tabular}{p{2.8in} r l}
%\textbf{\class} & \textbf{ФИО:} & \makebox[2.5in]{\hrulefill}\\
\textbf{\class} & \textbf{ФИО:} &Фоменко Сергей Юрьевич
\\

\textbf{\examdate} &&\\
%\textbf{Time Limit: \timelimit} & Teaching Assistant & \makebox[2in]{\hrulefill}
\end{tabular}\\
\end{flushright}
\rule[1ex]{\textwidth}{.1pt}


\begin{questions}
\question
Найдите и упростите P:
\begin{equation*}
\overline{P} = \overline{A} \cap B \cup \overline{A} \cap C \cup A \cap \overline{B} \cup \overline{B} \cap C
\end{equation*}
Затем найдите элементы множества P, выраженного через множества:
\begin{equation*}
A = \{0, 3, 4, 9\}; 
B = \{1, 3, 4, 7\};
C = \{0, 1, 2, 4, 7, 8, 9\};
I = \{0, 1, 2, 3, 4, 5, 6, 7, 8, 9\}.
\end{equation*}\question
Упростите следующее выражение с учетом того, что $A\subset B \subset C \subset D \subset U; A \neq \O$
\begin{equation*}
\overline{A} \cap \overline{B} \cup B \cap \overline{C} \cup \overline{C} \cap D
\end{equation*}

Примечание: U — универсум\question
Дано отношение на множестве $\{1, 2, 3, 4, 5\}$ 
\begin{equation*}
aRb \iff  \text{НОД}(a,b) =1
\end{equation*}
Напишите обоснованный ответ какими свойствами обладает или не обладает отношение и почему:   
\begin{enumerate} [a)]\setcounter{enumi}{0}
\item рефлексивность
\item антирефлексивность
\item симметричность
\item асимметричность
\item антисимметричность
\item транзитивность
\end{enumerate}

Обоснуйте свой ответ по каждому из приведенных ниже вопросов:
\begin{enumerate} [a)]\setcounter{enumi}{0}
    \item Является ли это отношение отношением эквивалентности?
    \item Является ли это отношение функциональным?
    \item Каким из отношений соответствия (одно-многозначным, много-многозначный и т.д.) оно является?
    \item К каким из отношений порядка (полного, частичного и т.д.) можно отнести данное отношение?
\end{enumerate}


\question
Установите, является ли каждое из перечисленных ниже отношений на А ($R \subseteq A \times A$) отношением эквивалентности (обоснование ответа обязательно). Для каждого отношения эквивалентности постройте классы 
эквивалентности и постройте граф отношения:
\begin{enumerate} [a)]\setcounter{enumi}{0}
\item На множестве $A = \{1; 2; 3\}$ задано отношение $R = \{(1; 1); (2; 2); (3; 3); (2; 1); (1; 2); (2; 3); (3; 2); (3; 1); (1; 3)\}$
\item На множестве $A = \{1; 2; 3; 4; 5\}$ задано отношение $R = \{(1; 2); (1; 3); (1; 5); (2; 3); (2; 4); (2; 5); (3; 4); (3; 5); (4; 5)\}$
\item А - множество целых чисел и отношение $R = \{(a,b)|a + b = 0\}$
\end{enumerate}\question Составьте полную таблицу истинности, определите, какие переменные являются фиктивными и проверьте, является ли формула тавтологией:
$((P \rightarrow Q) \lor R) \leftrightarrow (P \rightarrow (Q \lor R))$

\end{questions}
\newpage
%%% begin test
\begin{flushright}
\begin{tabular}{p{2.8in} r l}
%\textbf{\class} & \textbf{ФИО:} & \makebox[2.5in]{\hrulefill}\\
\textbf{\class} & \textbf{ФИО:} &Ханджян Ованес Ованесович
\\

\textbf{\examdate} &&\\
%\textbf{Time Limit: \timelimit} & Teaching Assistant & \makebox[2in]{\hrulefill}
\end{tabular}\\
\end{flushright}
\rule[1ex]{\textwidth}{.1pt}


\begin{questions}
\question
Найдите и упростите P:
\begin{equation*}
\overline{P} = A \cap \overline{C} \cup A \cap \overline{B} \cup B \cap \overline{C} \cup A \cap C
\end{equation*}
Затем найдите элементы множества P, выраженного через множества:
\begin{equation*}
A = \{0, 3, 4, 9\}; 
B = \{1, 3, 4, 7\};
C = \{0, 1, 2, 4, 7, 8, 9\};
I = \{0, 1, 2, 3, 4, 5, 6, 7, 8, 9\}.
\end{equation*}\question
Упростите следующее выражение с учетом того, что $A\subset B \subset C \subset D \subset U; A \neq \O$
\begin{equation*}
A \cap  \overline{C} \cup B \cap \overline{D} \cup  \overline{A} \cap C \cap  \overline{D}
\end{equation*}

Примечание: U — универсум\question
Дано отношение на множестве $\{1, 2, 3, 4, 5\}$ 
\begin{equation*}
aRb \iff b > a
\end{equation*}
Напишите обоснованный ответ какими свойствами обладает или не обладает отношение и почему:   
\begin{enumerate} [a)]\setcounter{enumi}{0}
\item рефлексивность
\item антирефлексивность
\item симметричность
\item асимметричность
\item антисимметричность
\item транзитивность
\end{enumerate}

Обоснуйте свой ответ по каждому из приведенных ниже вопросов:
\begin{enumerate} [a)]\setcounter{enumi}{0}
    \item Является ли это отношение отношением эквивалентности?
    \item Является ли это отношение функциональным?
    \item Каким из отношений соответствия (одно-многозначным, много-многозначный и т.д.) оно является?
    \item К каким из отношений порядка (полного, частичного и т.д.) можно отнести данное отношение?
\end{enumerate}

\question
Установите, является ли каждое из перечисленных ниже отношений на А ($R \subseteq A \times A$) отношением эквивалентности (обоснование ответа обязательно). Для каждого отношения эквивалентности постройте классы 
эквивалентности и постройте граф отношения:
\begin{enumerate} [a)]\setcounter{enumi}{0}
\item А - множество целых чисел и отношение $R = \{(a,b)|a + b = 5\}$
\item Пусть A – множество имен. $A = \{ $Алексей, Иван, Петр, Александр, Павел, Андрей$ \}$. Тогда отношение $R $ верно на парах имен, начинающихся с одной и той же буквы, и только на них.
\item На множестве $A = \{1; 2; 3; 4; 5\}$ задано отношение $R = \{(1; 2); (1; 3); (1; 5); (2; 3); (2; 4); (2; 5); (3; 4); (3; 5); (4; 5)\}$
\end{enumerate}\question Составьте полную таблицу истинности, определите, какие переменные являются фиктивными и проверьте, является ли формула тавтологией:
$(P \rightarrow (Q \rightarrow R)) \rightarrow ((P \rightarrow Q) \rightarrow (P \rightarrow R))$

\end{questions}
\newpage
%%% begin test
\begin{flushright}
\begin{tabular}{p{2.8in} r l}
%\textbf{\class} & \textbf{ФИО:} & \makebox[2.5in]{\hrulefill}\\
\textbf{\class} & \textbf{ФИО:} &Цэдашиев Амар Зориктоевич
\\

\textbf{\examdate} &&\\
%\textbf{Time Limit: \timelimit} & Teaching Assistant & \makebox[2in]{\hrulefill}
\end{tabular}\\
\end{flushright}
\rule[1ex]{\textwidth}{.1pt}


\begin{questions}
\question
Найдите и упростите P:
\begin{equation*}
\overline{P} = A \cap \overline{B} \cup \overline{B} \cap C \cup \overline{A} \cap \overline{B} \cup \overline{A} \cap C
\end{equation*}
Затем найдите элементы множества P, выраженного через множества:
\begin{equation*}
A = \{0, 3, 4, 9\}; 
B = \{1, 3, 4, 7\};
C = \{0, 1, 2, 4, 7, 8, 9\};
I = \{0, 1, 2, 3, 4, 5, 6, 7, 8, 9\}.
\end{equation*}\question
Упростите следующее выражение с учетом того, что $A\subset B \subset C \subset D \subset U; A \neq \O$
\begin{equation*}
\overline{A} \cap \overline{B} \cup B \cap \overline{C} \cup \overline{C} \cap D
\end{equation*}

Примечание: U — универсум\question
Дано отношение на множестве $\{1, 2, 3, 4, 5\}$ 
\begin{equation*}
aRb \iff b > a
\end{equation*}
Напишите обоснованный ответ какими свойствами обладает или не обладает отношение и почему:   
\begin{enumerate} [a)]\setcounter{enumi}{0}
\item рефлексивность
\item антирефлексивность
\item симметричность
\item асимметричность
\item антисимметричность
\item транзитивность
\end{enumerate}

Обоснуйте свой ответ по каждому из приведенных ниже вопросов:
\begin{enumerate} [a)]\setcounter{enumi}{0}
    \item Является ли это отношение отношением эквивалентности?
    \item Является ли это отношение функциональным?
    \item Каким из отношений соответствия (одно-многозначным, много-многозначный и т.д.) оно является?
    \item К каким из отношений порядка (полного, частичного и т.д.) можно отнести данное отношение?
\end{enumerate}

\question
Установите, является ли каждое из перечисленных ниже отношений на А ($R \subseteq A \times A$) отношением эквивалентности (обоснование ответа обязательно). Для каждого отношения эквивалентности постройте классы 
эквивалентности и постройте граф отношения:
\begin{enumerate} [a)]\setcounter{enumi}{0}
\item На множестве $A = \{1; 2; 3\}$ задано отношение $R = \{(1; 1); (2; 2); (3; 3); (2; 1); (1; 2); (2; 3); (3; 2); (3; 1); (1; 3)\}$
\item На множестве $A = \{1; 2; 3; 4; 5\}$ задано отношение $R = \{(1; 2); (1; 3); (1; 5); (2; 3); (2; 4); (2; 5); (3; 4); (3; 5); (4; 5)\}$
\item А - множество целых чисел и отношение $R = \{(a,b)|a + b = 0\}$
\end{enumerate}\question Составьте полную таблицу истинности, определите, какие переменные являются фиктивными и проверьте, является ли формула тавтологией:
$(P \rightarrow (Q \rightarrow R)) \rightarrow ((P \rightarrow Q) \rightarrow (P \rightarrow R))$

\end{questions}
\newpage
%%% begin test
\begin{flushright}
\begin{tabular}{p{2.8in} r l}
%\textbf{\class} & \textbf{ФИО:} & \makebox[2.5in]{\hrulefill}\\
\textbf{\class} & \textbf{ФИО:} &Чан Хоанг Нам
\\

\textbf{\examdate} &&\\
%\textbf{Time Limit: \timelimit} & Teaching Assistant & \makebox[2in]{\hrulefill}
\end{tabular}\\
\end{flushright}
\rule[1ex]{\textwidth}{.1pt}


\begin{questions}
\question
Найдите и упростите P:
\begin{equation*}
\overline{P} = A \cap \overline{B} \cup \overline{B} \cap C \cup \overline{A} \cap \overline{B} \cup \overline{A} \cap C
\end{equation*}
Затем найдите элементы множества P, выраженного через множества:
\begin{equation*}
A = \{0, 3, 4, 9\}; 
B = \{1, 3, 4, 7\};
C = \{0, 1, 2, 4, 7, 8, 9\};
I = \{0, 1, 2, 3, 4, 5, 6, 7, 8, 9\}.
\end{equation*}\question
Упростите следующее выражение с учетом того, что $A\subset B \subset C \subset D \subset U; A \neq \O$
\begin{equation*}
A \cap  \overline{C} \cup B \cap \overline{D} \cup  \overline{A} \cap C \cap  \overline{D}
\end{equation*}

Примечание: U — универсум\question
Дано отношение на множестве $\{1, 2, 3, 4, 5\}$ 
\begin{equation*}
aRb \iff a \leq b
\end{equation*}
Напишите обоснованный ответ какими свойствами обладает или не обладает отношение и почему:   
\begin{enumerate} [a)]\setcounter{enumi}{0}
\item рефлексивность
\item антирефлексивность
\item симметричность
\item асимметричность
\item антисимметричность
\item транзитивность
\end{enumerate}

Обоснуйте свой ответ по каждому из приведенных ниже вопросов:
\begin{enumerate} [a)]\setcounter{enumi}{0}
    \item Является ли это отношение отношением эквивалентности?
    \item Является ли это отношение функциональным?
    \item Каким из отношений соответствия (одно-многозначным, много-многозначный и т.д.) оно является?
    \item К каким из отношений порядка (полного, частичного и т.д.) можно отнести данное отношение?
\end{enumerate}


\question
Установите, является ли каждое из перечисленных ниже отношений на А ($R \subseteq A \times A$) отношением эквивалентности (обоснование ответа обязательно). Для каждого отношения эквивалентности постройте классы 
эквивалентности и постройте граф отношения:
\begin{enumerate} [a)]\setcounter{enumi}{0}
\item А - множество целых чисел и отношение $R = \{(a,b)|a + b = 5\}$
\item Пусть A – множество имен. $A = \{ $Алексей, Иван, Петр, Александр, Павел, Андрей$ \}$. Тогда отношение $R $ верно на парах имен, начинающихся с одной и той же буквы, и только на них.
\item На множестве $A = \{1; 2; 3; 4; 5\}$ задано отношение $R = \{(1; 2); (1; 3); (1; 5); (2; 3); (2; 4); (2; 5); (3; 4); (3; 5); (4; 5)\}$
\end{enumerate}\question Составьте полную таблицу истинности, определите, какие переменные являются фиктивными и проверьте, является ли формула тавтологией:
$(P \rightarrow (Q \rightarrow R)) \rightarrow ((P \rightarrow Q) \rightarrow (P \rightarrow R))$

\end{questions}
\newpage
%%% begin test
\begin{flushright}
\begin{tabular}{p{2.8in} r l}
%\textbf{\class} & \textbf{ФИО:} & \makebox[2.5in]{\hrulefill}\\
\textbf{\class} & \textbf{ФИО:} &Шевченко Александр Вадимович
\\

\textbf{\examdate} &&\\
%\textbf{Time Limit: \timelimit} & Teaching Assistant & \makebox[2in]{\hrulefill}
\end{tabular}\\
\end{flushright}
\rule[1ex]{\textwidth}{.1pt}


\begin{questions}
\question
Найдите и упростите P:
\begin{equation*}
\overline{P} = A \cap \overline{B} \cup \overline{B} \cap C \cup \overline{A} \cap \overline{B} \cup \overline{A} \cap C
\end{equation*}
Затем найдите элементы множества P, выраженного через множества:
\begin{equation*}
A = \{0, 3, 4, 9\}; 
B = \{1, 3, 4, 7\};
C = \{0, 1, 2, 4, 7, 8, 9\};
I = \{0, 1, 2, 3, 4, 5, 6, 7, 8, 9\}.
\end{equation*}\question
Упростите следующее выражение с учетом того, что $A\subset B \subset C \subset D \subset U; A \neq \O$
\begin{equation*}
\overline{A} \cap \overline{B} \cup B \cap \overline{C} \cup \overline{C} \cap D
\end{equation*}

Примечание: U — универсум\question
Дано отношение на множестве $\{1, 2, 3, 4, 5\}$ 
\begin{equation*}
aRb \iff a \geq b^2
\end{equation*}
Напишите обоснованный ответ какими свойствами обладает или не обладает отношение и почему:   
\begin{enumerate} [a)]\setcounter{enumi}{0}
\item рефлексивность
\item антирефлексивность
\item симметричность
\item асимметричность
\item антисимметричность
\item транзитивность
\end{enumerate}

Обоснуйте свой ответ по каждому из приведенных ниже вопросов:
\begin{enumerate} [a)]\setcounter{enumi}{0}
    \item Является ли это отношение отношением эквивалентности?
    \item Является ли это отношение функциональным?
    \item Каким из отношений соответствия (одно-многозначным, много-многозначный и т.д.) оно является?
    \item К каким из отношений порядка (полного, частичного и т.д.) можно отнести данное отношение?
\end{enumerate}


\question
Установите, является ли каждое из перечисленных ниже отношений на А ($R \subseteq A \times A$) отношением эквивалентности (обоснование ответа обязательно). Для каждого отношения эквивалентности постройте классы 
эквивалентности и постройте граф отношения:
\begin{enumerate} [a)]\setcounter{enumi}{0}
\item А - множество целых чисел и отношение $R = \{(a,b)|a + b = 5\}$
\item Пусть A – множество имен. $A = \{ $Алексей, Иван, Петр, Александр, Павел, Андрей$ \}$. Тогда отношение $R $ верно на парах имен, начинающихся с одной и той же буквы, и только на них.
\item На множестве $A = \{1; 2; 3; 4; 5\}$ задано отношение $R = \{(1; 2); (1; 3); (1; 5); (2; 3); (2; 4); (2; 5); (3; 4); (3; 5); (4; 5)\}$
\end{enumerate}\question Составьте полную таблицу истинности, определите, какие переменные являются фиктивными и проверьте, является ли формула тавтологией:
$ P \rightarrow (Q \rightarrow ((P \lor Q) \rightarrow (P \land Q)))$

\end{questions}
\newpage
%%% begin test
\begin{flushright}
\begin{tabular}{p{2.8in} r l}
%\textbf{\class} & \textbf{ФИО:} & \makebox[2.5in]{\hrulefill}\\
\textbf{\class} & \textbf{ФИО:} &Юлдашев Алишер\\

\textbf{\examdate} &&\\
%\textbf{Time Limit: \timelimit} & Teaching Assistant & \makebox[2in]{\hrulefill}
\end{tabular}\\
\end{flushright}
\rule[1ex]{\textwidth}{.1pt}


\begin{questions}
\question
Найдите и упростите P:
\begin{equation*}
\overline{P} = A \cap C \cup \overline{A} \cap \overline{C} \cup \overline{B} \cap C \cup \overline{A} \cap \overline{B}
\end{equation*}
Затем найдите элементы множества P, выраженного через множества:
\begin{equation*}
A = \{0, 3, 4, 9\}; 
B = \{1, 3, 4, 7\};
C = \{0, 1, 2, 4, 7, 8, 9\};
I = \{0, 1, 2, 3, 4, 5, 6, 7, 8, 9\}.
\end{equation*}\question
Упростите следующее выражение с учетом того, что $A\subset B \subset C \subset D \subset U; A \neq \O$
\begin{equation*}
A \cap  \overline{C} \cup B \cap \overline{D} \cup  \overline{A} \cap C \cap  \overline{D}
\end{equation*}

Примечание: U — универсум\question
Дано отношение на множестве $\{1, 2, 3, 4, 5\}$ 
\begin{equation*}
aRb \iff |a-b| = 1
\end{equation*}
Напишите обоснованный ответ какими свойствами обладает или не обладает отношение и почему:   
\begin{enumerate} [a)]\setcounter{enumi}{0}
\item рефлексивность
\item антирефлексивность
\item симметричность
\item асимметричность
\item антисимметричность
\item транзитивность
\end{enumerate}

Обоснуйте свой ответ по каждому из приведенных ниже вопросов:
\begin{enumerate} [a)]\setcounter{enumi}{0}
    \item Является ли это отношение отношением эквивалентности?
    \item Является ли это отношение функциональным?
    \item Каким из отношений соответствия (одно-многозначным, много-многозначный и т.д.) оно является?
    \item К каким из отношений порядка (полного, частичного и т.д.) можно отнести данное отношение?
\end{enumerate}

\question
Установите, является ли каждое из перечисленных ниже отношений на А ($R \subseteq A \times A$) отношением эквивалентности (обоснование ответа обязательно). Для каждого отношения эквивалентности постройте классы 
эквивалентности и постройте граф отношения:
\begin{enumerate} [a)]\setcounter{enumi}{0}
\item Пусть A – множество имен. $A = \{ $Алексей, Иван, Петр, Александр, Павел, Андрей$ \}$. Тогда отношение $R$ верно на парах имен, начинающихся с одной и той же буквы, и только на них.
\item $A = \{-10, -9, … , 9, 10\}$ и отношение $ R = \{(a,b)|a^{2} = b^{2}\}$
\item На множестве $A = \{1; 2; 3\}$ задано отношение $R = \{(1; 1); (2; 2); (3; 3); (3; 2); (1; 2); (2; 1)\}$
\end{enumerate}\question Составьте полную таблицу истинности, определите, какие переменные являются фиктивными и проверьте, является ли формула тавтологией:
$(P \rightarrow (Q \rightarrow R)) \rightarrow ((P \rightarrow Q) \rightarrow (P \rightarrow R))$

\end{questions}
\newpage



\end{document}
